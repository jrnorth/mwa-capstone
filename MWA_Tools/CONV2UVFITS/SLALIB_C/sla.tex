%%%%%%%%%%%%%%%%%%%%%%%%%%%%%%%%%%
%                                %
%  SLALIB/C Programmer's Manual  %
%                                %
%%%%%%%%%%%%%%%%%%%%%%%%%%%%%%%%%%

% ----------------
% Version and date
% ----------------

\newcommand{\docrev} {3.0}
\newcommand{\revdate} {18th December 2005}

% ----------------
% Document variant
% ----------------
 
\def\vfull{vfull}             % full (i.e. normal) version
\def\vshort{vshort}           % abridged version
 
\def\vdoc{vfull}

% ------------------------------------
% Number of pages in unabridged manual
% ------------------------------------
 
\newcommand{\npages} {243}

% -------------------
% Number of functions
% -------------------
 
\newcommand{\nroutines} {186}

% ----------
% Stationery
% ----------
\def\Europaper{Europaper}     % European A4
\def\USpaper{USpaper}         % US letter
 
\def\papersize{Europaper}
 
% ------------------
% Document licensing
% ------------------
 
\def\TPS{TPS}                 % Tpoint Software
\def\RAL{RAL}                 % Rutherford Appleton Lab
 
\def\license{RAL}

%-------------------------------------------------------------------------------
 
\documentclass[11pt,fleqn,twoside]{article}
\pagestyle{headings}
\pagenumbering{roman}
 
\renewcommand{\_}{{\tt\char'137}}     % re-centres the underscore
 
%-------------------------------------------------------------------------------
 
% Definitions to suit stationery
\ifx\papersize\Europaper
\setlength{\textwidth}{160mm}       %
\setlength{\textheight}{230mm}      % European A4
\setlength{\topmargin}{5mm}         %
\else
\setlength{\textwidth}{167mm}       %
\setlength{\textheight}{220mm}      % US Letter
\setlength{\topmargin}{5mm}         %
\fi
 
%-------------------------------------------------------------------------------
 
\setlength{\oddsidemargin}{0mm}
\setlength{\evensidemargin}{0mm}
\setlength{\parindent}{0mm}
\setlength{\parskip}{\medskipamount}
\setlength{\unitlength}{1mm}
 
%----------------------------------------------------------
% Document specific \newcommand or \newenvironment commands
%----------------------------------------------------------
 
\newcommand{\tpoint}    {{\it Published by:~~~}\\[1mm]
                         {\Large \tt Tpoint}
                         \begin{picture}(7,7)(-3.5,-1.2)
                         \put(0,0){\circle{6}}
                         \put(-4,0){\line(1,0){2.5}}
                         \put(4,0){\line(-1,0){2.5}}
                         \put(-1.08,-0.94){\mbox{\boldmath$\ast$}}
                         \put(0,-4){\line(0,1){2.5}}
                         \put(0,4){\line(0,-1){2.5}}
                         \end{picture}
                         {\Large \tt Software} \\[2mm]
                         19~Sutton~Wick~Lane, Drayton, \\
                         Abingdon, Oxfordshire, OX14~4HH \\
                         United~Kingdom\\[3mm]
                         {\it www.tpsoft.demon.co.uk}}
\newcommand{\radec}     {$[\,\alpha,\delta\,]$}
\newcommand{\hadec}     {$[\,h,\delta\,]$}
\newcommand{\xieta}     {$[\,\xi,\eta\,]$}
\newcommand{\azel}      {$[\,Az,El~]$}
\newcommand{\ecl}       {$[\,\lambda,\beta~]$}
\newcommand{\gal}       {$[\,l^{I\!I},b^{I\!I}\,]$}
\newcommand{\xy}        {$[\,x,y\,]$}
\newcommand{\xyz}       {$[\,x,y,z\,]$}
\newcommand{\xyzd}      {$[\,\dot{x},\dot{y},\dot{z}\,]$}
\newcommand{\xyzxyzd}   {$[\,x,y,z,\dot{x},\dot{y},\dot{z}\,]$}
\newcommand{\degree}[2] {$#1^{\circ}
                        \hspace{-0.37em}.\hspace{0.02em}#2$}
\newcommand{\arcseci}[1] {$#1\hspace{0.05em}$\raisebox{-0.5ex}
                         {$^{'\hspace{-0.1em}'\hspace{0.0em}}$}}
\newcommand{\arcsec}[2] {\arcseci{#1}$\hspace{-0.35em}.#2$}
\newcommand{\dms}[4]    {$#1^{\circ}\,#2\raisebox{-0.5ex}
                        {$^{'}$}\,$\arcsec{#3}{#4}}
\newcommand{\tseci}[1]   {$#1$\mbox{$^{\rm s}$}}
\newcommand{\tsec}[2]    {\tseci{#1}$\hspace{-0.3em}.#2$}
\newcommand{\hms}[4]    {$#1^{\rm h}\,#2^{\rm m}\,$\tsec{#3}{#4}}
\newcommand{\callhead}[1]{\goodbreak\vspace{\bigskipamount}{\large\bf{#1}}}
\newenvironment{callset}{\begin{list}{}{\setlength{\leftmargin}{2cm}
                         \setlength{\parsep}{\smallskipamount}}}{\end{list}}
\newcommand{\subp}[1]{\item\hspace{-1cm}{\tt #1}\\}
\newcommand{\subq}[2]{\item\hspace{-1cm}{\tt #1}\\\hspace*{-1cm}{\tt #2}\\}
\newcommand{\fortvar}[1]{\mbox{\em #1}}
 
\newcommand{\routine}[3]
{\hbadness=10000
  \vbox
  {
    \rule{\textwidth}{0.3mm}\\
    {\Large {\bf #1} \hfill #2 \hfill {\bf #1}}\\
    \setlength{\oldspacing}{\topsep}
    \setlength{\topsep}{0.3ex}
    \begin{description}
      #3
    \end{description}
    \setlength{\topsep}{\oldspacing}
  }
}
 
\newcommand{\action}[1]
{\item[ACTION]: #1}
 
\newcommand{\call}[1]
{\item[CALL]: \hspace{0.4em}{\tt #1}}
\newlength{\oldspacing}
 
\newcommand{\args}[2]
{
  \goodbreak
  \setlength{\oldspacing}{\topsep}
  \setlength{\topsep}{0.3ex}
  \begin{description}
  \item[#1]:\\[1.5ex]
    \begin{tabular}{p{6.7em}p{6.8em}p{22em}}
      #2
    \end{tabular}
  \end{description}
  \setlength{\topsep}{\oldspacing}
}
 
\newcommand{\spec}[3]
{
  {\tt \mbox{#1}} & {\tt \mbox{#2}} & {#3}
}
 
\newcommand{\specel}[2]
{
  {} & \multicolumn{1}{c}{\hspace{-2em} \tt [#1]} & {$\circ$ #2}
}
 
\newcommand{\anote}[1]
{
  \goodbreak
  \setlength{\oldspacing}{\topsep}
  \setlength{\topsep}{0.3ex}
  \begin{description}
    \item[NOTE]:
        #1
  \end{description}
  \setlength{\topsep}{\oldspacing}
}
 
\newcommand{\notes}[1]
{
  \goodbreak
  \setlength{\oldspacing}{\topsep}
  \setlength{\topsep}{0.3ex}
  \begin{description}
    \item[NOTES]:
        #1
  \end{description}
  \setlength{\topsep}{\oldspacing}
}
 
\newcommand{\aref}[1]
{
  \goodbreak
  \setlength{\oldspacing}{\topsep}
  \setlength{\topsep}{0.3ex}
  \begin{description}
    \item[REFERENCE]:
        #1
  \end{description}
  \setlength{\topsep}{\oldspacing}
}
 
\newcommand{\refs}[1]
{
  \goodbreak
  \setlength{\oldspacing}{\topsep}
  \setlength{\topsep}{0.3ex}
  \begin{description}
    \item[REFERENCES]:
        #1
  \end{description}
  \setlength{\topsep}{\oldspacing}
}
 
\newcommand{\exampleitem}{\item [EXAMPLE]:}
 
\newcommand{\fstring}[1]{\hspace{0.05em}\mbox{$'${\tt#1}\hspace{0.09em}$'$}}
 
\newcommand{\cstring}[1]{\mbox{{\tt "#1"}}}

\newcommand{\abridged}  {\vspace{3ex} \begin{center}
                        \framebox(50,14){\vspace*{1mm}
                        \large \it material omitted}
                        \end{center}\vspace{3ex}}
 
%----------------------------------
% End of document specific commands
%----------------------------------
 
%%%%%%%%%%%%%%%%%%%%%%%%%%%%%%%%%%%%%%%%%%%%%%%%%%%%%%%%%%%%%%%%%%%%%%%%%%%%%%%%
 
\begin{document}
\thispagestyle{empty}
\begin{center}
\vspace*{30mm}~
 
\framebox(40,18){\vspace*{-1mm} \Huge SLALIB\,}
 
\vspace{20mm}
 
{\huge \bf SLALIB/C Programmer's Manual}
 
\vspace{10mm}
{\Large \bf Patrick Wallace}
 
\ifx\vdoc\vshort
\vspace{5mm}
{\Large \it Abridged Version}
\fi
 
\vspace{20mm}
 
Version \docrev \\
\revdate
 
\vfill
 
\ifx\license\TPS
\tpoint
\else
{\it SLALIB/C is licensed to:}\\[3mm]
{\bf Space Science \& Technology Department, \\
Rutherford Appleton Laboratory, \\
United Kingdom}
\fi
 
\vspace{8mm}
 
{\footnotesize \bf TCSpk (TM) and TPOINT (TM) are trademarks.}
\end{center}
 
\newpage
\thispagestyle{empty}
~
\vfill
\begin{center}
\ifx\license\RAL
\tpoint
\vfill
\fi
 
\parbox{20em}{\small
Information in this document is subject to change without notice.
 
\vspace{4mm}
 
\copyright\ Copyright 2005 Tpoint Software, UK, and
P.\,T.\,Wallace. All Rights Reserved. Reproduction, adaptation, or
translation without prior written permission is prohibited, except
as allowed under the copyright laws.  Transmission of text or
content to third parties, including disclosure via the Internet,
is prohibited. }
\end{center}
~
\newpage
 
\setlength{\parskip}{0mm}
\thispagestyle{empty}
\tableofcontents
\setlength{\parskip}{\bigskipamount}
 
 
%------------------------------------------------------------------------------
 
\newenvironment{tabs}{\goodbreak\begin{tabbing}}{\end{tabbing}}
\newenvironment{cmnds}{\begin{tabs}
XXX \= XX \= XXXXXXXXXXXXXXXXXXXX \= \kill}{\end{tabs}}
\newenvironment{cmd}{\begin{tabbing}
XXXXXX \= XX \= XXXXXXXXXXXX \= \kill}{\end{tabbing}}
 
\cleardoublepage
\section{INTRODUCTION}
\pagenumbering{arabic}
 
\ifx\vdoc\vshort
This is {\it abridged version}\/ of the SLALIB/C Programmer's Manual.  It
omits substantial amounts of technical reference material but retains
the descriptive sections, including demonstration applications.
 
The unabridged version is \npages\ pages in length.
\fi

\subsection{Purpose}
SLALIB/C\footnote{The name SLALIB isn't an acronym;
it just stands for ``Subprogram Library~A''.}
is a library of ANSI C functions
that make accurate and reliable positional-astronomy
applications easier to write.
Most SLALIB functions
are concerned with astronomical position and time, but a
number have wider trigonometrical, numerical or general applications.
The Tpoint Software applications TPOINT and TCSpk
make extensive use of SLALIB/C
functions, and the library plays a key role in pointing
and tracking large telescopes and radio antennas around the world.
 
\subsection{Example Application}
\label{exDafin}
Here is a simple example of an application program written
using SLALIB/C calls:
\vspace{-3ex}
\begin{verbatim}
   #include <stdio.h>
   #include <string.h>
   #include <slalib.h>
 
   /*
   **  Read a B1950 position from stdin and reply on stdout
   **  with the J2000 equivalent.  Enter "q" to quit.
   **
   */
 
   int main ( )
   {
      char b[80], s;
      int bad, i, j, i4[4];
      double r4, d4, r5, d5;
 
 
   /* Loop indefinitely. */
      for ( ; ; ) {
 
      /* Read a string, either h m s d ' ", or q to exit. */
         gets ( b );
 
      /* Exit if "q" entered. */
         if ( b[0] == (char) 'q' ) return 0;
 
      /* Preset the status to fail. */
         bad = 1;
 
      /* Decode the FK4 right ascension. */
         i = 1;
         slaDafin ( b, &i, &r4, &j );
         if ( !j ) {
            r4 *= 15.0;
 
         /* Decode the FK4 declination. */
            slaDafin ( b, &i, &d4, &j );
            if ( !j ) {
 
            /* FK4 to FK5. */
               slaFk45z ( r4, d4, 1950.0, &r5, &d5 );
 
            /* Format and output the result. */
               slaDr2tf ( 2, r5, &s, i4 );
               printf ( "%2.2d %2.2d %2.2d.%2.2d",
                        i4[0],i4[1],i4[2],i4[3] );
               slaDr2af ( 1, d5, &s, i4 );
               printf ( "  %c%2.2d %2.2d %2.2d.%1d\n",
                        s,i4[0],i4[1],i4[2],i4[3] );
 
            /* Success. */
               bad = 0;
            }
         }
 
      /* Warn if malformed input. */
         if ( bad ) printf ( "?\n" );
 
      /* Next RA,Dec. */
      }
   }
\end{verbatim}
\vspace{-3ex}
In this example, SLALIB not only provides the complicated FK4 to
FK5 transformation but also
simplifies the tedious and error-prone tasks
of decoding and formatting angles
expressed as hours, minutes {\it etc}.  The
example incorporates range checking, and avoids the
notorious ``minus zero'' problem (an often-perpetrated bug where
declinations between $0^{\circ}$ and $-1^{\circ}$ lose their minus
sign).
With a little extra elaboration and a few more calls to SLALIB,
defaulting can be provided (enabling unused fields to
be replaced with commas to avoid retyping), proper motions
can be handled, different epochs can be specified, and
so on.
 
\subsection{Scope}
SLALIB contains \nroutines\ functions covering the following topics:
\begin{itemize}
\item string decoding,
      sexagesimal conversions
\item angles, vectors \& rotation matrices,
\item calendars and time-scales,
\item sidereal time,
\item precession \& nutation,
\item proper motion,
\item FK4/FK5/Hipparcos,
      elliptic aberration,
\item geocentric coordinates,
\item apparent \& observed place,
\item azimuth \& elevation\footnote{The correct astronomical
term for distance above the horizontal is {\it altitude,}\/ but the
present document follows the now well-established convention of
using the word {\it elevation}\/ instead.  Although the latter is
really an artillery term, it does have the advantage
for computer applications of not sharing an
initial with ``azimuth''.},
\item refraction \& air mass,
\item ecliptic,
      galactic,
      supergalactic coordinates,
\item ephemerides,
\item astrometry,
\item numerical methods.
\end{itemize}
 
\subsection{Objectives}
SLALIB was designed to give application programmers
a basic set of positional-astronomy tools which are
accurate and easy to use.  To this end, the library is:
\begin{itemize}
\item Available as source code, and documented.
\item Supported, maintained and actively developed.
\item Portable -- coded in ANSI standard C, and consequently
usable on essentially all computers and operating systems.
Arguments followed the usual conventions: by value
(except for arrays and strings of course)
for given arguments and by pointer for returned arguments.
All the C functions are re-entrant: SLALIB/C can be run in a multi-threaded
application and under such operating systems as VxWorks.
\item Thoroughly commented, so that the source code can be used as
a learning aid.
\item Stable -- changes are rare (and almost always improvements to
internal documentation only) and calling conventions are never changed.
\item Trustworthy -- great care has gone into
testing SLALIB, both by comparison with published data and
by checks for internal consistency.
\item As far as possible, rigorous -- corners are not cut,
even where the practical consequences would, as a rule, be
negligible.
\item Comprehensive, without including too many esoteric features
required only by specialists.
\item Practical -- almost all the functions have been written to
satisfy real needs encountered during the development of
real-life applications.
\item Environment-independent -- the package is
completely free of pauses, stops, I/O {\it etc.}.
\item Self-contained -- SLALIB calls no other libraries.
\end{itemize}
A few {\it caveats}:
\begin{itemize}
\item SLALIB does not pretend to be canonical.  It is in essence
an anthology, and the adopted algorithms are liable
to change as more up-to-date ones become available.\footnote{For
IAU-approved algorithms, see {\it www.iau-sofa.rl.ac.uk}.}
\item The functions aren't orthogonal -- there are several
cases of different
ones doing similar things, and many examples where
sequences of SLALIB calls have simply been packaged, all to
make applications less trouble to write.
\item There are gaps -- for example there are no functions for
calculating physical ephemerides of Solar-System bodies.
\item SLALIB is not homogeneous, though important subsets
(for example the FK4/FK5 functions) are.
\item The library is not foolproof.  You have to know what
you are trying to do ({\it e.g.}\ by reading textbooks on positional
astronomy), and it is the caller's responsibility to supply
sensible arguments (although enough internal validation is done to
avoid arithmetic errors).
\item Without being written in a wasteful
manner, SLALIB is nonetheless optimized for maintainability
rather than speed.  In addition, there are many places
where considerable simplification would be possible if some
specified amount of accuracy could be sacrificed;  such
compromises are left to the individual programmer and
are not allowed to limit SLALIB's value as a source
of comparison results.
\end{itemize}
 
\subsection{New Functions}
In a package like SLALIB it is difficult to know how far to go.  Is it
enough to provide the primitive operations, or should more
complicated functions be packaged?  Is it worth encroaching on
specialist areas, where individual experts have all written their
own software already?  To what extent should CPU efficiency be
an issue?  How much support of different numerical precisions is
required?  And so on.
 
In practice, almost all the functions in SLALIB are there because they were
needed for some specific application, and this is likely to remain the
pattern for any enhancements in the future.
Suggestions for additional SLALIB functions should be addressed to the
author.
 
\subsection{History and acknowledgements}
SLALIB is descended from a package of Fortran~66 routines written
by the author
for the Anglo-Australian Observatory 16-bit minicomputers
in the mid-1970s.  These routines were mainly for controlling the
3.9m~AAT and for pulsar observing.  By the end of the 1970s, the
DEC VAX/VMS computer was fast becoming the standard in
astronomy.  With essentially unlimited memory, and fast
double-precision floating point, a much more comprehensive and thorough
positional astronomy package could be contemplated, and this led to the
first SLALIB.  The work was carried out during the early 1980s at the
Rutherford Appleton Laboratory, UK, at a time when the adoption
of the IAU 1976 resolutions meant that astronomers were having to cope
with a mixture of reference systems, time-scales and nomenclature.
 
Preparatory work on this Fortran SLALIB was done by
Althea~Wilkinson of Manchester University, under the auspices of
the UK Starlink Project.
Subsequently,
Andrew~Murray,
Catherine~Hohenkerk,
Andrew~Sinclair,
Bernard~Yallop
and
Brian~Emerson of Her Majesty's Nautical Almanac Office were consulted
on many occasions and provided indispensable advice.
A number of enhancements to SLALIB were at the suggestion of
Russell~Owen, University of Washington,
the late Phil~Hill, St~Andrews University,
Bill~Vacca, JILA, Boulder and
Ron~Maddalena, NRAO and
Mark~Calabretta, CSIRO Radiophysics, Sydney.
Derek~Jones (RGO) advised on the
``universal variables'' method of calculating orbits.
 
The first C version of SLALIB was a hand-coded transcription
of the Starlink Fortran version carried out by
Steve~Eaton (University of Leeds) in the course of
MSc work in the late 1980s.  This was
subsequently enhanced by John~Straede (AAO)
and Martin~Shepherd (Caltech).
The current C SLALIB is a now distant descendant of
a complete {\it ab initio}\/
rewrite that was carried out privately by the present author,
including the development of a comprehensive validation suite.
Useful comments on this early version came from Bob~Payne (NRAO) and
Jeremy~Bailey (AAO).
 
Since its inception, the C SLALIB has become
the principal SLALIB product, though the Fortran version was kept
up to date and continued to be distributed
by the Starlink project until its closure in June 2005.
Permission for Tpoint Software to develop SLALIB/C commercially
was granted by RAL in 1996.
 
\section{COMPILING AND LINKING}
 
Code modules using SLALIB/C must {\tt \#include} the header file
{\tt <slalib.h>}.
 
To compile a program on Unix systems
(Linux, Sun, HP-UX, MacOS-X, DEC Alpha {\it etc.})
with the GNU gcc compiler installed:
\begin{verse}
{\tt \%~~gcc progname.c -I./include -L./lib -lsla -lm -o progname}
\end{verse}
(It may be necessary to point at different -I and -L
directories.)

\vfill

\pagebreak

\section{EXPLANATION AND EXAMPLES}
To guide the writer of positional-astronomy applications software,
this introductory chapter puts the SLALIB functions into the context of
astronomical phenomena and techniques, and presents a few
``cookbook'' examples
of the SLALIB calls in action.  The astronomical content of the chapter
is not, of course, intended to be a substitute for specialist text-books on
positional astronomy, but may help bridge the gap between
such books and the SLALIB functions.  For further reading, the following
cover a wide range of material and styles:
\begin{itemize}
\item {\it Explanatory Supplement to the Astronomical Almanac},
      ed.\ P.\,Kenneth~Seidelmann (1992), University Science Books.
\item {\it Vectorial Astrometry}, C.\,A.\,Murray (1983), Adam Hilger.
\item {\it Spherical Astronomy}, Robin~M.\,Green (1985), Cambridge
      University Press.
\item {\it Spacecraft Attitude Determination and Control},
      ed.\ James~R.\,Wertz (1986), Reidel.
\item {\it Practical Astronomy with your Calculator},
      Peter~Duffett-Smith (1981), Cambridge University Press.
\end{itemize}
Also of considerable value, though out of date in places, are:
\begin{itemize}
\item {\it Explanatory Supplement to the Astronomical Ephemeris
      and the American Ephemeris and Nautical Almanac}, RGO/USNO (1974),
      HMSO.
\item {\it Textbook on Spherical Astronomy}, W.\,M.\,Smart (1977),
      Cambridge University Press.
\end{itemize}
For full details of the latest standard algorithms, see:
\begin{itemize}
\item {\it IERS Conventions (2003)}, Dennis~D.~McCarthy \&
      G\'{e}rard~Petit (eds.), Verlag des Bundesamts f\"{u}r
      Kartographie und Geod\"{a}sie, Frankfurt am Main (2004).
\item {\it IAU Standards Of Fundamental Astronomy},
      {\tt http://www.iau-sofa.rl.ac.uk/}.
\end{itemize}
Only brief details of individual SLALIB functions are given here, and
readers will find it useful to refer to the subprogram specifications
elsewhere in this document.  The source code for the SLALIB functions
is also intended to be used as documentation.
 
\subsection {Spherical Trigonometry}
Celestial phenomena occur at such vast distances from the
observer that for most practical purposes there is no need to
work in 3D;  only the direction
of a source matters, not how far away it is.  Things can
therefore be viewed as if they were happening
on the inside of sphere with the observer at the centre --
the {\it celestial sphere}.  Problems involving
positions and orientations in the sky can then be solved by
using the formulae of {\it spherical trigonometry}, which
apply to {\it spherical triangles}, the sides of which are
{\it great circles}.
 
Positions on the celestial sphere may be specified by using
a spherical polar coordinate system, defined in terms of
some fundamental plane and a direction in that plane chosen to
represent zero longitude.  Mathematicians usually work with the
co-latitude, with zero at the principal pole, whereas most
astronomical coordinate systems use latitude, reckoned plus and
minus from the equator.
Astronomical coordinate systems may be either right-handed
({\it e.g.}\ right ascension and declination \radec,
galactic longitude and latitude \gal)
or left-handed ({\it e.g.}\ hour angle and
declination \hadec).  In some cases
different conventions have been used in the past, a fruitful source of
mistakes.  Azimuth and geographical longitude are examples;  azimuth
is now generally reckoned north through east
(making a left-handed system);  geographical longitude is now usually
taken to increase eastwards (a right-handed system) but astronomers
used to employ a west-positive convention.  In reports
and program comments it is wise to spell out what convention
is being used, if there is any possibility of confusion.
 
When applying spherical trigonometry formulae, attention must be
paid to
rounding errors (for example it is a bad idea to find a
small angle through its cosine) and to the possibility of
problems close to poles.
Also, if a formulation relies on inspection to establish
the quadrant of the result, it is a sure sign that a vector-related
method will be preferable.
 
As well as providing many functions which work in terms of specific
spherical coordinates such as \radec, SLALIB provides
two functions which operate directly on generic spherical
coordinates:
{\tt slaSep}
computes the separation between
two points (the distance along a great circle) and
{\tt slaBear}
computes the bearing (or {\it position angle})
of one point seen from the other.  The functions
{\tt slaDsep}
and
{\tt slaDbear}
are double precision equivalents.\footnote{The provision
in some cases of {\tt float}
precision functions in addition to the more usual {\tt double} is to
some extent a product of SLALIB's long history and its Fortran
background.  However, the {\tt float} versions sometimes have a role
in applications with very tight memory and CPU constraints such
as spaceborne instrumentation.}
As a simple demonstration
of SLALIB/C, we will use these facilities to estimate the distance from
London to Sydney and the initial compass heading:
\goodbreak
\vspace{-3ex}
\begin{verbatim}
      #include <stdio.h>
      #include <slalib.h>
      #define R2D 57.2957795     /* radians to degrees */
      #define RKM 6375.0         /* Earth radius in km */
      int main ( )
      {
      /* Longitudes and latitudes (radians) for London and Sydney. */
         double al =  -0.2/R2D, bl =  51.5/R2D,
                as = 151.2/R2D, bs = -33.9/R2D;

      /* Great-circle distance and initial heading. */
         printf ( "%5.0f km,%4.0f deg\n", slaDsep(al,bl,as,bs)*RKM,
                                          slaDbear(al,bl,as,bs)*R2D );
         return 0;
      }
\end{verbatim}
\vspace{-3ex}
(The result is 17011~km, $61^\circ$.)
 
The functions
{\tt slaSepv},
{\tt slaDsepv},
{\tt slaPav},
{\tt slaDpav}
are equivalents of
{\tt slaSep},
{\tt slaDsep},
{\tt slaBear} and
{\tt slaDbear}
but starting from vectors instead of spherical coordinates.

\subsubsection{Formatting angles}
SLALIB has functions for decoding decimal numbers
from character form and for converting angles to and from
sexagesimal form (hours, minutes, seconds or degrees,
arcminutes, arcseconds).  These apparently straightforward
operations contain hidden traps which the SLALIB functions avoid.
 
There are five functions for decoding numbers from a character
string, such as might be entered using a keyboard.
They all work in the same style, and successive calls
can work their way along a single string, decoding
a sequence of numbers of assorted types.  Number
fields can be separated by spaces or commas, and can be defaulted
to previous values or to preset defaults.
Four of the functions decode single numbers:
{\tt slaInt2in}
(integer),
{\tt slaIntin}
(long integer),
{\tt slaFlotin}
(single precision floating point) and
{\tt slaDfltin}
(double precision).  A minus sign can be
detected even when the number is zero;  this avoids
the frequently-encountered ``minus zero'' bug, where
declinations {\it etc.}\ in
the range $0^{\circ}$ to $-1^{\circ}$ mysteriously migrate to
the range $0^{\circ}$ to $+1^{\circ}$.

Here is an example C program where we wish to
read up to ten double precision numbers, defaulting to the
previous values each time.  The status is reported, showing
among other things whether the field was defaulted or was
negative even when zero.  (Fields can be defaulted by omission
or, individually, by using a comma.)
\goodbreak
\vspace{-3ex}
\begin{verbatim}
      #include <stdio.h>
      #include <slalib.h>

      /* Read a string and decode it into doubles; "q" to quit. */

      #define N 10    /* maximum number of fields */
      int main ( )
      {
         char b[100];
         int n, i, j;
         double d[N];


      /* Initial default values. */
         for ( n = 0; n < N; d[n++] = (double) n );

      /* Read strings until q entered. */
         while ( 1 ) {
            gets ( b );
            if ( b[0] == (char) 'q' ) return 0;

         /* Decode and report N numbers. */
            i = 1;
            for ( n = 0; n < N; n++ ) {
               slaDfltin ( b, &i, &d[n], &j );
               printf ( "value = %.20g, status = %d\n", d[n], j );
            }
         }
      }
\end{verbatim}
\vspace{-3ex}
\goodbreak
Two additional functions decode a 3-field sexagesimal number:
{\tt slaAfin}
(degrees, arcminutes, arcseconds to single
precision radians) and
{\tt slaDafin}
(the same but double precision).  They also
work using other units such as hours {\it etc}.\ if
you multiply the result by the appropriate factor.  An example
C program which uses
{\tt slaDafin}
was given earlier, in Section~\ref{exDafin}.

SLALIB provides four functions for expressing an angle in radians
in a preferred range.  The function
{\tt slaRange}
expresses an angle
in the range $\pm\pi$;
{\tt slaRanorm}
expresses an angle in the range
0-2$\pi$.  The functions
{\tt slaDrange}
and
{\tt slaDranrm}
are double precision versions.

Several functions
({\tt slaCtf2d},
{\tt slaCr2af}
{\it etc.}) are provided to convert
angles to and from
sexagesimal form (hours, minute, seconds or degrees,
arcminutes and arcseconds).
They avoid the common
``inconsistent rounding''
bug (as well as the ``minus zero'' bug),
which produces angles like \hms{24}{59}{59}{999}.
Here is a program which displays an hour angle
stored in radians, using two different resolutions:
\goodbreak
\vspace{-3ex}
\begin{verbatim}
      #include <stdio.h>
      #include <slalib.h>
      int main ( )
      {
         double ha;
         char sign;
         int ihmsf[4];

         ha = -0.261799315;
         slaDr2tf ( 3, ha, &sign, ihmsf );
         printf ( "%c%2.2d %2.2d %2.2d.%3.3d\n",
                  sign, ihmsf[0], ihmsf[1], ihmsf[2], ihmsf[3] );
         slaDr2tf ( 2, ha, &sign, ihmsf );
         printf ( "%c%2.2d %2.2d %2.2d.%2.2d\n",
                  sign, ihmsf[0], ihmsf[1], ihmsf[2], ihmsf[3] );
         return 0;
      }
\end{verbatim}
\vspace{-3ex}
The output is:
\vspace{-3ex}
\begin{verbatim}
      -00 59 59.999
      -01 00 00.00
\end{verbatim}
\vspace{-3ex}
\goodbreak

\subsection {Vectors and Matrices}
As an alternative to employing a spherical polar coordinate system,
the direction of an object can be defined in terms of the sum of any
three vectors as long as they are different and
not coplanar.  In practice, three vectors at right angles are
chosen, forming a system
of {\it Cartesian coordinates}.  The {\it x}- and {\it y}-axes
lie in the fundamental plane ({\it e.g.}\ the equator in the
case of \radec), with the {\it x}-axis pointing to zero longitude.
The {\it z}-axis is normal to the fundamental plane and points
towards positive latitudes.  The {\it y}-axis can lie in either
of the two possible directions, depending on whether the
coordinate system is right-handed or left-handed.
The three axes are sometimes called
a {\it triad}.  For most applications involving arbitrarily
distant objects such as stars, the vector which defines
the direction concerned is constrained to have unit length.
The {\it x}-, {\it y-} and {\it z-}components
can be regarded as the scalar (dot) product of this vector
onto the three axes of the triad in turn.  Because the vector
is a unit vector,
each of the three dot-products is simply the cosine of the angle
between the unit vector and the axis concerned, and the
{\it x}-, {\it y-} and {\it z-}components are sometimes
called {\it direction cosines}.

For some applications, including those involving objects
within the Solar System, unit vectors are inappropriate, and
it is necessary to use vectors scaled in length-units such as
AU, km {\it etc.}
In these cases the origin of the coordinate system might not be
the observer, but instead might be the Sun, the Solar-System
barycentre, the centre of the Earth {\it etc.}  But whatever the application,
the final direction in which the observer sees the object can be
expressed as direction cosines.

But where has this got us?  Instead of two numbers---a longitude and
a latitude---we now have three numbers to look after
-- the {\it x}-, {\it y-} and
{\it z-}components, whose quadratic sum we have somehow to contrive to
be unity.  And, in addition to this apparent redundancy,
most people find it harder to visualize
problems in terms of \xyz\ than in $[\,\theta,\phi~]$.
Despite these objections, the vector approach turns out to have
significant advantages over the spherical trigonometry approach:
\begin{itemize}
\item Vector formulae tend to be much more succinct;  one vector
      operation is the equivalent of strings of sines and cosines.
\item The formulae are as a rule rigorous, even at the poles.
\item Accuracy is maintained all over the celestial sphere.
      When one Cartesian component is nearly unity and
      therefore insensitive to direction, the others become small
      and therefore more precise.
\item Formulations usually deliver the quadrant of the result
      without the need for any inspection (except within the
      library function {\tt atan2}).
\end{itemize}
A number of important transformations in positional
astronomy turn out to be nothing more than changes of coordinate
system, something which is especially convenient if
the vector approach is used.  A direction with respect
to one triad can be expressed relative to another triad simply
by multiplying the \xyz\ column vector by the appropriate
$3\times3$ orthogonal matrix
(a tensor of Rank~2, or {\it dyadic}).  The three rows of this
{\it rotation matrix}\/
are the vectors in the old coordinate system of the three
new axes, and the transformation amounts to obtaining the
dot-product of the direction-vector with each of the three
new axes.  Precession, nutation, \hadec\ to \azel,
\radec\ to \gal\ and so on are typical examples of the
technique.  A useful property of the rotation matrices
is that they can be inverted simply by taking the transpose.
 
The elements of these vectors and matrices are assorted combinations of
the sines and cosines of the various angles involved (hour angle,
declination and so on, depending on which transformation is
being applied).  If you write out the matrix multiplications
in full you get expressions which are essentially the same as the
equivalent spherical trigonometry formulae.  Indeed, many of the
standard formulae of spherical trigonometry are most easily
derived by expressing the problem initially in
terms of vectors.

\subsubsection{Using vectors}
SLALIB provides transformations between spherical and vector
form
({\tt slaCs2c},
{\tt slaCc2s}
{\it etc.}), plus an assortment
of standard vector and matrix operations
({\tt slaVdv},
{\tt slaMxv}
{\it etc.}).
There are also functions
({\tt slaEuler}
{\it etc.}) for creating a rotation matrix
from three {\it Euler angles}\/ (successive rotations about
specified Cartesian axes).  Instead of Euler angles, a rotation
matrix can be expressed as an {\it axial vector}\/\footnote{The
name ``axial vector'' is specific to SLALIB;  the IAU SOFA
software calls them ``rotation vectors'' or ``r-vectors''.  Indeed
there seems to be little mention of this concise 3-number
parametrization in the literature.
The reason is probably that the closely-related ``Euler symmetric
parameters'' parametrization offers some of the advantages and
is computationally exceptionally efficient:  many applications
involving complicated sequences of rotations can be done
without computing trig functions.  However, the axial vector
approach has advantages of its own, including intuitive appeal and the
ability to express multi-turn rotations.} (the pole of the rotation,
scaled by the amount of rotation), and functions are provided for this
({\tt slaAv2m},
{\tt slaM2av}
{\it etc.}).

Here is a fragment of C code
where spherical coordinates [{\tt p},{\tt q}]
undergo a coordinate transformation;
the transformation consists of a rotation of the coordinate system
through angles {\tt a}, {\tt b} and {\tt c} radians about the
{\it z}, {\it y}\/ and {\it z}\/ axes respectively
({\it i.e.}~${\bf R}=R_3({\tt c}) \cdot R_2({\tt b}) \cdot R_3({\tt a})$):
\goodbreak
\vspace{-3ex}
\begin{verbatim}
          :
         double p, q, a, b, c, r[3][3], v[3];
          :
      /* Create rotation matrix. */
         slaEuler ( "zyz", a, b, c, r );
 
      /* Transform position (p,q) from spherical to Cartesian. */
         slaCs2c ( p, q, v );
 
      /* Apply the rotation. */
         slaMxv ( e, v, v );
 
      /* Back to spherical. */
         slaCc2s ( v, &p, &q );
          :
\end{verbatim}
\vspace{-3ex}
\goodbreak
Small adjustments to the direction of a position
vector are often most conveniently described in terms of
$[\,\Delta x,\Delta y, \Delta z\,]$.  Adding the correction
vector needs careful handling if the position
vector is to remain of length unity, an advisable precaution which
ensures that
the \xyz\ components are always available to mean the cosines of
the angles between the vector and the axis concerned.  Two types
of shifts are commonly used,
the first where a small vector of arbitrary direction is
added to the unit vector, and the second where there is a displacement
in the latitude coordinate (declination, elevation {\it etc.}) alone.
 
For a shift produced by adding a small \xyz\ vector ${\bf d}$ to a
unit vector ${\bf v}_1$, the resulting vector ${\bf v}_2$ has direction
$<{\bf v}_1+{\bf d}>$ but is no longer of unit length.  A better approximation
is available if the result is multiplied by a scaling factor of
$(1-{\bf d}.{\bf v}_1)$, where the dot
means scalar product.  In C:
\goodbreak
\vspace{-3ex}
\begin{verbatim}
         f = (1.0-(dx*v1x+dy*v1y+dz*v1z));
         v2x = f*(v1x+dx);
         v2y = f*(v1y+dy);
         v2z = f*(v1z+dz);
\end{verbatim}
\vspace{-3ex}
\goodbreak
The correction for diurnal aberration (discussed later) is
an example of this form of shift.

As an example of the second kind of displacement
we will apply a small change in elevation $\Delta E$ to an
\azel\ direction vector.  The direction of the
result can be obtained by making the allowable approximation
${\tan \Delta E\approx\Delta E}$ and adding a adjustment
vector of length $\Delta E$ normal
to the direction vector in the vertical plane containing the direction
vector.  The $z$-component of the adjustment vector is
$\Delta E \cos E$,
and the horizontal component is
$\Delta E \sin E$ which has then to be
resolved into $x$ and $y$ in proportion to their current sizes. To
approximate a unit vector more closely, a correction factor of
$\cos \Delta E$ can then be applied, which is nearly
$(1-\Delta E^2 /2)$ for
small $\Delta E$.  Expressed in C code, for initial vector
{\tt v1x,v1y,v1z}, change in elevation {\tt del}
(positive $\equiv$ upwards), and result
vector {\tt v2x,v2y,v2z}:
\goodbreak
\vspace{-3ex}
\begin{verbatim}
         cosdel = 1.0-del*del/2.0;
         r1 = sqrt(v1x*v1x+v1y*v1y);
         f = cosdel*(r1-del*v1z)/r1;
         v2x = f*v1x;
         v2y = f*v1Y;
         v2z = cosdel*(v1z+del*r1);
\end{verbatim}
\vspace{-3ex}
\goodbreak
An example of this type of shift is the correction for atmospheric
refraction (discussed later).
Depending upon the relationship between $\Delta E$ and $E$, special
handling at the pole (the zenith for our example) may be required.

SLALIB includes functions for the case where both a position
and a velocity are involved.  The functions
{\tt slaCs2c6}
and
{\tt slaCc62s}
convert from $[\theta,\phi,\dot{\theta},\dot{\phi}]$
to \xyzxyzd\ and back;
{\tt slaDs2c6}
and
{\tt slaDc62s}
are double precision equivalents.

\subsection {Celestial Coordinate Systems}
SLALIB has functions to perform transformations
of celestial positions between different spherical
coordinate systems, including those shown in the following table:

\begin{center}
\begin{tabular}{|l|c|c|c|c|c|c|} \hline
{\it system} & {\it symbols} & {\it longitude} & {\it latitude} &
          {\it x-y plane} & {\it long.\ zero} & {\it RH/LH}
\\ \hline \hline
horizon & -- & azimuth & elevation & horizontal & north & L
\\ \hline
equatorial & $\alpha,\delta$ & R.A.\ & Dec.\ & equator & equinox & R
\\ \hline
local equ.\ & $h,\delta$ & H.A.\ & Dec.\ & equator & meridian & L
\\ \hline
ecliptic & $\lambda,\beta$ & ecl.\ long.\ & ecl.\ lat.\ &
                                       ecliptic & equinox & R
\\ \hline
galactic & $l^{I\!I},b^{I\!I}$ & gal.\ long.\ & gal.\ lat.\ &
                                       gal.\ equator & gal.\ centre & R
\\ \hline
supergalactic & SGL,SGB & SG long.\ & SG lat.\ &
                                       SG equator & node w.\ gal.\ equ.\ & R
\\ \hline
\end{tabular}
\end{center}

Transformations between \hadec\ and \azel\ can be performed by
calling
{\tt slaE2h}
and
{\tt slaH2e},
or, in double precision,
{\tt slaDe2h}
and
{\tt slaDh2e}.
There is also a function for obtaining
zenith distance alone for a given \hadec,
{\tt slaZd},
and one for determining the parallactic angle,
{\tt slaPa}.
Three functions are included which relate to altazimuth telescope
mountings.  For a given \hadec\ and latitude,
{\tt slaAltaz}
returns the azimuth, elevation and parallactic angle, plus
velocities and accelerations for sidereal tracking.
The functions
{\tt slaPda2h}
and
{\tt slaPdq2h}
predict at what hour angle a given azimuth or
parallactic angle will be reached.  ({\it n.b.}~both ``elevation''
and ``parallactic angle'' are somewhat contentious terms, though
firmly entrenched.  ``Elevation'' is borrowed from gunnery;
the correct astronomical term is
``altitude''. ``Parallactic angle'', which now means the
angle between the direction to the zenith and
the direction to equatorial north and should really be
``vertical angle'', seems originally to have meant
instead the angle between the direction to ecliptic north and the
direction to equatorial north.)

The functions
{\tt slaEqecl}
and
{\tt slaEcleq}
transform between ecliptic
coordinates \ecl\ and equatorial coordinates
\radec\/; there is also a function for generating the
equatorial to ecliptic rotation matrix for a given date:
{\tt slaEcmat}.
 
For transformation between galactic coordinates \gal\
and equatorial coordinates \radec\ there are
two sets of functions, depending on whether the \radec\ is
old-style, B1950, or new-style, J2000;
{\tt slaEg50}
and
{\tt slaGe50}
are \radec\ to \gal\ and {\it vice versa}\/ for the B1950 case, while
{\tt slaEqgal}
and
{\tt slaGaleq}
are the J2000 equivalents.
 
Finally, the functions
{\tt slaGalsup}
and
{\tt slaSupgal}
transform \gal\ to de~Vaucouleurs supergalactic longitude and latitude
and {\it vice versa.}
 
It should be appreciated that the table, above, constitutes
a gross oversimplification.   Apparently
simple concepts such as equator, equinox
{\it etc.}\ are apt to be very hard to
pin down precisely (polar motion, orbital perturbations \ldots) and
some have several interpretations, all subtly different.  The various
frames move in complicated ways with respect to one another or to
the stars (themselves in motion).  And in some instances the
coordinate system is slightly distorted, so that the
ordinary rules of spherical trigonometry no longer strictly apply.

These {\it caveats}\/
apply particularly to the bewildering variety of different
\radec\ systems that are in use.  Figure~\ref{figclass} shows how
some of these systems are related, to one another and
to the direction in which a celestial source actually
appears in the sky.  At the top of the diagram are
the various sorts of {\it mean place}\/
found in star catalogues and papers;\footnote{One reference
system not included in
Figure~1 is that of the Hipparcos catalogue.  This is currently the
best available implementation in the optical of the {\it International
Celestial Reference System}\/ (ICRS), which is based on extragalactic
radio sources observed by VLBI.  The distinction between FK5 J2000
and Hipparcos coordinates only becomes important when accuracies of
50~mas or better are required.  More details are given in
Section~4.14.} at the bottom is the
{\it observed}\/ \azel, where a perfect theodolite would
be pointed to see the source;  and in the body of
the diagram are
the intermediate processing steps and coordinate
systems.  To help
understand this diagram, and the SLALIB functions that can
be used to carry out the various calculations, we will look at the coordinate
systems involved, and the astronomical phenomena that
affect them.

\begin{figure}
\begin{center}
\begin{tabular}{|cccccc|}   \hline
& & & & & \\
\hspace{5em} & \hspace{5em} & \hspace{5em} &
   \hspace{5em} & \hspace{5em} & \hspace{5em}  \\
\multicolumn{2}{|c}{\hspace{0em}\fbox{\parbox{8.5em}{\center \vspace{-2ex}
                                                   mean \radec, FK4, \\
                                                   any equinox
                                                   \vspace{0.5ex}}}} &
 \multicolumn{2}{c}{\hspace{0em}\fbox{\parbox{8.5em}{\center \vspace{-2ex}
                                                   mean \radec, FK4,
                                                   no $\mu$, any equinox
                                                   \vspace{0.5ex}}}} &
\multicolumn{2}{c|}{\hspace{0em}\fbox{\parbox{8.5em}{\center \vspace{-2ex}
                                                   mean \radec, FK5, \\
                                                   any equinox
                                                   \vspace{0.5ex}}}} \\
& \multicolumn{2}{|c|}{} & \multicolumn{2}{c|}{} & \\
\multicolumn{2}{|c}{\it space motion} & \multicolumn{1}{c|}{} & &
   \multicolumn{2}{c|}{\it space motion} \\
\multicolumn{2}{|c}{\it $-$ E-terms} &
   \multicolumn{2}{c}{\it $-$ E-terms} & \multicolumn{1}{c|}{} & \\
\multicolumn{2}{|c}{\it precess to B1950} &
                                 \multicolumn{2}{c}{\it precess to B1950} &
   \multicolumn{2}{c|}{\it precess to J2000} \\
\multicolumn{2}{|c}{\it + E-terms} &
   \multicolumn{2}{c}{\it + E-terms} & \multicolumn{1}{c|}{} & \\
\multicolumn{2}{|c}{\it FK4 to FK5, no $\mu$} &
   \multicolumn{2}{c}{\it FK4 to FK5, no $\mu$} & \multicolumn{1}{c|}{} & \\
\multicolumn{2}{|c}{\it parallax} & \multicolumn{1}{c|}{} & &
   \multicolumn{2}{c|}{\it parallax} \\
& \multicolumn{2}{|c|}{} & \multicolumn{2}{c|}{} & \\ \cline{2-5}
\multicolumn{3}{|c|}{} & & & \\
& \multicolumn{4}{c}{\fbox{\parbox{18em}{\center \vspace{-2ex}
                                   FK5, J2000, current epoch, geocentric
                                          \vspace{0.5ex}}}} & \\
\multicolumn{3}{|c|}{} & & & \\
& \multicolumn{4}{c}{\it light deflection} & \\
& \multicolumn{4}{c}{\it annual aberration} & \\
& \multicolumn{4}{c}{\it precession-nutation} & \\
\multicolumn{3}{|c|}{} & & & \\
& \multicolumn{4}{c}{\fbox{apparent \radec}} & \\
\multicolumn{3}{|c|}{} & & & \\
& \multicolumn{4}{c}{\it local apparent sidereal time} & \\
\multicolumn{3}{|c|}{} & & & \\
& \multicolumn{4}{c}{\fbox{apparent \hadec}} & \\
\multicolumn{3}{|c|}{} & & & \\
& \multicolumn{4}{c}{\it diurnal aberration} & \\
\multicolumn{3}{|c|}{} & & & \\
& \multicolumn{4}{c}{\fbox{topocentric \hadec}} & \\
\multicolumn{3}{|c|}{} & & & \\
& \multicolumn{4}{c}{\it \hadec\ to \azel} & \\
\multicolumn{3}{|c|}{} & & & \\
& \multicolumn{4}{c}{\fbox{topocentric \azel}} & \\
\multicolumn{3}{|c|}{} & & & \\
& \multicolumn{4}{c}{\it refraction} & \\
\multicolumn{3}{|c|}{} & & & \\
& \multicolumn{4}{c}{\fbox{observed \azel}} & \\
& & & & & \\
& & & & & \\ \hline
\end{tabular}
\end{center}
\vspace{-0.5ex}
\caption{\bf Relationship between celestial coordinates, pre-ICRS}
\label{figclass}
{\small Prior to the introduction of the
International Celestial Reference System,
star positions were published or catalogued using
one of the mean \radec\ systems shown at
the top, each with its own specific precession and
aberration models.  The ``FK4'' systems
were used before about 1980 and were usually
equinox B1950.  The ``FK5'' system, equinox J2000 (to which
the ICRS is nominally aligned), then took over.
The figure relates a star's mean \radec\ to the actual
line-of-sight to the star.
Note that for the conventional choices of equinox, namely
B1950 or J2000, all of the precession and E-terms corrections
are superfluous.}
\end{figure}

\subsection{Precession and Nutation}
{\it Right ascension and declination}, (\radec), are the names
of the longitude and latitude in a spherical
polar coordinate system based on the Earth's axis of rotation.
The zero point of $\alpha$ is the point of intersection of
the {\it celestial
equator}\/ and the {\it ecliptic}\/ (the apparent path of the Sun
through the year) where the Sun moves into the northern
hemisphere.  This point is called the
{\it first point of Aries},
the {\it vernal equinox}\/ (with apologies to
southern-hemisphere readers) or simply the {\it equinox}.\footnote{With
the introduction of the International Celestial Reference System (ICRS), the
connection between (i)~star coordinates and (ii)~the Earth's orientation
and orbit has been broken.  However, the orientation of the
International Celestial Reference System (ICRS) axes was, for convenience,
chosen to match J2000 FK5, and for most practical purposes ICRS coordinates
(for example entries in the Hipparcos catalogue) can be regarded as
synonymous with J2000 FK5.  See Section~\ref{hipicrs} for further details.}
 
This simple picture is unfortunately
complicated by the difficulty of defining
a suitable equator and equinox.  One problem is that the
Sun's apparent diurnal and annual motions
are not completely regular, due to the
ellipticity of the Earth's orbit and its continuous disturbance
by the Moon and planets.  This is dealt with by
separating the motion into (i)~a smooth and steady {\it mean Sun}\/
and (ii)~a set of periodic corrections and perturbations; only the former
is involved in establishing reference systems and time-scales.
A second, far larger problem, is that
the celestial equator and the ecliptic
are both moving with respect to the stars.
These motions arise because of the gravitational
interactions between the Earth and the other solar-system bodies.
 
By far the largest effect is the
so-called ``precession of the equinoxes'', where the Earth's
rotation axis sweeps out a cone centred on the ecliptic
pole, completing one revolution in about 26,000 years.  The
cause of the motion is the torque exerted on the distorted and
spinning Earth by the Sun and the Moon.  Consider the effect of the
Sun alone, at or near the northern summer solstice.  The Sun
`sees' the top (north pole) of the Earth tilted towards it
(by about \degree{23}{5}, the {\it obliquity of the
ecliptic}\/),
and sees the nearer part of the Earth's equatorial bulge
below centre and the further part above centre.
Although the Earth is in free fall,
the gravitational force on the nearer part of the
equatorial bulge is greater than that on the further part, and
so there is a net torque acting
as if to eliminate the tilt.  Six months later the same thing
is happening in reverse, except that the torque is still
trying to eliminate the tilt.  In between (at the equinoxes) the
torque shrinks to zero.  A torque acting on a spinning body
is gyroscopically translated
into a precessional motion of the spin axis at right-angles to the torque,
and this happens to the Earth.
The motion varies during the
year, going through two maxima, but always acts in the
same direction.  The Moon produces the same effect,
adding a contribution to the precession which peaks twice
per month.  The Moon's proximity to the Earth more than compensates
for its smaller mass and gravitational attraction, so that it
in fact contributes most of the precessional effect.
 
The complex interactions between the three bodies produce a
precessional motion that is wobbly rather than completely smooth.
However, the main 26,000-year component is on such a grand scale that
it dwarfs the remaining terms, the biggest of
which has an amplitude of only \arcseci{11} and a period of
about 18.6~years.  This difference of scale makes it convenient to treat
these two components of the motion separately.  The main 26,000-year
effect is called {\it luni-solar precession};  the smaller,
faster, periodic terms are called the {\it nutation}.
 
Note that precession and nutation are simply
different frequency components of the same physical effect.  It is
a common misconception that precession is caused
by the Sun and nutation is caused by the Moon.  In fact
the Moon is responsible for two-thirds of the precession, and,
while it is true that much of the complex detail of the nutation is
a reflection of the intricacies of the lunar orbit, there are
nonetheless important solar terms in the nutation.
 
In addition to and quite separate
from the precession-nutation effect, the orbit of the Earth-Moon system
is not fixed in orientation, a result of the attractions of the
planets.  This slow (about \arcsec{0}{5}~per~year)
secular rotation of the ecliptic about a slowly-moving diameter is called,
confusingly, {\it planetary
precession}\/ and, along with the luni-solar precession is
included in the {\it general precession}.  The equator and
ecliptic as affected by general precession
are what define the various ``mean'' \radec\ reference frames.

The models for precession and nutation come from a combination
of observation and theory, and are subject to continuous
refinement.  Nutation models in particular have reached a high
degree of sophistication, taking into account such things as
the non-rigidity of the Earth and the effects of
the planets; SLALIB's nutation
model (SF2001) involves 194 terms in each of $\psi$ (longitude)
and $\epsilon$ (obliquity), some as small as a few microarcseconds.

\subsubsection{SLALIB support for precession and nutation}
SLALIB offers a choice of three precession models:
\begin{itemize}
\item The old Bessel-Newcomb, pre IAU~1976, ``FK4'' model, used for B1950
      star positions and other pre-1984.0 purposes
({\tt slaPrebn}).
\item The newer Fricke, IAU~1976, ``FK5'' model, used for J2000 star
      positions and other post-1984.0 purposes
({\tt slaPrec}).
\item A model published by Simon {\it et al.}\ which is more accurate than
      the IAU~1976 model and which is suitable for long
      periods of time
({\tt slaPrecl}).
\end{itemize}
In each case, the named SLALIB function generates the $(3\times3)$
{\it precession
matrix}\/ for a given start and finish epoch.  For example,
here is C code for generating the rotation
matrix which describes the precession between the epochs
J2000 and J1985.372 (IAU~1976 model):
\goodbreak
\vspace{-3ex}
\begin{verbatim}
         double pmat[3][3];
          :
         slaPrec ( 2000.0, 1985.372, pmat );
\end{verbatim}
\vspace{-3ex}
\goodbreak
It is instructive to examine the resulting matrix:
\goodbreak
\vspace{-3ex}
\begin{verbatim}
         +0.9999936402  +0.0032709208  +0.0014214694
         -0.0032709208  +0.9999946505  -0.0000023247
         -0.0014214694  -0.0000023248  +0.9999989897
\end{verbatim}
\vspace{-3ex}
\goodbreak
Note that the diagonal elements are close to unity, and the
other elements are small.  This shows that over an interval as
short as 15~years the precession isn't going to move a
position vector very far---in this case about \degree{0}{2}.
 
For convenience, a direct \radec\ to \radec\ precession function is
also provided
({\tt slaPreces}),
suitable for either the FK4 or the FK5 system (but not a
mixture of the two).

SLALIB provides two nutation models, the old IAU~1980 model,
implemented in the function
{\tt slaNutc80}, and a much more accurate newer theory, SF2001,
implemented in the function
{\tt slaNutc}.
Both return the components of nutation
in longitude and latitude (and also provide the obliquity) from
which a nutation matrix can be generated by calling
{\tt slaDeuler}
(and from which the {\it equation of the equinoxes}, described
later, can be found).  Alternatively,
the SF2001 nutation matrix can be generated in a single call by using
{\tt slaNut}.  The SF2001 nutation theory includes components that correct
for errors in the IAU~1976 precession and also for the
$\sim 23\,$mas
displacement between the mean J2000 and ICRS coordinate systems,
achieving a final accuracy well under 1\,mas in the present era.
 
A rotation matrix for applying the entire precession-nutation
transformation in one go can be generated by calling
{\tt slaPrenut}.

\subsection{Mean Places}
From a classical standpoint,
the main effect of the precession-nutation is an increase of about
\arcseci{50}/year in the ecliptic longitudes of the stars.  It is therefore
essential, when reporting the position of an astronomical target with
respect to an equinox-based reference system, to
qualify the coordinates with a date, or {\it epoch}.
Specifying the epoch ties down the equator and
equinox which define the \radec\ coordinate system that is
being used.\footnote{An equinox is,
however, not required for coordinates
in the International Celestial Reference System.  Such coordinates must
be labelled simply ``ICRS'', or the specific catalogue can be mentioned,
such as ``Hipparcos'';  constructions such as ``Hipparcos, J2000'' are
redundant and misleading.}  For simplicity, only
the smooth and steady ``precession''
portion of the complete precession-nutation effect is
included, thereby defining what is called the {\it mean}\/
equator and equinox for the epoch concerned.  We say a star
has a mean place of (for example)
\hms{12}{07}{58}{09}~\dms{-19}{44}{37}{1} ``with respect to the mean equator
and equinox of epoch J2000''.  The short way of saying
this is ``\radec\ equinox J2000'' ({\bf not} ``\radec\ epoch J2000'',
which means something different to do with
proper motion).

\subsection{Epoch}
The word ``epoch'' just means a
significant moment in time, and can be supplied
in a variety of forms, using different calendar systems and time-scales.

For the purpose of specifying the epochs associated with the
mean place of a star, two conventions exist.  Both sorts of epoch
superficially resemble years AD but are not tied to the civil
(Gregorian) calendar;  to distinguish them from ordinary calendar-years
there is often
a ``.0'' suffix (as in ``1950.0''), although any other fractional
part is perfectly legal ({\it e.g.}\ 1987.5).

The older system,
{\it Besselian epoch}, is defined in such a way that its units are
tropical years of about 365.2422~days and its time-scale is the
obsolete {\it Ephemeris Time}.
The start of the Besselian year is the moment
when the ecliptic longitude of the mean Sun is
$280^{\circ}$;  this happens near the start of the
calendar year (which is why $280^{\circ}$ was chosen).
 
The new system, {\it Julian epoch}, was adopted as
part of the IAU~1976 revisions (about which more will be said
in due course) and came formally into use at the
beginning of 1984.  It uses the Julian year of exactly
365.25~days; Julian epoch 2000 is defined to be 2000~January~1.5 in the
TT time-scale.

For specifying mean places, various standard epochs are in use, the
most common ones being Besselian epoch 1950.0 and Julian epoch 2000.0.
To distinguish the two systems, Besselian epochs
are now prefixed ``B'' and Julian epochs are prefixed ``J''.
Epochs without an initial letter can be assumed to be Besselian
if before 1984.0, otherwise Julian.  These details are supported by
the SLALIB functions
{\tt slaDbjin}
(decodes numbers from a
character string, accepting an optional leading B or J),
{\tt slaKbj}
(decides whether B or J depending on prefix or range) and
{\tt slaEpco}
(converts one epoch to match another).
 
SLALIB has four functions for converting
Besselian and Julian epochs into other forms.
The functions
{\tt slaEpb2d}
and
{\tt slaEpj2d}
convert Besselian and Julian epochs into MJD; the functions
{\tt slaEpb}
and
{\tt slaEpj}
do the reverse.  For example, to express B1950 as a Julian epoch:
\goodbreak
\vspace{-3ex}
\begin{verbatim}
         printf ( "J%10.5f\n", slaEpj(slaEpb2d(1950.0)) );
\end{verbatim}
\vspace{-3ex}
\goodbreak
(The answer is J1949.99979.)

\subsection{Proper Motion}
Stars in catalogues usually have, in addition to the
\radec\  coordinates, a {\it proper motion} $[\mu_\alpha,\mu_\delta]$.
This is an intrinsic motion
of the star across the background.  Very few stars have a
proper motion which exceeds \arcseci{1}/year, and most are
far below this level.  A star observed as part of normal
astronomy research will, as a rule, have a proper motion
which is unknown.
 
Mean \radec\ and the proper motion $[\,\dot{\alpha},\dot{\delta}\,]$
are not sufficient to pin
down a star: the epoch at which the \radec\ was or will
be correct is also needed.  Note the distinction
between the epoch which specifies the
coordinate system and the epoch at which the star passed
through the given \radec.  The full specification for a star
is \radec, proper motions, equinox and epoch, plus something to
identify which set of models for the precession {\it etc.}\ is
being used (and, in the case of nearby stars, parallax and
radial velocity).
For convenience, coordinates given in star catalogues are almost
always adjusted to make the equinox and epoch the same---for
example B1950 in the case of the SAO~catalogue.  However, there
are some notable exceptions to this rule, in particular the
Hipparcos and Tycho\,1 catalogues, which have an epoch of
1991.25.

SLALIB provides one function to handle proper motion on its own,
{\tt slaPm}.
Proper motion is also allowed for in various other
functions as appropriate, for example
{\tt slaMap}
and
{\tt slaFk425}.
Note that in all SLALIB functions which involve proper motion
the units are radians per year and the
$\alpha$ component is in the form $\dot{\alpha}$ ({\it i.e.}~big
numbers near the poles).
Some star catalogues have proper motion per century, and
in most recent catalogues the $\alpha$ component is in the form
$\dot{\alpha}\cos\delta$ ({\it i.e.}~angle on the sky).  If
an RA proper motion is quoted using either arcseconds or milliarcseconds
as the angular measure, it can assumed to be the latter variety.

\subsection{Parallax and Radial Velocity}
For the utmost accuracy and the nearest stars, allowance can
be made for {\it annual parallax}\/ and for the effects of perspective
on the proper motion.
 
Parallax is appreciable only for nearby stars;  even
the nearest, Proxima Centauri, is displaced from its average
position by less than
an arcsecond as the Earth revolves in its orbit.
 
For stars with a known parallax, knowledge of the radial velocity
allows the proper motion to be expressed as an actual space
motion in 3~dimensions.  The proper motion is,
in fact, a snapshot of the transverse component of the
space motion, and in the case of nearby stars will
change with time due to perspective.
 
SLALIB does not provide facilities for handling parallax
and radial-velocity on their own, but their contribution is
allowed for in such functions as
{\tt slaPm},
{\tt slaMap}
and
{\tt slaFk425}.
Catalogue mean
places do not include the effects of parallax and are therefore
{\it barycentric};  when pointing telescopes {\it etc.}\ it is
usually most efficient to apply the slowly-changing
parallax correction to the mean place of the target early on
and to work with the {\it geocentric}\/ mean place.  This latter
approach is implied in Figure~\ref{figclass}.

\subsection{Aberration}
The finite speed of light combined with the motion of the observer
around the Sun during the year causes apparent displacements of
the positions of the stars.  The effect is called
the {\it annual aberration} (or ``stellar''
aberration).  Its maximum size, about \arcsec{20}{5},
occurs for stars $90^{\circ}$ from the point towards which
the Earth is headed as it orbits the Sun;  a star exactly in line with
the Earth's motion is not displaced.  To receive the light of
a star, the telescope has to be offset slightly in the direction of
the Earth's motion.  A familiar analogy is the need to tilt your
umbrella forward when on the move, to avoid getting wet.  This
classical model is,
in fact, misleading in the context of light as opposed
to rain, but happens to give the same answer as a relativistic
treatment to first order (better than 1~milliarcsecond).

Before the IAU 1976 resolutions, different
values for the approximately
\arcsec{20}{5} {\it aberration constant}\/ were employed
at different times, and this can complicate comparisons
between different catalogues.  Another complication comes from
the so-called {\it E-terms of aberration},
that small part of the annual aberration correction that is a
function of the eccentricity of the Earth's orbit.  The E-terms,
maximum amplitude about \arcsec{0}{3},
happen to be approximately constant for a given star, and so they
used to be incorporated in the catalogue \radec\/
to reduce the labour of converting to and from apparent place.
The E-terms can be removed from a catalogue \radec\/ by
calling
{\tt slaSubet}
or applied (for example to allow a pulsar
timing-position to be plotted on a B1950 finding chart)
by calling
{\tt slaAddet};
the E-terms vector itself can be obtained by calling
{\tt slaEtrms}.
Star positions post IAU 1976 are free of these distortions, and to
apply corrections for annual aberration involves the actual
barycentric velocity of the Earth rather than the use of
canonical circular-orbit models.
 
The annual aberration is the aberration correction for
an imaginary observer at the Earth's centre.
The motion of a real observer around the Earth's rotation axis in
the course of the day makes a small extra contribution to the total
aberration effect called the {\it diurnal aberration}.  Its
maximum amplitude is about \arcsec{0}{3}.

No SLALIB function is provided for calculating the aberration on
its own, though the required velocity vectors can be
generated using
{\tt slaEvp} (or
{\tt slaEpv})
and
{\tt slaGeoc}.
Annual and diurnal aberration are allowed for where required, for example in
{\tt slaMap}
{\it etc}.\ and
{\tt slaAop}
{\it etc}.  Note that this sort
of aberration is different from the {\it planetary
aberration}, which is the apparent displacement of a solar-system
body, with respect to the ephemeris position, as a consequence
of the motion of {\it both}\/ the Earth and the source.  The
planetary aberration can be computed either by correcting the
position of the solar-system body for light-time, followed by
the ordinary stellar aberration correction, or more
directly by expressing the position and velocity of the source
in the observer's frame and correcting for light-time alone.

\subsection{Different Sorts of Mean Place}
A confusing aspect of the mean places used in the
pre-ICRS era is that they
are sensitive to the precise way they were determined.  A mean
place is not directly observable, even with fundamental
instruments such as transit circles, and to produce one
will involve working with respect to some existing star catalogue,
for example the fundamental catalogues FK4 and FK5,
and applying specific mathematical models of precession, nutation,
aberration and so on.
Note in particular that no star catalogue,
even a fundamental catalogue such as FK4 or
FK5, nor even the International Celestial
Reference Frame itself,
defines a coordinate system, strictly speaking;
it is merely a list of star positions and proper motions.
However, once the stars from a given catalogue
are used as position calibrators, {\it e.g.}\ for
transit-circle observations or for plate reductions, then a
broader sense of there being a coordinate space naturally
arises, and such phrases as ``in the system of
the FK4'' can legitimately be employed.  However,
there is no formal link between the
two concepts: no ``standard least squares fit'' between
reality and the inevitably flawed catalogues.
All such
catalogues suffer at some level from systematic, zonal distortions
of both the star positions and of the proper motions,
and include measurement errors peculiar to individual
stars.

Many of these complications are of little significance except to
specialists.  However, astronomers who need to refer to
20th-century observations cannot
escape exposure to at least the two main varieties of
mean place, loosely called
FK4 and FK5, and should be aware of
certain pitfalls.  For most practical purposes the more recent
system, FK5, is free of surprises and tolerates naive
use well---it can even for many purposes be regarded
as equivalent to ICRS.  FK4, in contrast, contains two important traps:
\begin{itemize}
\item The FK4 system rotates at about
      \arcsec{0}{5} per century relative to distant galaxies.
      This is manifested as a systematic distortion in the
      proper motions of all FK4-derived catalogues, which will
      in turn pollute any astrometry done using those catalogues.
      For example, FK4-based astrometry of a QSO using plates
      taken decades apart will reveal a non-zero {\it fictitious proper
      motion}, and any FK4 star which happens to have zero proper
      motion is, in fact, slowly moving against the distant
      background.  The FK4 system rotates because it was
      established before the nature of the Milky Way, and hence the
      existence of systematic motions of nearby stars, had been
      recognized.
\item Star positions in the FK4 system are part-corrected for
      annual aberration (see above) and embody the so-called
      E-terms of aberration.
\end{itemize}
The change from the old FK4-based system to FK5
occurred at the beginning
of 1984 as part of a package of resolutions made by the IAU in 1976,
along with the adoption of J2000 as the reference epoch.  Star
positions in the newer, FK5, system are free from the E-terms, and
the system is a much better approximation to an
inertial frame---about five times better (and ICRS is hundreds of times
better still).

It may occasionally be convenient to specify the FK4 fictitious proper
motion directly.  In FK4, the centennial proper motion of (for example)
a QSO is:
 
$\mu_\alpha=-$\tsec{0}{015869}$
          +(($\tsec{0}{029032}$~\sin \alpha
            +$\tsec{0}{000340}$~\cos \alpha ) \sin \delta
            -$\tsec{0}{000105}$~\cos \alpha
            -$\tsec{0}{000083}$~\sin \alpha ) \sec \delta $ \\
$\mu_\delta\,=+$\arcsec{0}{43549}$~\cos \alpha
              -$\arcsec{0}{00510}$~\sin \alpha +
              ($\arcsec{0}{00158}$~\sin \alpha
              -$\arcsec{0}{00125}$~\cos \alpha ) \sin \delta
              -$\arcsec{0}{00066}$~\cos \delta $

\subsection{Mean Place Transformations}
Figure~\ref{figclass}
is based upon three varieties of mean \radec\ all of which are
of practical significance to observing astronomers in the present era:
\begin{itemize}
   \item Old style (FK4) with known proper motion in the FK4
         system, and with parallax and radial velocity either
         known or assumed zero.
   \item Old style (FK4) with zero proper motion in FK5,
         and with parallax and radial velocity assumed zero.
   \item New style (FK5 or, loosely, ICRS)
         with proper motion, parallax and
         radial velocity either known or assumed zero.
\end{itemize}
The figure outlines the steps required to convert positions in
any of these systems to a J2000 \radec\ for the current
date, as might be required in a telescope-control
program for example.
Most of the steps can be carried out by calling a single
SLALIB function;  there are other SLALIB functions which
offer set-piece end-to-end transformation functions for common cases.
Note, however, that SLALIB does not set out to provide the capability
for arbitrary transformations of star-catalogue data
between all possible systems of mean \radec.
Only in the (common) cases of FK4, equinox and epoch B1950,
to FK5, equinox and epoch J2000, and {\it vice versa}\/ are
proper motion, parallax and radial velocity transformed
along with the star position itself, the
focus of SLALIB support.

As an example of using SLALIB to transform mean places, here is
C code thate implements the top-left path of Figure~\ref{figclass}.
An FK4 \radec\ of arbitrary equinox and epoch and with
known proper motion and
parallax is transformed into an FK5 J2000 \radec\ for the current
epoch.  As a test star we will use $\alpha=$\hms{16}{09}{55}{13},
$\delta=$\dms{-75}{59}{27}{2}, equinox 1900, epoch 1963.087,
$\mu_\alpha=$\tsec{-0}{0312}$/y$, $\mu_\delta=$\arcsec{+0}{103}$/y$,
parallax = \arcsec{0}{062}, radial velocity = $-34.22$~km/s.  The
date of observation is 1994.35.
\goodbreak
\vspace{-3ex}
\begin{verbatim}
          :
         int j, i;
         double r0, d0, eq0, ep0, pr, pd, px, rv, ep1, r1, d1, r2, d2,
                r3, d3, r4, d4, r5, d5, r6, d6, ep1d, ep1b,
                w[3], eb[3], pxr, v[3];

      /* RA, Dec etc. of example star. */
         slaDtf2r ( 16, 9, 55.13, &r0, &j );
         slaDaf2r ( 75, 59, 27.2, &d0, &j );
         d0 = -d0;
         eq0 = 1900.0;
         ep0 = 1963.087;
         pr = -0.0312 * S2R;
         pd = 0.103 * AS2R;
         px = 0.062;
         rv = -34.22;
         ep1 = 1994.35;

      /* Date of observation as MJD and Besselian epoch. */
         ep1d = slaEpj2d ( ep1 );
         ep1b = slaEpb ( ep1d );

      /* Space motion to the current epoch. */
         slaPm ( r0, d0, pr, pd, px, rv, ep0, ep1b, &r1, &d1 );

      /* Remove E-terms of aberration for the original equinox. */
         slaSubet ( r1, d1, eq0, &r2, &d2 );

      /* Precess to B1950. */
         r3 = r2;
         d3 = d2;
         slaPreces ( "FK4", eq0, 1950.0, &r3, &d3 );

      /* Add E-terms for the standard equinox B1950. */
         slaAddet ( r3, d3, 1950.0, &r4, &d4 );

      /* Transform to J2000, no proper motion. */
         slaFk45z ( r4, d4, ep1b, &r5, &d5 );

      /* Parallax. */
         slaEvp ( slaEpj2d ( ep1 ), 2000.0, w, eb, w, w );
         pxr = px * AS2R;
         slaDcs2c ( r5, d5, v );
         for ( i = 0; i < 3; i++ ) {
            v[i] -= pxr * eb[i];
         }
         slaDcc2s ( v, &r6, &d6 );
          :
\end{verbatim}
\vspace{-3ex}
\goodbreak
It is interesting to look at how the \radec\ changes during the
course of the calculation:
\begin{tabbing}
xxxxxxxxxxxxxx \= xxxxxxxxxxxxxxxxxxxxxxxxx \= x \= \kill
\> {\tt 16 09 55.130 -75 59 27.20} \> \> {\it original equinox and epoch} \\
\> {\tt 16 09 54.155 -75 59 23.98} \> \> {\it with space motion} \\
\> {\tt 16 09 54.229 -75 59 24.18} \> \> {\it with old E-terms removed} \\
\> {\tt 16 16 28.213 -76 06 54.57} \> \> {\it precessed to 1950.0} \\
\> {\tt 16 16 28.138 -76 06 54.37} \> \> {\it with new E-terms} \\
\> {\tt 16 23 07.901 -76 13 58.87} \> \> {\it J2000, current epoch} \\
\> {\tt 16 23 07.907 -76 13 58.92} \> \> {\it including parallax}
\end{tabbing}

Other remarks about the above (deliberately complicated) example:
\begin{itemize}
\item If the original equinox and epoch were B1950, as is quite
      likely, then it would be unnecessary to treat space motions
      and E-terms explicitly.  Transformation to FK5 J2000 could
      be accomplished simply by calling
{\tt slaFk425}, after which
      a call to
{\tt slaPm} and the parallax code would complete the
      work.
\item The rigorous treatment of the E-terms
      has only a small effect on the result.  Such refinements
      are, nevertheless, worthwhile in order to facilitate comparisons and
      to increase the chances that star positions from different
      suppliers are compatible.
\item The FK4 to FK5 transformations,
{\tt slaFk425}
      and
{\tt slaFk45z},
      are not as is sometimes assumed simply 50 years of precession,
      though this indeed accounts for most of the change.  The
      transformations also include adjustments
      to the equinox, a revised precession model, elimination of the
      E-terms, a change to the proper-motion time unit and so on.
      The reason there are two functions rather than just one
      is that the FK4 system rotates relative to the background, whereas
      the FK5 system is a much better approximation to an
      inertial frame, and zero proper
      motion in FK4 does not, therefore, mean zero proper motion in FK5.
      SLALIB also provides two functions,
{\tt slaFk524}
      and
{\tt slaFk54z},
      to perform the inverse transformations.
\item Some star catalogues (FK4 itself is one) were constructed using slightly
      different procedures for the polar regions compared with
      elsewhere.  SLALIB ignores this inhomogeneity and always
      applies the standard
      transformations, irrespective of location on the celestial sphere.
\end{itemize}

\subsection {Mean Place to Apparent Place}
The {\it geocentric apparent place}\/ of a source, or {\it apparent place}\/
for short,
is the \radec\ if viewed from the centre of the Earth,
with respect to the true equator and equinox of date.
Transformation of an FK5 mean \radec, equinox J2000,
current epoch, to apparent place involves the following effects:
\goodbreak
\begin{itemize}
   \item Light deflection: the gravitational lens effect of
         the sun.
   \item Annual aberration.
   \item Precession-nutation.
\end{itemize}
The {\it light deflection}\/ is insignificant for most
applications.  Its value
at the limb of the Sun is about
\arcsec{1}{74};  it falls off rapidly with distance from the
Sun and at an elongation of $20^\circ$ has shrunk to about
\arcsec{0}{02}. (SLALIB neglects light deflection from the planets,
which can reach about 20\,mas.)
 
As already described, the {\it annual aberration}\/
is a function of the Earth's velocity
relative to the solar system barycentre (available through the
SLALIB functions
{\tt slaEvp} and
{\tt slaEpv})
and produces shifts of up to about \arcsec{20}{5}.

The {\it precession-nutation}, from J2000 to the current epoch, is
expressed by a rotation matrix which is available through the
SLALIB function
{\tt slaPrenut}.

The whole mean-to-apparent transformation can be done using the SLALIB
function
slaMap.  As a demonstration, here is a program which lists the
{\it North Polar Distance}\/ ($90^\circ-\delta$) of Polaris for
the decade of closest approach to the pole:
\goodbreak
\vspace{-3ex}
\begin{verbatim}
      #include <stdio.h>
      #include <slalib.h>

      #define PI 3.141592653589793238462643
      #define D2R (PI/180.0)
      #define PIBY2 (PI/2.0)
      #define S2R (PI/(12.0*3600.0))
      #define AS2R (PI/(180.0*3600.0))

      int main ( )
      {
         double rm, dm, pr, pd, date, ra, da;
         int j, iymdf[4];
         long ids, ide, id;


      /* Polaris J2000 RA,Dec and proper motion. */
         slaDtf2r ( 2, 31, 49.8131, &rm, &j );
         slaDaf2r ( 89, 15, 50.661, &dm, &j );
         pr = 0.217272 * S2R;
         pd = -0.01571 * AS2R;

      /* Report co-declination every 10 days from 2096 to 2105. */
         printf ( "Polaris north polar distance (deg) 2096-2105\n\n"
                  "   date       NPD\n\n" );
         slaCldj ( 2096, 1, 1, &date, &j );
         ids = (long) ( date + 0.5 );
         slaCldj ( 2105, 12, 31, &date, &j );
         ide = (long) ( date + 0.5 );
         for ( id = ids; id <= ide; id += 10L ) {
            date = (double) id;
            slaDjcal ( 0, date, iymdf, &j );
            slaMap ( rm, dm, pr, pd, 0.0, 0.0, 2000.0, date, &ra, &da );
            printf ( "%4d %2.2d %2.2d %8.5f\n",
                     iymdf[0], iymdf[1], iymdf[2], (PIBY2-da)/D2R );
         }

         return 0;
      }
\end{verbatim}
\vspace{-3ex}
\goodbreak
For cases where the transformation has to be repeated for different
times or for more than one star, the straightforward
{\tt slaMap}
approach is apt to be
wasteful as both the Earth velocity and the
precession-nutation matrix can be re-calculated relatively
infrequently without ill effect.  A more efficient method is to
perform the target-independent calculations only when necessary,
by calling
{\tt slaMappa},
and then to use either
{\tt slaMapqkz},
when only the \radec\/ is known, or
{\tt slaMapqk},
when full catalogue positions, including proper motion, parallax and
radial velocity, are available.  How frequently to call
{\tt slaMappa}
depends on the accuracy objectives;  once per
night will deliver sub-arcsecond accuracy for example.
 
The functions
{\tt slaAmp}
and
{\tt slaAmpqk}
allow the reverse transformation, from apparent to mean place.

\subsection{Apparent Place to Observed Place}
The {\it observed place}\/ of a source is its position as
seen by a perfect theodolite at the location of the
observer.  Transformation of an apparent \radec\ to observed
place involves the following effects:
\goodbreak
\begin{itemize}
   \item \radec\ to \hadec.
   \item Diurnal aberration.
   \item \hadec\ to \azel.
   \item Refraction.
\end{itemize}
The transformation from apparent \radec\ to
apparent \hadec\ is made by allowing for
{\it Earth rotation}\/ through the {\it sidereal time}, $\theta$:
\[ h = \theta - \alpha \]
For this equation to work, $\alpha$ must be the apparent right
ascension for the time of observation, and $\theta$ must be
the {\it local apparent sidereal time}.  The latter is obtained
as follows:
\begin{enumerate}
\item From civil time obtain the coordinated universal time, UTC.
      (More later on this.)
\item Add the UT1$-$UTC (typically a few tenths of a second) to
      give the UT.
\item From the UT compute the Greenwich mean sidereal time (using
{\tt slaGmst}).
\item Add the observer's (east) longitude, giving the local mean
      sidereal time.
\item Add the equation of the equinoxes (using
{\tt slaEqeqx}) to give local apparent sidereal time.
\end{enumerate}
The {\it equation of the equinoxes}\/~($=\Delta\psi\cos\epsilon$ plus
small terms) is the distance between the mean and true equinox
at the date concerned and hence is
nutation's contribution to the sidereal time.
Its value is typically a second (of time) or less.

Note that for very precise work the observer's longitude should
be corrected for {\it polar motion}.  This can be done with
{\tt slaPolmo}.
The corrections are always less than about \arcsec{0}{3}, and
are futile unless the position of the observer's telescope is known
to better than a few metres.
 
Bulletins containing tables of observed and
predicted UT1$-$UTC corrections and polar motion data
are published every few weeks by the International Earth Rotation and
reference frames Service (IERS).
 
The transformation from apparent \hadec\ to {\it topocentric}\/
\hadec\ consists of allowing for
{\it diurnal aberration} (and,in the case of
solar-system objects, diurnal parallax).
The diurnal aberration, maximum amplitude \arcsec{0}{2},
was described earlier.  There is no specific SLALIB function
for computing the diurnal aberration,
though the functions
{\tt slaAop} {\it etc.}\
include it, and the required velocity vector can be
determined by calling
{\tt slaGeoc}.
 
The next stage is the major coordinate rotation from local equatorial
coordinates \hadec\ into horizon coordinates.  The SLALIB functions
{\tt slaE2h}
{\it etc.}\ can be used for this.  For high-precision
applications the mean geodetic latitude should be corrected for polar
motion.

\subsubsection{Refraction}
The final correction is for atmospheric refraction.
This effect, which depends on local meteorological
conditions\footnote{The refraction down to zenith distances of about
$70^\circ$ depends almost entirely upon the
conditions at the observer and is hardly affected by the
conditions along the incoming beam.  The plausibility of this
convenient but perhaps surprising result can be seen by considering
the simple case of an atmosphere that consists
of plane-parallel homogeneous layers;  the individual
deviations suffered by the beam as it passes through the
successive layers all cancel out and so the only refractive
indices of consequence are those at the observer and in empty space.}
and the effective colour of the source/detector combination,
increases the observed elevation of the source by a
significant effect even at moderate zenith distances, and near the
horizon by over \degree{0}{5}.  The amount of refraction can by
computed by calling the SLALIB function
{\tt slaRefro}, which performs a ray-trace through
a model atmosphere starting from the observer and out into space;
however,
this requires as input the observed zenith distance, which is what
we are trying to predict.  For high precision it is
therefore necessary to iterate, using the topocentric
zenith distance as the initial estimate of the
observed zenith distance.
 
The full
{\tt slaRefro} refraction calculation is onerous, and for
zenith distances of less than, say, $75^{\circ}$ the following
model can be used instead:
 
\[ \zeta _{vac} \approx \zeta _{obs}
                     + A \tan \zeta _{obs}
                     + B \tan ^{3}\zeta _{obs} \]
where $\zeta _{vac}$ is the topocentric
zenith distance ({\it i.e.~in vacuo}),
$\zeta _{obs}$ is the observed
zenith distance ({\it i.e.}~affected by refraction), and $A$ and $B$ are
constants, about \arcseci{60}
and \arcsec{-0}{06} respectively for a sea-level site.
The two constants can be calculated for a given set of conditions
by calling either
{\tt slaRefco} or
{\tt slaRefcoq}.\footnote{The $A$ term is essentially the
contribution from a flat atmosphere and the $B$ term the
correction for curvature.  At zenith distances smaller than about
$40^\circ$, the $B$ term can be neglected.  Much beyond $\zeta=70^\circ$,
the two-term formulation becomes insufficient, at about the
same level at which
the conditions in the atmosphere along the beam
become increasingly important, making simple formulations based only on
the ambient conditions untrustworthy.}
 
{\tt slaRefco} works by calling
{\tt slaRefro} for two zenith distances and fitting $A$ and $B$
to match.  The calculation is onerous, but delivers accurate
results whatever the conditions.
{\tt slaRefcoq} uses a direct formulation of $A$ and $B$ and
is much faster;  it is slightly less accurate than
{\tt slaRefco} but more than adequate for most practical purposes.
 
Like the full refraction model, the two-term formulation works in the wrong
direction for our purposes, predicting
the {\it in vacuo}\/ (topocentric) zenith distance
given the refracted (observed) zenith distance,
rather than {\it vice versa}.  The obvious approach of
interchanging $\zeta _{vac}$ and $\zeta _{obs}$ and
reversing the signs, though approximately
correct, gives avoidable errors which are just significant in
some applications;  for
example about \arcsec{0}{2} at $70^\circ$ zenith distance.  A
much better result can easily be obtained, by using one Newton-Raphson
iteration as follows:
 
\[ \zeta _{obs} \approx \zeta _{vac}
    - \frac{A \tan \zeta _{vac} + B \tan ^{3}\zeta _{vac}}
    {1 + ( A + 3 B \tan ^{2}\zeta _{vac} ) \sec ^{2}\zeta _{vac}}\]
 
The effect of refraction can be applied to an unrefracted
zenith distance by calling
{\tt slaRefz} or to an unrefracted
\xyz\ by calling
{\tt slaRefv}.
Over most of the sky these two functions deliver almost identical
results, but beyond $\zeta=83^\circ$
{\tt slaRefv}
becomes unacceptably inaccurate while
{\tt slaRefz}
remains usable.  (However
{\tt slaRefv}
is significantly faster, which may be important in some applications.)
SLALIB also provides a function for computing the airmass, the function
{\tt slaAirmas}.
 
The refraction ``constants'' returned by
{\tt slaRefco} and
{\tt slaRefcoq}
are slightly affected by colour, especially at the blue end
of the spectrum.  Where values for more than one
wavelength are needed, rather than calling
{\tt slaRefco}
several times it is more efficient to call
{\tt slaRefco}
just once, for a selected ``base'' wavelength, and then to call
{\tt slaAtmdsp}
once for each wavelength of interest.
 
All the SLALIB refraction functions work for radio wavelengths as well
as the optical/IR band.  The radio refraction is very dependent on
humidity, and an accurate value must be supplied.  There is no
wavelength dependence, however.  The choice of optical/IR or
radio is made by specifying a wavelength greater than $100\mu {\rm m}$
for the radio case.

\subsubsection{Efficiency considerations}
The complete apparent place to observed place transformation
can be carried out by calling
{\tt slaAop}.
For improved efficiency
in cases of more than one star or a sequence of times, the
target-independent calculations can be done once by
calling
{\tt slaAoppa},
the time can be updated by calling
{\tt slaAoppat},
and
{\tt slaAopqk}
can then be used to perform the
apparent-to-observed transformation.  The reverse transformation
is available through
{\tt slaOap}
and
{\tt slaOapqk}.
({\it n.b.}\ These functions use accurate but computationally-expensive
refraction algorithms for zenith distances beyond about $76^\circ$.
For many purposes, in-line code tailored to the accuracy requirements
of the application will be preferable, for example ignoring polar motion,
omitting diurnal aberration and using
{\tt slaRefz}
to apply the refraction.)

\subsection{The Hipparcos Catalogue and the ICRS}
\label{hipicrs}
With effect from the beginning of 1998, the IAU adopted a new
reference system to replace FK5 J2000.  The new system, called the
International Celestial Reference System (ICRS), differs profoundly
from all predecessors in that the link with solar-system dynamics
was broken;  the ICRS axes are defined in terms of the coordinates
of a set of extragalactic sources, not in terms of the mean equator and
equinox at a given reference epoch.  Although the ICRS and FK5 coordinates
of any given object are almost the same, the orientation of the new system
was essentially arbitrary, and the close match to FK5 J2000
(about 23\,mas, it turned out) was contrived
purely for reasons of continuity and convenience.
 
A distinction is made between the reference {\it system}\/ (the ICRS)
and {\it frame}\/ (ICRF).  The ICRS is the set of prescriptions and
conventions together with the modelling required to define, at any
time, a triad of axes.  The ICRF is a practical realization, and
currently consists of a catalogue of equatorial coordinates for 608
extragalactic radio sources observed by VLBI.
 
The best optical realization of the ICRS currently available is the
Hipparcos catalogue.  The extragalactic sources were not directly
observable by the Hipparcos satellite and so the link from Hipparcos
to ICRS was established through a variety of indirect techniques: VLBI and
conventional interferometry of radio stars, photographic astrometry
and so on.  The Hipparcos system is aligned to the ICRS to within about
0.5~mas and 0.5~mas/year (at epoch 1991.25).
 
The Hipparcos catalogue includes all of the FK5 stars, which has enabled
the orientation and spin of the latter to be studied.  At epoch J2000,
the misalignment of the FK5 system with respect to Hipparcos
(and hence ICRS) are about 32~mas and 1~mas/year respectively.
Consequently, for many practical purposes, including pointing
telescopes, the IAU 1976-1982 conventions on reference frames and
Earth orientation remain adequate and there is no need to change to
Hipparcos coordinates, new precession-nutation models and so on.
However, for the most exacting astrometric applications, SLALIB
provides some support for Hipparcos coordinates in the form of
four new functions:
{\tt slaFk52h} and
{\tt slaH2fk5},
which transform FK5 positions and proper motions to the Hipparcos system
and {\it vice versa,}\/ and
{\tt slaFk5hz} and
{\tt slaHfk5z},
where the transformations are for stars whose Hipparcos proper motion is
zero.
 
Further information on the ICRS can be found in the paper by M.\,Feissel
and F.\,Mignard, {\it Astron.Astrophys.}~{\bf 331}, L33-L36 (1988).

\subsection{Time-scales}
SLALIB provides transformations between several time-scales, and involves
use of one or two others:
\begin{itemize}
\item TAI: International Atomic Time
\item UTC: Coordinated Universal Time
\item TT: Terrestrial Time
\item TDB: Barycentric Dynamical Time.
\item UT: Universal Time
\item GMST: Greenwich mean sidereal time
\item GAST (or GST): Greenwich apparent sidereal time.
\item LAST: local apparent sidereal time
\end{itemize}
Strictly speaking, UT and the sidereal times are not {\it times}\/ in
the physics sense, but {\it angles}\/ that describe Earth rotation.

Three obsolete time-scales should be mentioned here to avoid confusion.
\begin{itemize}
\item GMT: Greenwich Mean Time -- can mean either UTC or UT.
\item ET: Ephemeris Time -- more or less the same as either TT or TDB.
\item TDT: Terrestrial Dynamical Time -- former name of TT.
\end{itemize}

Time-scales that have no SLALIB support at present:
\begin{itemize}
\item Any form of local civil time (BST, PDT {\it etc.})
\item TCG: geocentric coordinate time.
\item TCB: barycentric coordinate time.
\item GPS time: a common starting point for real-time applications,
      exactly 19\,s behind TAI.
\end{itemize}

The relationships between the various time-scales and Earth rotation
measures are shown in Figure~\ref{timescales}.

\begin{figure}
\begin{center}
\begin{picture}(160,145)(0,-5)
\put(  40.00, 137.50){\framebox(  24.00,   7.00){{\bf GPS time}}}
\put(  52.00, 137.50){\line( 0,-1){   7.00}}
\put(  52.00, 127.00){\oval(  14.00,   7.00)}
\put(  52.00, 127.00){\makebox(0,0){$+19^{\rm s}$}}
\put(  52.00, 123.50){\vector( 0,-1){   7.00}}
\put(  45.00, 109.50){\framebox(  14.00,   7.00){{\bf TAI}}}
\put(  52.00, 109.50){\line( 0,-1){   7.00}}
\put(  52.00,  99.00){\oval(  16.00,   7.00)}
\put(  52.00,  99.00){\makebox(0,0){$-\mathit\Delta{\rm AT}$}}
\put(  52.00,  95.50){\vector( 0,-1){   7.00}}
\put(  44.00,  81.50){\framebox(  16.00,   7.00){{\bf UTC}}}
\put(  52.00,  81.50){\line( 0,-1){   7.00}}
\put(  52.00,  71.00){\oval(  16.00,   7.00)}
\put(  52.00,  71.00){\makebox(0,0){$+\mathit\Delta{\rm UT}$}}
\put(  52.00,  67.50){\vector( 0,-1){   7.00}}
\put(  45.00,  53.50){\framebox(  14.00,   7.00){{\bf UT1}}}
\put(  52.00,  53.50){\line( 0,-1){   7.00}}
\put(  52.00,  43.00){\oval(  18.00,   7.00)}
\put(  52.00,  43.00){\makebox(0,0){{\tt slaGmst}}}
\put(  52.00,  39.50){\vector( 0,-1){   7.00}}
\put(  43.00,  25.50){\framebox(  18.00,   7.00){{\bf GMST}}}
\put(  52.00,  25.50){\line( 0,-1){   7.00}}
\put(  52.00,  15.00){\oval(  42.00,   7.00)}
\put(  52.00,  15.00){\makebox(0,0){+{\tt slaEqeqx({\rm TDB})}+$\lambda$}}
\put(  52.00,  11.50){\vector( 0,-1){   7.00}}
\put(  44.00,  -2.50){\framebox(  16.00,   7.00){{\bf LAST}}}
\put(  45.00, 113.00){\line(-1, 0){  18.00}}
\put(  27.00, 108.00){\oval(10,10)[tl]}
\put(  22.00, 108.00){\line( 0,-1){  19.50}}
\put(  22.00,  85.00){\oval(  30.00,   7.00)}
\put(  22.00,  85.00){\makebox(0,0){$+\mathit\Delta{\rm UT}-\mathit\Delta{\rm AT}$}}
\put(  22.00,  81.50){\line( 0,-1){  19.50}}
\put(  27.00,  62.00){\oval(10,10)[bl]}
\put(  27.00,  57.00){\vector( 1, 0){  18.00}}
\put(  60.00,  85.00){\line( 1, 0){  14.00}}
\put(  82.00,  85.00){\oval(  16.00,   7.00)}
\put(  82.00,  85.00){\makebox(0,0){$+\mathit\Delta{\rm TT}$}}
\put(  90.00,  85.00){\line( 1, 0){  17.00}}
\put( 107.00,  80.00){\oval(10,10)[tr]}
\put( 112.00,  80.00){\vector( 0,-1){  19.50}}
\put(  59.00,  57.00){\line( 1, 0){  16.00}}
\put(  82.00,  57.00){\oval(  14.00,   7.00)}
\put(  82.00,  57.00){\makebox(0,0){$+\mathit\Delta{\rm T}$}}
\put(  89.00,  57.00){\vector( 1, 0){  17.00}}
\put( 106.00,  53.50){\framebox(  12.00,   7.00){{\bf TT}}}
\put( 112.00,  53.50){\line( 0,-1){   7.00}}
\put( 112.00,  43.00){\oval(  38.00,   7.00)}
\put( 112.00,  43.00){\makebox(0,0){{\tt slaRcc({\rm UT1},$\lambda$,$\phi$)}}}
\put( 112.00,  39.50){\vector( 0,-1){   7.00}}
\put( 105.00,  25.50){\framebox(  14.00,   7.00){{\bf TDB}}}
\put(  59.00, 113.00){\line( 1, 0){  78.00}}
\put( 137.00, 108.00){\oval(10,10)[tr]}
\put( 142.00, 108.00){\line( 0,-1){  19.50}}
\put( 142.00,  85.00){\oval(  18.00,   7.00)}
\put( 142.00,  85.00){\makebox(0,0){$+32^{\rm s}\hspace{-0.3em}.184$}}
\put( 142.00,  81.50){\line( 0,-1){  19.50}}
\put( 137.00,  62.00){\oval(10,10)[br]}
\put( 137.00,  57.00){\vector(-1, 0){  19.00}}
\end{picture}
\end{center}
\caption{\bf Relationships Between Time-Scales}
\label{timescales}
\begin{quote}
\small
The diagram shows how the various time-scales (and Earth
rotation measures) supported by SLALIB are
interrelated.
GPS time is also shown, as this is a common input to real-time
applications.
In some cases the time-scales are offset from one another by
specified amounts, usually designated by standard symbols such as
``$\mathit\Delta{\rm AT}$''.
In other cases the transformation involves canonical formulae, and
here these are indicated by quoting the name of the appropriate
SLALIB function, for example ``{\tt slaRcc}''.
It is possible to reach some of the timescales via more than one
route.  An important example of this is the transformation from
TAI to UT1, where there is the choice of going directly or via UTC.
In a practical application (such as a telescope control system),
the direct route has the important advantage that the offset varies
in a continuous manner; the route via UTC introduces the complication
of dealing with leap seconds.
\end{quote}
\end{figure}

\subsubsection{Atomic Time: TAI}
{\it International Atomic Time,} TAI, is a ``laboratory''
time-scale with no link to astronomical observations
except in an historical sense.  Its
unit is the SI second, which is defined in terms of a
specified number
of wavelengths of the radiation produced by a certain electronic
transition in the caesium 133 atom.  It
is realized through a changing
population of high-precision atomic clocks held
at standards institutes in various countries.  There is an
elaborate process of continuous intercomparison, leading to
a weighted average of all the clocks involved.
 
Though TAI shares the same second as the more familiar UTC, the
two time-scales are noticeably separated in epoch because of the
build-up of leap seconds (see Section~\ref{UTC}).
At the time of writing, UTC
lags over half a minute behind TAI.
 
For any given date, the difference TAI$-$UTC
can be obtained by calling the SLALIB function
{\tt slaDat}.
Note, however, that an up-to-date copy of the function must be used if
the most recent leap seconds are required.  For applications
where this is critical, mechanisms independent of SLALIB
and under local control must
be set up;  in such cases
{\tt slaDat}
can be useful as an
independent check, for test dates within the range of the
available version.  Up-to-date information on TAI$-$UTC is available
from {\tt ftp://maia.usno.navy.mil/ser7/tai-utc.dat}.
 
\subsubsection{Universal Time: UT, UTC}
\label{UTC}
{\it Universal Time,} UT, or more specifically UT1,
is in effect mean solar time and is really an expression
of Earth rotation rather than a measure of time.
Originally
defined in terms of a point in the sky called ``the fictitious
mean Sun'', UT is now defined through its relationship
with Earth rotation angle
(formerly sidereal time).
Because the Earth's rotation rate is slightly irregular and
gradually decreasing,\footnote{The Earth is slowing
down because of tidal effects.  The SI
second reflects the length-of-day in the mid-19th century, when
the astronomical observations that established modern timekeeping
were being made.  Since then,
the average length-of-day has increased by roughly 2~ms.
Superimposed in this gradual slowdown are
variations (seasonal and decadal) that are geophysical in origin,
notably due to large scale movements of water and atmosphere.
Because of
conservation of angular momentum, as the Earth's rotation-rate
decreases, the Moon moves farther away.  In 50 billion years the
distance of the Moon will be at a maximum, 44\% greater than now, at
which stage day and month will both equal 47 present days.}
the UT second is not precisely
matched to the SI second.  This makes UT itself unsuitable for
use as a time-scale.

That role is instead taken by
{\it Coordinated Universal Time,} UTC, which is clock-based and
is the foundation of civil timekeeping.
Most time zones differ from UTC by an integer number
of hours, though a few ({\it e.g.}\ parts of Canada and Australia) differ
by $n+0.5$~hours.  Since its introduction, UTC has been kept
roughly in step with UT by a variety of adjustments that are
agreed in advance and then carried out in a coordinated manner by
the providers of time services---hence the
name.  Though rate
changes were used in the past, nowadays all such adjustments
are made by occasionally inserting
a whole second.  This procedure is called
a {\it leap second}.  Because the day length is now slightly longer
than 86400 SI seconds, a leap second amounts to stopping the UTC
clock for a second to let the Earth catch up.

You need UT1 in order to point a telescope or antenna at a
clestial target.  To obtain it
starting from UTC, you
have to look up the value of UT1$-$UTC for the date concerned
in tables published by the International Earth Rotation and
reference frames
Service (IERS);  this quantity, kept in the range
$\pm$\tsec{0}{9} by means of leap
seconds, is then added to the UTC.  The quantity UT1$-$UTC,
which typically changes by of order 1~ms per day,
can be obtained only by observation (VLBI using
extragalactic radio sources), though seasonal trends
are well known and the IERS listings are able to predict some way into
the future with adequate accuracy for pointing telescopes.

UTC leap seconds are introduced as necessary,
usually at the end of December or June.
Because on the average the solar day is slightly longer
than the nominal 86,400~SI~seconds, leap seconds are always positive;
however, provision exists for negative leap seconds if needed.
The form of a leap second can be seen from the
following description of the end of June~1994:

\hspace{3em}
\begin{tabular}{clrccc} \\
     &      &    &   UTC    & UT1$-$UTC  &    UT1       \\ \\
1994 & June & 30 & 23 59 58 & $-0.218$ & 23 59 57.782 \\
     &      &    & 23 59 59 & $-0.218$ & 23 59 58.782 \\
     &      &    & 23 59 60 & $-0.218$ & 23 59 59.782 \\
     & July &  1 & 00 00 00 & $+0.782$ & 00 00 00.782 \\
     &      &    & 00 00 01 & $+0.782$ & 00 00 01.782 \\
\end{tabular}
\goodbreak

Note that UTC has to be expressed as hours, minutes and
seconds (or at least in seconds for a given date) if leap seconds
are to be taken into account in the
correct manner.
It is improper to express a UTC as a
Julian Date, for example, because there will be an ambiguity
during a leap second (in the above example,
1994~June~30 \hms{23}{59}{60}{0} and
1994~July~1 \hms{00}{00}{00}{0} would {\it both}\/ come out as
MJD~49534.00000).  Although in the vast majority of
cases this won't matter, there are potential problems in
on-line data acquisition systems and in applications involving
taking the difference between two times.  Note that although the functions
{\tt slaDat}
and
{\tt slaDtt}
expect UTC in the form of an MJD, the meaning here is really a
whole-number {\it date}\/ rather than a time.
Though the functions will accept
a fractional part and will almost always function correctly, on a day
which ends with a leap
second incorrect results would be obtained during the leap second
itself because by then the MJD would have moved into the next day.

\subsubsection{Sidereal Time: GMST, LAST {\it etc.}}
Sidereal time is like the time of day but relative to the
stars rather than to the Sun.  After
one sidereal day the stars come back to the same place in the
sky, apart from sub-arcsecond precession effects.  Because the Earth
rotates faster relative to the stars than to the Sun by one day
per year, the sidereal second is shorter than the solar
second; the ratio is about 0.9973.
 
The {\it Greenwich mean sidereal time,} GMST, is
linked to UT1 by a numerical formula which
is implemented in the SLALIB functions
{\tt slaGmst}
and
{\tt slaGmsta}.
There are, of course, no leap seconds in GMST, but the sidereal
second (measured in SI seconds)
changes in length along with the UT1 second, and also varies
over long periods of time because of slow changes in the Earth's
orbit.  This makes sidereal time unsuitable for everything except
predicting the apparent directions of celestial sources, in other
words as an angle rather than a time.
 
The {\it local apparent sidereal time,} LAST, is the apparent right
ascension of the local meridian, from which the hour angle of any
star can be determined knowing its right
ascension.  LAST can be obtained from the
GMST by adding the east longitude (corrected for polar motion
in precise work) and the {\it equation of the equinoxes}.  The
latter, already described, is an aspect of the nutation effect
and can be predicted by calling the SLALIB function
{\tt slaEqeqx}
or, neglecting certain very small terms, by calling
{\tt slaNutc}
and using the expression $\Delta\psi\cos\epsilon$.

GAST, or plain GST, is GMST plus the equation of the equinoxes.

\subsubsection{Dynamical Time: TT, TDB}
Dynamical time (formerly Ephemeris Time, ET)
is the independent variable in the theories
which describe the motions of bodies in the solar system.  When
using published formulae or
tables that model the position of the
Earth in its orbit, for example, or look up
the Moon's position in a precomputed ephemeris, the date and time
must be in terms of one of the dynamical time-scales.  It
is a common but understandable mistake to use UTC directly, in which
case the results will be over a minute out (at the time of writing).
 
It is not hard to see why such time-scales are necessary.
UTC would clearly be unsuitable as the argument of an
ephemeris because of leap seconds.
A solar-system ephemeris based on UT1 or sidereal time would somehow
have to include the unpredictable variations of the Earth's rotation.
TAI would work, but in principle
the ephemeris and the ensemble of atomic clocks would
eventually drift apart.
In effect, the ephemeris {\it is}\/ a clock, with the bodies of
the solar system the hands from which the ephemeris time is read.
 
Only two of the dynamical time-scales are of any great importance to
observational astronomers, TT and TDB.
 
{\it Terrestrial Time,} TT, is
the theoretical time-scale of apparent geocentric ephemerides of solar
system bodies.  It applies to clocks at sea-level, and for practical purposes
it is tied to
Atomic Time TAI through the formula TT~$=$~TAI~$+$~\tsec{32}{184}.
In practice, therefore, the units of TT are ordinary SI seconds, and
the offset of \tsec{32}{184} with respect to TAI is fixed.
The SLALIB function
{\tt slaDtt}
returns TT$-$UTC for a given UTC
({\it n.b.}~{\tt slaDtt}
calls
{\tt slaDat},
and the latter must be an up-to-date version if recent leap seconds are
to be taken into account).
 
{\it Barycentric Dynamical Time,} TDB, is a
{\it coordinate time,} suitable
for labelling events that are most simply described in a context
where the bodies of the solar system
are absent.  Applications include
the emission of pulsar radiation and the motions of the
solar-system bodies themselves.  When the readings of the
observer's TT clock are labelled using such a
coordinate time, differences
are seen because the clock is affected by its
speed in the barycentric coordinate system
and the gravitational potential in which it is immersed.  Equivalently,
observations of pulsars
expressed in TT would display similar variations (quite
apart from the familiar light-time effects).

TDB is defined in such a way that it keeps close to TT
on the average, with the relativistic effects emerging as
quasi-periodic differences of maximum amplitude rather less
than 2\,ms.  This is
negligible for many purposes, so that TT can act as
a perfectly adequate surrogate for TDB in most cases,
but unless taken into
account would swamp
long-term analysis of pulse arrival times from the
millisecond pulsars.

Most of the variation between TDB and TT comes from the ellipticity of
the Earth's orbit;  the TT clock's speed and
gravitational potential vary slightly
during the course of the year, and as a consequence
its rate as seen from an outside observer
varies due to transverse Doppler effect and gravitational
redshift.  The main component is a sinusoidal variation of
amplitude \tsec{0}{0017};  higher harmonics, and terms
caused by Moon and planets, lie two orders of magnitude below
this dominant annual term.  Diurnal (topocentric) terms, a
function of UT, are $2\,\mu$s or less.

The IAU 1976 resolution defined TDB by
stipulating that TDB$-$TT consists of periodic terms only.
This provided
a good qualitative description, but turned out to
contain hidden assumptions about the form of the
solar-system ephemeris and hence lacked dynamical
rigour.  A later resolution, in 1991, introduced new
coordinate time-scales, TCG and TCB, and identified TDB as a
linear transformation of one of them (TCB) with a rate
chosen not to drift from TT on the average.  Unfortunately
even this improved definition has proved to contain
ambiguities.  The SLALIB
{\tt slaRcc} function implements TDB in the way that is
most consistent with the 1976 definition and
with existing practice.  It provides a model of
TDB$-$TT accurate to a few nanoseconds.

Unlike TDB, the IAU 1991 coordinate time-scales TCG and TCB
(not supported by SLALIB functions at present)
do not have their rates adjusted to track TT and consequently
gain on TT and TDB, by about
\tsec{0}{02}/year and \tsec{0}{5}/year respectively.
 
As already pointed out, the distinction between TT and TDB is
of no practical importance for most purposes.  For
example when calling
{\tt slaPrenut}
to generate a precession-nutation matrix, or when calling
{\tt slaEvp} or
{\tt slaEpv}
to predict the
Earth's position and velocity, the time argument is strictly
TDB, but TT is entirely adequate and will require much
less computation.

The time-scale used by the JPL solar-system ephemerides is called
$T_{eph}$ and numerically is essentially the same as TDB.
 
Predictions of topocentric solar-system phenomena such as
occultations and eclipses require solar time UT as well as dynamical
time.  TT/TDB/ET is all that is required in order to compute the geocentric
circumstances, but if horizon coordinates or geocentric parallax
are to be tackled UT is also needed.  A rough estimate
of $\Delta {\rm T} = {\rm ET} - {\rm UT}$ is
available via the function
{\tt slaDt}.
For a given epoch ({\it e.g.}\ 1650) this returns an approximation
to $\Delta {\rm T}$ in seconds.

\subsection{Calendars}
The ordinary {\it Gregorian Calendar Date},
together with a time of day, can be
used to express a point in time in any desired time-scale.
For many purposes,
however, a continuous count of days is more convenient, and for
this purpose the system of {\it Julian Day Number}\/ can be used.
JD zero is located about 7000~years ago, well before the
historical era, and is formally defined in terms of Greenwich noon;
for example Julian Day Number 2449444 began at noon
on 1994 April~1.  {\it Julian Date}\/
is the same system but with a fractional part appended;
Julian Date 2449443.5 was the midnight on which 1994 April~1
commenced.  Because of the unwieldy size of Julian Dates
and the awkwardness of the half-day offset, it is
accepted practice to remove the leading `24' and the trailing `.5',
producing what is called the {\it Modified Julian Date}:
MJD\,=\,JD$-2400000.5$.  SLALIB functions use MJD, as opposed to
JD, throughout, largely to avoid loss of precision.
1994 April~1 commenced at MJD~49443.0.
 
Despite JD (and hence MJD) being defined in terms of (in effect)
UT, the system can be used in conjunction with other time-scales
such as TAI, TT and TDB (and even sidereal time through the
concept of {\it Greenwich Sidereal Date}).  However, it is improper
to express a UTC as a JD or MJD because of leap seconds.
 
SLALIB has six functions for converting to and from dates in
the Gregorian calendar.  The functions
{\tt slaCldj}
and
{\tt slaCaldj}
both convert a calendar date into an MJD, the former interpreting
years between 0 and 99 as 1st century and the latter as late 20th or
early 21st century.  The functions slaDjcl
and
{\tt slaDjcal}
both convert an MJD into calendar year, month, day and fraction of a day;
the latter performs rounding to a specified precision.
Some of SLALIB's low-precision ephemeris functions
({\tt slaEarth},
{\tt slaMoon}
and
{\tt slaEcor})
work in terms of year plus day-in-year (where
day~1~=~January~1st, at least for the modern era).
This form of date can be generated by
calling
{\tt slaCalyd}
(which defaults years 0-99 into 1950-2049)
or
{\tt slaClyd}
(which covers the full range from prehistoric times).

\subsection{Geocentric Coordinates}
The location of the observer on the Earth is significant in a
number of ways.  The most obvious, of course, is the effect of
longitude and latitude
on the observed \azel\ of a star.  Less obvious is the need to
allow for geocentric parallax when finding the Moon with a
telescope (and when doing high-precision work involving the
Sun or planets), and the need to correct observed radial
velocities and apparent pulsar periods for the effects
of the Earth's rotation.
 
The SLALIB function
{\tt slaObs}
supplies details of groundbased observatories from an internal
list.  This is useful when writing applications that apply to
more than one observatory;  the user can enter a brief name,
or browse through a list, and be spared the trouble of typing
in the full latitude, longitude {\it etc}.  The following
C program reports the full name, longitude and latitude
of a specified observatory
({\it n.b.}~beware of the longitude sign convention, which is west positive
for historical reasons):
\goodbreak
\vspace{-3ex}
\begin{verbatim}
      #include <stdio.h>
      #include <string.h>
      #include <slalib.h>

      #define R2D 57.295779513

      int main ( )
      {
         char ident[11], name[41];
         double w, p, h;

         strcpy ( ident, "AAT" );
         slaObs ( 0, ident, name, &w, &p, &h );
         printf ( "%s:  E%g, %g, %gm\n", name[0] == (char) '?' ? "?" : name,
                                       -w*R2D, p*R2D, h );
         return 0;
      }
\end{verbatim}
\vspace{-3ex}
\goodbreak
The output is:
\goodbreak
\vspace{-3ex}
\begin{verbatim}
      Anglo-Australian 3.9m Telescope:  E149.066, -31.277, 1164m
\end{verbatim}
\vspace{-3ex}
\goodbreak
The following C program lists all the supported observatories:
\goodbreak
\vspace{-3ex}
\begin{verbatim}
      #include <stdio.h>
      #include <slalib.h>

      int main ( )
      {
         char ident[11], name[41];
         int n;
         double w, p, h;

         n = 1;
         name[0] = (char) 0;
         while ( name[0] != (char) '?' ) {
            slaObs ( n, ident, name, &w, &p, &h );
            if ( name[0] != (char) '?' ) {
               printf ( "%3d    %-10s    %s\n", n, ident, name );
            }
         n++;
         }
         return 0;
      }
\end{verbatim}
\vspace{-3ex}
\goodbreak
The function
{\tt slaGeoc}
converts a {\it geodetic latitude}\/
(one referred to the local horizon) to a geocentric position,
taking into account the Earth's oblateness and also the height
above sea level of the observer.  The results are expressed in
vector form, namely as the distance of the observer from
the spin axis and equator respectively.  The {\it geocentric
latitude}\/ can be found be evaluating {\tt atan2} of the
two numbers.  A full 3-D vector description of the position
and velocity of the observer is available through the function
{\tt slaPvobs}.
For a specified geodetic latitude, height above
sea level, and local sidereal time,
{\tt slaPvobs}
generates a 6-element vector containing the position and
velocity with respect to the true equator and equinox of
date ({\it i.e.}~compatible with apparent \radec).  For
some applications it will be necessary to convert to a
mean \radec\ system (notably FK5, J2000) by multiplying
elements 1-3 and 4-6 respectively with the appropriate
precession matrix.  (In theory an additional correction to the
velocity vector is needed to allow for differential precession,
but this correction is always negligible.)
 
See also the discussion of the function
{\tt slaRverot},
later.

\label{ephem}
\subsection{Ephemerides}
SLALIB includes functions for generating positions and
velocities of Solar-System bodies.  The accuracy objectives are
modest, and the SLALIB facilities do not attempt
to compete with precomputed ephemerides such as
those provided by JPL, or with models containing
thousands of terms.  It is also worth noting
that SLALIB's very accurate star coordinate transformation
functions are not strictly applicable to solar-system cases,
though they are adequate for most practical purposes.
 
Earth/Sun ephemerides can be generated using the functions
{\tt slaEvp} and
{\tt slaEpv},
each of which predict Earth position and velocity with respect to both the
solar-system barycentre and the
Sun.  The two functions offer different trade-offs between
accuracy and execution time.  For most purposes,
{\tt slaEvp} is adequate:
maximum velocity error is 0.42~metres per second;  maximum
heliocentric position error is 1600~km (equivalent to
about \arcseci{2} at 1~AU),
with barycentric position errors about 4 times worse.
The larger and slower
{\tt slaEpv}
delivers $3\sigma$ results of 0.005~metres per second in velocity
and 15~km in position, and is particularly useful when predicting
apparent directions of near-Earth objects.
(The Sun's position as
seen from the Earth can, of course, be obtained simply by
reversing the signs of the Cartesian components of the
Earth\,:\,Sun vector.)
 
Geocentric Moon ephemerides are available from
{\tt slaDmoon},
which predicts the Moon's position and velocity with respect to
the Earth's centre.  Direction accuracy is usually better than
10~km (\arcseci{5}) and distance accuracy a little worse.

Lower-precision but faster predictions for the Sun and Moon
can be made by calling
{\tt slaEarth}
and
{\tt slaMoon}.
Both are {\tt float} precision and accept dates in the form of
year, day-in-year and fraction of day
(starting from a calendar date you need to call
{\tt slaClyd}
or
{\tt slaCalyd}
to get the required year and day).
The
{\tt slaEarth}
function returns the heliocentric position and velocity
of the Earth's centre for the mean equator and
equinox of date.  The accuracy is better than 20,000~km in position
and 10~metres per second in speed.
The
position and velocity of the Moon with respect to the
Earth's centre for the mean equator and ecliptic of date
can be obtained by calling
{\tt slaMoon}.
The positional accuracy is better than \arcseci{30} in direction
and 1000~km in distance.
 
Approximate ephemerides for all the major planets
can be generated by calling
{\tt slaPlanet}
or
{\tt slaRdplan}.  These functions offer arcminute accuracy (much
better for the inner planets and for Pluto) over a span of several
millennia (but only $\pm100$ years for Pluto).
The function
{\tt slaPlanet} produces heliocentric position and
velocity in the form of equatorial \xyzxyzd\ for the
mean equator and equinox of J2000.  The vectors
produced by
{\tt slaPlanet}
can be used in a variety of ways according to the
requirements of the application concerned.  The function
{\tt slaRdplan}
uses
{\tt slaPlanet}
and
{\tt slaDmoon}
to deal with the common case of predicting
a planet's apparent \radec\ and angular size as seen by a
terrestrial observer.

Note that in predicting the position in the sky of a solar-system body
it is necessary to allow for geocentric parallax.  This correction
is {\it essential}\/ in the case of the Moon, where the observer's
position on the Earth can affect the Moon's \radec\ by up to
$1^\circ$.  The calculation can most conveniently be done by calling
{\tt slaPvobs} and subtracting the resulting 6-vector from the
one produced by
{\tt slaDmoon}, as is demonstrated by the following example:
\goodbreak
\vspace{-3ex}
\begin{verbatim}
      #include <stdio.h>
      #include <slalib.h>

      int main ( )

      /* Demonstrate the size of the geocentric parallax correction */
      /* in the case of the Moon.  The test example is for the AAT, */
      /* before midnight, in summer, near first quarter.            */

      {
         char name[41], s;
         int j, i, m[4];
         double slongw, slat, h, djutc, fdutc, djut1, djtt, stl,
                rmatn[3][3], pmm[6], pmt[6], rm, dm, pvo[6], tl;


      /* Get AAT longitude and latitude in radians and height in metres. */
         slaObs ( 0, "AAT", name, &slongw, &slat, &h );

      /* UTC (1992 January 13, 11 13 59) to MJD. */
         slaCldj ( 1992, 1, 13, &djutc, &j );
         slaDtf2d ( 11, 13, 59.0, &fdutc, &j );
         djutc += fdutc;

      /* UT1 (UT1-UTC value of -0.152 sec is from IERS Bulletin B). */
         djut1 = djutc + (-0.152) / 86400.0;

      /* TT. */
         djtt = djutc + slaDtt ( djutc ) / 86400.0;

      /* Local apparent sidereal time. */
         stl = slaGmst( djut1 ) - slongw + slaEqeqx ( djtt );

      /* Geocentric position/velocity of Moon (mean of date). */
         slaDmoon ( djtt, pmm );

      /* Nutation to true equinox of date. */
         slaNut ( djtt, rmatn );
         slaDmxv ( rmatn, pmm, pmt );
         slaDmxv ( rmatn, pmm+3, pmt+3 );

      /* Report geocentric HA,Dec. */
         slaDcc2s ( pmt, &rm, &dm );
         slaDr2tf ( 2, slaDranrm(stl-rm), &s, m );
         printf ( " geocentric:  %c%2.2d %2.2d %2.2d.%2.2d",
                                                 s, m[0], m[1],m[2],m[3] );
         slaDr2af ( 1, dm, &s, m );
         printf ( "  %c%2.2d %2.2d %2.2d.%1d\n", s, m[0], m[1],m[2],m[3] );

      /* Geocentric position of observer (true equator and equinox of date). */
         slaPvobs ( slat, h, stl, pvo );

      /* Place origin at observer. */
         for ( i = 0; i < 6; i++ ) {
            pmt[i] -= pvo[i];
         }

      /* Allow for planetary aberration. */
         tl = 499.004782 * sqrt ( pmt[0]*pmt[0]
                                + pmt[1]*pmt[1]
                                + pmt[2]*pmt[2] );
         for ( i = 0; i < 3; i++ ) {
            pmt[i] -= tl * pmt[i+3];
         }

      /* Report topocentric HA,Dec. */
         slaDcc2s ( pmt, &rm, &dm );
         slaDr2tf ( 2, slaDranrm(stl-rm), &s, m );
         printf ( "topocentric:  %c%2.2d %2.2d %2.2d.%2.2d",
                                                 s, m[0], m[1],m[2],m[3] );
         slaDr2af ( 1, dm, &s, m );
         printf ( "  %c%2.2d %2.2d %2.2d.%1d\n", s, m[0], m[1],m[2],m[3] );

         return 0;
      }
\end{verbatim}
\vspace{-3ex}
\goodbreak
The output produced is as follows:
\goodbreak
\begin{verbatim}
       geocentric:  +03 06 55.55  +15 03 38.8
      topocentric:  +03 09 23.76  +15 40 51.4
\end{verbatim}
\goodbreak
(An easier but
less instructive method of estimating the topocentric apparent place of the
Moon is to call the function
{\tt slaRdplan}.)

As an example of using
{\tt slaPlanet},
the following program estimates the geocentric separation
between Venus and Jupiter during a close conjunction
in 2\,BC, which is a star-of-Bethlehem candidate:
\goodbreak
\vspace{-3ex}
\begin{verbatim}
#include <stdio.h>
#include <slalib.h>

int main ( )

/* Compute time and minimum geocentric apparent separation         */
/* between Venus and Jupiter during the close conjunction of 2 BC. */

{
   double sepmin, djd0, fd, djd, djdm, pv[6], rmatp[3][3],
          pvm[6], pve[6], tl, dx, dy, dz, rv, dv, rj, dj, sep;
   int ihour, imin, j, i, ihmin, immin;


/* Search for closest approach on the given day. */
   djd0 = 1720859.5;
   sepmin = 1e10;
   for ( ihour = 20; ihour <= 22; ihour++ ) {
      for ( imin = 0; imin < 60; imin++ ) {
         slaDtf2d ( ihour, imin, 0.0, &fd, &j );

      /* Julian date and MJD. */
         djd = djd0 + fd;
         djdm = djd - 2400000.5;

      /* Earth to Moon (mean of date). */
         slaDmoon ( djdm, pv );

      /* Precess Moon position to J2000. */
         slaPrecl ( slaEpj ( djdm ), 2000.0, rmatp );
         slaDmxv ( rmatp, pv, pvm );

      /* Sun to Earth-Moon Barycentre (mean J2000). */
         slaPlanet ( djdm, 3, pve, &j );

      /* Correct from EMB to Earth. */
         for ( i = 0; i < 3; i++ ) {
            pve[i] -= 0.012150581 * pvm[i];
         }

      /* Sun to Venus. */
         slaPlanet ( djdm, 2, pv, &j );

      /* Earth to Venus. */
         for ( i = 0; i < 6; i++ ) {
            pv[i] -= pve[i];
         }

      /* Light time to Venus (sec). */
         dx = pv[0] - pve[0];
         dy = pv[1] - pve[1];
         dz = pv[2] - pve[2];
         tl = 499.004782 * sqrt ( dx*dx +dy*dy + dz*dz );

      /* Extrapolate backwards in time by that much. */
         for ( i = 0; i < 3; i++ ) {
            pv[i] -= tl * pv[i+3];
         }

      /* To RA,Dec. */
         slaDcc2s ( pv, &rv, &dv );

      /* Same for Jupiter. */
         slaPlanet ( djdm, 5, pv, &j );
         for ( i = 0; i < 6; i++ ) {
            pv[i] -= pve[i];
         }
         dx = pv[0] - pve[0];
         dy = pv[1] - pve[1];
         dz = pv[2] - pve[2];
         tl = 499.004782 * sqrt ( dx*dx +dy*dy + dz*dz );
         for ( i = 0; i < 3; i++ ) {
            pv[i] -= tl * pv[i+3];
         }
         slaDcc2s ( pv, &rj, &dj );

      /* Separation (arcsec) */
         sep = slaDsep ( rv, dv, rj, dj );

      /* Keep if smallest so far. */
         if ( sep < sepmin ) {
            ihmin = ihour;
            immin = imin;
            sepmin = sep;
         }
      }
   }

/* Report. */
   printf ( "%2.2d:%2.2d %5.1f\n", ihmin, immin, 206264.8062 * sepmin );

   return 0;
}
\end{verbatim}
\vspace{-3ex}
\goodbreak

The output produced (the Ephemeris Time on the day in question, and
the closest approach in arcseconds) is as follows:
\goodbreak
\vspace{-3ex}
\begin{verbatim}
      21:16  33.3
\end{verbatim}
\vspace{-3ex}
\goodbreak
For comparison, accurate JPL predictions
give a separation \arcseci{8} less than
the above estimate, occurring $30^{\rm m}$ earlier
(see {\it Sky and Telescope,}\/ April~1987, p\,357).

The following program demonstrates
{\tt slaRdplan}.
\goodbreak
\vspace{-3ex}
\begin{verbatim}
      #include <stdio.h>
      #include <ctype.h>
      #include <string.h>
      #include <slalib.h>

      #define R2AS 206264.80625              /* radians to arcsec */
      #define D15B2P 2.3873241463784300365   /* 15/2pi            */

      int main ( )

      /* For a given date, time and geographical location, output */
      /* a table of planetary positions and diameters.            */

      {
         static char *pnames[] = { "Sun", "Mercury", "Venus", "Moon",
                                   "Mars", "Jupiter", "Saturn",
                                   "Uranus", "Neptune",  "Pluto" };
         char b[100], s;
         int n, i, np, iy, j, im, id, m[4];
         double fd, djm, elong, phi, ra, dec, diam;


      /* Loop until "end" typed. */
         n = strlen ( b );
         for ( i = 0; i < n; i++ ) b[i] = (char) toupper ( b[i] );
         while ( strcmp ( b, "END" ) ) {

         /* Get date, time and observer's location */
            printf ( "Date? (Y,M,D, Gregorian)\n" );
            gets ( b );
            n = strlen ( b );
            for ( i = 0; i < n; i++ ) b[i] = (char) toupper ( b[i] );
            if ( strcmp ( b, "END" ) ) {
               i = 1;
               slaInt2in ( b, &i, &iy, &j );
               slaInt2in ( b, &i, &im, &j );
               slaInt2in ( b, &i, &id, &j );
               printf ( "Time? (H,M,S, dynamical)\n" );
               gets ( b );
               i = 1;
               slaDafin ( b, &i, &fd, &j );
               fd *= D15B2P;
               slaCldj ( iy, im, id, &djm, &j );
               djm += fd;
               printf ( "Longitude? (d,',\", east +ve)\n" );
               gets ( b );
               i = 1;
               slaDafin ( b, &i, &elong, &j );
               printf  ( "Latitude? (d,',\", geodetic)\n" );
               gets ( b );
               i = 1;
               slaDafin ( b, &i, &phi, &j );
               printf ( "\n" );

            /* Loop planet by planet. */
               for ( np = 0; np <= 9; np++ ) {

               /* Get RA,Dec and diameter. */
                  slaRdplan ( djm, np, elong, phi, &ra, &dec, &diam );

               /* One line of report. */
                  slaDr2tf ( 2, ra, &s, m );
                  printf ( "%-9s %2.2d %2.2d %2.2d.%2.2d",
                                      pnames[np], m[0], m[1], m[2], m[3] );
                  slaDr2af ( 1, dec, &s, m );
                  printf ( "  %c%2.2d %2.2d %2.2d.%1d %7.1f\n",
                                    s, m[0], m[1], m[2], m[3], R2AS*diam );

               /* Next planet. */
               }
               printf ( "\n" );
            }

         /* Next case. */
         }

         return 0;
      }
\end{verbatim}
\vspace{-3ex}
\goodbreak
Entering the following data (for 1927~June~29 at $5^{\rm h}\,25^{\rm m}$~ET
and the position of Preston, UK):
\goodbreak
\vspace{-3ex}
\begin{verbatim}
      1927 6 29
      5 25
      -2 42
      53 46
\end{verbatim}
\vspace{-3ex}
\goodbreak
produces the following report:
\goodbreak
\vspace{-3ex}
\begin{verbatim}
      Sun       06 28 14.03  +23 17 17.3  1887.8
      Mercury   08 08 58.60  +19 20 57.1     9.3
      Venus     09 38 53.61  +15 35 32.8    22.8
      Moon      06 28 15.95  +23 17 21.3  1902.3
      Mars      09 06 49.34  +17 52 26.6     4.0
      Jupiter   00 11 12.08  -00 10 57.5    41.1
      Saturn    16 01 43.35  -18 36 55.9    18.2
      Uranus    00 13 33.54  +00 39 36.1     3.5
      Neptune   09 49 35.76  +13 38 40.8     2.2
      Pluto     07 05 29.51  +21 25 04.2     0.1
\end{verbatim}
\vspace{-3ex}
\goodbreak
Inspection of the Sun and Moon data reveals that
a total solar eclipse is in progress.
 
SLALIB also provides for the case where orbital elements (with respect
to the J2000 equinox and ecliptic)
are available.  This allows predictions to be made for minor-planets and
(if you ignore non-gravitational effects)
comets.  Furthermore, if major-planet elements for an epoch close to the date
in question are available, more accurate predictions can be made than
are offered by
{\tt slaRdplan} and
{\tt slaPlanet}.
 
The SLALIB planetary-prediction
functions that work with orbital elements are
{\tt slaPlante} (the orbital-elements equivalent of
{\tt slaRdplan}), which predicts the topocentric \radec, and
{\tt slaPlanel} (the orbital-elements equivalent of
{\tt slaPlanet}), which predicts the
heliocentric \xyzxyzd\ with respect to the
J2000 equinox and equator.  In addition, the function
{\tt slaPv2el} does the inverse of
{\tt slaPlanel}, transforming \xyzxyzd\ into {\it osculating elements.}
 
Osculating elements describe the unperturbed 2-body orbit.  Depending
on accuracy requirements, this unperturbed orbit is an
adequate approximation to the actual orbit for a few weeks either
side of the specified epoch, outside which perturbations due to
the other bodies of the Solar System lead to
increasing errors.  Given a minor planet's osculating elements for
a particular date, predictions for a date only
100 days earlier or later
are likely to be in error by several arcseconds.
These errors can
be reduced if new elements are generated which take account of
the perturbations of the major planets, and this is what the function
{\tt slaPertel} does.  Once
{\tt slaPertel} has been called, to provide osculating elements
close to the required date, the elements can be passed to
{\tt slaPlanel} or
{\tt slaPlante} in the normal way.  Predictions of arcsecond accuracy
over a span of a decade or more are available using this
technique.

Three different combinations of orbital elements are
provided for, matching the usual conventions
for major planets, minor planets and
comets respectively.  The choice is made through the
argument {\tt jform}:
 
\vspace{1ex}
\hspace{3em}
\begin{tabular}{|c|c|c|} \hline
{\tt jform=1} & {\tt jform=2} & {\tt jform=3} \\
\hline \hline
$t_0$ & $t_0$ & $T$ \\
\hline
$i$ & $i$ & $i$ \\
\hline
$\Omega$ & $\Omega$ & $\Omega$ \\
\hline
$\varpi$ & $\omega$ & $\omega$ \\
\hline
$a$ & $a$ & $q$ \\
\hline
$e$ & $e$ & $e$ \\
\hline
$L$ & $M$ & \\
\hline
$n$ & & \\
\hline
\end{tabular}\\[2ex]
The symbols have the following meanings:
 
\vspace{-1ex}
\begin{tabbing}
xxxxxxx \= xxxx \= \kill
\> $t_0$ \> epoch of osculation \\
\> $T$ \> epoch of perihelion passage \\
\> $i$ \> inclination of the orbit \\
\> $\Omega$ \> longitude of the ascending node \\
\> $\varpi$ \> longitude of perihelion ($\varpi = \Omega + \omega$) \\
\> $\omega$ \> argument of perihelion \\
\> $a$ \> semi-major axis of the orbital ellipse \\
\> $q$ \> perihelion distance \\
\> $e$ \> orbital eccentricity \\
\> $L$ \> mean longitude ($L = \varpi + M$) \\
\> $M$ \> mean anomaly \\
\> $n$ \> mean motion \\
\end{tabbing}
 
The mean motion, $n$, tells
{\tt slaPlanel} the mass of the planet.
If it is not available, it should be calculated
from $n^2 a^3 = k^2 (1+m)$, where $k = 0.01720209895$ and
m is the mass of the planet ($M_\odot = 1$); $a$ is in AU.
 
Note that for any given problem there are up to three different
epochs in play, and it is vital to distinguish clearly between
them:
\begin{itemize}
\item The epoch of observation:  the moment in time for which the
      position of the body is to be predicted.
\item The epoch defining the position of the body:  the moment in time
      at which, in the absence of purturbations, the specified
      position---mean longitude, mean anomaly, or perihelion---is
      reached.
\item The epoch of osculation:  the moment in time at which the given
      elements precisely specify the body's position and velocity.
\end{itemize}
 
For the major-planet and minor-planet cases it is usual to make
the epoch that defines the position of the body the same as the
epoch of osculation.  Thus, for planets (major and
minor) only two different epochs are
involved:  the epoch of the elements and the epoch of observation.
For comets, the epoch of perihelion fixes the position in the
orbit and in general a different epoch of osculation will be
chosen.  Thus, for comets all three types of epoch are involved.
How many of the three elements are present in a given SLALIB
argument list depends on the function concerned.
 
Two important sources for orbital elements are the {\it Horizons}\/
service, operated by the Jet Propulsion Laboratory, Pasadena,
and the Minor Planet Center, operated by the Center for
Astrophysics, Harvard.
The JPL elements (heliocentric, J2000 ecliptic and
equinox) and MPC elements
correspond to SLALIB arguments as shown in the following table,
where ``(rad)'' means conversion from degrees to radians, and
``(MJD)'' means ``subtract {\tt 2400000.5}'':
 
\vspace{2ex}
 
\begin{small}
\begin{tabular}{|c||c|c|c||c|c|} \hline
{\it SLALIB } & \multicolumn{3}{c||}{\it JPL}
 & \multicolumn{2}{c|}{\it MPC} \\
{\it argument} & {\it major planet} & {\it minor planet} &
{\it comet} & {\it minor planet} & {\it comet} \\
\hline \hline
{\tt jform} & {\tt 1} & {\tt 2} & {\tt 3} & {\tt 2} & {\tt 3} \\
{\tt epoch} & {\tt JDCT} (MJD) & {\tt JDCT} (MJD) & {\tt Tp} (MJD) &
                                    {\tt Epoch} (MJD) & {\tt T} (MJD) \\
{\tt orbinc} & {\tt IN} (rad) & {\tt IN} (rad) & {\tt IN} (rad) &
                                {\tt Incl.}\ (rad) & {\tt Incl.}\ (rad) \\
{\tt anode} & {\tt OM} (rad) & {\tt OM} (rad) & {\tt OM} (rad) &
                                 {\tt Node} (rad) & {\tt Node.}\ (rad) \\
{\tt perih} & {\tt OM+W} (rad) & {\tt W} (rad) & {\tt W} (rad) &
                              {\tt Perih.}\ (rad) & {\tt Perih.}\ (rad) \\
{\tt aorq} & {\tt A} & {\tt A} & {\tt QR} & {\tt a} & {\tt q} \\
{\tt e} & {\tt EC} & {\tt EC} & {\tt EC} & {\tt e} & {\tt e} \\
{\tt aorl} & {\tt MA+OM+W} (rad) & {\tt MA} (rad) & & {\tt M} (rad) & \\
{\tt dm} & {\tt N} (rad) & & & & \\  \hline
{\it epoch of osculation} & {\tt JDCT} (MJD)
                          & {\tt JDCT} (MJD)
                          & {\tt JDCT} (MJD)
                          & {\tt Epoch} (MJD)
                          & {\tt Epoch} (MJD) \\
\hline
\end{tabular}
\end{small}\\[3ex]
 
Conventional elements are not the only way of specifying an orbit.
The \xyzxyzd\ state vector is an equally valid specification,
and the so-called {\it method of universal variables}\/ allows
orbital calculations to be made directly, bypassing angular
quantities and avoiding Kepler's Equation.  The universal-variables
approach has various advantages, including better handling of
near-parabolic cases and greater efficiency.
SLALIB uses universal variables for its internal
calculations and also offers a number of functions which
applications can call.
 
The universal elements are the \xyzxyzd\ and its epoch, plus the mass
of the body.  The SLALIB functions supplement these elements with
certain redundant values in order to
avoid unnecessary recomputation when the elements are next used.
 
The functions
{\tt slaEl2ue} and
{\tt slaUe2el} transform conventional elements into the
universal form and {\it vice versa.}
The function
{\tt slaPv2ue} takes an \xyzxyzd\ and forms the set of universal
elements;
{\tt slaUe2pv} takes a set of universal elements and predicts the \xyzxyzd\
for a specified epoch.
The function
{\tt slaPertue} provides updated universal elements,
taking into account perturbations from the major planets.
Starting with universal elements, the function
{\tt slaPlantu} (the universal elements equivalent of
{\tt slaPlante}) predicts topocentric \radec.

The following C program demonstrates the use of the
{\tt slaPertel} and
{\tt slaPlante} functions for
predicting the astrometric place of a minor planet seen from a specified
place at a specified time.  The prediction is carried out twice,
first with elements that are up to date and then with elements several
years old:
\goodbreak
\vspace{-3ex}
\begin{verbatim}
      #include <stdio.h>
      #include <slalib.h>
      void predict ( double, double, double, int, int, int, double,
                     double, double, double, double, double, double );

      #define D2R 0.017453292519943  /* degrees to radians */

      int main ( )

      /*  Two predictions of astrometric RA,Dec for a minor    */
      /*  planet, using both stale and fresh orbital elements. */

      {
         int j;
         double el, phi, utc, fday, tt;


      /* UTC date and time: 2006 Apr 1 04:10:23. */
         slaCldj ( 2006, 4, 1, &utc, &j );
         slaDtf2d ( 4, 10, 23.0, &fday, &j );
         utc += fday;

      /* TT. */
         tt = utc + slaDtt ( utc ) / 86400.0;

      /* Site: 50 45 N, 01 42 W. */
         el = -1.7*D2R;
         phi = 50.75*D2R;

      /*  Orbital elements for a minor planet, from the MPCORBcr.DAT file.

      A) Fresh elements, close to the date of observation:

      21903   12.7   0.15 K0636  16.72954   72.75790   29.18688    9.73664...
        0.0439154  0.18492020   3.0512642  0 MPO 27572   165   5 1991-2002...
       0.50 M-v 38h MPC        0000  (21903)Wallace             20020220

      B) Stale elements, several years old:

      21903   12.7   0.15 K0256 118.68597   72.04680   29.21207    9.73712...
        0.0448818  0.18492704   3.0511890  2 MPO  8901   118   4 1991-2001...
       0.51 M-v 38h Williams   0000 (21903) Wallace             20010203

      */

      /* Prediction using up-to-date elements. */
         predict ( tt, el, phi, 2, 2006, 3, 6.0, 9.73664*D2R, 29.18688*D2R,
                   72.75790*D2R, 3.0512642, 0.0439154, 16.72954*D2R );

      /* Prediction using stale elements. */
         predict ( tt, el, phi, 2, 2002, 5, 6.0, 9.73712*D2R, 29.21207*D2R,
                   72.04680*D2R, 3.0511890, 0.0448818, 118.68597*D2R );

         return 0;
      }

      void predict ( double tt, double el, double phi, int jform,
                     int iyear, int imonth, double days,
                     double orbinc, double anode, double perih,
                     double aorq, double e, double am )
      /*
      **  Perform the prediction and report the results.
      **
      **  Given:
      **     tt      d   date of observation (TT MJD)
      **     el      d   longitude, east +ve (radians)
      **     phi     d   latitude (radians)
      **     jform   i   2 = minor planet, 3 = comet
      **     iyear   i   year            } epoch of elements;
      **     imonth  i   month           } for comets, also date
      **     days    d   day + fraction  } of perihelion passage
      **     orbinc  d   inclination (radians)
      **     anode   d   longitude of ascending node (radians)
      **     perih   d   argument of perihelion (radians)
      **     aorq    d   a, for minor planets, or q, for comets (AU)
      **     e       d   eccentricity
      **     am      d   mean anomaly
      **
      **  n.b. Validation has been omitted.  A real application would check
      **       all returned status values.
      */

      {
         int iday, j, m[4];
         double fday, epoch, ep, orbi, ano, ph, aq, ecc, anom,
                rt, dt, r, ra, da, rm, dm;
         char s;


      /* Transform epoch from calendar format to MJD. */
         iday = (int) days;
         fday = days - (double) iday;
         slaCldj ( iyear, imonth, iday, &epoch, &j );
         epoch += fday;

      /* Update the elements for planetary perturbations. */
         slaPertel ( jform, epoch, tt,
                     epoch, orbinc, anode, perih, aorq, e, am,
                     &ep, &orbi, &ano, &ph, &aq, &ecc, &anom,
                     &j );

      /* Predict topocentric (i.e. no refraction) apparent place. */
         slaPlante ( tt, el, phi, jform, ep, orbi, ano, ph, aq, ecc, anom,
                     0.0, &rt, &dt, &r, &j );

      /* To astrometric J2000 (UT=TT, J2000=ICRS, near enough). */
         slaOap ( "r", rt, dt, tt, 0.0, el, phi, 0.0, 0.0, 0.0,
                  0.0, 0.0, 0.0, 0.0, 0.0, &ra, &da );
         slaAmp ( ra, da, tt, 2000.0, &rm, &dm );

      /* Report. */
         slaDr2tf ( 2, rm, &s, m );
         printf ( "J2000 astrometric: %2.2i %2.2i %2.2i.%2.2i  ",
                                                    m[0], m[1], m[2], m[3] );
         slaDr2af ( 1, dm, &s, m );
         printf ( "%c%2.2i %2.2i %2.2i.%1.1i\n", s, m[0], m[1], m[2], m[3] );

      }
\end{verbatim}
\vspace{-3ex}
\goodbreak
The program produces the following output; the first line is the
prediction using up to date elements, the second using stale elements:
\goodbreak
\vspace{-3ex}
\begin{verbatim}
      J2000 astrometric: 07 12 20.27  +33 03 13.9
      J2000 astrometric: 07 12 20.31  +33 03 13.9
\end{verbatim}
\vspace{-3ex}
\goodbreak
The results from the two sets of elements are very similar, showing that
in the second case
{\tt slaPertel} has successfully compensated for perturbations in the
four-year interval between the orbital elements and the date of observation.

If the program is modified so that the call to
{\tt slaPertel} is omitted,
the results using fresh elements (first line) are little affected,
but the results using four-year-old elements (second line)
are markedly worse:
\goodbreak
\vspace{-3ex}
\begin{verbatim}
      J2000 astrometric: 07 12 20.16  +33 03 14.2
      J2000 astrometric: 07 13 41.72  +33 01 58.8
\end{verbatim}
\vspace{-3ex}
\goodbreak

\subsection{Radial Velocity and Light-Time Corrections}
When publishing high-resolution spectral observations
it is necessary to refer them to a specified standard of rest.
This involves knowing the component in the direction of the
source of the velocity of the observer.  SLALIB provides a number
of functions for this purpose, allowing observations to be
referred to the Earth's centre, the Sun, a Local Standard of Rest
(either dynamical or kinematical), the centre of the Galaxy, and
the mean motion of the Local Group.
 
The function
{\tt slaRverot}
corrects for the diurnal rotation of
the observer around the Earth's axis.  This is always less than 0.5\,km/s.
 
No specific function is provided to correct a radial velocity
from geocentric to heliocentric, but this can easily be done by calling
{\tt slaEvp}
as follows:
\goodbreak
\vspace{-3ex}
\begin{verbatim}
         double rm, dm, v[3], dvb[3], dpb[3], dvh[3], dph[3], vcorb;
          :
      /* Star vector, J2000. */
         slaDcs2c ( rm, dm, v );
 
      /* Earth/Sun velocity and position, J2000. */
         slaEvp ( tdb, 2000.0, dvb, dpb, dvh, dph );
 
      /* Radial velocity correction due to Earth orbit (km/s). */
         vcorb = - slaDvdv ( v, dvh ) * 149.597870e6;
          :
\end{verbatim}
\vspace{-3ex}
\goodbreak
The maximum value of this correction is the Earth's orbital speed
of about 30\,km/s.  A related function,
{\tt slaEcor},
computes the light-time correction with respect to the Sun.  It
would be used when reducing observations of a rapid variable-star
for instance.
For pulsar work the
{\tt slaEvp} function is not sufficiently accurate for
phase predictions, being limited to about 25\,ms.  The
alternative
{\tt slaEpv} function will deliver pulse arrival times
accurate to 50\,$\mu$s, but is significantly slower.
 
To remove the intrinsic $\sim20$\,km/s motion of the Sun relative
to other stars in the solar neighbourhood,
a velocity correction to a
{\it local standard of rest}\/ (LSR) is required.  There are
opportunities for mistakes here.  There are two sorts of LSR,
{\it dynamical}\/ and {\it kinematical}, and
multiple definitions exist for the latter.  The
dynamical LSR is a point near the Sun which is in a circular
orbit around the Galactic centre;  the Sun has a ``peculiar''
motion relative to the dynamical LSR.  A kinematical LSR is
the mean standard of rest of specified star catalogues or stellar
populations, and its precise definition depends on which
catalogues or populations were used and how the analysis was
carried out.  The Sun's motion with respect to a kinematical
LSR is called the ``standard'' solar motion.  Radial
velocity corrections to the dynamical LSR are produced by the function
{\tt slaRvlsrd}
and to the adopted kinematical LSR by
{\tt slaRvlsrk}.
See the individual specifications for these functions for the
precise definition of the LSR in each case.
 
For extragalactic sources, the centre of the Galaxy can be used as
a standard of rest.  The radial velocity correction from the
dynamical LSR to the Galactic centre can be obtained by calling
{\tt slaRvgalc}.
Its maximum value is 220\,km/s.
 
For very distant sources it is appropriate to work relative
to the mean motion of the Local Group.  The function for
computing the radial velocity correction in this case is
{\tt slaRvlg}.
Note that in this case the correction is with respect to the
dynamical LSR, not the Galactic centre as might be expected.
This conforms to the IAU definition, and confers immunity from
revisions of the Galactic rotation speed.

\subsection{Focal-Plane Astrometry}
The relationship between the position of a star image in
the focal plane of a telescope and the star's celestial
coordinates is usually described in terms of the {\it tangent plane}\/
or {\it gnomonic}\/ projection.  This is the projection produced
by a pin-hole camera and is a good approximation to the projection
geometry of a traditional large {\it f}\/-ratio astrographic refractor.
SLALIB includes a group of functions which transform
star positions between their observed places on the celestial
sphere and their \xy\ coordinates in the tangent plane.  The
spherical coordinate system does not have to be \radec\ but
usually is.  The so-called {\it standard coordinates}\/ of a star
are the tangent plane \xy, in radians, with respect to an origin
at the tangent point, with the $y$-axis pointing north and
the $x$-axis pointing east (in the direction of increasing $\alpha$).
The factor relating the standard coordinates to
the actual \xy\ coordinates in, say, millimetres is simply
the focal length of the telescope.
 
Given the \radec\ of the {\it plate centre}\/ (the tangent point)
and the \radec\ of a star within the field, the standard
coordinates can be determined by calling
{\tt slaS2tp}
(single precision) or
{\tt slaDs2tp}
(double precision).  The reverse transformation, where the
\xy\ is known and we wish to find the \radec, is carried out by calling
{\tt slaTp2s}
or
{\tt slaDtp2s}.
Occasionally we know the both the \xy\ and the \radec\ of a
star and need to deduce the \radec\ of the tangent point;
this can be done by calling
{\tt slaTps2c}
or
{\tt slaDtps2c}.
(All of these transformations apply not just to \radec\ but to
other spherical coordinate systems, of course.)
Equivalent (and faster)
functions are provided which work directly in \xyz\ instead of
spherical coordinates:
{\tt slaV2tp} and
{\tt slaDv2tp},
{\tt slaTp2v} and
{\tt slaDtp2v},
{\tt slaTpv2c} and
{\tt slaDtpv2c}.
 
Even at the best of times, the tangent plane projection is merely an
approximation.  Some telescopes and cameras exhibit considerable pincushion
or barrel distortion and some have a curved focal surface.
For example, neither Schmidt cameras nor (especially)
large reflecting telescopes with wide-field corrector lenses
are adequately modelled by tangent-plane geometry.  In such
cases, however, it is still possible to do most of the work
using the (mathematically convenient) tangent-plane
projection by inserting an extra step which applies or
removes the distortion peculiar to the system concerned.
A simple $r_1=r_0(1+kr_0^2)$ law works well in the
majority of cases; $r_0$ is the radial distance in the
tangent plane, $r_1$ is the radial distance after adding
the distortion, and $k$ is a constant which depends on the
telescope; $\theta$ is unaffected.  The function
{\tt slaPcd}
applies the distortion to an \xy\ and
{\tt slaUnpcd}
removes it.  For \xy\ in radians, $k$ values range from $-1/3$ for the
tiny amount of barrel distortion in Schmidt geometry to several
hundred for the serious pincushion distortion
produced by wide-field correctors in big reflecting telescopes
(the AAT prime focus triplet corrector is about $k=+178.6$).
 
SLALIB includes a group of functions which can be put together
to build a simple plate-reduction program.  The heart of the group is
{\tt slaFitxy},
which fits a linear model to relate two sets of \xy\ coordinates,
in the case of a plate reduction the measured positions of the
images of a set of
reference stars and the standard
coordinates derived from their catalogue positions.  The
model is of the form:
\begin{eqnarray}
x_{p} & = & a + bx_{m} + cy_{m} \nonumber \\
y_{p} & = & d + ex_{m} + fy_{m} \nonumber
\end{eqnarray}
where the {\it p}\/ subscript indicates ``predicted'' coordinates
(the model's approximation to the ideal ``expected'' coordinates) and the
{\it m}\/ subscript indicates ``measured coordinates''.  The
six coefficients {\it a,b,c,d,e,f}\/ can optionally be
constrained to represent a ``solid body rotation'' free of
any squash or shear distortions.  Without this constraint
the model can, to some extent, accommodate effects like refraction,
allowing mean places to be used directly and
avoiding the extra complications of a
full mean-apparent-observed transformation for each star.
Having obtained the linear model,
{\tt slaPxy}
can be used to process the set of measured and expected
coordinates, giving the predicted coordinates and determining
the RMS residuals in {\it x}\/ and {\it y}.
The function
{\tt slaXy2xy}
transforms one \xy\ into another using the linear model.  A model
can be inverted by calling
{\tt slaInvf},
and decomposed into zero points, scales, $x/y$ nonperpendicularity
and orientation by calling
{\tt slaDcmpf}.

\subsection{Numerical Methods}
SLALIB contains a small number of simple, general-purpose
numerical-methods functions.

Applications which perform a least-squares fit using a traditional
normal-equations methods can accomplish the required matrix-inversion
by calling either
{\tt slaSmat}
({\tt float} precision) or
{\tt slaDmat}
({\tt double}).  A generally better way to perform such fits is
to use singular value decomposition.  SLALIB provides a function
to do the decomposition itself,
{\tt slaSvd},
and two functions to use the results:
{\tt slaSvdsol}
generates the solution, and
{\tt slaSvdcov}
produces the covariance matrix.
A simple demonstration of the use of the SLALIB SVD
functions is given below.  It generates 500 simulated data
points and fits them to a model which has four unknown coefficients.
The arrays in the example are sized to accept up to 1000
points and 10 unknowns.  The model is:
\[ y = c_{1} +c_{2}x +c_{3}sin{x} +c_{4}cos{x} \]
The test values for the four coefficients are
$c_1\!=\!+50.0$,
$c_2\!=\!-2.0$,
$c_3\!=\!-10.0$ and
$c_4\!=\!+25.0$.
Gaussian noise, $\sigma=2.0$, is added to each
``observation''.\footnote{The internal functions {\tt Gresid}
and {\tt Random} are useful in their own right.  Functions like them
are not included in SLALIB because they contain
{\tt static} variables and hence are not re-entrant.}
\vspace{-3ex}
\goodbreak
\begin{verbatim}
      #include <stdio.h>
      #include <slalib.h>
      double Gresid ( double );
      double Random ( double );

      int main ( )

      /* Sizes of arrays, physical and logical */
      #define MP 1000 /* maximum number of samples allowed for  */
      #define NP 10   /* maximum number of unknowns allowed for */
      #define NC NP   /* dimension of covariance matrix         */
      #define M 500   /* actual number of samples               */
      #define N 4     /* actual number of unknowns              */

      /* The unknowns we are going to solve for. */
      #define C1 50.0
      #define C2 -2.0
      #define C3 -10.0
      #define C4 25.0

      {
         double a[MP][NP], w[NP], v[NP][NP], work[NP], b[MP], x[NP], cvm[NC][NC];
         double val, bf1, bf2, bf3, bf4, sd2, d, var;
         int i, j;

      /* Fill the design matrix. */
         for ( i = 0; i < M; i++ ) {

         /* Dummy independent variable. */
            val = ( (double) i ) / 10.0;

         /* Fill one row of the design matrix with the basis functions. */
            a[i][0] = bf1 = 1.0;
            a[i][1] = bf2 = val;
            a[i][2] = bf3 = sin ( val );
            a[i][3] = bf4 = cos ( val );

         /* The observed value, including noise. */
            b[i] = C1*bf1 + C2*bf2 + C3*bf3 + C4*bf4 + Gresid(2.0);
         }

      /* Factorize the design matrix, solve and generate covariance matrix. */
         slaSvd ( M, N, MP, NP, (double*) a, w, (double*) v, work, &j );
         slaSvdsol ( M, N, MP, NP, b, (double*) a, w, (double*) v, work, x );
         slaSvdcov ( N, NP, NC, w, (double*) v, work, (double*) cvm );

      /* Compute the variance. */
         sd2 = 0.0;
         for ( i = 0; i < M; i++ ) {
            val = ( (double) i ) / 10.0;
            bf1 = 1.0;
            bf2 = val;
            bf3 = sin ( val );
            bf4 = cos ( val );
            d = b[i] - ( x[0]*bf1 + x[1]*bf2 + x[2]*bf3 + x[3]*bf4 );
            sd2 += d*d;
         }
         var = sd2 / ( (double) M );

      /* Report the RMS and the solution. */
         printf ( "RMS = %5.2f\n\n", sqrt ( var ) );
         for ( i = 0; i < N; i++ ) {
            printf ( "C%1d = %+7.3f +/- %6.3f\n", i+1, x[i], sqrt(var*cvm[i][i]) );
         }

         return 0;
      }

      /* Pseudo-random normal deviate by Box-Muller method. */
      double Gresid ( double s )
      {
         static double seed = 1.0;
         static int jftf = 1;
         static double x, y;
         double w, rsq;

         if ( jftf ) {
            do {
               x = 2.0*Random(seed) - 1.0;
               y = 2.0*Random(seed) - 1.0;
               rsq = x*x + y*y;
            } while ( rsq >= 1.0 );
            w = ( rsq > 1e-30 ) ? sqrt(-2.0*log(rsq)/rsq) : 0.0;
            x *= w;
            y *= w;
            jftf = 0;
         } else {
            x = y;
            jftf = 1;
         }
         return x * s;
      }

      /* Generate random number between 0.0 and 1.0. */
      #include <stdlib.h>
      #include <limits.h>
      #include <slamac.h>
      double Random ( double seed )
      {
         static int jftf = 1;

         if ( jftf )
         {
            jftf = 0;
            srand ( gmin(UINT_MAX, (unsigned int) gmax(1.0,fabs(seed))) );
         }
         return (double) rand() / RAND_MAX;
      }
\end{verbatim}
\vspace{-3ex}
\goodbreak
The program produces output like (but not necessarily
identical to, because of the platform-dependent vagaries of the random
number generator) the following:
\goodbreak
\vspace{-3ex}
\begin{verbatim}
      RMS =  2.01

      C1 = +49.759 +/-  0.180
      C2 =  -1.992 +/-  0.006
      C3 =  -9.962 +/-  0.127
      C4 = +25.186 +/-  0.127
\end{verbatim}
\vspace{-3ex}
\goodbreak
In this above example, essentially
identical results would be obtained if the more
commonplace normal-equations method had been used,
and the relatively large
$1000\times 10$ array would have been avoided.  However, the SVD method
comes into its own when the opportunity is taken to edit the W-matrix
(the so-called ``singular values'') in order to control
possible ill-conditioning.  The procedure involves replacing with
zeroes any W-elements smaller than a nominated value, for example
0.001 times the largest W-element.  Small W-elements indicate
ill-conditioning, which in the case of the normal-equations
method would produce spurious large coefficient values and
possible arithmetic overflows.  Using SVD, the effect on the solution
of setting suspiciously small W-elements to zero is to restrain
the offending coefficients from moving very far.  The
fact that action was taken can be reported to show the program user that
something is amiss.  Furthermore, if element W({\it j}\/) was set to zero,
the row numbers of the two biggest elements in the
{\it j}\/th column of the
V-matrix identify the pair of solution coefficients that are
dependent.
 
A more detailed description of SVD and its use in least-squares
problems would be out of place here, and the reader is urged
to refer to the relevant sections of the book {\it Numerical Recipes}
(Press {\it et al.}, Cambridge University Press, 1987).
 
The functions
{\tt slaCombn}
and
{\tt slaPermut}
are useful for problems which involve combinations (different subsets)
and permutations (different orders).
Both return the next in a sequence of results, cycling through all the
possible results as the function is called repeatedly.

\vfill

\pagebreak
 
\section{SUBPROGRAM SPECIFICATIONS}
\ifx\vdoc\vshort
\abridged
\else
%-----------------------------------------------------------------------
\routine{slaAddet}{add E-terms of aberration}
{
 \action{Add the E-terms (elliptic component of annual aberration) to a
  pre IAU 1976 mean place to conform to the old catalogue convention.}
 \call{slaAddet( rm, dm, eq, \&rc, \&dc );}
}
\args{GIVEN}
{
 \spec{rm,dm}{double}{\radec\ without E-terms (radians)} \\
 \spec{eq}{\tt double}{Besselian epoch of mean equator and equinox}
}
\args{RETURNED}
{
 \spec{rc,dc}{double*}{\radec\ with E-terms included (radians)}
}
\anote{Most star positions from pre-1984 optical catalogues (or
       obtained by astrometry with respect to such stars) have the
       E-terms built-in.  If it is necessary to convert a formal mean
       place (for example a pulsar timing position) to one
       consistent with such a star catalogue, then the
       \radec\ should be adjusted using this function.}
\aref{{\it Explanatory Supplement to the Astronomical Ephemeris},
 section 2D, page 48.}
%-----------------------------------------------------------------------
\routine{slaAfin}{sexagesimal character string to angle}
{
 \action{Decode a free-format sexagesimal string (degrees, arcminutes,
         arcseconds) into a single precision floating point
         number (radians).}
 \call{slaAfin( string, \&nstrt, \&reslt, \&j );}
}
\args{GIVEN}
{
 \spec{string}{char*}{string containing deg, arcmin, arcsec fields} \\
 \spec{nstrt}{int*}{index to start of decode (beginning of string = 1)}
}
\args{RETURNED}
{
 \spec{nstrt}{int*}{advanced past the decoded angle} \\
 \spec{reslt}{float*}{angle in radians} \\
 \spec{j}{int*}{status:} \\
 \spec{}{}{\hspace{1.5em}   0 = OK} \\
 \spec{}{}{\hspace{0.7em} $+1$ = default, {\tt reslt} unchanged (Note~2)} \\
 \spec{}{}{\hspace{0.7em} $-1$ = bad degrees (Note~3)} \\
 \spec{}{}{\hspace{0.7em} $-2$ = bad arcminutes (Note~3)} \\
 \spec{}{}{\hspace{0.7em} $-3$ = bad arcseconds (Note~3)} \\
}
\goodbreak
\setlength{\oldspacing}{\topsep}
\setlength{\topsep}{0.3ex}
\begin{description}
 \exampleitem \\ [1.5ex]
  \begin{tabular}{p{7em}p{15em}p{12em}}
   {\it argument} & {\it before} & {\it after} \\ \\
   string & $'$\verb*}-57 17 44.806  12 34 56.7}$'$ & unchanged \\
   nstrt & 1 & 16 ({\it i.e.}~pointing to 12\ldots) \\
   reslt & - & $-1.00000$ \\
   j & - & 0
  \end{tabular}
\end{description}
A further call to
{\tt slaAfin}, without adjustment of {\tt nstrt},
will decode the second angle, \dms{12}{34}{56}{7}.
\setlength{\topsep}{\oldspacing}
\notes
{
 \begin{enumerate}
  \item The first three ``fields'' in {\tt string} are degrees, arcminutes,
   arcseconds, separated by spaces or commas.  The degrees field
   may be signed, but not the others.  The decoding is carried
   out by the {\tt slaDfltin} function and is free-format.
  \item Successive fields may be absent, defaulting to zero.  For
   zero status, the only combinations allowed are degrees alone,
   degrees and arcminutes, and all three fields present.  If all
   three fields are omitted, a status of +1 is returned and {\tt reslt} is
   unchanged.  In all other cases {\tt reslt} is changed.
  \item Range checking:
   \begin{itemize}
    \item The degrees field is not range checked.  However, it is
     expected to be integral unless the other two fields are absent.
    \item The arcminutes field is expected to be 0-59, and integral if
     the arcseconds field is present.  If the arcseconds field
     is absent, the arcminutes is expected to be 0-59.9999\ldots
    \item The arcseconds field is expected to be 0-59.9999\ldots
    \item Decoding continues even when a check has failed.  Under these
     circumstances the field takes the supplied value, defaulting to
     zero, and the result {\tt reslt} is computed and returned.
   \end{itemize}
   \item Further fields after the three expected ones are not treated as
    an error.  The index {\tt nstrt} is left in the correct state for
    further decoding with the present function or with {\tt slaDfltin}
    {\it etc}.  See the example, above.
   \item If {\tt string} contains hours, minutes, seconds instead of
    degrees {\it etc.},
    or if the required units are turns (or days) instead of radians,
    the result {\tt reslt} should be multiplied as follows: \\ [1.5ex]
    \begin{tabular}{p{6em}p{5em}p{15em}}
    {\it for {\tt string}} & {\it to obtain} & {\it multiply {\tt reslt} by} \\ \\
    ${\circ}$~~\raisebox{-0.7ex}{$'$}~~\raisebox{-0.7ex}{$''$}
     & radians & $1.0$ \\
    ${\circ}$~~\raisebox{-0.7ex}{$'$}~~\raisebox{-0.7ex}{$''$}
     & turns & $1/{2 \pi} = 0.1591549430918953358$ \\
    h m s & radians & $15.0$ \\
    h m s & days & $15/{2\pi} = 2.3873241463784300365$ \\
   \end{tabular}
 \end{enumerate}
}
%-----------------------------------------------------------------------
\routine{slaAirmas}{air mass}
{
 \action{Air mass at given zenith distance.}
 \call{d~=~slaAirmas( zd );}
}
\args{GIVEN}
{
 \spec{zd}{double}{observed zenith distance (radians)}
}
\args{RETURNED}
{
 \spec{}{double}{air mass (1 at zenith)}
}
\notes
{
 \begin{enumerate}
  \item The {\it observed}\/ zenith distance referred to above means
        ``as affected by refraction''.
  \item The function uses Hardie's (1962) polynomial fit to Bemporad's
        data for the relative air mass, $X$, in units of thickness at the
        zenith as tabulated by Schoenberg (1929). This is adequate for all
        normal needs as it is accurate to better than
        0.1\% up to $X = 6.8$ and better than 1\% up to $X = 10$.
        Bemporad's tabulated values are unlikely to be trustworthy
        to such accuracy
        because of variations in density, pressure and other
        conditions in the atmosphere from those assumed in his work.
  \item The sign of the ZD is ignored.
  \item At zenith distances greater than about $\zeta = 87^{\circ}$ the
        air mass is held constant to avoid arithmetic overflows.
 \end{enumerate}
}
\refs
{
 \begin{enumerate}
  \item Hardie, R.H.\ 1962, in {\it Astronomical Techniques}\,
        ed. W.A.~Hiltner, University of Chicago Press, p180.
  \item Schoenberg, E.\ 1929, {\it Hdb.d.Ap.},
        Berlin, Julius Springer, 2, 268.
 \end{enumerate}
}
%-----------------------------------------------------------------------
\routine{slaAltaz}{velocities {\it etc.}\ for altazimuth mount}
{
 \action{Positions, velocities and accelerations for an altazimuth
         telescope mount that is tracking a star.}
 \call{slaAltaz( \vtop{
         \hbox{ha, dec, phi,}
         \hbox{\&az, \&azd, \&azdd,
               \&el, \&eld, \&eldd,
               \&pa, \&pad, \&padd );}}}
}
\args{GIVEN}
{
 \spec{ha}{double}{hour angle} \\
 \spec{dec}{double}{declination} \\
 \spec{phi}{double}{observatory latitude}
}
\args{RETURNED}
{
 \spec{az}{double*}{azimuth} \\
 \spec{azd}{double*}{azimuth velocity} \\
 \spec{azdd}{double*}{azimuth acceleration} \\
 \spec{el}{double*}{elevation} \\
 \spec{eld}{double*}{elevation velocity} \\
 \spec{eldd}{double*}{elevation acceleration} \\
 \spec{pa}{double*}{parallactic angle} \\
 \spec{pad}{double*}{parallactic angle velocity} \\
 \spec{padd}{double*}{parallactic angle acceleration}
}
\notes
{
 \begin{enumerate}
  \setlength{\parskip}{\medskipamount}
  \item Natural units are used throughout.
        The arguments {\tt ha}, {\tt dec}, {\tt phi}, {\tt az}, {\tt el}
        and {\tt zd} are in radians.  The velocities and accelerations
        assume constant declination and constant rate of change of
        hour angle (as for tracking a star);  the units of
        {\tt azd}, {\tt eld} and {\tt pad} are radians per radian of HA,
        while the units of {\tt azdd}, {\tt eldd} and {\tt padd} are
        radians per radian of HA squared.  To
        convert into practical degree- and second-based units:
 
        \begin{center}
        \begin{tabular}{rlcl}
                  angles & $\times 360/2\pi$ & $\rightarrow$ & degrees \\
              velocities & $\times (2\pi/86400) \times (360/2\pi)$
                                             & $\rightarrow$ & degree/sec \\
           accelerations & $\times (2\pi/86400)^2 \times (360/2\pi)$
                                             & $\rightarrow$ & degree/sec/sec \\
        \end{tabular}
        \end{center}
 
        Note that the seconds here are sidereal rather than SI.  One
        sidereal second is about 0.99727 SI seconds.
 
        The velocity and acceleration factors assume the sidereal
        tracking case.  Their respective numerical values are (exactly)
        1/240 and (approximately) 1/3300236.9.
  \item Azimuth is returned in the range $[\,0,2\pi\,]$;  north is zero,
        and east is $+\pi/2$.  Elevation and parallactic angle are
        returned in the range $\pm\pi$.  Position angle is positive
        for a star west of the meridian and is the angle NP--star--zenith.
  \item The latitude is geodetic as opposed to geocentric.  The
        hour angle and declination are topocentric.  Refraction and
        deficiencies in the telescope mounting are ignored.  The
        purpose of the function is to give the general form of the
        quantities.  The details of a real telescope could profoundly
        change the results, especially close to the zenith.
  \item No range checking of arguments is carried out.
  \item In applications which involve many such calculations, rather
        than calling the present function it will be more efficient to
        use inline code, having previously computed fixed terms such
        as sine and cosine of latitude, and (for tracking a star)
        sine and cosine of declination.
 \end{enumerate}
}
%-----------------------------------------------------------------------
\routine{slaAmp}{apparent to mean}
{
 \action{Convert star \radec\ from geocentric apparent to mean place.}
 \call{slaAmp( ra, da, date, eq, \&rm, \&dm );}
}
\args{GIVEN}
{
 \spec{ra,da}{double}{apparent \radec\ (radians)} \\
 \spec{date}{double}{TDB for apparent place (JD$-$2400000.5)} \\
 \spec{eq}{double}{equinox: Julian epoch of mean place}
}
\args{RETURNED}
{
 \spec{rm,dm}{double*}{mean \radec\ (radians)}
}
\notes
{
 \begin{enumerate}
  \item The distinction between the required TDB and TT is
        always negligible.  Moreover, for all but the most
        critical applications UTC is adequate.
  \item Iterative techniques are used for the aberration and
        light deflection corrections so that the functions
        {\tt slaAmp} (or {\tt slaAmpqk}) and
        {\tt slaMap} (or {\tt slaMapqk}) are
        accurate inverses;  even at the edge of the Sun's disc
        the discrepancy is only about 1~nanoarcsecond.
  \item Where multiple apparent places are to be converted to
        mean places, for a fixed date and equinox, it is more
        efficient to use the {\tt slaMappa} function to compute the
        required parameters once, followed by one call to
        {\tt slaAmpqk} per star.
  \item For {\tt eq}\,=\,2000.0, the agreement with ICRS
        is sub-mas, limited by the
        precession-nutation model (IAU 1976 precession, Shirai \&
        Fukushima 2001 forced nutation and precession corrections).
  \item The accuracy is further limited by the function
        {\tt slaEvp}, called
        by {\tt slaMappa}, which computes the Earth position and
        velocity using the methods of Stumpff.  The maximum
        error is about 0.3~milliarcsecond.
 \end{enumerate}
}
\refs
{
 \begin{enumerate}
 \item 1984 {\it Astronomical Almanac}, pp B39-B41.
 \item Lederle \& Schwan 1984, {\it Astr.Astrophys.}\ {\bf 134}, 1-6.
 \item Shirai, T. \& Fukushima, T.\ 2001, {\it Astron.J.}, {\bf 121},
       3270-3283.
 \end{enumerate}
}
%-----------------------------------------------------------------------
\routine{slaAmpqk}{quick apparent to mean}
{
 \action{Convert star \radec\ from geocentric apparent to mean place.
         Use of this function is appropriate when
         efficiency is important and where many star positions are
         all to be transformed for one epoch and equinox.  The
         star-independent parameters can be obtained by calling
         the {\tt slaMappa} function.}
 \call{slaAmpqk( ra, da, amprms, \&rm, \&dm );}
}
\args{GIVEN}
{
 \spec{ra,da}{double}{apparent \radec\ (radians)} \\
 \spec{amprms}{double[21]}{star-independent mean-to-apparent parameters:} \\
 \specel{0}     {time interval for proper motion (Julian years)} \\
 \specel{1-3}   {barycentric position of the Earth (AU)} \\
 \specel{4-6}   {heliocentric direction of the Earth (unit vector)} \\
 \specel{7}     {(gravitational radius of
                   Sun)$\times 2 / $(Sun-Earth distance)} \\
 \specel{8-10}  {{\bf v}: barycentric Earth velocity in units of c} \\
 \specel{11}    {$(1-\left|\mbox{\bf v}\right|^2)^{1/2}$} \\
 \specel{12-20} {precession-nutation $3\times3$ matrix}
}
\args{RETURNED}
{
 \spec{rm,dm}{double*}{mean \radec\ (radians)}
}
\anote{Iterative techniques are used for the aberration and
       light deflection corrections so that the functions
       {\tt slaAmp} (or {\tt slaAmpqk}) and
       {\tt slaMap} (or {\tt slaMapqk}) are
       accurate inverses;  even at the edge of the Sun's disc
       the discrepancy is only about 1~nanoarcsecond.}
\refs
{
 \begin{enumerate}
 \item 1984 {\it Astronomical Almanac}, pp B39-B41.
 \item Lederle \& Schwan, 1984.\ {\it Astr.Astrophys.}\ {\bf 134}, 1-6.
 \end{enumerate}
}
%-----------------------------------------------------------------------
\routine{slaAop}{apparent to observed}
{
 \action{Apparent to observed place, for sources distant from
         the solar system.}
 \call{slaAop( \vtop{
         \hbox{rap, dap, utc, dut, elongm, phim, hm, xp, yp,}
         \hbox{tdk, pmb, rh, wl, tlr, \&aob, \&zob, \&hob, \&dob, \&rob);}}}
}
\args{GIVEN}
{
 \spec{rap,dap}{double}{geocentric apparent \radec\ (radians)} \\
 \spec{utc}{double}{UTC date/time (Modified Julian Date, JD$-$2400000.5)} \\
 \spec{dut}{double}{$\Delta$UT:  UT1$-$UTC (UTC seconds)} \\
 \spec{elongm}{double}{observer's mean longitude (radians, east +ve)} \\
 \spec{phim}{double}{observer's mean geodetic latitude (radians)} \\
 \spec{hm}{double}{observer's height above sea level (metres)} \\
 \spec{xp,yp}{double}{polar motion \xy\ coordinates (radians)} \\
 \spec{tdk}{double}{local ambient temperature (K; std=273.15)} \\
 \spec{pmb}{double}{local atmospheric pressure (mb; std=1013.25)} \\
 \spec{rh}{double}{local relative humidity (in the range 0.0\,--\,1.0)} \\
 \spec{wl}{double}{effective wavelength ($\mu{\rm m}$, {\it e.g.}\ 0.55)} \\
 \spec{tlr}{double}{tropospheric lapse rate (K per metre,
                                              {\it e.g.}\ 0.0065)}
}
\args{RETURNED}
{
 \spec{aob}{double*}{observed azimuth (radians: N=0, E=$90^{\circ}$)} \\
 \spec{zob}{double*}{observed zenith distance (radians)} \\
 \spec{hob}{double*}{observed hour angle (radians)} \\
 \spec{dob}{double*}{observed $\delta$ (radians)} \\
 \spec{rob}{double*}{observed $\alpha$ (radians)}
}
\notes
{
 \begin{enumerate}
  \item This function returns zenith distance rather than elevation
        in order to reflect the fact that no allowance is made for
        depression of the horizon.
  \item The accuracy of the result is limited by the corrections for
        refraction.  Providing the meteorological parameters are
        known accurately and there are no gross local effects, the
        predicted azimuth and elevation should be within about
        \arcsec{0}{1} for $\zeta<70^{\circ}$.  Even
        at a topocentric zenith distance of
        $90^{\circ}$, the accuracy in elevation should
        normally be better than
        1~arcminute;  useful results are available for a further
        $3^{\circ}$, beyond which the {\tt slaRefro} function returns a
        fixed value of the refraction.  The complementary
        functions {\tt slaAop} (or {\tt slaAopqk})
        and {\tt slaOap} (or {\tt slaOapqk})
        are self-consistent to better than 1~microarcsecond all over
        the celestial sphere.
  \item It is advisable to take great care with units, as even
        unlikely values of the input parameters are accepted and
        processed in accordance with the models used.
  \item {\it Apparent}\/ \radec\ means the geocentric apparent
        right ascension
        and declination, which is obtained from a catalogue mean place
        by allowing for space motion, parallax, the Sun's gravitational
        lens effect, annual aberration, and precession-nutation.  For
        star positions in the FK5 system ({\it i.e.}~J2000), these
        effects can
        be applied by means of the {\tt slaMap} {\it etc.}\ functions.
        Starting from
        other mean place systems, additional transformations will be
        needed;  for example, FK4 ({\it i.e.}~B1950) mean places would
        first have to be converted to FK5, which can be done with the
        {\tt slaFk425} {\it etc.}\ functions.
  \item {\it Observed}\/ \azel\ means the position that would be seen by a
        perfect theodolite located at the observer.  This is obtained
        from the geocentric apparent \radec\ by allowing for Earth
        orientation and diurnal aberration, rotating from equator
        to horizon coordinates, and then adjusting for refraction.
        The \hadec\ is obtained by rotating back into equatorial
        coordinates, using the geodetic latitude corrected for polar
        motion, and is the position that would be seen by a perfect
        equatorial located at the observer and with its polar axis
        aligned to the Earth's axis of rotation ({\it n.b.}\ not to the
        refracted pole).  Finally, the $\alpha$ is obtained by subtracting
        the {\it h}\/ from the local apparent ST.
  \item To predict the required setting of a real telescope, the
        observed place produced by this function would have to be
        adjusted for the tilt of the azimuth or polar axis of the
        mounting (with appropriate corrections for mount flexures),
        for non-perpendicularity between the mounting axes, for the
        position of the rotator axis and the pointing axis relative
        to it, for tube flexure, for gear and encoder errors, and
        finally for encoder zero points.  Some telescopes would, of
        course, exhibit other properties which would need to be
        accounted for at the appropriate point in the sequence.
  \item This function takes time to execute, due mainly to the
        rigorous integration used to evaluate the refraction.
        For processing multiple stars for one location and time,
        call {\tt slaAoppa} once followed by
        one call per star to {\tt slaAopqk}.
        Where a range of times within a limited period of a few hours
        is involved, and the highest precision is not required, call
        {\tt slaAoppa} once, followed by a call
        to {\tt slaAoppat} each time the
        time changes, followed by one call per star to {\tt slaAopqk}.
  \item The {\tt utc} argument is UTC expressed as an MJD.  This is,
        strictly speaking, wrong, because of leap seconds.  However,
        as long as the $\Delta$UT and the UTC are consistent there
        are no difficulties, except during a leap second.  In this
        case, the start of the 61st second of the final minute should
        begin a new MJD day and the old pre-leap $\Delta$UT should
        continue to be used.  As the 61st second completes, the MJD
        should revert to the start of the day as, simultaneously,
        the $\Delta$UT changes by one second to its post-leap new value.
  \item The $\Delta$UT (UT1$-$UTC) is tabulated in IERS circulars and
        elsewhere.  It increases by exactly one second at the end of
        each UTC leap second, introduced in order to keep $\Delta$UT
        within $\pm$\tsec{0}{9}.
  \item {\bf Important : take care with the longitude sign convention.}  The
        longitude required by the present function is {\bf east-positive},
        in accordance with geographical convention (and right-handed).
        In particular, note that the longitudes returned by the
        {\tt slaObs} function are west-positive (as in the {\it Astronomical
        Almanac}\/ before 1984) and must be reversed in sign before use
        in the present function.
  \item The polar coordinates {\tt XP},{\tt YP} can be obtained from IERS
        circulars and equivalent publications.  The maximum amplitude is
        about \arcsec{0}{3}.  If {\tt XP},{\tt YP} values
        are unavailable, use {\tt xp=yp=0.0}.  See page B60 of the 1988
        {\it Astronomical Almanac}\/ for a definition of the two angles.
  \item The height above sea level of the observing station, {\tt hm},
        can be obtained from the {\it Astronomical Almanac}\/ (Section J
        in the 1988 edition), or via the function {\tt slaObs}.  If {\tt p},
        the pressure in millibars, is available, an adequate
        estimate of {\tt hm} can be obtained from the following expression:
        \begin{quote}
         {\tt hm=-29.3*tsl*log(p/1013.25);}
        \end{quote}
        where {\tt tsl} is the approximate sea-level air temperature in K
        (see {\it Astrophysical Quantities}, C.W.Allen, 3rd~edition,
        \S 52).  Similarly, if the pressure {\tt p} is not known,
        it can be estimated from the height of the observing
        station, {\tt hm} as follows:
        \begin{quote}
         {\tt p=1013.25*exp(-hm/(29.3*tsl));}
        \end{quote}
        Note, however, that the refraction is nearly proportional to the
        pressure and that an accurate {\tt p} value is important for
        precise work.
  \item The azimuths {\it etc.}\ used by the present function are with
        respect to the celestial pole.
        The correction needed to refer instead to terrestrial north
        can be computed using {\tt slaPolmo}.
 \end{enumerate}
}
%-----------------------------------------------------------------------
\routine{slaAoppa}{apparent-to-observed parameters}
{
 \action{Pre-compute the set of apparent to observed place parameters
         required by the ``quick'' functions
         {\tt slaAopqk} and {\tt slaOapqk}.}
 \call{slaAoppa( \vtop{
          \hbox{utc, dut, elongm, phim, hm, xp, yp,}
          \hbox{tdk, pmb, rh, wl, tlr, aoprms );}}}
}
\args{GIVEN}
{
 \spec{utc}{double}{UTC date/time (Modified Julian Date, JD$-$2400000.5)} \\
 \spec{dut}{double}{$\Delta$UT:  UT1$-$UTC (UTC seconds)} \\
 \spec{elongm}{double}{observer's mean longitude (radians, east +ve)} \\
 \spec{phim}{double}{observer's mean geodetic latitude (radians)} \\
 \spec{hm}{double}{observer's height above sea level (metres)} \\
 \spec{xp,yp}{double}{polar motion \xy\ coordinates (radians)} \\
 \spec{tdk}{double}{local ambient temperature (K; std=273.15)} \\
 \spec{pmb}{double}{local atmospheric pressure (mb; std=1013.25)} \\
 \spec{rh}{double}{local relative humidity (in the range 0.0\,--\,1.0)} \\
 \spec{wl}{double}{effective wavelength ($\mu{\rm m}$, {\it e.g.}\ 0.55)} \\
 \spec{tlr}{double}{tropospheric lapse rate (K per metre,
                                              {\it e.g.}\ 0.0065)}
}
\args{RETURNED}
{
 \spec{aoprms}{double[14]}{star-independent apparent-to-observed parameters:} \\
 \specel{0}     {geodetic latitude (radians)} \\
 \specel{1,2}   {sine and cosine of geodetic latitude} \\
 \specel{3}     {magnitude of diurnal aberration vector} \\
 \specel{4}     {height above sea level (metres)} \\
 \specel{5}     {ambient temperature (K; std=273.15)} \\
 \specel{6}     {pressure (mb = hPa)} \\
 \specel{7}     {relative humidity (in the range 0.0\,--\,1.0)} \\
 \specel{8}     {wavelength ($\mu{\rm m}$)} \\
 \specel{9}     {tropospheric lapse rate (K per metre)} \\
 \specel{10,11} {refraction constants A and B (radians)} \\
 \specel{12}    {longitude + eqn of equinoxes +
                   ``sidereal $\Delta$UT'' (radians)} \\
 \specel{13}    {local apparent sidereal time (radians)}
}
\notes
{
 \begin{enumerate}
  \item It is advisable to take great care with units, as even
        unlikely values of the input parameters are accepted and
        processed in accordance with the models used.
  \item The {\tt utc} argument is UTC expressed as an MJD.  This is,
        strictly speaking, wrong, because of leap seconds.  However,
        as long as the $\Delta$UT and the UTC are consistent there
        are no difficulties, except during a leap second.  In this
        case, the start of the 61st second of the final minute should
        begin a new MJD day and the old pre-leap $\Delta$UT should
        continue to be used.  As the 61st second completes, the MJD
        should revert to the start of the day as, simultaneously,
        the $\Delta$UT changes by one second to its post-leap new value.
  \item The $\Delta$UT (UT1$-$UTC) is tabulated in IERS circulars and
        elsewhere.  It increases by exactly one second at the end of
        each UTC leap second, introduced in order to keep $\Delta$UT
        within $\pm$\tsec{0}{9}.  The ``sidereal $\Delta$UT'' which forms
        part of {\tt aoprms[12]} is the
        same quantity, but converted from solar
        to sidereal seconds and expressed in radians.
  \item {\bf Important : take care with the longitude sign convention.}  The
        longitude required by the present function is {\bf east-positive},
        in accordance with geographical convention (and right-handed).
        In particular, note that the longitudes returned by the
        {\tt slaObs} function are west-positive (as in the {\it Astronomical
        Almanac}\/ before 1984) and must be reversed in sign before use in
        the present function.
  \item The polar coordinates {\tt xp},{\tt yp} can be obtained from IERS
        circulars and equivalent publications.  The
        maximum amplitude is about \arcsec{0}{3}.
        If {\tt xp},{\tt yp} values
        are unavailable, use {\tt xp}\,=\,{\tt yp}\,=\,0.0.
        See page B60 of the 1988
        {\it Astronomical Almanac}\/ for a definition of the two angles.
  \item The height above sea level of the observing station, {\tt hm},
        can be obtained from the {\it Astronomical Almanac}\/ (Section J
        in the 1988 edition), or via the function {\tt slaObs}.  If {\tt p},
        the pressure in millibars (the same as hPa),
        is available, an adequate
        estimate of {\tt hm} can be obtained from the following expression:
        \begin{quote}
         {\tt hm = -29.3*tsl*log(p/1013.25);}
        \end{quote}
        where {\tt tsl} is the approximate sea-level air temperature in K
        (see {\it Astrophysical Quantities}, C.W.Allen, 3rd~edition,
        \S 52).  Similarly, if the pressure {\tt p} is not known,
        it can be estimated from the height of the observing
        station, {\tt hm} as follows:
        \begin{quote}
         {\tt p = 1013.25*exp(-hm/(29.3*tsl));}
        \end{quote}
        Note, however, that the refraction is nearly proportional to the
        pressure and that an accurate {\tt p} value is important for
        precise work.
  \item Repeated, computationally-expensive, calls to {\tt slaAoppa} for
        times that are very close together can be avoided by calling
        {\tt slaAoppa} just once and then using
        {\tt slaAoppat} for the subsequent
        times.  Fresh calls to {\tt slaAoppa}
        will be needed only when changes
        in the precession have grown to unacceptable levels or when
        anything affecting the refraction has changed.
 \end{enumerate}
}
%-----------------------------------------------------------------------
\routine{slaAoppat}{update apparent-to-observed parameters}
{
 \action{Recompute the sidereal time in the apparent to observed place
         star-independent parameter block.}
 \call{slaAoppat( utc, aoprms );}
}
\args{GIVEN}
{
 \spec{utc}{double}{UTC date/time (Modified Julian Date, JD$-$2400000.5)} \\
 \spec{aoprms}{double[14]}{star-independent apparent-to-observed parameters:} \\
 \specel{0-11}{not required} \\
 \specel{12}{longitude + eqn of equinoxes +
               ``sidereal $\Delta$UT'' (radians)} \\
 \specel{13}{not required}
}
\args{RETURNED}
{
 \spec{aoprms}{double[14]}{star-independent
                           apparent-to-observed parameters:} \\
 \specel{0-12}{not changed} \\
 \specel{13}{local apparent sidereal time (radians)}
}
\anote{For more information, see {\tt slaAoppa}.}
%-----------------------------------------------------------------------
\routine{slaAopqk}{quick apparent to observed}
{
 \action{Quick apparent to observed place (but see Note~8, below).}
 \call{slaAopqk( rap, dap, aoprms, \&aob, \&zob, \&hob, \&dob, \&rob );}
}
\args{GIVEN}
{
 \spec{rap,dap}{double}{geocentric apparent \radec\ (radians)} \\
 \spec{aoprms}{double[14]}{star-independent apparent-to-observed parameters:} \\
 \specel{0}     {geodetic latitude (radians)} \\
 \specel{1,2}   {sine and cosine of geodetic latitude} \\
 \specel{3}     {magnitude of diurnal aberration vector} \\
 \specel{4}     {height above sea level (metres)} \\
 \specel{5}     {ambient temperature (K; std=273.15)} \\
 \specel{6}     {pressure (mb = hPa)} \\
 \specel{7}     {relative humidity (in the range 0.0\,--\,1.0)} \\
 \specel{8}     {wavelength ($\mu{\rm m}$)} \\
 \specel{9}     {tropospheric lapse rate (K per metre)} \\
 \specel{10,11} {refraction constants A and B (radians)} \\
 \specel{12}    {longitude + eqn of equinoxes +
                   ``sidereal $\Delta$UT'' (radians)} \\
 \specel{13}    {local apparent sidereal time (radians)}
}
\args{RETURNED}
{
 \spec{aob}{double*}{observed azimuth (radians: N=0, E=$90^{\circ}$)} \\
 \spec{zob}{double*}{observed zenith distance (radians)} \\
 \spec{hob}{double*}{observed hour angle (radians)} \\
 \spec{dob}{double*}{observed declination (radians)} \\
 \spec{rob}{double*}{observed right ascension (radians)}
}
\notes
{
 \begin{enumerate}
  \item This function returns zenith distance rather than elevation
        in order to reflect the fact that no allowance is made for
        depression of the horizon.
  \item The accuracy of the result is limited by the corrections for
        refraction.  Providing the meteorological parameters are
        known accurately and there are no gross local effects, the
        predicted azimuth and elevation should be within about
        \arcsec{0}{1} for $\zeta<70^{\circ}$.  Even
        at a topocentric zenith distance of
        $90^{\circ}$, the accuracy in elevation
        should normally be better than
        1~arcminute;  useful results are available for a further
        $3^{\circ}$, beyond which the {\tt slaRefro} function returns a
        fixed value of the refraction.  The complementary
        functions {\tt slaAop} (or {\tt slaAopqk}) and {\tt slaOap} (or {\tt slaOapqk})
        are self-consistent to better than 1~microarcsecond all over
        the celestial sphere.
  \item It is advisable to take great care with units, as even
        unlikely values of the input parameters are accepted and
        processed in accordance with the models used.
  \item {\it Apparent}\/ \radec\ means the geocentric apparent right ascension
        and declination, which is obtained from a catalogue mean place
        by allowing for space motion, parallax, the Sun's gravitational
        lens effect, annual aberration and precession-nutation.  For
        star positions in the FK5 system ({\it i.e.}~J2000), these effects can
        be applied by means of the {\tt slaMap} {\it etc.}\ functions.  Starting from
        other mean place systems, additional transformations will be
        needed;  for example, FK4 ({\it i.e.}~B1950) mean places would first
        have to be converted to FK5, which can be done with the
        {\tt slaFk425} {\it etc.}\ functions.
  \item {\it Observed}\/ \azel\ means the position that would be seen by a
        perfect theodolite located at the observer.  This is obtained
        from the geocentric apparent \radec\ by allowing for Earth
        orientation and diurnal aberration, rotating from equator
        to horizon coordinates, and then adjusting for refraction.
        The \hadec\ is obtained by rotating back into equatorial
        coordinates, using the geodetic latitude corrected for polar
        motion, and is the position that would be seen by a perfect
        equatorial located at the observer and with its polar axis
        aligned to the Earth's axis of rotation ({\it n.b.}\ not to the
        refracted pole).  Finally, the $\alpha$ is obtained by subtracting
        the {\it h}\/ from the local apparent ST.
  \item To predict the required setting of a real telescope, the
        observed place produced by this function would have to be
        adjusted for the tilt of the azimuth or polar axis of the
        mounting (with appropriate corrections for mount flexures),
        for non-perpendicularity between the mounting axes, for the
        position of the rotator axis and the pointing axis relative
        to it, for tube flexure, for gear and encoder errors, and
        finally for encoder zero points.  Some telescopes would, of
        course, exhibit other properties which would need to be
        accounted for at the appropriate point in the sequence.
  \item The star-independent apparent-to-observed-place parameters
        in {\tt aoprms} may be computed by means of the {\tt slaAoppa} function.
        If nothing has changed significantly except the time, the
        {\tt slaAoppat} function may be used to perform the requisite
        partial recomputation of {\tt aoprms}.
  \item At zenith distances beyond about $76^\circ$, the need for
        special care with the corrections for refraction causes a
        marked increase in execution time.  Moreover, the effect
        gets worse with increasing zenith distance.  Adroit
        programming in the calling application may allow the
        problem to be reduced.  Prepare an alternative {\tt aoprms} array,
        computed for zero air-pressure;  this will disable the
        refraction corrections and cause rapid execution.  Using
        this {\tt aoprms} array, a preliminary call to the present function
        will, depending on the application, produce a rough position
        which may be enough to establish whether the full, slow
        calculation (using the real {\tt aoprms} array) is worthwhile.
        For example, there would be no need for the full calculation
        if the preliminary call had already established that the
        source was well below the elevation limits for a particular
        telescope.
  \item The azimuths {\it etc.}\ used by the present function are with
        respect to the celestial pole.
        The correction needed to refer instead to terrestrial north
        can be computed using {\tt slaPolmo}.
 \end{enumerate}
}
%-----------------------------------------------------------------------
\routine{slaAtmdsp}{atmospheric dispersion}
{
 \action{Apply atmospheric-dispersion adjustments to refraction coefficients.}
 \call{slaAtmdsp( tdk, pmb, rh, wl1, a1, b1, wl2, \&a2, \&b2 );}
}
\args{GIVEN}
{
 \spec{tdk}{double}{ambient temperature at the observer (K)} \\
 \spec{pmb}{double}{pressure at the observer (mb)} \\
 \spec{rh}{double}{relative humidity at the observer (range 0\,--\,1)} \\
 \spec{wl1}{double}{base wavelength ($\mu{\rm m}$)} \\
 \spec{a1}{double}{refraction coefficient A for wavelength
                   {\tt wl1} (radians)} \\
 \spec{b1}{double}{refraction coefficient B for wavelength
                   {\tt wl1} (radians)} \\
 \spec{wl2}{double}{wavelength for which adjusted A,B required ($\mu{\rm m}$)}
}
\args{RETURNED}
{
 \spec{a2}{double*}{refraction coefficient A for wavelength
                    {\tt wl2} (radians)} \\
 \spec{b2}{double*}{refraction coefficient B for wavelength
                    {\tt wl2} (radians)}
}
\notes
{
 \begin{enumerate}
  \item To use this function, first call {\tt slaRefco} specifying {\tt wl1} as the
        wavelength.  This yields refraction coefficients
        {\tt b1}, {\tt b1}, correct
        for that wavelength.  Subsequently, calls to {\tt slaAtmdsp} specifying
        different wavelengths will produce new, slightly adjusted
        refraction coefficients {\tt a2}, {\tt b2},
        which apply to the specified wavelength.
  \item Most of the atmospheric dispersion happens between $0.7\,\mu{\rm m}$
        and the UV atmospheric cutoff, and the effect increases strongly
        towards the UV end.  For this reason a blue reference wavelength
        is recommended, for example $0.4\,\mu{\rm m}$.
  \item The accuracy, for this set of conditions: \\[1pc]
   \hspace*{5ex} \begin{tabular}{rcl}
        height above sea level & ~ & 2000\,m \\
                      latitude & ~ & $29^\circ$ \\
                      pressure & ~ & 793\,mb \\
                   temperature & ~ & $290^\circ$\,K \\
                      humidity & ~ & 0.5 (50\%) \\
                    lapse rate & ~ & $0.0065^\circ m^{-1}$ \\
          reference wavelength & ~ & $0.4\,\mu{\rm m}$ \\
                star elevation & ~ & $15^\circ$ \\
                  \end{tabular}\\[1pc]
        is about 2.5\,mas RMS between 0.3 and $1.0\,\mu{\rm m}$, and stays
        within 4\,mas for the whole range longward of $0.3\,\mu{\rm m}$
        (compared with a total dispersion from 0.3 to $20\,\mu{\rm m}$
        of about \arcseci{11}).  These errors are typical for ordinary
        conditions;  in extreme conditions values a few times this size
        may occur.
  \item If either wavelength exceeds $100\,\mu{\rm m}$, the radio case
        is assumed and the returned refraction coefficients are the
        same as the given ones. Note that radio refraction coefficients
        cannot be turned into optical values using this function, nor
        vice versa.
  \item The algorithm consists of calculation of the refractivity of the
        air at the observer for the two wavelengths, using the methods
        of the {\tt slaRefro} function, and then scaling of the two refraction
        coefficients according to classical refraction theory.  This
        amounts to scaling the A coefficient in proportion to $(\mu-1)$ and
        the B coefficient almost in the same ratio (see R.M.Green,
        {\it Spherical Astronomy,}\/ Cambridge University Press, 1985).
 \end{enumerate}
}
%-----------------------------------------------------------------------
\routine{slaAv2m}{rotation matrix from axial vector}
{
 \action{Form the rotation matrix corresponding to a given axial vector
         (single precision).}
 \call{slaAv2m( axvec, rmat );}
}
\args{GIVEN}
{
 \spec{axvec}{float[3]}{axial vector (radians)}
}
\args{RETURNED}
{
 \spec{rmat}{float[3][3]}{rotation matrix}
}
\notes
{
 \begin{enumerate}
  \item A rotation matrix describes a rotation about some
        arbitrary axis, called the Euler axis.  The
        {\it axial vector} supplied to this function
        has the same direction as the Euler axis, and its
        magnitude is the amount of rotation in radians.
  \item If {\tt axvec} is null, the unit matrix is returned.
  \item The reference frame rotates clockwise as seen looking along
        the axial vector from the origin.
 \end{enumerate}
}
%-----------------------------------------------------------------------
\routine{slaBear}{direction between points on a sphere}
{
 \action{Returns the bearing (position angle) of one point on a
         sphere seen from another (single precision).}
 \call{r~=~slaBear( a1, b1, a2, b2 );}
}
\args{GIVEN}
{
 \spec{a1,b1}{float}{spherical coordinates of one point} \\
 \spec{a2,b2}{float}{spherical coordinates of the other point}
}
\args{RETURNED}
{
 \spec{}{float}{bearing from first point to second}
}
\notes
{
 \begin{enumerate}
 \item The spherical coordinates are \radec,
       \ecl\ {\it etc.}, in radians.
 \item The result is the bearing (position angle), in radians,
       of point [{\tt a2,b2}] as seen
       from point [{\tt a1,b1}].  It is in the range $\pm \pi$.  The sense
       is such that if [{\tt a2,b2}]
       is a small distance due east of [{\tt a1,b1}] the result
       is about $+\pi/2$. Zero is returned
       if the two points are coincident.
 \item If either b-coordinate is outside the range $\pm\pi/2$, the
       result may correspond to ``the long way round''.
 \item The function {\tt slaPav} performs an equivalent function except
       that the points are specified in the form of Cartesian unit
       vectors.
 \end{enumerate}
}
%-----------------------------------------------------------------------
\routine{slaCaf2r}{deg,arcmin,arcsec to radians}
{
 \action{Convert degrees, arcminutes, arcseconds to radians
         (single precision).}
 \call{slaCaf2r( ideg, iamin, asec, \&rad, \&j );}
}
\args{GIVEN}
{
 \spec{ideg}{int}{degrees} \\
 \spec{iamin}{int}{arcminutes} \\
 \spec{asec}{float}{arcseconds}
}
\args{RETURNED}
{
 \spec{rad}{float*}{angle in radians} \\
 \spec{j}{int*}{status:} \\
 \spec{}{}{\hspace{1.5em} 1 = {\tt ideg} outside range 0$-$359} \\
 \spec{}{}{\hspace{1.5em} 2 = {\tt iamin} outside range 0$-$59} \\
 \spec{}{}{\hspace{1.5em} 3 = {\tt asec} outside range 0$-$59.999$\cdots$}
}
\notes
{
 \begin{enumerate}
  \item The result is computed even if any of the range checks fail.
  \item The sign must be dealt with outside this function.
 \end{enumerate}
}
%-----------------------------------------------------------------------
\routine{slaCaldj}{calendar date to MJD}
{
 \action{Gregorian Calendar to Modified Julian Date, with century default.}
 \call{slaCaldj( iy, im, id, \&djm, \&j );}
}
\args{GIVEN}
{
 \spec{iy,im,id}{int}{year, month, day in Gregorian calendar}
}
\args{RETURNED}
{
 \spec{djm}{double*}{modified Julian Date (JD$-$2400000.5) for $0^{\rm h}$} \\
 \spec{j}{int*}{status:} \\
 \spec{}{}{\hspace{1.5em} 0 = OK} \\
 \spec{}{}{\hspace{1.5em} 1 = bad year   (MJD not computed)} \\
 \spec{}{}{\hspace{1.5em} 2 = bad month  (MJD not computed)} \\
 \spec{}{}{\hspace{1.5em} 3 = bad day    (MJD computed)} \\
}
\notes
{
 \begin{enumerate}
  \item This function supports the {\it century default}\/ feature.
        Acceptable years are:
        \begin{itemize}
         \item 00-49, interpreted as 2000\,--\,2049,
         \item 50-99, interpreted as 1950\,--\,1999, and
         \item 100 upwards, interpreted literally.
        \end{itemize}
        For 1-100\,AD use the function {\tt slaCldj} instead.
  \item For year $n$\,BC use {\tt iy}\,$=-(n-1)$.
  \item When an invalid year or month is supplied
        (status {\tt j}\,=\,1~or~2)
        the MJD is {\bf not} computed.  When an invalid day is supplied
        (status {\tt j}\,=\,3) the MJD {\bf is} computed.
 \end{enumerate}
}
%-----------------------------------------------------------------------
\routine{slaCalyd}{calendar date to year,day}
{
 \action{Gregorian calendar date to year and day in year, in a Julian
         calendar aligned to the 20th/21st century Gregorian calendar,
         with century default.}
 \call{slaCalyd( iy, im, id, \&ny, \&nd, \&j );}
}
\args{GIVEN}
{
 \spec{iy,im,id}{int}{year, month, day in Gregorian calendar:
                    year may optionally omit the century}
}
\args{RETURNED}
{
 \spec{ny}{int*}{year (re-aligned Julian calendar)} \\
 \spec{nd}{int*}{day in year (1 = January 1st)} \\
 \spec{j}{int*}{status:} \\
 \spec{}{}{\hspace{1.5em}  0 = OK} \\
 \spec{}{}{\hspace{1.5em}  1 = bad year (before $-4711$)} \\
 \spec{}{}{\hspace{1.5em}  2 = bad month} \\
 \spec{}{}{\hspace{1.5em}  3 = bad day}
}
\notes
{
 \begin{enumerate}
  \item This function supports the {\it century default}\/ feature.
        Acceptable years are:
        \begin{itemize}
         \item 00-49, interpreted as 2000\,--\,2049,
         \item 50-99, interpreted as 1950\,--\,1999, and
         \item other years after $-4712$, interpreted literally.
        \end{itemize}
        Use {\tt slaClyd} for years before 100AD.
  \item The purpose of {\tt slaCaldj} is to support
        {\tt slaEarth}, {\tt slaMoon} and {\tt slaEcor}.
  \item Between 1900~March~1 and 2100~February~28 it returns answers
        which are consistent with the ordinary Gregorian calendar.
        Outside this range there will be a discrepancy which increases
        by one day for every non-leap century year.
  \item When an invalid year or month is supplied
        (status {\tt j}\,=\,1 or {\tt j}\,=\,2)
        the results are {\bf not} computed.  When a day is
        supplied which is outside the conventional range
        (status {\tt j}\,=\,3)
        the results {\bf are} computed.
 \end{enumerate}
}
%-----------------------------------------------------------------------
\routine{slaCc2s}{Cartesian to spherical}
{
 \action{Cartesian coordinates to spherical coordinates (single precision).}
 \call{slaCc2s( v, \&a, \&b );}
}
\args{GIVEN}
{
 \spec{v}{float[3]}{\xyz\ vector}
}
\args{RETURNED}
{
 \spec{a,b}{float*}{spherical coordinates in radians}
}
\notes
{
 \begin{enumerate}
  \item The spherical coordinates are longitude (positive anticlockwise
        looking from the positive latitude pole) and latitude.  The
        Cartesian coordinates are right handed, with the {\it x}-axis
        at zero longitude and latitude, and the {\it z}-axis at the
        positive latitude pole.
  \item If {\tt v} is null, zero {\tt a} and {\tt b} are returned.
  \item At either pole, zero {\tt a} is returned.
 \end{enumerate}
}
%-----------------------------------------------------------------------
\routine{slaCc62s}{Cartesian 6-vector to spherical}
{
 \action{Transformation of position \& velocity in Cartesian coordinates
         to spherical coordinates (single precision).}
 \call{slaCc62s( v, \&a, \&b, \&r, \&ad, \&bd, \&rd );}
}
\args{GIVEN}
{
 \spec{v}{float[6]}{\xyzxyzd}
}
\args{RETURNED}
{
 \spec{a}{float*}{longitude (radians) -- for example $\alpha$} \\
 \spec{b}{float*}{latitude (radians) -- for example $\delta$} \\
 \spec{r}{float*}{radial coordinate} \\
 \spec{ad}{float*}{longitude derivative (radians per unit time)} \\
 \spec{bd}{float*}{latitude derivative (radians per unit time)} \\
 \spec{rd}{float*}{radial derivative}
}
\anote{The spherical coordinates are longitude (positive anticlockwise
       looking from the positive latitude pole) and latitude.  The
       Cartesian coordinates are right handed, with the {\it x}-axis
       at zero longitude and latitude, and the {\it z}-axis at the
       positive latitude pole.}
%-----------------------------------------------------------------------
\routine{slaCd2tf}{days to hour,min,sec}
{
 \action{Convert an interval in days to hours, minutes, seconds
         (single precision).}
 \call{slaCd2tf( ndp, days, \&sign, ihmsf );}
}
\args{GIVEN}
{
 \spec{ndp}{int}{number of decimal places of seconds} \\
 \spec{days}{float}{interval in days}
}
\args{RETURNED}
{
 \spec{sign}{char*}{\fstring{+} or \fstring{-}} \\
 \spec{ihmsf}{int[4]}{hours, minutes, seconds, fraction}
}
\notes
{
 \begin{enumerate}
  \item {\tt ndp} less than zero is interpreted as zero.
  \item The largest useful value for {\tt ndp} is determined by the size of
        {\tt days}, the format of {\tt float} numbers on the target
        machine, and the risk of overflowing {\tt ihmsf[3]}.  On some
        architectures, for {\tt days} up to 1.0,
        the available floating-point
        precision corresponds roughly to {\tt ndp}\,=\,3.  This is well
        below the ultimate limit of {\tt ndp}\,=\,9 set by the capacity of a
        typical 32-bit {\tt ihmsf[3]}.
  \item The absolute value of {\tt days} may exceed 1.0.  In cases where it
        does not, it is up to the caller to test for and handle the
        case where {\tt days} is very nearly 1.0 and rounds up to 24~hours,
        by testing for {\tt ihmsf[0]}\,=\,24
        and setting {\tt ihmsf[0-3]} to zero.
\end{enumerate}
}
%-----------------------------------------------------------------------
\routine{slaCldj}{calendar date to MJD}
{
 \action{Gregorian Calendar to Modified Julian Date.}
 \call{slaCldj( iy, im, id, \&djm, \&j );}
}
\args{GIVEN}
{
 \spec{iy,im,id}{int}{year, month, day in Gregorian calendar}
}
\args{RETURNED}
{
 \spec{djm}{double*}{modified Julian Date (JD$-$2400000.5) for $0^{\rm h}$} \\
 \spec{j}{int*}{status:} \\
 \spec{}{}{\hspace{1.5em} 0 = OK} \\
 \spec{}{}{\hspace{1.5em} 1 = bad year} \\
 \spec{}{}{\hspace{1.5em} 2 = bad month} \\
 \spec{}{}{\hspace{1.5em} 3 = bad day}
}
\notes
{
 \begin{enumerate}
  \item When an invalid year or month is supplied
        (status {\tt j}\,=\,1~or~2)
        the MJD is {\bf not} computed.  When an invalid day is supplied
        (status {\tt j}\,=\,3) the MJD {\bf is} computed.
  \item The year must be $-$4699 ({\it i.e.}~4700\,BC) or later.
        For year $n$\,BC use {\tt iy}\,$=-(n-1)$.
  \item An alternative to the present function is {\tt slaCaldj}, which
        accepts a year with the century missing.
 \end{enumerate}
}
\aref{The algorithm is adapted from Hatcher 1984,
      {\it Q.\,Jl.\,R.\,astr.\,Soc.}\ {\bf 25}, 53-55.}
%-----------------------------------------------------------------------
\routine{slaClyd}{calendar date to year,day}
{
 \action{Gregorian calendar date to year and day in year, in a Julian
         calendar aligned to the 20th/21st century Gregorian calendar.}
 \call{slaClyd( iy, im, id, \&ny, \&nd, \&j );}
}
\args{GIVEN}
{
 \spec{iy,im,id}{int}{year, month, day in Gregorian calendar}
}
\args{RETURNED}
{
 \spec{ny}{int*}{year (re-aligned Julian calendar)} \\
 \spec{nd}{int*}{day in year (1 = January 1st)} \\
 \spec{j}{int*}{status:} \\
 \spec{}{}{\hspace{1.5em}  0 = OK} \\
 \spec{}{}{\hspace{1.5em}  1 = bad year (before $-4711$)} \\
 \spec{}{}{\hspace{1.5em}  2 = bad month} \\
 \spec{}{}{\hspace{1.5em}  3 = bad day}
}
\notes
{
 \begin{enumerate}
  \item The purpose of {\tt slaClyd} is to support {\tt slaEarth},
        {\tt slaMoon} and {\tt slaEcor}.
  \item Between 1900~March~1 and 2100~February~28 it returns answers
        which are consistent with the ordinary Gregorian calendar.
        Outside this range there will be a discrepancy which increases
        by one day for every non-leap century year.
  \item When an invalid year or month is supplied
        (status {\tt j}\,=\,1 or {\tt j}\,=\,2)
        the results are {\bf not} computed.  When a day is
        supplied which is outside the conventional range
        (status {\tt j}\,=\,3)
        the results {\bf are} computed.
 \end{enumerate}
}
%-----------------------------------------------------------------------
\routine{slaCombn}{next combination}
{
 \action{Generate the next combination, a subset of a specified size chosen
         from a specified number of items.}
 \call{slaCombn( nsel, ncand, list, \&j );}
}
\args{GIVEN}
{
 \spec{nsel}{int}{number of items (subset size)} \\
 \spec{ncand}{int}{number of candidates (set size)}
}
\args{GIVEN and RETURNED}
{
 \spec{list}{I(NSEL)}{latest combination, LIST(1)=0 to initialize}
}
\args{RETURNED}
{
 \spec{j}{int*}{status:} \\
 \spec{}{}{\hspace{1.5em} $-$1 = illegal {\tt nsel} or {\tt ncand}} \\
 \spec{}{}{\hspace{2.3em}    0 = OK} \\
 \spec{}{}{\hspace{1.5em} $+$1 = no more combinations available}
}
\notes
{
 \begin{enumerate}
  \item {\tt nsel} and {\tt ncand} must both be at least 1,
        and {\tt nsel} must be less
        than or equal to {\tt ncand}.
  \item This function returns, in the LIST array,
        a subset of {\tt nsel} integers
        chosen from the range 1 to {\tt ncand} inclusive, in ascending order.
        Before calling the function for the first time, the caller must set
        the first element of the {\tt list} array to zero
        (any value less than 1
        will do) to cause initialization.
  \item The first combination to be generated is:
        \begin{verse}
           {\tt list[0]=1, list[1]=2, \ldots, list[nsel-1]=nsel}
        \end{verse}
        This is also the combination returned for the ``finished''
        ({\tt j}\,=\,1) case.
        The final permutation to be generated is:
        \begin{verse}
           {\tt list[0]=ncand, list[1]=ncand$-$1, \ldots,
                               list[nsel-1]=ncand$-$nsel+1}
        \end{verse}
  \item If the ``finished'' ({\tt j}\,=\,1)
        status is ignored, the function
        continues to deliver combinations, the pattern repeating
        every {\tt ncand!/(nsel!(ncand$-$nsel)!)} calls.
  \item The algorithm is by R.\,F.\,Warren-Smith (private communication).
 \end{enumerate}
}
%-----------------------------------------------------------------------
\routine{slaCr2af}{radians to deg,arcmin,arcsec}
{
 \action{Convert an angle in radians to degrees, arcminutes,
         arcseconds (single precision).}
 \call{slaCr2af( ndp, angle, \&sign, idmsf );}
}
\args{GIVEN}
{
 \spec{ndp}{int}{number of decimal places of arcseconds} \\
 \spec{angle}{float}{angle in radians}
}
\args{RETURNED}
{
 \spec{sign}{char*}{\fstring{+} or \fstring{-}} \\
 \spec{idmsf}{int[4]}{degrees, arcminutes, arcseconds, fraction}
}
\notes
{
 \begin{enumerate}
  \item {\tt ndp} less than zero is interpreted as zero.
  \item The largest useful value for {\tt ndp}
        is determined by the size of
        {\tt angle}, the format of {\tt float} numbers on the target
        machine, and the risk of overflowing {\tt idmsf[3]}.  On some
        architectures, for {\tt angle} up to $2\pi$,
        the available floating-point
        precision corresponds roughly to {\tt ndp}\,=\,3.  This is well
        below the ultimate limit of {\tt ndp}\,=\,9
        set by the capacity of a
        typical 32-bit {\tt idmsf[3]}.
  \item The absolute value of {\tt angle}
        may exceed $2\pi$.  In cases where it
        does not, it is up to the caller to test for and handle the
        case where {\tt angle} is very nearly $2\pi$
        and rounds up to $360^{\circ}$,
        by testing for {\tt idmsf[0]}\,=\,360
        and setting {\tt idmsf[0-3]} to zero.
 \end{enumerate}
}
%-----------------------------------------------------------------------
\routine{slaCr2tf}{radians to hour,min,sec}
{
 \action{Convert an angle in radians to hours, minutes, seconds
         (single precision).}
 \call{slaCr2tf( ndp, angle, \&sign, ihmsf );}
}
\args{GIVEN}
{
 \spec{ndp}{int}{number of decimal places of seconds} \\
 \spec{angle}{float}{angle in radians}
}
\args{RETURNED}
{
 \spec{sign}{char*}{\fstring{+} or \fstring{-}} \\
 \spec{ihmsf}{int[4]}{hours, minutes, seconds, fraction}
}
\notes
{
 \begin{enumerate}
  \item {\tt ndp} less than zero is interpreted as zero.
  \item The largest useful value for {\tt ndp}
        is determined by the size of
        {\tt angle}, the format of {\tt float} numbers on the target
        machine, and the risk of overflowing {\tt ihmsf[3]}.  On some
        architectures, for {\tt angle} up to $2\pi$, the available
        floating-point precision corresponds roughly
        to {\tt ndp}\,=\,3.  This is well
        below the ultimate limit of {\tt ndp}\,=\,9
        set by the capacity of a
        typical 32-bit {\tt ihmsf[3]}.
  \item The absolute value of {\tt angle}
        may exceed $2\pi$.  In cases where it
        does not, it is up to the caller to test for and handle the
        case where {\tt angle} is very nearly $2\pi$
        and rounds up to 24~hours,
        by testing for {\tt ihmsf[0]}\,=\,24
        and setting {\tt ihmsf[0-3]} to zero.
 \end{enumerate}
}
%-----------------------------------------------------------------------
\routine{slaCs2c}{spherical to Cartesian}
{
 \action{Spherical coordinates to Cartesian coordinates (single precision).}
 \call{slaCs2c( a, b, v );}
}
\args{GIVEN}
{
 \spec{a,b}{float}{spherical coordinates in radians: \radec\ {\it etc.}}
}
\args{RETURNED}
{
 \spec{v}{float[3]}{\xyz\ unit vector}
}
\anote{The spherical coordinates are longitude (positive anticlockwise
       looking from the positive latitude pole) and latitude.  The
       Cartesian coordinates are right handed, with the {\it x}-axis
       at zero longitude and latitude, and the {\it z}-axis at the
       positive latitude pole.}
%-----------------------------------------------------------------------
\routine{slaCs2c6}{spherical position and velocity to Cartesian}
{
 \action{Transformation of position \& velocity in spherical coordinates
         to Cartesian coordinates (single precision).}
 \call{slaCs2c6( a, b, r, ad, bd, rd, v );}
}
\args{GIVEN}
{
 \spec{a}{float}{longitude (radians) -- for example $\alpha$} \\
 \spec{b}{float}{latitude (radians) -- for example $\delta$} \\
 \spec{float}{float}{radial coordinate} \\
 \spec{ad}{float}{longitude derivative (radians per unit time)} \\
 \spec{bd}{float}{latitude derivative (radians per unit time)} \\
 \spec{rd}{float}{radial derivative}
}
\args{RETURNED}
{
 \spec{v}{float[6]}{\xyzxyzd}
}
\anote{The spherical coordinates are longitude (positive anticlockwise
       looking from the positive latitude pole) and latitude.  The
       Cartesian coordinates are right handed, with the {\it x}-axis
       at zero longitude and latitude, and the {\it z}-axis at the
       positive latitude pole.}
%-----------------------------------------------------------------------
\routine{slaCtf2d}{hour,min,sec to days}
{
 \action{Convert hours, minutes, seconds to days (single precision).}
 \call{slaCtf2d( ihour, imin, sec, \&days, \&j );}
}
\args{GIVEN}
{
 \spec{ihour}{int}{hours} \\
 \spec{imin}{int}{minutes} \\
 \spec{sec}{float}{seconds}
}
\args{RETURNED}
{
 \spec{days}{float*}{interval in days} \\
 \spec{j}{int}{status:} \\
 \spec{}{}{\hspace{1.5em} 0 = OK} \\
 \spec{}{}{\hspace{1.5em} 1 = {\tt ihour} outside range 0-23} \\
 \spec{}{}{\hspace{1.5em} 2 = {\tt imin} outside range 0-59} \\
 \spec{}{}{\hspace{1.5em} 3 = {\tt sec} outside range 0-59.999$\cdots$}
}
\notes
{
 \begin{enumerate}
  \item The result is computed even if any of the range checks fail.
  \item The sign must be dealt with outside this function.
 \end{enumerate}
}
%-----------------------------------------------------------------------
\routine{slaCtf2r}{hour,min,sec to radians}
{
 \action{Convert hours, minutes, seconds to radians (single precision).}
 \call{slaCtf2r( ihour, imin, sec, \&rad, \&j );}
}
\args{GIVEN}
{
 \spec{ihour}{int}{hours} \\
 \spec{imin}{int}{minutes} \\
 \spec{sec}{float}{seconds}
}
\args{RETURNED}
{
 \spec{rad}{float*}{angle in radians} \\
 \spec{j}{int*}{status:} \\
 \spec{}{}{\hspace{1.5em} 0 = OK} \\
 \spec{}{}{\hspace{1.5em} 1 = {\tt ihour} outside range 0-23} \\
 \spec{}{}{\hspace{1.5em} 2 = {\tt imin} outside range 0-59} \\
 \spec{}{}{\hspace{1.5em} 3 = {\tt sec} outside range 0-59.999$\cdots$}
}
\notes
{
 \begin{enumerate}
  \item The result is computed even if any of the range checks fail.
  \item The sign must be dealt with outside this function.
 \end{enumerate}
}
%-----------------------------------------------------------------------
\routine{slaDaf2r}{deg,arcmin,arcsec to radians}
{
 \action{Convert degrees, arcminutes, arcseconds to radians
  (double precision).}
 \call{slaDaf2r( ideg, iamin, asec, \&rad, \&j );}
}
\args{GIVEN}
{
 \spec{ideg}{int}{degrees} \\
 \spec{iamin}{int}{arcminutes} \\
 \spec{asec}{double}{arcseconds}
}
\args{RETURNED}
{
 \spec{rad}{double*}{angle in radians} \\
 \spec{j}{int*}{status:} \\
 \spec{}{}{\hspace{1.5em} 1 = {\tt ideg} outside range 0$-$359} \\
 \spec{}{}{\hspace{1.5em} 2 = {\tt iamin} outside range 0$-$59} \\
 \spec{}{}{\hspace{1.5em} 3 = {\tt asec} outside range 0$-$59.999$\cdots$}
}
\notes
{
 \begin{enumerate}
  \item The result is computed even if any of the range checks fail.
  \item The sign must be dealt with outside this function.
 \end{enumerate}
}
%-----------------------------------------------------------------------
\routine{slaDafin}{sexagesimal character string to angle}
{
 \action{Decode a free-format sexagesimal string (degrees, arcminutes,
         arcseconds) into a double precision floating point
         number (radians).}
 \call{slaDafin( string, \&nstrt, \&dreslt, \&jf );}
}
\args{GIVEN}
{
 \spec{string}{char*}{string containing deg, arcmin, arcsec fields} \\
 \spec{nstrt}{int}{index to start of decode (beginning of {\tt string}~=~1)}
}
\args{RETURNED}
{
 \spec{nstrt}{int*}{advanced past the decoded angle} \\
 \spec{dreslt}{double*}{angle in radians} \\
 \spec{jf}{int*}{status:} \\
 \spec{}{}{\hspace{1.5em}   0 = OK} \\
 \spec{}{}{\hspace{0.7em} $+1$ = default, {\tt dreslt} unchanged (Note~2)} \\
 \spec{}{}{\hspace{0.7em} $-1$ = bad degrees (Note~3)} \\
 \spec{}{}{\hspace{0.7em} $-2$ = bad arcminutes (Note~3)} \\
 \spec{}{}{\hspace{0.7em} $-3$ = bad arcseconds (Note~3)} \\
}
\goodbreak
\setlength{\oldspacing}{\topsep}
\setlength{\topsep}{0.3ex}
\begin{description}
 \item [EXAMPLE]: \\ [1.5ex]
  \begin{tabular}{p{7em}p{15em}p{12em}}
   {\it argument} & {\it before} & {\it after} \\ \\
   {\tt string} & $'$\verb*}-57 17 44.806  12 34 56.7}$'$ & unchanged \\
   {\tt nstrt} & 1 & 16 ({\it i.e.}~pointing to 12...) \\
   {\tt reslt} & - & $-1.00000$ \\
   {\tt jf} & - & 0
  \end{tabular}
 \item A further call to {\tt slaDafin},
       without adjustment of {\tt nstrt}, will
       decode the second angle, \dms{12}{34}{56}{7}.
\end{description}
\setlength{\topsep}{\oldspacing}
\notes
{
 \begin{enumerate}
  \item The first three ``fields'' in {\tt string} are degrees, arcminutes,
   arcseconds, separated by spaces or commas.  The degrees field
   may be signed, but not the others.  The decoding is carried
   out by the {\tt slaDfltin} function and is free-format.
  \item Successive fields may be absent, defaulting to zero.  For
   zero status, the only combinations allowed are degrees alone,
   degrees and arcminutes, and all three fields present.  If all
   three fields are omitted, a status of +1 is returned and {\tt dreslt} is
   unchanged.  In all other cases {\tt dreslt} is changed.
  \item Range checking:
   \begin{itemize}
    \item The degrees field is not range checked.  However, it is
     expected to be integral unless the other two fields are absent.
    \item The arcminutes field is expected to be 0-59, and integral if
     the arcseconds field is present.  If the arcseconds field
     is absent, the arcminutes is expected to be 0-59.9999...
    \item The arcseconds field is expected to be 0-59.9999...
    \item Decoding continues even when a check has failed.  Under these
     circumstances the field takes the supplied value, defaulting to
     zero, and the result {\tt dreslt} is computed and returned.
   \end{itemize}
   \item Further fields after the three expected ones are not treated as
    an error.  The index {\tt nstrt} is left in the correct state for
    further decoding with the present function or with {\tt slaDfltin}
    {\it etc}.  See the example, above.
   \item If {\tt string} contains hours, minutes, seconds instead of
    degrees {\it etc},
    or if the required units are turns (or days) instead of radians,
    the result {\tt dreslt} should be multiplied as follows: \\ [1.5ex]
    \begin{tabular}{p{6em}p{5em}p{18em}}
    {\it for {\tt string}} & {\it to obtain} &
                             {\it multiply {\tt dreslt} by} \\ \\
    ${\circ}$~~\raisebox{-0.7ex}{$'$}~~\raisebox{-0.7ex}{$''$}
     & radians & $1.0$ \\
    ${\circ}$~~\raisebox{-0.7ex}{$'$}~~\raisebox{-0.7ex}{$''$}
     & turns & $1/{2 \pi} = 0.1591549430918953358$ \\
    h m s & radians & $15.0$ \\
    h m s & days & $15/{2\pi} = 2.3873241463784300365$
   \end{tabular}
 \end{enumerate}
}
%------------------------------------------------------------------------------
\routine{slaDat}{TAI$-$UTC}
{
 \action{Increment to be applied to Coordinated Universal Time UTC to give
         International Atomic Time TAI.}
 \call{d~=~slaDat( utc );}
}
\args{GIVEN}
{
 \spec{utc}{double}{UTC date as a modified JD (JD$-$2400000.5)}
}
\args{RETURNED}
{
 \spec{}{double*}{TAI$-$UTC in seconds}
}
\notes
{
 \begin{enumerate}
 \item The UTC is specified to be a date rather than a time to indicate
       that care needs to be taken not to specify an instant which lies
       within a leap second.  Although
       in most cases {\tt utc} can include the
       fractional part, correct behaviour on the day of a leap second
       can be guaranteed only up to the end of the second
       $23^{\rm h}\,59^{\rm m}\,59^{\rm s}$.
 \item For dates from 1961~January~1 onwards, the expressions from the
       file \\
       {\tt ftp://maia.usno.navy.mil/ser7/tai-utc.dat} are used.
       A 5\,ms time step at \\
       1961~January~1 is taken from 2.58.1 (p87) of
       the 1992 Explanatory Supplement.
 \item UTC began at 1960~January~1.0 (JD 2436934.5) and it is improper
       to call the function with an earlier date.  However, if this
       is attempted, the TAI$-$UTC expression for the year 1960 is used.
 \item This function has to be updated on each occasion that a
       leap second is announced, and programs using it relinked.
       Refer to the program source code for information on when the
       most recent leap second was added.
 \end{enumerate}
}
%-----------------------------------------------------------------------
\routine{slaDav2m}{rotation matrix from axial vector}
{
 \action{Form the rotation matrix corresponding to a given axial vector
         (double precision).}
 \call{slaDav2m( axvec, rmat );}
}
\args{GIVEN}
{
 \spec{axvec}{double[3]}{axial vector (radians)}
}
\args{RETURNED}
{
 \spec{rmat}{double[3][3]}{rotation matrix}
}
\notes
{
 \begin{enumerate}
  \item A rotation matrix describes a rotation about some
        arbitrary axis, called the Euler axis.  The
        {\it axial vector} supplied to this function
        has the same direction as the Euler axis, and its
        magnitude is the amount of rotation in radians.
  \item If {\tt axvec} is null, the unit matrix is returned.
  \item The reference frame rotates clockwise as seen looking along
        the axial vector from the origin.
 \end{enumerate}
}
%-----------------------------------------------------------------------
\routine{slaDbear}{direction between points on a sphere}
{
 \action{Returns the bearing (position angle) of one point on a
         sphere relative to another (double precision).}
 \call{d~=~slaDbear( a1, b1, a2, b2 );}
}
\args{GIVEN}
{
 \spec{a1,b1}{double}{spherical coordinates of one point} \\
 \spec{a2,b2}{double}{spherical coordinates of the other point}
}
\args{RETURNED}
{
 \spec{}{double*}{bearing from first point to second}
}
\notes
{
 \begin{enumerate}
 \item The spherical coordinates are \radec,
       \ecl\ {\it etc.}, in radians.
 \item The result is the bearing (position angle), in radians,
       of point [{\tt a2,b2}] as seen
       from point [{\tt a1,b1}].  It is in the range $\pm \pi$.
       The sense
       is such that if [{\tt a2,b2}]
       is a small distance due east of [{\tt a1,b1}] the result
       is about $+\pi/2$. Zero is returned
       if the two points are coincident.
 \item If either B-coordinate is outside the range $\pm\pi/2$, the
       result may correspond to ``the long way round''.
 \item The function {\tt slaDpav} performs an equivalent function except
       that the points are specified in the form of Cartesian
       vectors.
 \end{enumerate}
}
%-----------------------------------------------------------------------
\routine{slaDbjin}{decode string to B/J epoch}
{
 \action{Decode a character string into a double precision number,
         with special provision for Besselian and Julian epochs.
         The string syntax is as for {\tt slaDfltin}, prefixed by
         an optional \fstring{B} or \fstring{J}.}
 \call{slaDbjin( string, \&nstrt, \&dreslt, \&j1, \&j2 );}
}
\args{GIVEN}
{
 \spec{string}{C}{string containing field to be decoded} \\
 \spec{nstrt}{int}{index to first character of field in string}
}
\args{RETURNED}
{
 \spec{nstrt}{int*}{incremented past the decoded field} \\
 \spec{dreslt}{double*}{result} \\
 \spec{j1}{int*}{{\tt slaDfltin} status:} \\
 \spec{}{}{\hspace{0.7em} $-$1 = $-$OK} \\
 \spec{}{}{\hspace{1.5em}   0 = +OK} \\
 \spec{}{}{\hspace{1.5em}   1 = null field} \\
 \spec{}{}{\hspace{1.5em}   2 = error} \\
 \spec{j2}{int*}{syntax flag:} \\
 \spec{}{}{\hspace{1.5em}   0 = normal {\tt slaDfltin} syntax} \\
 \spec{}{}{\hspace{1.5em}   1 = \fstring{B} or \fstring{b}} \\
 \spec{}{}{\hspace{1.5em}   2 = \fstring{J} or \fstring{j}}
}
\notes
{
 \begin{enumerate}
  \item The purpose of the syntax extensions is to help cope with mixed
        FK4 and FK5 data, allowing fields such as
        \fstring{B1950} or \fstring{J2000} to be decoded.
  \item In addition to the syntax accepted by {\tt slaDfltin},
        the following two extensions are recognized by {\tt slaDbjin}:
        \begin{enumerate}
         \item A valid non-null field preceded by the character
               \fstring{B} (or \fstring{b}) is accepted.
         \item A valid non-null field preceded by the character
               \fstring{J} (or \fstring{j}) is accepted.
         \end{enumerate}
  \item The calling program is told of the
        \fstring{B} or \fstring{J} through an
        supplementary status argument.  The rest of
        the arguments are as for {\tt slaDfltin}.
 \end{enumerate}
}
%-----------------------------------------------------------------------
\routine{slaDc62s}{Cartesian 6-vector to spherical}
{
 \action{Transformation of position \& velocity in Cartesian coordinates
         to spherical coordinates (double precision).}
 \call{slaDc62s( v, \&a, \&b, \&r, \&ad, \&bd, \&rd );}
}
\args{GIVEN}
{
 \spec{v}{double[6]}{\xyzxyzd}
}
\args{RETURNED}
{
 \spec{a}{double*}{longitude (radians)} \\
 \spec{b}{double*}{latitude (radians)} \\
 \spec{r}{double*}{radial coordinate} \\
 \spec{ad}{double*}{longitude derivative (radians per unit time)} \\
 \spec{bd}{double*}{latitude derivative (radians per unit time)} \\
 \spec{rd}{double*}{radial derivative}
}
\anote{The spherical coordinates are longitude (positive anticlockwise
       looking from the positive latitude pole) and latitude.  The
       Cartesian coordinates are right handed, with the {\it x}-axis
       at zero longitude and latitude, and the {\it z}-axis at the
       positive latitude pole.}
%-----------------------------------------------------------------------
\routine{slaDcc2s}{Cartesian to spherical}
{
 \action{Cartesian coordinates to spherical coordinates (double precision).}
 \call{slaDcc2s( v, \&a, \&b );}
}
\args{GIVEN}
{
 \spec{v}{double[3]}{\xyz\ vector}
}
\args{RETURNED}
{
 \spec{a,b}{double*}{spherical coordinates in radians}
}
\notes
{
 \begin{enumerate}
  \item The spherical coordinates are longitude (positive anticlockwise
        looking from the positive latitude pole) and latitude.  The
        Cartesian coordinates are right handed, with the {\it x}-axis
        at zero longitude and latitude, and the {\it z}-axis at the
        positive latitude pole.
  \item If V is null, zero {\tt a} and {\tt b} are returned.
  \item At either pole, zero {\tt a} is returned.
 \end{enumerate}
}
%-----------------------------------------------------------------------
\routine{slaDcmpf}{interpret linear fit}
{
 \action{Decompose an \xy\ linear fit into its constituent parameters:
         zero points, scales, nonperpendicularity and orientation.}
 \call{slaDcmpf( coeffs, \&xz, \&yz, \&xs, \&ys, \&perp, \&orient );}
}
\args{GIVEN}
{
 \spec{coeffs}{double[6]}{transformation coefficients (see note)}
}
\args{RETURNED}
{
 \spec{xz}{double*}{{\it x} zero point} \\
 \spec{yz}{double*}{{\it y} zero point} \\
 \spec{xs}{double*}{{\it x} scale} \\
 \spec{ys}{double*}{{\it y} scale} \\
 \spec{perp}{double}{nonperpendicularity (radians)} \\
 \spec{orient}{double}{orientation (radians)}
}
\notes
{
 \begin{enumerate}
  \item The model relates two sets of \xy\ coordinates as follows.
        Naming the six elements of {\tt coeffs} $a,b,c,d,e$ \& $f$,
        the model transforms coordinates $[x_{1},y_{1}\,]$ into coordinates
        $[x_{2},y_{2}\,]$ as follows:
        \begin{verse}
         $x_{2} = a + bx_{1} + cy_{1}$ \\
         $y_{2} = d + ex_{1} + fy_{1}$
        \end{verse}
        The {\tt slaDcmpf} function decomposes this transformation
        into four steps:
        \begin{enumerate}
        \item Zero points:
              \begin{verse}
               $x' = x_{1} + {\tt xz}$ \\
               $y' = y_{1} + {\tt yz}$
              \end{verse}
        \item Scales:
              \begin{verse}
               $x'' = x' \times {\tt xs}$ \\
               $y'' = y' \times {\tt ys}$
              \end{verse}
        \item Nonperpendicularity:
              \begin{verse}
               $x''' = + x'' \cos({\tt perp}/2) + y'' \sin({\tt perp}/2)$ \\
               $y''' = + x'' \sin({\tt perp}/2) + y'' \cos({\tt perp}/2)$
              \end{verse}
        \item Orientation:
              \begin{verse}
               $x_{2} = + x''' \cos({\tt orient}) +
                          y''' \sin({\tt orient})$ \\
               $y_{2} = - x''' \sin({\tt orient}) +
                          y''' \cos({\tt orient})$
              \end{verse}
        \end{enumerate}
  \item See also {\tt slaFitxy}, {\tt slaPxy}, {\tt slaInvf}, {\tt slaXy2xy}.
 \end{enumerate}
}
%-----------------------------------------------------------------------
\routine{slaDcs2c}{spherical to Cartesian}
{
 \action{Spherical coordinates to Cartesian coordinates (double precision).}
 \call{slaDcs2c( a, b, v );}
}
\args{GIVEN}
{
 \spec{a,b}{double}{spherical coordinates in radians: \radec\ {\it etc.}}
}
\args{RETURNED}
{
 \spec{v}{double[3]}{\xyz\ unit vector}
}
\anote{The spherical coordinates are longitude (positive anticlockwise
       looking from the positive latitude pole) and latitude.  The
       Cartesian coordinates are right handed, with the {\it x}-axis
       at zero longitude and latitude, and the {\it z}-axis at the
       positive latitude pole.}
%-----------------------------------------------------------------------
\routine{slaDd2tf}{days to hour,min,sec}
{
 \action{Convert an interval in days into hours, minutes, seconds
         (double precision).}
 \call{slaDd2tf( ndp, days, \&sign, ihmsf );}
}
\args{GIVEN}
{
 \spec{ndp}{int}{number of decimal places of seconds} \\
 \spec{days}{double}{interval in days}
}
\args{RETURNED}
{
 \spec{sign}{char*}{\fstring{+} or \fstring{-}} \\
 \spec{ihmsf}{int[4]}{hours, minutes, seconds, fraction}
}
\notes
{
 \begin{enumerate}
  \item {\tt ndp} less than zero is interpreted as zero.
  \item The largest useful value for {\tt ndp}
        is determined by the size of
        {\tt days}, the format of {\tt double} numbers on the target
        machine, and the risk of overflowing {\tt ihmsf[3]}.  On some
        architectures, for {\tt days} up to 1.0, the available
        floating-point precision
        corresponds roughly to {\tt ndp}\,=\,12.
        However, the practical limit is {\tt ndp}\,=\,9,
        set by the capacity of a
        typical 32-bit {\tt ihmsf[3]}.
  \item The absolute value of {\tt days}
        may exceed 1.  In cases where it
        does not, it is up to the caller to test for and handle the
        case where {\tt days} is very nearly 1.0
        and rounds up to 24~hours,
        by testing for {\tt ihmsf[0]}\,=\,24
        and setting {\tt ihmsf[0-3]} to zero.
\end{enumerate}
}
%-----------------------------------------------------------------------
\routine{slaDe2h}{\hadec\ to \azel}
{
 \action{Equatorial to horizon coordinates
         (double precision).}
 \call{slaDe2h( ha, dec, phi, \&az, \&el );}
}
\args{GIVEN}
{
 \spec{ha}{double}{hour angle (radians)} \\
 \spec{dec}{double}{declination (radians)} \\
 \spec{phi}{double}{latitude (radians)}
}
\args{RETURNED}
{
 \spec{az}{double*}{azimuth (radians)} \\
 \spec{el}{double*}{elevation (radians)}
}
\notes
{
 \begin{enumerate}
  \item Azimuth is returned in the range $0\!-\!2\pi$;  north is zero,
        and east is $+\pi/2$.  Elevation is returned in the range
        $\pm\pi$.
  \item The latitude must be geodetic.  In critical applications,
        corrections for polar motion should be applied.
  \item In some applications it will be important to specify the
        correct type of hour angle and declination in order to
        produce the required type of azimuth and elevation.  In
        particular, it may be important to distinguish between
        elevation as affected by refraction, which would
        require the {\it observed} \hadec, and the elevation
        {\it in vacuo}, which would require the {\it topocentric}\/
        \hadec.
        If the effects of diurnal aberration can be neglected, the
        {\it apparent}\/ \hadec\ may be used instead of the topocentric
        \hadec.
  \item No range checking of arguments is carried out.
  \item In applications which involve many such calculations, rather
        than calling the present function it will be more efficient to
        use inline code, having previously computed fixed terms such
        as sine and cosine of latitude, and (for tracking a star)
        sine and cosine of declination.
 \end{enumerate}
}
%-----------------------------------------------------------------------
\routine{slaDeuler}{Euler angles to rotation matrix}
{
 \action{Form a rotation matrix from the Euler angles -- three
         successive rotations about specified Cartesian axes
         (double precision).}
 \call{slaDeuler( order, phi, theta, psi, rmat );}
}
\args{GIVEN}
{
 \spec{order}{char*}{specifies about which axes the rotations occur} \\
 \spec{phi}{double}{first rotation (radians)} \\
 \spec{theta}{double}{second rotation (radians)} \\
 \spec{psi}{double}{third rotation (radians)}
}
\args{RETURNED}
{
 \spec{rmat}{double[3][3]}{rotation matrix}
}
\notes
{
 \begin{enumerate}
 \item The characters of {\tt order}
       define which axes the three successive
       rotations are about.  A typical value is \cstring{zxz},
       indicating that {\tt rmat}
       is to become the direction cosine matrix corresponding to
       rotations of the reference frame through {\tt phi}
       radians about the
       old {\it z}-axis, followed by
       {\tt theta} radians about the resulting
       {\it x}-axis,
       then {\tt psi} radians about the resulting {\it z}-axis.
 \item The axis names can be any of the following, in any order or
       combination:  \fstring{X}, \fstring{Y}, \fstring{Z},
       uppercase or lowercase, \fstring{1}, \fstring{2}, \fstring{3}.
       Thus, the
       \cstring{zxz} example given above
       could be written \cstring{ZXZ} or
       \cstring{313} (or even \cstring{ZxZ} or
       \cstring{3xZ}).
 \item {\tt order}
       is terminated by length or by the first unrecognized character.
       Fewer than three rotations are acceptable, in which case the later
       angle arguments are ignored.
       Zero rotations produces the identity {\tt rmat}.
 \item Normal axis labelling/numbering conventions apply;
       the {\it xyz} ($\equiv$\cstring{123}) triad is right-handed.
 \item A rotation is positive when the reference frame rotates
       anticlockwise as seen looking towards the origin from the
       positive region of the specified axis.
 \end{enumerate}
}
%-----------------------------------------------------------------------
\routine{slaDfltin}{decode a double precision number}
{
 \action{Convert free-format input into double precision floating point.}
 \call{slaDfltin( string, \&nstrt, \&dreslt, \&jflag );}
}
\args{GIVEN}
{
 \spec{string}{char*}{string containing number to be decoded} \\
 \spec{nstrt}{int*}{index to where decoding is to commence} \\
 \spec{dreslt}{double*}{current value of result}
}
\args{RETURNED}
{
 \spec{nstrt}{int*}{advanced to next number} \\
 \spec{dreslt}{double*}{result} \\
 \spec{jflag}{int*}{status: $-$1~=~$-$OK, 0~=~+OK, 1~=~null result, 2~=~error}
}
\notes
{
 \begin{enumerate}
 \item The reason {\tt slaDfltin} has separate ``OK'' status values
       for `+' and `$-$'
       is to enable minus zero to be detected.
       This is of crucial importance
       when decoding mixed-radix numbers.  For example, an angle
       expressed as degrees, arcminutes and arcseconds may have a
       leading minus sign but a zero degrees field.
 \item A TAB is interpreted as a space, and lowercase characters are
       interpreted as uppercase.
 \item The basic format is the sequence of fields $\pm n.n x \pm n$,
       where $\pm$ is a sign
       character \fstring{+} or \fstring{-},
       $n$ means a string of decimal digits,
       `.' is a decimal point, and $x$, which indicates an exponent,
       means \fstring{D} or \fstring{E}.
       Various combinations of these fields can be
       omitted, and embedded blanks are permissible in certain places.
 \item Spaces:
       \begin{itemize}
       \item Leading spaces are ignored.
       \item Embedded spaces are allowed only after
             {\tt +},
             {\tt -},
             {\tt D} or
             {\tt E},
             and after the decimal point if the first sequence of
             digits is absent.
       \item Trailing spaces are ignored;  the first signifies
             end of decoding and subsequent ones are skipped.
       \end{itemize}
 \item Delimiters:
       \begin{itemize}
       \item Any character other than
             {\tt +},
             {\tt -},
             {\tt 0}-{\tt 9},
             period,
             {\tt D},
             {\tt E} or space may be
             used to signal the end of the number and terminate decoding.
       \item Comma is recognized by {\tt slaDfltin} as a special case; it
             is skipped, leaving the index on the next character.  See
             13, below.
       \item Decoding will in all cases terminate if end of string
             is reached.
       \end{itemize}
 \item Both signs are optional.  The default is \fstring{+}.
 \item The mantissa $n.n$ defaults to unity.
 \item The exponent $x\!\pm\!n$ defaults to \cstring{D0}.
 \item The strings of decimal digits may be of any length.
 \item The decimal point is optional for whole numbers.
 \item A {\it null result}\/ occurs when the string of characters
       being decoded does not begin with
             {\tt +},
             {\tt -},
             {\tt 0}-{\tt 9},
             period,
             {\tt D} or
             {\tt E}
       or consists entirely of spaces.  When this condition is
       detected, {\tt jflag} is set to 1 and
       {\tt dreslt} is left untouched.
 \item {\tt nstrt}\,=\,1 for the first character in the string.
 \item On return from {\tt slaDfltin}, {\tt nstrt} is set ready for the next
       decode -- following trailing blanks and any comma.  If a
       delimiter other than comma is being used, {\tt nstrt} must be
       incremented before the next call to {\tt slaDfltin}, otherwise
       all subsequent calls will return a null result.
 \item Errors ({\tt jflag}\,=\,2) occur when:
       \begin{itemize}
       \item a
             {\tt +},
             {\tt -},
             {\tt D} or
             {\tt E}
       is left unsatisfied; or
       \item the decimal point is present without at least
             one decimal digit before or after it; or
       \item an exponent more than 100 has been presented.
       \end{itemize}
 \item When an error has been detected, {\tt nstrt} is left
       pointing to the character following the last
       one used before the error came to light.  This
       may be after the point at which a more sophisticated
       program could have detected the error.  For example,
       {\tt slaDfltin} does not detect that \cstring{1e999} is unacceptable
       (on a platform where this is so) until the entire number
       has been decoded.
 \item Certain highly unlikely combinations of mantissa and
       exponent can cause arithmetic faults during the
       decode, in some cases despite the fact that they
       together could be construed as a valid number.
 \item Decoding is left to right, one pass.
 \item See also {\tt slaFlotin}, {\tt slaInt2in} and {\tt slaIntin}.
 \end{enumerate}
}
%-----------------------------------------------------------------------
\routine{slaDh2e}{\azel\ to \hadec}
{
 \action{Horizon to equatorial coordinates
         (double precision).}
 \call{slaDh2e( az, el, phi, \&ha, \&dec );}
}
\args{GIVEN}
{
 \spec{az}{double}{azimuth (radians)} \\
 \spec{el}{double}{elevation (radians)} \\
 \spec{phi}{double}{latitude (radians)}
}
\args{RETURNED}
{
 \spec{ha}{double*}{hour angle (radians)} \\
 \spec{dec}{double*}{declination (radians)}
}
\notes
{
 \begin{enumerate}
  \item The sign convention for azimuth is north zero, east $+\pi/2$.
  \item Hour angle is returned in the range $\pm\pi$.
        Declination is returned in the range $\pm\pi/2$.
  \item The latitude is (in principle) geodetic.  In critical
        applications, corrections for polar motion should be applied
        (see {\tt slaPolmo}).
  \item In some applications it will be important to specify the
        correct type of elevation in order to produce the required
        type of \hadec.  In particular, it may be important to
        distinguish between the elevation as affected by refraction,
        which will yield the {\it observed} \hadec, and the elevation
        {\it in vacuo}, which will yield the {\it topocentric}\/
        \hadec.  If the
        effects of diurnal aberration can be neglected, the
        topocentric \hadec\ may be used as an approximation to the
        {\it apparent}\/ \hadec.
  \item No range checking of arguments is carried out.
  \item In applications which involve many such calculations, rather
        than calling the present function it will be more efficient to
        use inline code, having previously computed fixed terms such
        as sine and cosine of latitude.
 \end{enumerate}
}
%-----------------------------------------------------------------------
\routine{slaDimxv}{apply 3D reverse rotation}
{
 \action{Multiply a 3-vector by the inverse of a rotation
         matrix (double precision).}
 \call{slaDimxv( dm, va, vb );}
}
\args{GIVEN}
{
 \spec{dm}{double[3][3]}{rotation matrix} \\
 \spec{va}{double[3]}{vector to be rotated}
}
\args{RETURNED}
{
 \spec{vb}{double[3]}{result vector}
}
\notes
{
 \begin{enumerate}
  \item This function performs the operation:
        \begin{verse}
         ${\bf b} = {\bf M}^T\times{\bf a}$
        \end{verse}
        where {\bf a} and {\bf b} are the 3-vectors
        {\tt va} and {\tt vb}
        respectively, and {\bf M} is the
        $3\times3$ matrix {\tt dm}.
  \item The purpose of this function is to apply an inverse
        rotation;  under these circumstances, ${\bf M}$ must be
        {\it orthogonal}, so that its inverse is the same as
        its transpose.
  \item The same array may be specified for both
        {\tt va} and {\tt vb}.
 \end{enumerate}
}
%-----------------------------------------------------------------------
\routine{slaDjcal}{MJD to Gregorian for output}
{
 \action{Modified Julian Date to Gregorian Calendar Date, expressed
         in a form convenient for formatting messages (namely
         rounded to a specified precision, and with the fields
         stored in a single array).}
 \call{slaDjcal( ndp, djm, iymdf, \&j );}
}
\args{GIVEN}
{
 \spec{ndp}{int}{number of decimal places of days in fraction} \\
 \spec{djm}{double}{modified Julian Date (JD$-$2400000.5)}
}
\args{RETURNED}
{
 \spec{iymdf}{int[4]}{year, month, day, fraction in Gregorian calendar} \\
 \spec{j}{int*}{status:  nonzero = out of range}
}
\notes
{
 \begin{enumerate}
  \item Any date after 4701\,BC March 1 is accepted.
  \item Large {\tt ndp} values risk internal overflows.
        It is typically safe to use up to {\tt ndp}\,=\,4.
 \end{enumerate}
}
\aref{The algorithm is adapted from Hatcher 1984,
      {\it Q.\,Jl.\,R.\,astr.\,Soc.}\ {\bf 25}, 53-55.}
%-----------------------------------------------------------------------
\routine{slaDjcl}{MJD to year,month,day,frac}
{
 \action{Modified Julian Date to Gregorian year, month, day,
         and fraction of a day.}
 \call{slaDjcl( djm, \&iy, \&im, \&id, \&fd, \&j );}
}
\args{GIVEN}
{
 \spec{djm}{double}{modified Julian Date (JD$-$2400000.5)}
}
\args{RETURNED}
{
 \spec{iy}{int*}{year} \\
 \spec{im}{int*}{month} \\
 \spec{id}{int*}{day} \\
 \spec{fd}{double*}{fraction of day} \\
 \spec{j}{int*}{status:} \\
 \spec{}{}{\hspace{1.5em}0~=~OK} \\
 \spec{}{}{\hspace{0.7em}$-$1~= unacceptable date} \\
 \spec{}{}{\hspace{0.7em}~~~~~~~~~~~~(before 4701\,BC~March~1)}
}
\aref{The algorithm is adapted from Hatcher 1984,
      {\it Q.\,Jl.\,R.\,astr.\,Soc.}\ {\bf 25}, 53-55.}
%-----------------------------------------------------------------------
\routine{slaDm2av}{rotation matrix to axial vector}
{
 \action{From a rotation matrix, determine the corresponding axial vector
        (double precision).}
 \call{slaDm2av( rmat, axvec );}
}
\args{GIVEN}
{
 \spec{rmat}{double[3][3]}{rotation matrix}
}
\args{RETURNED}
{
 \spec{axvec}{double[3]}{axial vector (radians)}
}
\notes
{
 \begin{enumerate}
  \item A rotation matrix describes a rotation about some arbitrary axis,
        called the Euler axis.  The {\it axial vector} returned by
        this function has the same direction as the Euler axis, and its
        magnitude is the amount of rotation in radians.
  \item The magnitude and direction of the axial vector can be separated
        by means of the function {\tt slaDvn}.
  \item The reference frame rotates clockwise as seen looking along
        the axial vector from the origin.
  \item If {\tt rmat} is null, so is the result.
 \end{enumerate}
}
%-----------------------------------------------------------------------
\routine{slaDmat}{solve simultaneous equations}
{
 \action{Matrix inversion and solution of simultaneous equations
         (double precision).}
 \call{slaDmat( n, a, y, d, \&jf, iw );}
}
\args{GIVEN}
{
 \spec{n}{int}{number of unknowns} \\
 \spec{a}{double[n][n]}{matrix} \\
 \spec{y}{double[n]}{vector}
}
\args{RETURNED}
{
 \spec{a}{double[n][n]}{matrix inverse} \\
 \spec{y}{double[n]}{solution} \\
 \spec{d}{double*}{determinant} \\
 \spec{jf}{int*}{singularity flag: 0\,=\,OK} \\
 \spec{iw}{int[n]}{workspace}
}
\notes
{
 \begin{enumerate}
  \item For the set of $n$ simultaneous linear equations in $n$ unknowns:
        \begin{verse}
         ${\bf A}\times{\bf y} = {\bf x}$
        \end{verse}
        where:
        \begin{itemize}
         \item {\bf A} is a non-singular $n \times n$ matrix,
         \item {\bf y} is the vector of $n$ unknowns, and
         \item {\bf x} is the known vector,
        \end{itemize}
        {\tt slaDmat} computes:
        \begin{itemize}
         \item the inverse of matrix {\bf A},
         \item the determinant of matrix {\bf A}, and
         \item the vector of $n$ unknowns {\bf y}.
        \end{itemize}
        Argument {\tt n} is the order $n$,
        {\tt a} (given) is the matrix {\bf A},
        {\tt y} (given) is the vector
        {\bf x} and {\tt y} (returned)
        is the vector {\bf y}.
        The argument {\tt a} (returned)
        is the inverse matrix ${\bf A}^{-1}$,
        and {\tt d} is $\det{\bf A}$.
  \item {\tt jf} is the singularity flag.
        If the matrix is non-singular,
        ${\tt jf}=0$ is returned.
        If the matrix is singular, ${\tt jf}=-1$
        and ${\tt d}=0.0$ are returned.
        In the latter case, the contents
        of array {\tt a} on return are undefined.
  \item The algorithm is Gaussian elimination with partial pivoting.
        This method is very fast;  some much slower algorithms can give
        better accuracy, but only by a small factor.
 \end{enumerate}
}
%-----------------------------------------------------------------------
\routine{slaDmoon}{approximate Moon position and velocity}
{
 \action{Approximate geocentric position and velocity of the Moon
         (double precision).}
 \call{slaDmoon( date, pv );}
}
\args{GIVEN}
{
 \spec{date}{double}{TDB as a Modified Julian Date (JD$-$2400000.5)
}
}
\args{RETURNED}
{
 \spec{pv}{double[6]}{Moon \xyzxyzd, mean equator and equinox
                 of date (AU, AU~s$^{-1}$)}
}
\notes
{
 \begin{enumerate}
  \item This function is a full implementation of the algorithm
        published by Meeus (see reference).
  \item Meeus quotes accuracies of \arcseci{10} in longitude,
        \arcseci{3} in latitude and \arcsec{0}{2} arcsec in HP
        (equivalent to about 20~km in distance).  Comparison with
        JPL~DE200 over the interval 1960-2025 gives RMS errors of
        \arcsec{3}{7} and 83~mas/hour in longitude,
        \arcsec{2}{3} arcsec and 48~mas/hour in latitude,
        11~km and 81~mm/s in distance.
        The maximum errors over the same interval are
        \arcseci{18} and \arcsec{0}{50}/hour in longitude,
        \arcseci{11} and \arcsec{0}{24}/hour in latitude,
        40~km and 0.29~m/s in distance.
  \item The original algorithm is expressed in terms of the obsolete
        time-scale {\it Ephemeris Time}.  Either TDB or TT can be used,
        but not UT without incurring significant errors (\arcseci{30} at
        the present time) due to the Moon's \arcsec{0}{5}/s movement.
  \item The algorithm is based on pre IAU~1976 standards.  However,
        the result has been moved onto the new (FK5) equinox, an
        adjustment which is in any case much smaller than the
        intrinsic accuracy of the procedure.
  \item Velocity is obtained by a complete analytical differentiation
        of the Meeus model.
 \end{enumerate}
}
\aref{Meeus, {\it l'Astronomie}, June 1984, p348.}
%-----------------------------------------------------------------------
\routine{slaDmxm}{product of two $3\times3$ matrices}
{
 \action{Product of two $3\times3$ matrices (double precision).}
 \call{slaDmxm( a, b, c );}
}
\args{GIVEN}
{
 \spec{a}{double[3][3]}{matrix {\bf A}} \\
 \spec{b}{double[3][3]}{matrix {\bf B}}
}
\args{RETURNED}
{
 \spec{c}{double[3][3]}{matrix result: {\bf A}$\times${\bf B}}
}
\anote{The arguments do not have to be different arrays.}
%-----------------------------------------------------------------------
\routine{slaDmxv}{apply 3D rotation}
{
 \action{Multiply a 3-vector by a rotation matrix (double precision).}
 \call{slaDmxv( dm, va, vb );}
}
\args{GIVEN}
{
 \spec{dm}{double[3][3]}{rotation matrix} \\
 \spec{va}{double[3]}{vector to be rotated}
}
\args{RETURNED}
{
 \spec{vb}{double[3]}{result vector}
}
\notes
{
 \begin{enumerate}
  \item This function performs the operation:
        \begin{verse}
           {\bf b} = {\bf M}$\times${\bf a}
        \end{verse}
        where {\bf a} and {\bf b} are the 3-vectors
        {\tt va} and {\tt vb}
        respectively, and {\bf M} is the $3\times3$ matrix {\tt dm}.
  \item The main function of this function is apply a
        rotation;  under these circumstances, {\bf M} is a
        {\it proper real orthogonal}\/ matrix.
  \item {\tt va} and {\tt vb} may be the same array.
 \end{enumerate}
}
%-----------------------------------------------------------------------
\routine{slaDpav}{position-angle between two directions}
{
 \action{Returns the bearing (position angle) of one celestial
         direction with respect to another (double precision).}
 \call{d~=~slaDpav( v1, v2 );}
}
\args{GIVEN}
{
 \spec{v1}{double[3]}{vector to one point} \\
 \spec{v2}{double[3]}{vector to the other point}
}
\args{RETURNED}
{
 \spec{}{double}{position-angle of second point with respect to first}
}
\notes
{
 \begin{enumerate}
 \item The coordinate frames correspond to \radec,
       \ecl\ {\it etc.}.
 \item The result is the bearing (position angle), in radians,
        of point {\tt v2} as seen
       from point {\tt v1}.  It is in the range $\pm \pi$.  The sense
       is such that if {\tt v2}
       is a small distance due east of {\tt v1} the result
       is about $+\pi/2$. Zero is returned
       if the two points are coincident.
 \item There is no requirement for either vector to be of unit length.
 \item The function {\tt slaDbear} performs an equivalent function except
       that the points are specified in the form of spherical coordinates.
 \end{enumerate}
}
%------------------------------------------------------------------------------
\routine{slaDr2af}{radians to deg,min,sec,frac}
{
 \action{Convert an angle in radians to degrees, arcminutes, arcseconds,
         fraction (double precision).}
 \call{slaDr2af( ndp, angle, \&sign, idmsf );}
}
\args{GIVEN}
{
 \spec{ndp}{int}{number of decimal places of arcseconds} \\
 \spec{angle}{double}{angle in radians}
}
\args{RETURNED}
{
 \spec{sign}{char*}{\fstring{+} or \fstring{-}} \\
 \spec{idmsf}{int[4]}{degrees, arcminutes, arcseconds, fraction}
}
\notes
{
 \begin{enumerate}
  \item {\tt ndp} less than zero is interpreted as zero.
  \item The largest useful value for {\tt ndp}
        is determined by the size of
        {\tt angle}, the format of {\tt double} numbers on the target
        machine, and the risk of overflowing {\tt idmsf[3]}.  On some
        architectures, for {\tt angle} up to $2\pi$,
        the available floating-point
        precision corresponds roughly to {\tt ndp}\,=\,12.
        However, the practical limit is {\tt ndp}\,=\,9
        set by the capacity of a
        typical 32-bit {\tt idmsf[3]}.
  \item The absolute value of {\tt angle}
        may exceed $2\pi$.  In cases where it
        does not, it is up to the caller to test for and handle the
        case where {\tt angle} is very nearly $2\pi$
        and rounds up to $360^{\circ}$,
        by testing for {\tt idmsf[0]}\,=\,360
        and setting {\tt idmsf[0-3]} to zero.
 \end{enumerate}
}
%-----------------------------------------------------------------------
\routine{slaDr2tf}{radians to hour,min,sec,frac}
{
 \action{Convert an angle in radians to hours, minutes, seconds,
         fraction (double precision).}
 \call{slaDr2tf( ndp, angle, \&sign, ihmsf );}
}
\args{GIVEN}
{
 \spec{ndp}{int}{number of decimal places of seconds} \\
 \spec{angle}{double}{angle in radians}
}
\args{RETURNED}
{
 \spec{sign}{char*}{\fstring{+} or \fstring{-}} \\
 \spec{ihmsf}{int[4]}{hours, minutes, seconds, fraction}
}
\notes
{
 \begin{enumerate}
  \item {\tt ndp} less than zero is interpreted as zero.
  \item The largest useful value for {\tt ndp}
        is determined by the size of
        {\tt angle}, the format of {\tt double} numbers on the target
        machine, and the risk of overflowing {\tt ihmsf[3]}.  On some
        architectures, for {\tt angle} up to $2\pi$, the available
        floating-point precision corresponds roughly
        to {\tt ndp}\,=\,12.  However, the practical limit is
        {\tt ndp}\,=\,9, set by the capacity of a
        typical 32-bit {\tt ihmsf[3]}.
  \item The absolute value of {\tt angle}
        may exceed $2\pi$.  In cases where it
        does not, it is up to the caller to test for and handle the
        case where {\tt angle} is very nearly $2\pi$
        and rounds up to 24~hours,
        by testing for {\tt ihmsf[0]}\,=\,24
        and setting {\tt ihmsf[0-3]} to zero.
 \end{enumerate}
}
%-----------------------------------------------------------------------
\routine{slaDrange}{put angle into range $\pm\pi$}
{
 \action{Normalize an angle into the range $\pm\pi$ (double precision).}
 \call{d~=~slaDrange( angle );}
}
\args{GIVEN}
{
 \spec{angle}{double}{angle in radians}
}
\args{RETURNED}
{
 \spec{}{double}{{\tt angle} expressed in the range $\pm\pi$.}
}
%-----------------------------------------------------------------------
\routine{slaDranrm}{put angle into range $0\!-\!2\pi$}
{
 \action{Normalize an angle into the range $0\!-\!2\pi$
         (double precision).}
 \call{d~=~slaDranrm( angle );}
}
\args{GIVEN}
{
 \spec{angle}{double}{angle in radians}
}
\args{RETURNED}
{
 \spec{}{double}{{\tt angle} expressed in the range $0\!-\!2\pi$}
}
%-----------------------------------------------------------------------
\routine{slaDs2c6}{spherical position and velocity to Cartesian}
{
 \action{Transformation of position \& velocity in spherical coordinates
         to Cartesian coordinates (double precision).}
 \call{slaDs2c6( a, b, r, ad, bd, rd, v );}
}
\args{GIVEN}
{
 \spec{a}{double}{longitude (radians) -- for example $\alpha$} \\
 \spec{b}{double}{latitude (radians) -- for example $\delta$} \\
 \spec{float}{double}{radial coordinate} \\
 \spec{ad}{double}{longitude derivative (radians per unit time)} \\
 \spec{bd}{double}{latitude derivative (radians per unit time)} \\
 \spec{rd}{double}{radial derivative}
}
\args{RETURNED}
{
 \spec{v}{double[6]}{\xyzxyzd}
}
\anote{The spherical coordinates are longitude (positive anticlockwise
       looking from the positive latitude pole) and latitude.  The
       Cartesian coordinates are right handed, with the {\it x}-axis
       at zero longitude and latitude, and the {\it z}-axis at the
       positive latitude pole.}
%-----------------------------------------------------------------------
\routine{slaDs2tp}{spherical to tangent plane}
{
 \action{Projection of spherical coordinates onto the tangent plane
         (double precision).}
 \call{slaDs2tp( ra, dec, raz, decz, \&xi, \&eta, \&j );}
}
\args{GIVEN}
{
 \spec{ra,dec}{double}{spherical coordinates of star (radians)} \\
 \spec{raz,decz}{double}{spherical coordinates of tangent point (radians)}
}
\args{RETURNED}
{
 \spec{xi,eta}{*double}{tangent plane coordinates (radians)} \\
 \spec{j}{int*}{status:} \\
 \spec{}{}{\hspace{1.5em} 0 = OK, star on tangent plane} \\
 \spec{}{}{\hspace{1.5em} 1 = error, star too far from axis} \\
 \spec{}{}{\hspace{1.5em} 2 = error, antistar on tangent plane} \\
 \spec{}{}{\hspace{1.5em} 3 = error, antistar too far from axis}
}
\notes
{
 \begin{enumerate}
  \item The projection is called the {\it gnomonic}\/ projection;  the
        Cartesian coordinates \xieta\ are called
        {\it standard coordinates.}\/  The latter
        are in units of the distance from the tangent plane to the projection
        point, {\it i.e.}~radians near the origin.
  \item When working in \xyz\ rather than spherical coordinates, the
        equivalent Cartesian function {\tt slaDv2tp} is available.
 \end{enumerate}
}
%-----------------------------------------------------------------------
\routine{slaDsep}{angle between two points on a sphere}
{
 \action{Angle between two points on a sphere (double precision).}
 \call{d~=~slaDsep( a1, b1, a2, b2 );}
}
\args{GIVEN}
{
 \spec{a1,b1}{double}{spherical coordinates of one point (radians)} \\
 \spec{a2,b2}{double}{spherical coordinates of the other point (radians)}
}
\args{RETURNED}
{
 \spec{}{double}{angle between [{\tt a1,b1}] and [{\tt a2,b2}] in radians}
}
\notes
{
 \begin{enumerate}
  \item The spherical coordinates are right ascension and declination,
        longitude and latitude, {\it etc.}, in radians.
  \item The result is always positive.
 \end{enumerate}
}
%-----------------------------------------------------------------------
\routine{slaDsepv}{angle between two vectors}
{
 \action{Angle between two vectors (double precision).}
 \call{d~=~slaDsepv( v1, v2 );}
}
\args{GIVEN}
{
 \spec{v1}{double[3]}{first vector} \\
 \spec{v2}{double[3]}{second vector}
}
\args{RETURNED}
{
 \spec{}{double}{angle between {\tt v1} and {\tt v2} in radians}
}
\notes
{
 \begin{enumerate}
  \item There is no requirement for either vector to be of unit length.
  \item If either vector is null, zero is returned.
  \item The result is always positive.
 \end{enumerate}
}
%-----------------------------------------------------------------------
\routine{slaDt}{approximate ET minus UT}
{
 \action{Estimate $\Delta$T, the offset between dynamical time
         and Universal Time, for a given historical epoch.}
 \call{d~=~slaDt( epoch );}
}
\args{GIVEN}
{
 \spec{epoch}{double}{(Julian) epoch ({\it e.g.}\ 1850.0)}
}
\args{RETURNED}
{
 \spec{}{double}{approximate ET$-$UT (after 1984, TT$-$UT) in seconds}
}
\notes
{
 \begin{enumerate}
  \item Depending on the epoch, one of three parabolic approximations
        is used:
\begin{tabbing}
xx \= xxxxxxxxxxxxxxxxxx \= \kill
\> before AD\,979 \> Stephenson \& Morrison's 390\,BC to AD\,948 model \\
\> AD\,979 to AD\,1708 \> Stephenson \& Morrison's AD\,948 to AD\,1600 model \\
\> after AD\,1708 \> McCarthy \& Babcock's post-1650 model
\end{tabbing}
        The breakpoints are chosen to ensure continuity:  they occur
        at places where the adjacent models give the same answer as
        each other.
  \item The accuracy is modest, with errors of up to $20^{\rm s}$ during
        the interval since 1650, rising to perhaps $30^{\rm m}$
        by 1000\,BC.  Comparatively accurate values from AD\,1600
        are tabulated in
        the {\it Astronomical Almanac}\/ (see section K8 of the 1995
        edition).
  \item The use of double precision for both argument and result is
        simply for compatibility with other SLALIB time functions.
  \item The models used are based on a lunar tidal acceleration value
        of \arcsec{-26}{00} per century.
 \end{enumerate}
}
\aref{Seidelmann, P.K.\ (ed), 1992, {\it Explanatory
      Supplement to the Astronomical Almanac,}\/ ISBN~0-935702-68-7.
      This contains references to the papers by Stephenson \& Morrison
      and by McCarthy \& Babcock which describe the models used here.}
%-----------------------------------------------------------------------
\routine{slaDtf2d}{hour,min,sec to days}
{
 \action{Convert hours, minutes, seconds to days (double precision).}
 \call{slaDtf2d( ihour, imin, sec, \&days, \&j );}
}
\args{GIVEN}
{
 \spec{ihour}{int}{hours} \\
 \spec{imin}{int}{minutes} \\
 \spec{sec}{double}{seconds}
}
\args{RETURNED}
{
 \spec{days}{double*}{interval in days} \\
 \spec{j}{int*}{status:} \\
 \spec{}{}{\hspace{1.5em} 0 = OK} \\
 \spec{}{}{\hspace{1.5em} 1 = {\tt ihour} outside range 0-23} \\
 \spec{}{}{\hspace{1.5em} 2 = {\tt imin} outside range 0-59} \\
 \spec{}{}{\hspace{1.5em} 3 = {\tt sec} outside range 0-59.999$\cdots$}
}
\notes
{
 \begin{enumerate}
  \item The result is computed even if any of the range checks fail.
  \item The sign must be dealt with outside this function.
 \end{enumerate}
}
%-----------------------------------------------------------------------
\routine{slaDtf2r}{hour,min,sec to radians}
{
 \action{Convert hours, minutes, seconds to radians (double precision).}
 \call{slaDtf2r( ihour, imin, sec, \&rad, \&j );}
}
\args{GIVEN}
{
 \spec{ihour}{int}{hours} \\
 \spec{imin}{int}{minutes} \\
 \spec{sec}{double}{seconds}
}
\args{RETURNED}
{
 \spec{rad}{double*}{angle in radians} \\
 \spec{j*}{int}{status:} \\
 \spec{}{}{\hspace{1.5em} 0 = OK} \\
 \spec{}{}{\hspace{1.5em} 1 = {\tt ihour} outside range 0-23} \\
 \spec{}{}{\hspace{1.5em} 2 = {\tt imin} outside range 0-59} \\
 \spec{}{}{\hspace{1.5em} 3 = {\tt sec} outside range 0-59.999$\cdots$}
}
\notes
{
 \begin{enumerate}
  \item The result is computed even if any of the range checks fail.
  \item The sign must be dealt with outside this function.
 \end{enumerate}
}
%-----------------------------------------------------------------------
\routine{slaDtp2s}{tangent plane to spherical}
{
 \action{Transform tangent plane coordinates into spherical
         coordinates (double precision)}
 \call{slaDtp2s( xi, eta, raz, decz, \&ra, \&dec );}
}
\args{GIVEN}
{
 \spec{xi,eta}{double}{tangent plane rectangular coordinates (radians)} \\
 \spec{raz,decz}{double}{spherical coordinates of tangent point (radians)}
}
\args{RETURNED}
{
 \spec{ra,dec}{double*}{spherical coordinates (radians)}
}
\notes
{
 \begin{enumerate}
  \item The projection is called the {\it gnomonic}\/ projection;  the
        Cartesian coordinates \xieta\ are called
        {\it standard coordinates.}\/  The latter
        are in units of the distance from the tangent plane to the projection
        point, {\it i.e.}~radians near the origin.
  \item When working in \xyz\ rather than spherical coordinates, the
        equivalent Cartesian function {\tt slaDtp2v} is available.
 \end{enumerate}
}
%-----------------------------------------------------------------------
\routine{slaDtp2v}{tangent plane to \xyz}
{
 \action{Given the tangent-plane coordinates of a star and the direction
         cosines of the tangent point, determine the direction cosines
         of the star
         (double precision).}
 \call{slaDtp2v( xi, eta, v0, v );}
}
\args{GIVEN}
{
 \spec{xi,eta}{double}{tangent plane coordinates of star (radians)} \\
 \spec{v0}{double[3]}{direction cosines of tangent point}
}
\args{RETURNED}
{
 \spec{v}{double[3]}{direction cosines of star}
}
\notes
{
 \begin{enumerate}
  \item If vector {\tt v0}
        is not of unit length, the returned vector {\tt v} will
        be wrong.
  \item If vector {\tt v0}
        points at a pole, the returned vector {\tt v} will be
        based on the arbitrary assumption that $\alpha=0$ at
        the tangent point.
  \item The projection is called the {\it gnomonic}\/ projection;  the
        Cartesian coordinates \xieta\ are called
        {\it standard coordinates.}\/  The latter
        are in units of the distance from the tangent plane to the projection
        point, {\it i.e.}~radians near the origin.
  \item This function is the Cartesian equivalent of the function {\tt slaDtp2s}.
 \end{enumerate}
}
%-----------------------------------------------------------------------
\routine{slaDtps2c}{plate centre from \xieta\ and \radec}
{
 \action{From the tangent plane coordinates of a star of known \radec,
        determine the \radec\ of the tangent point (double precision)}
 \call{slaDtps2c( xi, eta, ra, dec, \&raz1, \&decz1, \&raz2, \&decz2, \&n );}
}
\args{GIVEN}
{
 \spec{xi,eta}{double}{tangent plane rectangular coordinates (radians)} \\
 \spec{ra,dec}{double}{spherical coordinates (radians)}
}
\args{RETURNED}
{
 \spec{raz1,decz1}{double*}{spherical coordinates of tangent point,
                      solution 1} \\
 \spec{raz2,decz2}{double*}{spherical coordinates of tangent point,
                      solution 2} \\
 \spec{n}{int*}{number of solutions:} \\
 \spec{}{}{\hspace{1em} 0 = no solutions returned  (Note~2)} \\
 \spec{}{}{\hspace{1em} 1 = only the first solution is useful (Note~3)} \\
 \spec{}{}{\hspace{1em} 2 = there are two useful solutions (Note~3)}
}
\notes
{
 \begin{enumerate}
  \item The {\tt raz1} and {\tt raz2}
        values returned are in the range $0\!-\!2\pi$.
  \item Cases where there is no solution can only arise near the poles.
        For example, it is clearly impossible for a star at the pole
        itself to have a non-zero $\xi$ value, and hence it is
        meaningless to ask where the tangent point would have to be
        to bring about this combination of $\xi$ and $\delta$.
  \item Also near the poles, cases can arise where there are two useful
        solutions.  The argument {\tt n}
        indicates whether the second of the
        two solutions returned is useful:  {\tt n}\,=\,1
        indicates only one useful solution, the usual case;  under
        these circumstances, the second solution corresponds to the
        ``beyond the pole'' case, and this is reflected in the values
        of {\tt raz2} and {\tt decz2} which are returned.
  \item The {\tt decz1} and {\tt decz2} values returned
        are in the range $\pm\pi$,
        but in the ordinary, non-pole-crossing, case, the range is
        $\pm\pi/2$.
  \item {\tt ra}, {\tt dec},
        {\tt raz1}, {\tt decz1},
        {\tt raz2}, {\tt decz2} are all in radians.
  \item The projection is called the {\it gnomonic}\/ projection;  the
        Cartesian coordinates \xieta\ are called
        {\it standard coordinates.}\/  The latter
        are in units of the distance from the tangent plane to the projection
        point, {\it i.e.}~radians near the origin.
  \item When working in \xyz\ rather than spherical coordinates, the
        equivalent Cartesian function {\tt slaDtpv2c} is available.
 \end{enumerate}
}
%-----------------------------------------------------------------------
\routine{slaDtpv2c}{plate centre from \xieta\ and \xyz}
{
 \action{From the tangent plane coordinates of a star of known
         direction cosines, determine the direction cosines
         of the tangent point (double precision)}
 \call{slaDtpv2c( xi, eta, v, v01, v02, \&n );}
}
\args{GIVEN}
{
 \spec{xi,eta}{double}{tangent plane coordinates of star (radians)} \\
 \spec{v}{double[3]}{direction cosines of star}
}
\args{RETURNED}
{
 \spec{v01}{double[3]}{direction cosines of tangent point, solution 1} \\
 \spec{v02}{double[3]}{direction cosines of tangent point, solution 2} \\
 \spec{n}{int*}{number of solutions:} \\
 \spec{}{}{\hspace{1em} 0 = no solutions returned  (Note~2)} \\
 \spec{}{}{\hspace{1em} 1 = only the first solution is useful (Note~3)} \\
 \spec{}{}{\hspace{1em} 2 = there are two useful solutions (Note~3)}
}
\notes
{
 \begin{enumerate}
  \item The vector {\tt v}
        must be of unit length or the result will be wrong.
  \item Cases where there is no solution can only arise near the poles.
        For example, it is clearly impossible for a star at the pole
        itself to have a non-zero {\tt xi} value.
  \item Also near the poles, cases can arise where there are two useful
        solutions.  The argument {\tt n}
        indicates whether the second of the
        two solutions returned is useful: {\tt n}\,=\,1
        indicates only one useful solution, the usual case;  under these
        circumstances, the second solution can be regarded as valid if
        the vector {\tt v02} is interpreted as the ``beyond the pole'' case.
  \item The projection is called the {\it gnomonic}\/ projection;  the
        Cartesian coordinates \xieta\ are called
        {\it standard coordinates.}\/  The latter
        are in units of the distance from the tangent plane to the projection
        point, {\it i.e.}~radians near the origin.
  \item This function is the Cartesian equivalent of the function {\tt slaDtps2c}.
 \end{enumerate}
}
%-----------------------------------------------------------------------
\routine{slaDtt}{TT minus UTC}
{
 \action{Compute $\Delta$TT, the increment to be applied to
         Coordinated Universal Time UTC to give
         Terrestrial Time TT.}
 \call{d~=~slaDtt( utc );}
}
\args{GIVEN}
{
 \spec{utc}{double}{UTC date as a modified JD (JD$-$2400000.5)}
}
\args{RETURNED}
{
 \spec{}{double}{TT$-$UTC in seconds}
}
\notes
{
 \begin{enumerate}
  \item The UTC is specified to be a date rather than a time to indicate
        that care needs to be taken not to specify an instant which lies
        within a leap second.  Though in most cases {\tt utc}
        can include the
        fractional part, correct behaviour on the day of a leap second
        can be guaranteed only up to the end of the second
        $23^{\rm h}\,59^{\rm m}\,59^{\rm s}$.
  \item Pre 1972 January 1 a fixed value of 10 + ET$-$TAI is returned.
  \item TT is one interpretation of the defunct time-scale
        {\it Ephemeris Time}, ET.
  \item See also the function {\tt slaDt}, which roughly estimates ET$-$UT for
        historical epochs.
 \end{enumerate}
}
%-----------------------------------------------------------------------
\routine{slaDv2tp}{\xyz\ to tangent plane}
{
 \action{Given the direction cosines of a star and of the tangent point,
         determine the star's tangent-plane coordinates
         (double precision).}
 \call{slaDv2tp( v, v0, \&xi, \&eta, \&j );}
}
\args{GIVEN}
{
 \spec{v}{double[3]}{direction cosines of star} \\
 \spec{v0}{double[3]}{direction cosines of tangent point}
}
\args{RETURNED}
{
 \spec{xi,eta}{double*}{tangent plane coordinates (radians)} \\
 \spec{j}{int*}{status:} \\
 \spec{}{}{\hspace{1.5em} 0 = OK, star on tangent plane} \\
 \spec{}{}{\hspace{1.5em} 1 = error, star too far from axis} \\
 \spec{}{}{\hspace{1.5em} 2 = error, antistar on tangent plane} \\
 \spec{}{}{\hspace{1.5em} 3 = error, antistar too far from axis}
}
\notes
{
 \begin{enumerate}
  \item If vector {\tt v0}
        is not of unit length, or if vector {\tt v} is of zero
        length, the results will be wrong.
  \item If {\tt v0} points at a pole, the returned $\xi,\eta$
        will be based on the
        arbitrary assumption that $\alpha=0$ at the tangent point.
  \item The projection is called the {\it gnomonic}\/ projection;  the
        Cartesian coordinates \xieta\ are called
        {\it standard coordinates.}\/  The latter
        are in units of the distance from the tangent plane to the projection
        point, {\it i.e.}~radians near the origin.
  \item This function is the Cartesian equivalent of the function {\tt slaDs2tp}.
 \end{enumerate}
}
%-----------------------------------------------------------------------
\routine{slaDvdv}{scalar product}
{
 \action{Scalar product of two 3-vectors (double precision).}
 \call{d~=~slaDvdv( va, vb );}
}
\args{GIVEN}
{
 \spec{va}{double[3]}{first vector} \\
 \spec{vb}{double[3]}{second vector}
}
\args{RETURNED}
{
 \spec{}{double}{scalar product {\tt va.vb}}
}
%-----------------------------------------------------------------------
\routine{slaDvn}{normalize vector}
{
 \action{Normalize a 3-vector, also giving the modulus (double precision).}
 \call{slaDvn( v, uv, \&vm );}
}
\args{GIVEN}
{
 \spec{v}{double[3]}{vector}
}
\args{RETURNED}
{
 \spec{uv}{double[3]}{unit vector $\hat{\tt v}$} \\
 \spec{vm}{double*}{modulus $\mid\!{\tt v}\!\mid$}
}
\anote{If the modulus of {\tt v} is zero,
       {\tt uv} is set to the null vector.}
%-----------------------------------------------------------------------
\routine{slaDvxv}{vector product}
{
 \action{Vector product of two 3-vectors (double precision).}
 \call{slaDvxv( va, vb, vc );}
}
\args{GIVEN}
{
 \spec{va}{double[3]}{first vector} \\
 \spec{vb}{double[3]}{second vector}
}
\args{RETURNED}
{
 \spec{vc}{double[3]}{vector product {\tt va}$\times${\tt vb}}
}
%-----------------------------------------------------------------------
\routine{slaE2h}{\hadec\ to \azel}
{
 \action{Equatorial to horizon coordinates
         (single precision).}
 \call{slaDe2h( ha, dec, phi, \&az, \&el );}
}
\args{GIVEN}
{
 \spec{ha}{float}{hour angle (radians)} \\
 \spec{dec}{float}{declination (radians)} \\
 \spec{phi}{float}{latitude (radians)}
}
\args{RETURNED}
{
 \spec{az}{float*}{azimuth (radians)} \\
 \spec{el}{float*}{elevation (radians)}
}
\notes
{
 \begin{enumerate}
  \item Azimuth is returned in the range $0\!-\!2\pi$;  north is zero,
        and east is $+\pi/2$.  Elevation is returned in the range
        $\pm\pi$.
  \item The latitude must be geodetic.  In critical applications,
        corrections for polar motion should be applied.
  \item In some applications it will be important to specify the
        correct type of hour angle and declination in order to
        produce the required type of azimuth and elevation.  In
        particular, it may be important to distinguish between
        elevation as affected by refraction, which would
        require the {\it observed} \hadec, and the elevation
        {\it in vacuo}, which would require the {\it topocentric}\/
        \hadec.
        If the effects of diurnal aberration can be neglected, the
        {\it apparent}\/ \hadec\ may be used instead of the topocentric
        \hadec.
  \item No range checking of arguments is carried out.
  \item In applications which involve many such calculations, rather
        than calling the present function it will be more efficient to
        use inline code, having previously computed fixed terms such
        as sine and cosine of latitude, and (for tracking a star)
        sine and cosine of declination.
 \end{enumerate}
}
%-----------------------------------------------------------------------
\routine{slaEarth}{approximate Earth position and velocity}
{
 \action{Approximate heliocentric position and velocity of the Earth
         (single precision).}
 \call{slaEarth( iy, id, fd, pv );}
}
\args{GIVEN}
{
 \spec{iy}{int}{year} \\
 \spec{id}{int}{day in year (1 = Jan 1st)} \\
 \spec{fd}{float}{fraction of day}
}
\args{RETURNED}
{
 \spec{pv}{float[6]}{Earth \xyzxyzd\ (AU, AU~s$^{-1}$)}
}
\notes
{
 \begin{enumerate}
  \item The date and time is TDB (TT, or even UTC, will do)
        in a Julian calendar
        which has been aligned to the ordinary Gregorian
        calendar for the interval 1900~March~1 to 2100~February~28.
        The year and day can be obtained by calling {\tt slaCalyd} or
        {\tt slaClyd}.
  \item The Earth heliocentric 6-vector is referred to the
        FK4 mean equator and equinox of date.
  \item Maximum/RMS errors 1950-2050:
        \begin{itemize}
         \item 13/5~$\times10^{-5}$~AU = 19200/7600~km in position
         \item 47/26~$\times10^{-10}$~AU~s$^{-1}$ =
               0.0070/0.0039~km~s$^{-1}$ in speed
        \end{itemize}
  \item More accurate results are obtainable with the functions {\tt slaEvp}
        and {\tt slaEpv}.
 \end{enumerate}
}
%-----------------------------------------------------------------------
\routine{slaEcleq}{ecliptic to equatorial}
{
 \action{Transformation from ecliptic longitude and latitude to
         J2000.0 \radec.}
 \call{slaEcleq( dl, db, date, \&dr, \&dd );}
}
\args{GIVEN}
{
 \spec{dl,db}{double}{ecliptic longitude and latitude
                          (mean of date, IAU~1980 theory, radians)} \\
 \spec{date}{double}{TDB as Modified Julian Date (JD$-$2400000.5)}
}
\args{RETURNED}
{
 \spec{dr,dd}{double*}{J2000.0 mean \radec\ (radians)}
}
\anote{TT, or even UTC, can be used instead of TDB.}
%-----------------------------------------------------------------------
\routine{slaEcmat}{form \radec$\rightarrow$\ecl\ matrix}
{
 \action{Form the equatorial to ecliptic rotation matrix (IAU 1980 theory).}
 \call{slaEcmat( date, rmat );}
}
\args{GIVEN}
{
 \spec{date}{double}{TDB as Modified Julian Date (JD$-$2400000.5)}
}
\args{RETURNED}
{
 \spec{rmat}{double[3][3]}{rotation matrix}
}
\notes
{
 \begin{enumerate}
  \item The distinction between the required TDB and TT
        (or even UTC) is always negligible.
  \item {\tt rmat} is matrix {\bf M} in the expression
        {\bf v}$_{ecl}$~=~{\bf M}$\cdot${\bf v}$_{equ}$.
  \item The equator, equinox and ecliptic are mean of date.
 \end{enumerate}
}
\aref{Murray, C.A., {\it Vectorial Astrometry}, section 4.3.}
%-----------------------------------------------------------------------
\routine{slaEcor}{radial velocity \& light-time corrections to Sun}
{
 \action{Component of Earth orbit velocity and heliocentric
         light time in a given direction.}
 \call{slaEcor( rm, dm, iy, id, fd, \&rv, \&tl );}
}
\args{GIVEN}
{
 \spec{rm,dm}{float}{mean \radec\ of date (radians)} \\
 \spec{iy}{int}{year} \\
 \spec{id}{int}{day in year (1 = Jan 1st)} \\
 \spec{fd}{float}{fraction of day}
}
\args{RETURNED}
{
 \spec{rv}{float*}{component of Earth orbital velocity (km~s$^{-1}$)} \\
 \spec{tl}{float*}{component of heliocentric light time (s)}
}
\notes
{
 \begin{enumerate}
  \item The date and time is TDB (TT, or even UTC, will do)
        in a Julian calendar
        which has been aligned to the ordinary Gregorian
        calendar for the interval 1900~March~1 to 2100~February~28.
        The year and day can be obtained by calling {\tt slaCalyd} or
        {\tt slaClyd}.
  \item Sign convention:
        \begin{itemize}
         \item The velocity component is positive when the
               Earth is receding from
               the given point on the sky.
         \item The light time component is positive
               when the Earth lies between the Sun and
               the given point on the sky.
        \end{itemize}
 \item Accuracy:
       \begin{itemize}
        \item The velocity component is usually within 0.004~km~s$^{-1}$
              of the correct value and is never in error by more than
              0.007~km~s$^{-1}$.
        \item The error in light time correction is about
              \tsec{0}{03} at worst,
              but is usually better than \tsec{0}{01}.
       \end{itemize}
       For applications requiring higher accuracy, see the {\tt slaEvp}
       and {\tt slaEpv} functions.
 \end{enumerate}
}
%-----------------------------------------------------------------------
\routine{slaEg50}{B1950 \radec\ to \gal}
{
 \action{Transformation from B1950.0 FK4 equatorial coordinates to
         IAU 1958 galactic coordinates.}
 \call{slaEg50( dr, dd, \&dl, \&db );}
}
\args{GIVEN}
{
 \spec{dr,dd}{double}{B1950.0 \radec\ (radians)}
}
\args{RETURNED}
{
 \spec{dl,db}{double*}{galactic longitude and latitude \gal\ (radians)}
}
\anote{The equatorial coordinates are B1950.0 FK4.  Use the
       function slaEqgal if transformation from J2000.0 FK5 coordinates
       is required.}
\aref{Blaauw {\it et al.}\ 1960, {\it Mon.Not.R.astr.Soc.},
      {\bf 121}, 123.}
%-----------------------------------------------------------------------
\routine{slaEl2ue}{conventional to universal elements}
{
 \action{Transform conventional osculating orbital elements
         into ``universal'' form.}
 \call{slaEl2ue( \vtop{
         \hbox{date, jform, epoch, orbinc, anode,}
         \hbox{perih, aorq, e, aorl, dm,}
         \hbox{u, \&jstat);}}}
}
\args{GIVEN}
{
 \spec{date}{double}{epoch (TDB MJD) of osculation (Note~3)} \\
 \spec{jform}{int}{choice of element set (1-3; Note~6)} \\
 \spec{epoch}{double}{epoch of elements ($t_0$ or $T$, TDB MJD)} \\
 \spec{orbinc}{double}{inclination ($i$, radians)} \\
 \spec{anode}{double}{longitude of the ascending node ($\Omega$, radians)} \\
 \spec{perih}{double}{longitude or argument of perihelion
                            ($\varpi$ or $\omega$,} \\
 \spec{}{}{\hspace{1.5em} radians)} \\
 \spec{aorq}{double}{mean distance or perihelion distance ($a$ or $q$, AU)} \\
 \spec{e}{double}{eccentricity ($e$)} \\
 \spec{aorl}{double}{mean anomaly or longitude
                               ($M$ or $L$, radians,} \\
 \spec{}{}{\hspace{1.5em} {\tt jform}\,=\,1,2 only)} \\
 \spec{dm}{double}{daily motion ($n$, radians, {\tt jform}\,=\,1 only)}
}
\args{RETURNED}
{
 \spec{u}{double[13]}{universal orbital elements (Note~1)} \\
 \specel{0}     {combined mass ($M+m$)} \\
 \specel{1}     {total energy of the orbit ($\alpha$)} \\
 \specel{2}     {reference (osculating) epoch ($t_0$)} \\
 \specel{3-5}   {position at reference epoch (${\rm \bf r}_0$)} \\
 \specel{6-8}   {velocity at reference epoch (${\rm \bf v}_0$)} \\
 \specel{9}     {heliocentric distance at reference epoch} \\
 \specel{10}    {${\rm \bf r}_0.{\rm \bf v}_0$} \\
 \specel{11}    {date ($t$)} \\
 \specel{12}    {universal eccentric anomaly ($\psi$) of date,
                    approx} \\ \\
 \spec{jstat}{int*}{status:} \\
 \spec{}{}{\hspace{1.95em}       0 = OK} \\
 \spec{}{}{\hspace{1.2em}  $-$1 = illegal {\tt jform}} \\
 \spec{}{}{\hspace{1.2em}   $-$2 = illegal {\tt e}} \\
 \spec{}{}{\hspace{1.2em}   $-$3 = illegal {\tt aorq}} \\
 \spec{}{}{\hspace{1.2em}   $-$4 = illegal {\tt dm}} \\
 \spec{}{}{\hspace{1.2em}   $-$5 = numerical error}
}
\notes
{
 \begin{enumerate}
  \item The ``universal'' elements are those which define the orbit for
        the purposes of the method of universal variables (see reference).
        They consist of the combined mass of the two bodies, an epoch,
        and the position and velocity vectors (arbitrary reference frame)
        at that epoch.  The parameter set used here includes also various
        quantities that can, in fact, be derived from the other
        information.  This approach is taken to avoiding unnecessary
        computation and loss of accuracy.  The supplementary quantities
        are (i)~$\alpha$, which is proportional to the total energy of the
        orbit, (ii)~the heliocentric distance at epoch,
        (iii)~the outwards component of the velocity at the given epoch,
        (iv)~an estimate of $\psi$, the ``universal eccentric anomaly'' at a
        given date and (v)~that date.
  \item The companion function is {\tt slaUe2pv}.  This takes the set of numbers
        that the present function outputs and uses them to derive the
        object's position and velocity.  A single prediction requires one
        call to the present function followed by one call to {\tt slaUe2pv};
        for convenience, the two calls are packaged as the function
        {\tt slaPlanel}.  Multiple predictions may be made by again calling the
        present function once, but then calling {\tt slaUe2pv} multiple times,
        which is faster than multiple calls to {\tt slaPlanel}.
  \item {\tt date} is the epoch of osculation.  It is in the TDB time-scale
        (TT will do) and is a Modified Julian Date (JD$-$2400000.5).
  \item The supplied orbital elements are with respect to the J2000
        ecliptic and equinox.  The position and velocity parameters
        returned in the array {\tt u}
        are with respect to the mean equator and
        equinox of epoch J2000, and are for the perihelion prior to the
        specified epoch.
  \item The universal elements returned in the array {\tt u}
        are in canonical units (solar masses, AU and canonical days).
  \item Three different element-format options are supported, as
        follows. \\
 
        ${\tt jform}=1$, suitable for the major planets:
 
        \begin{tabbing}
        xxx \= xxxxxxxx \= xx \= \kill
        \> {\tt epoch}  \> = \> epoch of elements $t_0$ (TDB MJD) \\
        \> {\tt orbinc} \> = \> inclination $i$ (radians) \\
        \> {\tt anode}  \> = \> longitude of the
                                ascending node $\Omega$ (radians) \\
        \> {\tt perih}  \> = \> longitude of perihelion
                                $\varpi$ (radians) \\
        \> {\tt aorq}   \> = \> mean distance $a$ (AU) \\
        \> {\tt e}      \> = \> eccentricity $e$ $( 0 \leq e < 1 )$ \\
        \> {\tt aorl}   \> = \> mean longitude $L$ (radians) \\
        \> {\tt dm}     \> = \> daily motion $n$ (radians)
        \end{tabbing}
 
        ${\tt jform}=2$, suitable for minor planets:
 
        \begin{tabbing}
        xxx \= xxxxxxxx \= xx \= \kill
        \> {\tt epoch}  \> = \> epoch of elements $t_0$ (TDB MJD) \\
        \> {\tt orbinc} \> = \> inclination $i$ (radians) \\
        \> {\tt anode}  \> = \> longitude of the ascending node
                                $\Omega$ (radians) \\
        \> {\tt perih}  \> = \> argument of perihelion
                                $\omega$ (radians) \\
        \> {\tt aorq}   \> = \> mean distance $a$ (AU) \\
        \> {\tt e}      \> = \> eccentricity $e$ $( 0 \leq e < 1 )$ \\
        \> {\tt aorl}   \> = \> mean anomaly $M$ (radians)
        \end{tabbing}
 
        ${\tt jform}=3$, suitable for comets:
 
        \begin{tabbing}
        xxx \= xxxxxxxx \= xx \= \kill
        \> {\tt epoch}  \> = \> epoch of perihelion $T$ (TDB MJD) \\
        \> {\tt orbinc} \> = \> inclination $i$ (radians) \\
        \> {\tt anode}  \> = \> longitude of the ascending node
                                $\Omega$ (radians) \\
        \> {\tt perih}  \> = \> argument of perihelion $\omega$ (radians) \\
        \> {\tt aorq}   \> = \> perihelion distance $q$ (AU) \\
        \> {\tt e}   \> = \> eccentricity $e$ $( 0 \leq e \leq 10 )$
        \end{tabbing}
  \item Unused elements ({\tt dm} for {\tt jform}\,=\,2,
        {\tt aorl} and {\tt dm} for {\tt jform}\,=\,3) are
        not accessed.
  \item The algorithm was originally adapted from the
        {\tt EPHSLA} Fortran program of
        D.\,H.\,P.\,Jones (private communication, 1996).  The method
        is based on Stumpff's Universal Variables.
 \end{enumerate}
}
\aref{Everhart, E. \& Pitkin, E.T.\ 1983, {\it Am.J.Phys.}\
      {\bf 51}, 712.}
%------------------------------------------------------------------------------
\routine{slaEpb}{MJD to Besselian epoch}
{
 \action{Transformation of Modified Julian Date to Besselian Epoch.}
 \call{d~=~slaEpb( date );}
}
\args{GIVEN}
{
 \spec{date}{double}{Modified Julian Date (JD$-$2400000.5)}
}
\args{RETURNED}
{
 \spec{}{double}{Besselian Epoch}
}
\aref{Lieske, J.H.\ 1979, {\it Astr.Astrophys.}\ {\bf 73}, 282.}
%-----------------------------------------------------------------------
\routine{slaEpb2d}{Besselian epoch to MJD}
{
 \action{Transformation of Besselian Epoch to Modified Julian Date.}
 \call{d~=~slaEpb2d( epb );}
}
\args{GIVEN}
{
 \spec{epb}{double}{Besselian Epoch}
}
\args{RETURNED}
{
 \spec{}{double}{Modified Julian Date (JD$-$2400000.5)}
}
\aref{Lieske, J.H.\ 1979, {\it Astr.Astrophys.}\ {\bf 73}, 282.}
%-----------------------------------------------------------------------
\routine{slaEpco}{convert epoch to B or J}
{
 \action{Convert an epoch to Besselian or Julian to match another one.}
 \call{d~=~slaEpco( k0, k, e );}
 
}
\args{GIVEN}
{
 \spec{k0}{C}{form of result: \fstring{B}\,=\,Besselian,
                              \fstring{J}\,=\,Julian} \\
 \spec{k}{C}{form of given epoch:  \fstring{B} or \fstring{J}} \\
 \spec{e}{double}{epoch}
}
\args{RETURNED}
{
 \spec{}{double}{the given epoch converted as necessary}
}
\notes
{
 \begin{enumerate}
  \item The result is always either equal to or very close to
        the given epoch {\tt e}.  The function is required only in
        applications where punctilious treatment of heterogeneous
        mixtures of star positions is necessary.
  \item {\tt k0} and {\tt k}
        are not validated.  They are interpreted as follows:
        \begin{itemize}
         \item If {\tt k0} and {\tt k} are the same,
               the result is {\tt e}.
         \item If {\tt k0} is \fstring{B} and {\tt k} isn't,
               the conversion is J to B.
         \item In all other cases, the conversion is B to J.
        \end{itemize}
 \end{enumerate}
}
%-----------------------------------------------------------------------
\routine{slaEpj}{MJD to Julian epoch}
{
 \action{Convert Modified Julian Date to Julian Epoch.}
 \call{d~=~slaEpj( date );}
}
\args{GIVEN}
{
 \spec{date}{double}{Modified Julian Date (JD$-$2400000.5)}
}
\args{RETURNED}
{
 \spec{}{double}{Julian Epoch}
}
\aref{Lieske, J.H.\ 1979, {\it Astr.Astrophys.}, {\bf 73}, 282.}
%-----------------------------------------------------------------------
\routine{slaEpj2d}{Julian epoch to MJD}
{
 \action{Convert Julian Epoch to Modified Julian Date.}
 \call{d~=~slaEpj2d( epj );}
}
\args{GIVEN}
{
 \spec{epj}{double}{Julian Epoch}
}
\args{RETURNED}
{
 \spec{}{double}{Modified Julian Date (JD$-$2400000.5)}
}
\aref{Lieske, J.H.\ 1979, {\it Astr.Astrophys.}, {\bf 73}, 282.}
%-----------------------------------------------------------------------
\routine{slaEpv}{Earth position \& velocity (high accuracy)}
{
 \action{Earth position and velocity, heliocentric and barycentric,
         with respect to the Barycentric Celestial Reference System.}
 \call{slaEpv( date, ph, vh, pb, vb );}
}
\args{GIVEN}
{
 \spec{date}{double}{TDB Modified Julian Date (Note~1)}
}
\args{RETURNED}
{
 \spec{ph}{double[3]}{heliocentric \xyz, AU} \\
 \spec{vh}{double[3]}{heliocentric \xyzd, AU~d$^{-1}$} \\
 \spec{pb}{double[3]}{barycentric \xyz, AU} \\
 \spec{vb}{double[3]}{barycentric \xyzd, AU~d$^{-1}$}
}
\notes
{
 \begin{enumerate}
  \item The date is TDB as MJD (=\,JD$-$2400000.5).  TT can be used
        instead of TDB in most applications.
  \item The vectors are with respect to the Barycentric Celestial
        Reference System (BCRS).  Positions are in AU;  velocities are in
        AU per TDB day.
  \item The function is a {\it simplified solution}\/ from the planetary
        theory VSOP2000 (X.\,Moisson, P.\,Bretagnon, 2001,
        Celes.\,Mechanics
        \& Dyn.\,Astron., {\bf 80}, 3/4, 205-213) and is an adaptation of
        original Fortran code supplied by P.\,Bretagnon (private
        communication, 2000).
  \item Comparisons over the time span 1900-2100 with this simplified
        solution and the JPL DE405 ephemeris give the following results:
        \begin{tabbing}
        xxx \= xxxxxxxxxxxxxxxxxxx \= xxxxxx \= xxxxxxx \= \kill
        \>               \> {\it RMS} \> {\it max} \\
        \> {\it Heliocentric:} \\
        \> ~~~~~position error \> 3.7 \> 11.2 \> km \\
        \> ~~~~~velocity error \> 1.4 \> ~5.0 \> mm/s \\
        \> {\it Barycentric:} \\
        \> ~~~~~position error \> 4.6 \> 13.4 \> km \\
        \> ~~~~~velocity error \> 1.4 \> ~4.9 \> mm/s
        \end{tabbing}
        The results deteriorate outside this time span.
  \item The function {\tt slaEvp} is faster but less accurate.
        The present function targets the case where high
        accuracy is more important
        than CPU time, yet the extra complication of reading a
        pre-computed ephemeris is not justified.
 \end{enumerate}
}
%-----------------------------------------------------------------------
\routine{slaEqecl}{J2000 \radec\ to \ecl}
{
 \action{Transformation from J2000.0 equatorial coordinates to
         ecliptic longitude and latitude.}
 \call{slaEqecl( dr, dd, date, \&dl, \&db );}
}
\args{GIVEN}
{
 \spec{dr,dd}{double}{J2000.0 mean \radec\ (radians)} \\
 \spec{date}{double}{TDB as Modified Julian Date (JD$-$2400000.5)}
}
\args{RETURNED}
{
 \spec{dl,db}{double*}{ecliptic longitude and latitude
                        (mean of date, IAU~1980 theory, radians)}
}
\anote{TT, or even UTC, can be used instead of TDB.}
%-----------------------------------------------------------------------
\routine{slaEqeqx}{equation of the equinoxes}
{
 \action{Equation of the equinoxes (IAU 1994).}
 \call{d~=~slaEqeqx( date );}
}
\args{GIVEN}
{
 \spec{date}{double}{TDB as Modified Julian Date (JD$-$2400000.5)}
}
\args{RETURNED}
{
 \spec{}{double}{The equation of the equinoxes (radians)}
}
\notes{
 \begin{enumerate}
  \item The equation of the equinoxes is defined here as GAST$-$GMST:
        it is added to a {\it mean}\/ sidereal time to give the
        {\it apparent}\/ sidereal time.
  \item The change from the classic ``textbook'' expression
        $\Delta\psi\,cos\,\epsilon$ occurred with IAU Resolution C7,
        Recommendation~3 (1994).  The new formulation takes into
        account cross-terms between the various precession and
        nutation quantities, amounting to about 3~milliarcsec.
        The transition from the old to the new model officially
        took place on 1997 February~27.
  \item The distinction between the required TDB and TT is
        always negligible.  Moreover, for all but the most
        critical applications UTC is adequate.
 \end{enumerate}
}
\aref{Capitaine, N.\ \& Gontier, A.-M.\ 1993,
      {\it Astron.Astrophys.}, {\bf 275}, 645-650.}
%-----------------------------------------------------------------------
\routine{slaEqgal}{J2000 \radec\ to \gal}
{
 \action{Transformation from J2000.0 FK5 equatorial coordinates to
 IAU 1958 galactic coordinates.}
 \call{slaEqgal( dr, dd, \&dl, \&db );}
}
\args{GIVEN}
{
 \spec{dr,dd}{double}{J2000.0 \radec\ (radians)}
}
\args{RETURNED}
{
 \spec{dl,db}{double*}{galactic longitude and latitude \gal\ (radians)}
}
\anote{The equatorial coordinates are J2000.0 FK5.  Use the function
       slaEg50 if transformation from B1950.0 FK4 coordinates is required.}
\aref{Blaauw {\it et al.}\ 1960, {\it Mon.Not.R.astr.Soc.},
      {\bf 121}, 123.}
%-----------------------------------------------------------------------
\routine{slaEtrms}{E-terms of aberration}
{
 \action{Compute the E-terms vector -- the part of the annual
         aberration which arises from the eccentricity of the
         Earth's orbit.}
 \call{slaEtrms( ep, ev );}
}
\args{GIVEN}
{
 \spec{ep}{double}{Besselian epoch}
}
\args{RETURNED}
{
 \spec{ev}{double[3]}{E-terms as $[\Delta x, \Delta y, \Delta z\,]$}
}
\anote{Note the use of the J2000 aberration constant (\arcsec{20}{49552}).
       This is a reflection of the fact that the E-terms embodied in
       existing star catalogues were computed from a variety of
       aberration constants.  Rather than adopting one of the old
       constants the latest value is used here.}
\refs
{
 \begin{enumerate}
  \item Smith, C.A.\ {\it et al.}\ 1989, {\it Astr.J.}\ {\bf 97}, 265.
  \item Yallop, B.D.\ {\it et al.}\ 1989, {\it Astr.J.}\ {\bf 97}, 274.
 \end{enumerate}
}
%-----------------------------------------------------------------------
\routine{slaEuler}{rotation matrix from Euler angles}
{
 \action{Form a rotation matrix from the Euler angles -- three
         successive rotations about specified Cartesian axes
         (single precision).}
 \call{slaEuler( order, phi, theta, psi, rmat );}
}
\args{GIVEN}
{
 \spec{order}{char*}{specifies about which axes the rotations occur} \\
 \spec{phi}{float}{first rotation (radians)} \\
 \spec{theta}{float}{second rotation (radians)} \\
 \spec{psi}{float}{third rotation (radians)}
}
\args{RETURNED}
{
 \spec{rmat}{float[3][3]}{rotation matrix}
}
\notes
{
 \begin{enumerate}
 \item The characters of {\tt order}
       define which axes the three successive
       rotations are about.  A typical value is \cstring{zxz},
       indicating that {\tt rmat}
       is to become the direction cosine matrix corresponding to
       rotations of the reference frame through {\tt phi}
       radians about the
       old {\it z}-axis, followed by
       {\tt theta} radians about the resulting
       {\it x}-axis,
       then {\tt psi} radians about the resulting {\it z}-axis.
 \item The axis names can be any of the following, in any order or
       combination:  \fstring{X}, \fstring{Y}, \fstring{Z},
       uppercase or lowercase, \fstring{1}, \fstring{2}, \fstring{3}.
       Thus, the
       \cstring{zxz} example given above
       could be written \cstring{ZXZ} or
       \cstring{313} (or even \cstring{ZxZ} or
       \cstring{3xZ}).
 \item {\tt order}
       is terminated by length or by the first unrecognized character.
       Fewer than three rotations are acceptable, in which case the later
       angle arguments are ignored.
       Zero rotations produces the identity {\tt rmat}.
 \item Normal axis labelling/numbering conventions apply;
       the {\it xyz} ($\equiv$\cstring{123}) triad is right-handed.
 \item A rotation is positive when the reference frame rotates
       anticlockwise as seen looking towards the origin from the
       positive region of the specified axis.
 \end{enumerate}
}
%-----------------------------------------------------------------------
\routine{slaEvp}{Earth position \& velocity}
{
 \action{Barycentric and heliocentric velocity and position of the Earth.}
 \call{slaEvp( date, deqx, dvb, dpb, dvh, dph );}
}
\args{GIVEN}
{
 \spec{date}{double}{TDB as a Modified Julian Date
                                        (JD$-$2400000.5)} \\
 \spec{deqx}{double}{Julian Epoch ({\it e.g.}\ 2000.0) of mean equator and
                equinox of the vectors returned.  If DEQX~$<0$,
                  all vectors are referred to the mean equator and
                  equinox (FK5) of date DATE.}
}
\args{RETURNED}
{
 \spec{dvb}{double[3]}{barycentric \xyzd, AU~s$^{-1}$} \\
 \spec{dpb}{double[3]}{barycentric \xyz, AU} \\
 \spec{dvh}{double[3]}{heliocentric \xyzd, AU~s$^{-1}$} \\
 \spec{dph}{double[3]}{heliocentric \xyz, AU}
}
\notes
{
 \begin{enumerate}
  \item This function is accurate enough for many purposes but faster
        and more compact than the {\tt slaEpv} function.  The maximum
        deviations from the JPL~DE96 ephemeris are as follows:
        \begin{itemize}
         \item velocity (barycentric or heliocentric): 420~mm~s$^{-1}$
         \item position (barycentric): 6900~km
         \item position (heliocentric): 1600~km
        \end{itemize}
  \item The function is adapted from the {\tt BARVEL} and {\tt BARCOR}
        subroutines of Stumpff (1980).
        Most of the changes are merely cosmetic and do not affect
        the results at all.  However, some adjustments have been
        made so as to give results that refer to the IAU 1976
        `FK5' equinox and precession, although the differences these
        changes make relative to the results from Stumpff's original
        `FK4' version are smaller than the inherent accuracy of the
        algorithm.  One minor shortcoming in the original functions
        that has {\bf not} been corrected is that slightly better
        numerical accuracy could be achieved if the various polynomial
        evaluations were to be so arranged that the smallest terms were
        computed first.
  \item The distinction between the required TDB and TT is
        always negligible.  Moreover, for all but the most
        critical applications UTC is adequate.
 \end{enumerate}
}
\aref {Stumpff, P., 1980.,  {\it Astron.Astrophys.Suppl.Ser.}\
       {\bf 41}, 1-8.}
%-----------------------------------------------------------------------
\routine{slaFitxy}{fit linear model to two \xy\ sets}
{
 \action{Fit a linear model to relate two sets of \xy\ coordinates.}
 \call{slaFitxy( itype, np, xye, xym, coeffs, \&j );}
}
\args{GIVEN}
{
 \spec{itype}{int}{type of model: 4 or 6 (Note~1)} \\
 \spec{np}{int}{number of samples (Note~2)} \\
 \spec{xye}{double[2][np]}{expected \xy\ for each sample} \\
 \spec{xym}{double[2][np]}{measured \xy\ for each sample}
}
\args{RETURNED}
{
 \spec{coeffs}{double[6]}{coefficients of model (Note~3)} \\
 \spec{j}{int*}{status:} \\
 \spec{}{}{\hspace{1.5em} 0 = OK} \\
 \spec{}{}{\hspace{0.7em} $-$1 = illegal {\tt itype}} \\
 \spec{}{}{\hspace{0.7em} $-$2 = insufficient data} \\
 \spec{}{}{\hspace{0.7em} $-$3 = singular solution}
}
\notes
{
 \begin{enumerate}
  \item {\tt itype}, which must be either 4 or 6, selects the type of model
        fitted.  Both allowed {\tt itype}
        values produce a model {\tt coeffs} which
        consists of six coefficients, namely the zero points and, for
        each of {\tt xe} and {\tt ye},
        the coefficient of {\tt xm} and {\tt ym}.
        For {\tt itype}\,=\,6,
        all six coefficients are independent, modelling squash and shear
        as well as origin, scale, and orientation.
        However, {\tt itype}\,=\,4
        selects the {\it solid body rotation}\/ option;
        the model {\tt coeffs}
        still consists of the same six coefficients, but now two of
        them are used twice (appropriately signed).  Origin, scale
        and orientation are still modelled, but not squash or shear --
        the units of $x$ and $y$ have to be the same.
  \item For {\tt nc}\,=\,4, {\tt np} must be at least 2.
        For {\tt nc}\,=\,6, {\tt np} must be at least 3.
  \item The model is returned in the array {\tt coeffs}.  Writing
        $x_m$ {\it etc.}\ for {\tt xm} {\it etc.}\ and naming the
        six elements of {\tt coeffs} $a,b,c,d,e$ \& $f$,
        the model transforms {\it measured}\/ coordinates
        $[x_{m},y_{m}\,]$ into {\it expected}\/ coordinates
        $[x_{e},y_{e}\,]$ as follows:
        \begin{verse}
         $x_{e} = a + bx_{m} + cy_{m}$ \\
         $y_{e} = d + ex_{m} + fy_{m}$
        \end{verse}
        For the {\it solid body rotation}\/ option ({\tt itype}\,=\,4), the
        magnitudes of $b$ and $f$, and of $c$ and $e$, are equal.  The
        signs of these coefficients depend on whether there is a
        sign reversal between $[x_{e},y_{e}]$ and $[x_{m},y_{m}]$;
        fits are performed
        with and without a sign reversal and the best one chosen.
  \item Error status values {\tt j}\,=\,$-$1 and $-$2 leave
        {\tt coeffs} unchanged;
        if {\tt j}\,=\,$-$3 {\tt coeffs} may have been changed.
  \item See also {\tt slaPxy}, {\tt slaInvf}, {\tt slaXy2xy}, {\tt slaDcmpf}.
 \end{enumerate}
}
%-----------------------------------------------------------------------
\routine{slaFk425}{FK4 to FK5}
{
 \action{Convert B1950.0 FK4 star data to J2000.0 FK5.
         This function converts stars from the old, Bessel-Newcomb, FK4
         system to the new, IAU~1976, FK5, Fricke system.  The precepts
         of Smith~{\it et~al.}\ (see reference~1) are followed,
         using the implementation
         by Yallop~{\it et~al.}\ (reference~2) of a matrix method
         due to Standish.
         Kinoshita's development of Andoyer's post-Newcomb precession is
         used.  The numerical constants from
         Seidelmann~{\it et~al.}\  (reference~3) are used canonically.}
 \call{slaFk425( \vtop{
         \hbox{r1950, d1950, dr1950, dd1950, p1950, v1950,}
         \hbox{\&r2000, \&d2000, \&dr2000, \&dd2000, \&p2000, \&v2000);}}}
}
\args{GIVEN}
{
 \spec{r1950}{double}{B1950.0 $\alpha$ (radians)} \\
 \spec{d1950}{double}{B1950.0 $\delta$ (radians)} \\
 \spec{dr1950}{double}{B1950.0 proper motion in $\alpha$
                              (radians per tropical year)} \\
 \spec{dd1950}{double}{B1950.0 proper motion in $\delta$
                              (radians per tropical year)} \\
 \spec{p1950}{double}{B1950.0 parallax (arcsec)} \\
 \spec{v1950}{double}{B1950.0 radial velocity (km~s$^{-1}$, +ve = moving away)}
}
\args{RETURNED}
{
 \spec{r2000}{double*}{J2000.0 $\alpha$ (radians)} \\
 \spec{d2000}{double*}{J2000.0 $\delta$ (radians)} \\
 \spec{dr2000}{double*}{J2000.0 proper motion in $\alpha$
                              (radians per Julian year)} \\
 \spec{dd2000}{double*}{J2000.0 proper motion in $\delta$
                              (radians per Julian year)} \\
 \spec{p2000}{double*}{J2000.0 parallax (arcsec)} \\
 \spec{v2000}{double*}{J2000.0 radial velocity (km~s$^{-1}$, +ve = moving away)}
}
\notes
{
 \begin{enumerate}
  \item The $\alpha$ proper motions are $\dot{\alpha}$ rather than
        $\dot{\alpha}\cos\delta$, and are per year rather than per century.
  \item Transformation from Besselian epoch 1950.0 to Julian epoch
        2000.0 only is provided for.  Transformations involving other
        epochs will require use of the appropriate precession,
        proper motion, and E-terms functions before and/or after {\tt slaFk425}
        is called.
  \item In the FK4 catalogue the proper motions of stars within
        $10^{\circ}$ of the poles do not include the {\it differential
        E-terms}\/ effect and should, strictly speaking, be handled
        in a different manner from stars outside these regions.
        However, given the general lack of homogeneity of the star
        data available for function astrometry, the difficulties of
        handling positions that may have been determined from
        astrometric fields spanning the polar and non-polar regions,
        the likelihood that the differential E-terms effect was not
        taken into account when allowing for proper motion in past
        astrometry, and the undesirability of a discontinuity in
        the algorithm, the decision has been made in this function to
        include the effect of differential E-terms on the proper
        motions for all stars, whether polar or not.  At epoch J2000,
        and measuring on the sky rather than in terms of $\Delta\alpha$,
        the errors resulting from this simplification are less than
        1~milliarcsecond in position and 1~milliarcsecond per
        century in proper motion.
  \item See also {\tt slaFk45z}, {\tt slaFk524}, {\tt slaFk54z}.
 \end{enumerate}
}
\refs
{
 \begin{enumerate}
  \item Smith, C.A.\ {\it et al.}\ 1989, {\it Astr.J.}\ {\bf 97}, 265.
  \item Yallop, B.D.\ {\it et al.}\ 1989, {\it Astr.J.}\ {\bf 97}, 274.
  \item Seidelmann, P.K.\ (ed), 1992, {\it Explanatory
        Supplement to the Astronomical Almanac,}\/ ISBN~0-935702-68-7.
 \end{enumerate}
}
%-----------------------------------------------------------------------
\routine{slaFk45z}{FK4 to FK5, no proper motion or parallax}
{
 \action{Convert B1950.0 FK4 star data to J2000.0 FK5 assuming zero
         proper motion in the FK5 system.
         This function converts stars from the old, Bessel-Newcomb, FK4
         system to the new, IAU~1976, FK5, Fricke system, in such a
         way that the FK5 proper motion is zero.  Because such a star
         has, in general, a non-zero proper motion in the FK4 system,
         the function requires the epoch at which the position in the
         FK4 system was determined.  The method is from appendix~2 of
         reference~1, but using the constants of reference~4.}
 \call{slaFk45z( r1950, d1950, bepoch, \&r2000, \&d2000 );}
}
\args{GIVEN}
{
 \spec{r1950}{double}{B1950.0 FK4 $\alpha$ at epoch BEPOCH (radians)} \\
 \spec{d1950}{double}{B1950.0 FK4 $\delta$ at epoch BEPOCH (radians)} \\
 \spec{bepoch}{double}{Besselian epoch ({\it e.g.}\ 1979.3)}
}
\args{RETURNED}
{
 \spec{r2000}{double*}{J2000.0 FK5 $\alpha$ (radians)} \\
 \spec{d2000}{double*}{J2000.0 FK5 $\delta$ (radians)}
}
\notes
{
 \begin{enumerate}
  \item The epoch {\tt bepoch} is strictly speaking Besselian, but
        if a Julian epoch is supplied the result will be
        affected only to a negligible extent.
  \item Transformation from Besselian epoch 1950.0 to Julian epoch
        2000.0 only is provided for.  Transformations involving other
        epochs will require use of the appropriate precession,
        proper motion, and E-terms functions before and/or
        after {\tt slaFk45z} is called.
  \item In the FK4 catalogue the proper motions of stars within
        $10^{\circ}$ of the poles do not include the {\it differential
        E-terms}\/ effect and should, strictly speaking, be handled
        in a different manner from stars outside these regions.
        However, given the general lack of homogeneity of the star
        data available for function astrometry, the difficulties of
        handling positions that may have been determined from
        astrometric fields spanning the polar and non-polar regions,
        the likelihood that the differential E-terms effect was not
        taken into account when allowing for proper motion in past
        astrometry, and the undesirability of a discontinuity in
        the algorithm, the decision has been made in this function to
        include the effect of differential E-terms on the proper
        motions for all stars, whether polar or not.  At epoch 2000,
        and measuring on the sky rather than in terms of $\Delta\alpha$,
        the errors resulting from this simplification are less than
        1~milliarcsecond in position and 1~milliarcsecond per
        century in proper motion.
  \item See also {\tt slaFk425}, {\tt slaFk524}, {\tt slaFk54z}.
 \end{enumerate}
}
\refs
{
 \begin{enumerate}
  \item Aoki, S., {\it et al.}\ 1983, {\it Astr.Astrophys.}, {\bf 128}, 263.
  \item Smith, C.A.\ {\it et al.}\ 1989,  {\it Astr.J.}\ {\bf 97}, 265.
  \item Yallop, B.D.\ {\it et al.}\ 1989, {\it Astr.J.}\ {\bf 97}, 274.
  \item Seidelmann, P.K.\ (ed), 1992, {\it Explanatory
        Supplement to the Astronomical Almanac,}\/ ISBN~0-935702-68-7.
 \end{enumerate}
}
%-----------------------------------------------------------------------
\routine{slaFk524}{FK5 to FK4}
{
 \action{Convert J2000.0 FK5 star data to B1950.0 FK4.
         This function converts stars from the new, IAU~1976, FK5, Fricke
         system, to the old, Bessel-Newcomb, FK4 system.
         The precepts of Smith~{\it et~al.}\ (reference~1) are followed,
         using the implementation by Yallop~{\it et~al.}\ (reference~2)
         of a matrix method due to Standish.  Kinoshita's development of
         Andoyer's post-Newcomb precession is used.  The numerical
         constants from Seidelmann~{\it et~al.}\ (reference~3) are
         used canonically.}
 \call{slaFk524( \vtop{
         \hbox{r2000, d2000, dr2000, dd2000, p2000, v2000,}
         \hbox{\&r1950, \&d1950, \&dr1950, \&dd1950, \&p1950, \&v1950);}}}
}
\args{GIVEN}
{
 \spec{r2000}{double}{J2000.0 $\alpha$ (radians)} \\
 \spec{d2000}{double}{J2000.0 $\delta$ (radians)} \\
 \spec{dr2000}{double}{J2000.0 proper motion in $\alpha$
                              (radians per Julian year)} \\
 \spec{dd2000}{double}{J2000.0 proper motion in $\delta$
                              (radians per Julian year)} \\
 \spec{p2000}{double}{J2000.0 parallax (arcsec)} \\
 \spec{v2000}{double}{J2000 radial velocity (km~s$^{-1}$, +ve = moving away)}
}
\args{RETURNED}
{
 \spec{r1950}{double*}{B1950.0 $\alpha$ (radians)} \\
 \spec{d1950}{double*}{B1950.0 $\delta$ (radians)} \\
 \spec{dr1950}{double*}{B1950.0 proper motion in $\alpha$
                              (radians per tropical year)} \\
 \spec{dd1950}{double*}{B1950.0 proper motion in $\delta$
                              (radians per tropical year)} \\
 \spec{p1950}{double*}{B1950.0 parallax (arcsec)} \\
 \spec{v1950}{double*}{radial velocity (km~s$^{-1}$, +ve = moving away)}
}
\notes
{
 \begin{enumerate}
  \item The $\alpha$ proper motions are $\dot{\alpha}$ rather than
        $\dot{\alpha}\cos\delta$, and are per year rather than per century.
  \item Note that transformation from Julian epoch 2000.0 to Besselian
        epoch 1950.0 only is provided for.  Transformations involving
        other epochs will require use of the appropriate precession,
        proper motion, and E-terms functions before and/or after
        {\tt slaFk524} is called.
  \item In the FK4 catalogue the proper motions of stars within
        $10^{\circ}$ of the poles do not include the {\it differential
        E-terms}\/ effect and should, strictly speaking, be handled
        in a different manner from stars outside these regions.
        However, given the general lack of homogeneity of the star
        data available for function astrometry, the difficulties of
        handling positions that may have been determined from
        astrometric fields spanning the polar and non-polar regions,
        the likelihood that the differential E-terms effect was not
        taken into account when allowing for proper motion in past
        astrometry, and the undesirability of a discontinuity in
        the algorithm, the decision has been made in this function to
        include the effect of differential E-terms on the proper
        motions for all stars, whether polar or not.  At epoch 2000,
        and measuring on the sky rather than in terms of $\Delta\alpha$,
        the errors resulting from this simplification are less than
        1~milliarcsecond in position and 1~milliarcsecond per
        century in proper motion.
  \item See also {\tt slaFk425}, {\tt slaFk45z}, {\tt slaFk54z}.
 \end{enumerate}
}
\refs
{
 \begin{enumerate}
  \item Smith, C.A.\ {\it et al.}\ 1989,  {\it Astr.J.}\ {\bf 97}, 265.
  \item Yallop, B.D.\ {\it et al.}\ 1989, {\it Astr.J.}\ {\bf 97}, 274.
  \item Seidelmann, P.K.\ (ed), 1992, {\it Explanatory
        Supplement to the Astronomical Almanac,}\/ ISBN~0-935702-68-7.
 \end{enumerate}
}
%-----------------------------------------------------------------------
\routine{slaFk52h}{FK5 to Hipparcos}
{
 \action{Transform an FK5 (J2000) position and proper motion
         into the system of the Hipparcos catalogue.}
 \call{slaFk52h( r5, d5, dr5, dd5, \&rh, \&dh, \&drh, \&ddh );}
}
\args{GIVEN}
{
 \spec{r5}{double}{J2000.0 FK5 $\alpha$ (radians)} \\
 \spec{d5}{double}{J2000.0 FK5 $\delta$ (radians)} \\
 \spec{dr5}{double}{J2000.0 FK5 proper motion in $\alpha$
                              (radians per Julian year)} \\
 \spec{dd5}{double}{J2000.0 FK5 proper motion in $\delta$
                              (radians per Julian year)}
}
\args{RETURNED}
{
 \spec{rh}{double*}{Hipparcos $\alpha$ (radians)} \\
 \spec{dh}{double*}{Hipparcos $\delta$ (radians)} \\
 \spec{drh}{double*}{Hipparcos proper motion in $\alpha$
                              (radians per Julian year)} \\
 \spec{ddh}{double*}{Hipparcos proper motion in $\delta$
                              (radians per Julian year)}
}
\notes
{
 \begin{enumerate}
  \item The $\alpha$ proper motions are $\dot{\alpha}$ rather than
        $\dot{\alpha}\cos\delta$, and are per year rather than per century.
  \item The FK5 to Hipparcos
        transformation consists of a pure rotation and spin;
        zonal errors in the FK5 catalogue are not taken into account.
  \item The adopted epoch J2000.0 FK5 to Hipparcos orientation and spin
        values are as follows (see reference):
 
        \vspace{2ex}
 
        ~~~~~~~~~~~~
        \begin{tabular}{|r|r|r|} \hline
        &
        \multicolumn{1}{|c}{\it orientation} &
        \multicolumn{1}{|c|}{\it ~~~spin~~~} \\ \hline
        $x$ & $-19.9$~~~~ & ~$-0.30$~~ \\
        $y$ &  $-9.1$~~~~ & ~$+0.60$~~ \\
        $z$ & $+22.9$~~~~ & ~$+0.70$~~ \\ \hline
        & {\it mas}~~~~~ & ~{\it mas/y}~ \\ \hline
        \end{tabular}
 
        \vspace{3ex}
 
        These orientation and spin components are interpreted as
        {\it axial vectors.}  An axial vector points at the pole of
        the rotation and its length is the amount of rotation in radians.
  \item See also {\tt slaFk5hz}, {\tt slaH2fk5}, {\tt slaHfk5z}.
 \end{enumerate}
}
\aref {Feissel, M.\ \& Mignard, F., 1998,  {\it Astron.Astrophys.}\
       {\bf 331}, L33-L36.}
%-----------------------------------------------------------------------
\routine{slaFk54z}{FK5 to FK4, no proper motion or parallax}
{
 \action{Convert a J2000.0 FK5 star position to B1950.0 FK4 assuming
         FK5 zero proper motion and parallax.
         This function converts star positions from the new, IAU~1976,
         FK5, Fricke system to the old, Bessel-Newcomb, FK4 system.}
 \call{slaFk54z( r2000, d2000, bepoch, \&r1950, \&d1950, \&dr1950, \&dd1950 );}
}
\args{GIVEN}
{
 \spec{r2000}{double}{J2000.0 FK5 $\alpha$ (radians)} \\
 \spec{d2000}{double}{J2000.0 FK5 $\delta$ (radians)} \\
 \spec{bepoch}{double}{Besselian epoch ({\it e.g.}\ 1950.0)}
}
\args{RETURNED}
{
 \spec{r1950}{double*}{B1950.0 FK4 $\alpha$ at epoch BEPOCH (radians)} \\
 \spec{d1950}{double*}{B1950.0 FK4 $\delta$ at epoch BEPOCH (radians)} \\
 \spec{dr1950}{double*}{B1950.0 FK4 proper motion in $\alpha$
                              (radians per tropical year)} \\
 \spec{dd1950}{double*}{B1950.0 FK4 proper motion in $\delta$
                              (radians per tropical year)}
}
\notes
{
 \begin{enumerate}
  \item The $\alpha$ proper motions are $\dot{\alpha}$ rather than
        $\dot{\alpha}\cos\delta$, and are per year rather than per century.
  \item Transformation from Julian epoch 2000.0 to Besselian epoch 1950.0
        only is provided for.  Transformations involving other epochs will
        require use of the appropriate precession functions before and
        after this function is called.
  \item Unlike in the {\tt slaFk524} function, the FK5 proper motions, the
        parallax and the radial velocity are presumed zero.
  \item It was the intention that FK5 should be a close approximation
        to an inertial frame, so that distant objects have zero proper
        motion;  such objects have (in general) non-zero proper motion
        in FK4, and this function returns those {\it fictitious proper
        motions}.
  \item The position returned by this function is in the B1950
        reference system but at Besselian epoch {\tt bepoch}.  For
        comparison with catalogues the {\tt bepoch} argument will
        frequently be 1950.0.
  \item See also {\tt slaFk425}, {\tt slaFk45z}, {\tt slaFk524}.
 \end{enumerate}
}
%-----------------------------------------------------------------------
\routine{slaFk5hz}{FK5 to Hipparcos, no proper motion}
{
 \action{Transform an FK5 (J2000) star position into the system of the
         Hipparcos catalogue, assuming zero Hipparcos proper motion.}
 \call{slaFk5hz( r5, d5, epoch, \&rh, \&dh );}
}
\args{GIVEN}
{
 \spec{r5}{double}{J2000.0 FK5 $\alpha$ (radians)} \\
 \spec{d5}{double}{J2000.0 FK5 $\delta$ (radians)} \\
 \spec{epoch}{double}{Julian epoch (TDB)}
}
\args{RETURNED}
{
 \spec{rh}{double*}{Hipparcos $\alpha$ (radians)} \\
 \spec{dh}{double*}{Hipparcos $\delta$ (radians)}
}
\notes
{
 \begin{enumerate}
  \item The $\alpha$ proper motions are $\dot{\alpha}$ rather than
        $\dot{\alpha}\cos\delta$, and are per year rather than per century.
  \item The FK5 to Hipparcos
        transformation consists of a pure rotation and spin;
        zonal errors in the FK5 catalogue are not taken into account.
  \item The adopted epoch J2000.0 FK5 to Hipparcos orientation and spin
        values are as follows (see reference):
 
        \vspace{2ex}
 
        ~~~~~~~~~~~~
        \begin{tabular}{|r|r|r|} \hline
        &
        \multicolumn{1}{|c}{\it orientation} &
        \multicolumn{1}{|c|}{\it ~~~spin~~~} \\ \hline
        $x$ & $-19.9$~~~~ & ~$-0.30$~~ \\
        $y$ &  $-9.1$~~~~ & ~$+0.60$~~ \\
        $z$ & $+22.9$~~~~ & ~$+0.70$~~ \\ \hline
        & {\it mas}~~~~~ & ~{\it mas/y}~ \\ \hline
        \end{tabular}
 
        \vspace{3ex}
 
        These orientation and spin components are interpreted as
        {\it axial vectors.}  An axial vector points at the pole of
        the rotation and its length is the amount of rotation in radians.
  \item TT, or even UTC, can be used instead of TDB.
  \item See also {\tt slaFk52h}, {\tt slaH2fk5}, {\tt slaHfk5z}.
 \end{enumerate}
}
\aref {Feissel, M.\ \& Mignard, F., 1998,  {\it Astron.Astrophys.}\
       {\bf 331}, L33-L36.}
%-----------------------------------------------------------------------
\routine{slaFlotin}{decode a single-precision real number}
{
 \action{Convert free-format input into single precision floating point.}
 \call{slaFlotin( string, \&nstrt, \&reslt, \&jflag );}
}
\args{GIVEN}
{
 \spec{string}{char*}{string containing number to be decoded} \\
 \spec{nstrt}{int*}{index to where decoding is to commence} \\
 \spec{reslt}{float*}{current value of result}
}
\args{RETURNED}
{
 \spec{nstrt}{int*}{advanced to next number} \\
 \spec{reslt}{float*}{result} \\
 \spec{jflag}{int*}{status: $-$1~=~$-$OK, 0~=~+OK, 1~=~null result, 2~=~error}
}
\notes
{
 \begin{enumerate}
 \item The reason {\tt slaFlotin} has separate ``OK'' status values
       for `+' and `$-$'
       is to enable minus zero to be detected.
       This is of crucial importance
       when decoding mixed-radix numbers.  For example, an angle
       expressed as degrees, arcminutes and arcseconds may have a
       leading minus sign but a zero degrees field.
 \item A TAB is interpreted as a space, and lowercase characters are
       interpreted as uppercase
 \item The basic format is the sequence of fields $\pm n.n x \pm n$,
       where $\pm$ is a sign
       character \fstring{+} or \fstring{-},
       $n$ means a string of decimal digits,
       `.' is a decimal point, and $x$, which indicates an exponent,
       means \fstring{D} or \fstring{E}.
       Various combinations of these fields can be
       omitted, and embedded blanks are permissible in certain places.
 \item Spaces:
       \begin{itemize}
       \item Leading spaces are ignored.
       \item Embedded spaces are allowed only after
             {\tt +},
             {\tt -},
             {\tt D} or
             {\tt E},
             and after the decimal point if the first sequence of
             digits is absent.
       \item Trailing spaces are ignored;  the first signifies
             end of decoding and subsequent ones are skipped.
       \end{itemize}
 \item Delimiters:
       \begin{itemize}
       \item Any character other than
             {\tt +},
             {\tt -},
             {\tt 0}-{\tt 9},
             period,
             {\tt D},
             {\tt E} or space may be
             used to signal the end of the number and terminate decoding.
       \item Comma is recognized by {\tt slaFlotin} as a special case; it
             is skipped, leaving the index on the next character.  See
             13, below.
       \item Decoding will in all cases terminate if end of string
             is reached.
       \end{itemize}
 \item Both signs are optional.  The default is \fstring{+}.
 \item The mantissa $n.n$ defaults to unity.
 \item The exponent $x\!\pm\!n$ defaults to \cstring{E0}.
 \item The strings of decimal digits may be of any length.
 \item The decimal point is optional for whole numbers.
 \item A {\it null result}\/ occurs when the string of characters
       being decoded does not begin with
             {\tt +},
             {\tt -},
             {\tt 0}-{\tt 9},
             period,
             {\tt D} or
             {\tt E}
       or consists entirely of spaces.  When this condition is
       detected, {\tt jflag} is set to 1 and
       {\tt reslt} is left untouched.
 \item {\tt nstrt}\,=\,1 for the first character in the string.
 \item On return from {\tt slaFlotin}, {\tt nstrt} is set ready for the next
       decode -- following trailing blanks and any comma.  If a
       delimiter other than comma is being used, {\tt nstrt} must be
       incremented before the next call to {\tt slaFlotin}, otherwise
       all subsequent calls will return a null result.
 \item Errors ({\tt jflag}\,=\,2) occur when:
       \begin{itemize}
       \item a
             {\tt +},
             {\tt -},
             {\tt D} or
             {\tt E}
       is left unsatisfied; or
       \item the decimal point is present without at least
             one decimal digit before or after it; or
       \item an exponent more than 100 has been presented.
       \end{itemize}
 \item When an error has been detected, {\tt nstrt} is left
       pointing to the character following the last
       one used before the error came to light.  This
       may be after the point at which a more sophisticated
       program could have detected the error.  For example,
       {\tt slaFlotin} does not detect that \cstring{1e999} is unacceptable
       (on a platform where this is so) until the entire number
       has been decoded.
 \item Certain highly unlikely combinations of mantissa and
       exponent can cause arithmetic faults during the
       decode, in some cases despite the fact that they
       together could be construed as a valid number.
 \item Decoding is left to right, one pass.
 \item See also {\tt slaDfltin}, {\tt slaInt2in} and {\tt slaIntin}.
 \end{enumerate}
}
%-----------------------------------------------------------------------
\routine{slaGaleq}{galactic to J2000 \radec}
{
 \action{Transformation from IAU 1958 galactic coordinates
         to J2000.0 FK5 equatorial coordinates.}
 \call{slaGaleq( dl, db, \&dr, \&dd );}
}
\args{GIVEN}
{
 \spec{dl,db}{double}{galactic longitude and latitude \gal}
}
\args{RETURNED}
{
 \spec{dr,dd}{double*}{J2000.0 \radec}
}
\notes
{
 \begin{enumerate}
  \item All arguments are in radians.
  \item The equatorial coordinates are J2000.0 FK5.  Use the function
        {\tt slaGe50} if transformation to B1950.0 FK4 coordinates is
                           required.
 \end{enumerate}
}
%-----------------------------------------------------------------------
\routine{slaGalsup}{galactic to supergalactic}
{
 \action{Transformation from IAU 1958 galactic coordinates to
         de Vaucouleurs supergalactic coordinates.}
 \call{slaGalsup( dl, db, \&dsl, \&dsb );}
}
\args{GIVEN}
{
 \spec{dl,db}{double}{galactic longitude and latitude \gal\ (radians)}
}
\args{RETURNED}
{
 \spec{dsl,dsb}{double*}{supergalactic longitude and latitude (radians)}
}
\refs
{
 \begin{enumerate}
  \item de Vaucouleurs, de Vaucouleurs, \& Corwin, 1976,
        {\it Second Reference Catalogue of Bright Galaxies},
        U.Texas, p8.
  \item Systems \& Applied Sciences Corp., documentation for the
        machine-readable version of the above catalogue,
        Contract NAS 5-26490.
 \end{enumerate}
 (These two references give different values for the galactic
 longitude of the supergalactic origin.  Both are wrong;  the
 correct value is $l^{I\!I}=137.37^\circ$.)
}
%-----------------------------------------------------------------------
\routine{slaGe50}{galactic to B1950 \radec}
{
 \action{Transformation from IAU 1958 galactic coordinates to
         B1950.0 FK4 equatorial coordinates.}
 \call{slaGe50( dl, db, \&dr, \&dd );}
}
\args{GIVEN}
{
 \spec{dl,db}{double}{galactic longitude and latitude \gal}
}
\args{RETURNED}
{
 \spec{dr,dd}{double*}{B1950.0 \radec}
}
\notes
{
 \begin{enumerate}
  \item All arguments are in radians.
  \item The equatorial coordinates are B1950.0 FK4.  Use the
        function {\tt slaGaleq} if transformation to J2000.0 FK5 coordinates
        is required.
 \end{enumerate}
}
\aref{Blaauw {\it et al.}\ 1960, {\it Mon.Not.R.astr.Soc.},
      {\bf 121}, 123.}
%-----------------------------------------------------------------------
\routine{slaGeoc}{geodetic to geocentric}
{
 \action{Convert geodetic position to geocentric.}
 \call{slaGeoc( p, h, \&r, \&z );}
}
\args{GIVEN}
{
 \spec{p}{double}{latitude (geodetic, radians)} \\
 \spec{h}{double}{height above reference spheroid (geodetic, metres)}
}
\args{RETURNED}
{
 \spec{r}{double*}{distance from Earth axis (AU)} \\
 \spec{z}{double*}{distance from plane of Earth equator (AU)}
}
\notes
{
 \begin{enumerate}
  \item Geocentric latitude can be obtained by evaluating {\tt atan2(z,r)}.
  \item IAU 1976 constants are used.
 \end{enumerate}
}
\aref{Green, R.M.\ 1985, {\it Spherical Astronomy}, Cambridge U.P., p98.}
%-----------------------------------------------------------------------
\routine{slaGmst}{UT to GMST}
{
 \action{Transformation from universal time UT1 to Greenwich mean
         sidereal time.}
 \call{d~=~slaGmst( ut1 );}
}
\args{GIVEN}
{
 \spec{ut1}{double}{universal time (strictly UT1) expressed as
                 modified Julian Date (JD$-$2400000.5)}
}
\args{RETURNED}
{
 \spec{}{double}{Greenwich mean sidereal time (radians)}
}
\notes{
 \begin{enumerate}
 \item The IAU~1982 expression
       (see page~S15 of the 1984 {\it Astronomical
       Almanac})\/ is used, but rearranged to reduce rounding errors.  This
       expression is always described as giving the GMST at $0^{\rm h}$UT;
       in fact, it gives the difference between the
       GMST and the UT, which happens to equal the GMST (modulo
       24~hours) at $0^{\rm h}$UT each day.  In {\tt slaGmst}, the
       entire UT is used directly as the argument for the
       canonical formula, and the fractional part of the UT is
       added separately;  note that the factor $1.0027379\cdots$ does
       not appear.
 \item See also the function {\tt slaGmsta}, which
       delivers better numerical
       precision by accepting the UT date and time as separate arguments.
 \end{enumerate}
}
%-----------------------------------------------------------------------
\routine{slaGmsta}{UT to GMST (extra precision)}
{
 \action{Transformation from universal time UT1 to Greenwich mean
         sidereal time, with rounding errors minimized.}
 \call{d~=~slaGmsta( date, ut1 );}
}
\args{GIVEN}
{
 \spec{date}{double}{UT1 date as Modified Julian Date (integer part
                of JD$-$2400000.5)} \\
 \spec{ut1}{double}{UT1 time (fraction of a day)}
}
\args{RETURNED}
{
 \spec{}{double}{Greenwich mean sidereal time (radians)}
}
\notes{
 \begin{enumerate}
 \item The algorithm is derived from the IAU 1982 expression
       (see page~S15 of the 1984 Astronomical Almanac).
       \item There is no restriction on how the UT
       is apportioned between the
       {\tt date} and {\tt ut1} arguments.
       Either of the two arguments could, for
       example, be zero and the entire date\,+\,time supplied in the other.
       However, the function is designed to deliver maximum accuracy when
       the {\tt date} argument is a whole number
       and the {\tt ut1} argument
       lies in the range $[\,0,\,1\,]$, or {\it vice versa}.
 \item See also the function {\tt slaGmst}, which accepts the UT1 as a single
       argument.  Compared with {\tt slaGmst}, the extra numerical precision
       delivered by the present function is unlikely to be important in
       an absolute sense, but may be useful when critically comparing
       algorithms and in applications where two sidereal times close
       together are differenced.
 \end{enumerate}
}
%-----------------------------------------------------------------------
\routine{slaH2e}{\azel\ to \hadec}
{
 \action{Horizon to equatorial coordinates
         (single precision).}
 \call{slaH2e( az, el, phi, \&ha, \&dec );}
}
\args{GIVEN}
{
 \spec{az}{float}{azimuth (radians)} \\
 \spec{el}{float}{elevation (radians)} \\
 \spec{phi}{float}{latitude (radians)}
}
\args{RETURNED}
{
 \spec{ha}{float*}{hour angle (radians)} \\
 \spec{dec}{float*}{declination (radians)}
}
\notes
{
 \begin{enumerate}
  \item The sign convention for azimuth is north zero, east $+\pi/2$.
  \item Hour angle is returned in the range $\pm\pi$.
        Declination is returned
        in the range $\pm\pi$.
  \item The latitude is (in principle) geodetic.  In critical
        applications, corrections for polar motion should be applied
        (see {\tt slaPolmo}).
  \item In some applications it will be important to specify the
        correct type of elevation in order to produce the required
        type of \hadec.  In particular, it may be important to
        distinguish between the elevation as affected by refraction,
        which will yield the {\it observed} \hadec, and the elevation
        {\it in vacuo}, which will yield the {\it topocentric}\/
        \hadec.  If the
        effects of diurnal aberration can be neglected, the
        topocentric \hadec\ may be used as an approximation to the
        {\it apparent}\/ \hadec.
  \item No range checking of arguments is carried out.
  \item In applications which involve many such calculations, rather
        than calling the present function it will be more efficient to
        use inline code, having previously computed fixed terms such
        as sine and cosine of latitude.
 \end{enumerate}
}
%-----------------------------------------------------------------------
\routine{slaH2fk5}{Hipparcos to FK5}
{
 \action{Transform a Hipparcos star position and proper motion
         into the FK5 (J2000) system.}
 \call{slaH2fk5( rh, dh, drh, ddh, \&r5, \&d5, \&dr5, \&dd5 );}
}
\args{GIVEN}
{
 \spec{rh}{double}{Hipparcos $\alpha$ (radians)} \\
 \spec{dh}{double}{Hipparcos $\delta$ (radians)} \\
 \spec{drh}{double}{Hipparcos proper motion in $\alpha$
                              (radians per Julian year)} \\
 \spec{ddh}{double}{Hipparcos proper motion in $\delta$
                              (radians per Julian year)}
}
\args{RETURNED}
{
 \spec{r5}{double*}{J2000.0 FK5 $\alpha$ (radians)} \\
 \spec{d5}{double*}{J2000.0 FK5 $\delta$ (radians)} \\
 \spec{dr5}{double*}{J2000.0 FK5 proper motion in $\alpha$
                              (radians per Julian year)} \\
 \spec{dd5}{double*}{FK5 J2000.0 proper motion in $\delta$
                              (radians per Julian year)}
}
\notes
{
 \begin{enumerate}
  \item The $\alpha$ proper motions are $\dot{\alpha}$ rather than
        $\dot{\alpha}\cos\delta$, and are per year rather than per century.
  \item The FK5 to Hipparcos
        transformation consists of a pure rotation and spin;
        zonal errors in the FK5 catalogue are not taken into account.
  \item The adopted epoch J2000.0 FK5 to Hipparcos orientation and spin
        values are as follows (see reference):
 
        \vspace{2ex}
 
        ~~~~~~~~~~~~
        \begin{tabular}{|r|r|r|} \hline
        &
        \multicolumn{1}{|c}{\it orientation} &
        \multicolumn{1}{|c|}{\it ~~~spin~~~} \\ \hline
        $x$ & $-19.9$~~~~ & ~$-0.30$~~ \\
        $y$ &  $-9.1$~~~~ & ~$+0.60$~~ \\
        $z$ & $+22.9$~~~~ & ~$+0.70$~~ \\ \hline
        & {\it mas}~~~~~ & ~{\it mas/y}~ \\ \hline
        \end{tabular}
 
        \vspace{3ex}
 
        These orientation and spin components are interpreted as
        {\it axial vectors.}  An axial vector points at the pole of
        the rotation and its length is the amount of rotation in radians.
  \item See also {\tt slaFk52h}, {\tt slaFk5hz}, {\tt slaHfk5z}.
 \end{enumerate}
}
\aref {Feissel, M.\ \& Mignard, F., 1998,  {\it Astron.Astrophys.}\
       {\bf 331}, L33-L36.}
%-----------------------------------------------------------------------
\routine{slaHfk5z}{Hipparcos to FK5, no proper motion}
{
 \action{Transform a Hipparcos star position
         into the FK5 (J2000) system assuming zero Hipparcos proper motion.}
 \call{slaHfk5z( rh, dh, epoch, \&r5, \&d5, \&dr5, \&dd5 );}
}
\args{GIVEN}
{
 \spec{rh}{double}{Hipparcos $\alpha$ (radians)} \\
 \spec{dh}{double}{Hipparcos $\delta$ (radians)} \\
 \spec{epoch}{double}{Julian epoch (TDB)}
}
\args{RETURNED}
{
 \spec{r5}{double*}{J2000.0 FK5 $\alpha$ (radians)} \\
 \spec{d5}{double*}{J2000.0 FK5 $\delta$ (radians)} \\
 \spec{dr5}{double*}{J2000.0 FK5 proper motion in $\alpha$
                              (radians per Julian year)} \\
 \spec{dd5}{double*}{FK5 J2000.0 proper motion in $\delta$
                              (radians per Julian year)}
}
\notes
{
 \begin{enumerate}
  \item The $\alpha$ proper motions are $\dot{\alpha}$ rather than
        $\dot{\alpha}\cos\delta$, and are per year rather than per century.
  \item The FK5 to Hipparcos
        transformation consists of a pure rotation and spin;
        zonal errors in the FK5 catalogue are not taken into account.
  \item The adopted epoch J2000.0 FK5 to Hipparcos orientation and spin
        values are as follows (see reference):
 
        \vspace{2ex}
 
        ~~~~~~~~~~~~
        \begin{tabular}{|r|r|r|} \hline
        &
        \multicolumn{1}{|c}{\it orientation} &
        \multicolumn{1}{|c|}{\it ~~~spin~~~} \\ \hline
        $x$ & $-19.9$~~~~ & ~$-0.30$~~ \\
        $y$ &  $-9.1$~~~~ & ~$+0.60$~~ \\
        $z$ & $+22.9$~~~~ & ~$+0.70$~~ \\ \hline
        & {\it mas}~~~~~ & ~{\it mas/y}~ \\ \hline
        \end{tabular}
 
        \vspace{3ex}
 
        These orientation and spin components are interpreted as
        {\it axial vectors.}  An axial vector points at the pole of
        the rotation and its length is the amount of rotation in radians.
  \item It was the intention that Hipparcos should be a close
        approximation to an inertial frame, so that distant objects
        have zero proper motion;  such objects have (in general)
        non-zero proper motion in FK5, and this function returns those
        {\it fictitious proper motions.}
  \item The position returned by this function is in the FK5 J2000
        reference system but at Julian epoch {\tt epoch}.
  \item TT, or even UTC, can be used instead of TDB.
  \item See also {\tt slaFk52h}, {\tt slaFk5hz}, {\tt slaH2fk5}.
 \end{enumerate}
}
\aref {Feissel, M.\ \& Mignard, F., 1998,  {\it Astron.Astrophys.}\
       {\bf 331}, L33-L36.}
%-----------------------------------------------------------------------
\routine{slaImxv}{apply 3D reverse rotation}
{
 \action{Multiply a 3-vector by the inverse of a rotation
         matrix (single precision).}
 \call{slaImxv( rm, va, vb );}
}
\args{GIVEN}
{
 \spec{rm}{float[3][3]}{rotation matrix} \\
 \spec{va}{float[3]}{vector to be rotated}
}
\args{RETURNED}
{
 \spec{vb}{float[3]}{result vector}
}
\notes
{
 \begin{enumerate}
  \item This function performs the operation:
        \begin{verse}
           ${\bf b} = {\bf M}^T\times{\bf a}$
        \end{verse}
        where {\bf a} and {\bf b} are the 3-vectors
        {\tt va} and {\tt vb}
        respectively, and {\bf M} is the $3\times3$ matrix {\tt rm}.
  \item The main function of this function is apply a
        rotation;  under these circumstances, {\bf M} is a
        {\it proper real orthogonal}\/ matrix.
  \item {\tt va} and {\tt vb} may be the same array.
 \end{enumerate}
}
%-----------------------------------------------------------------------
\routine{slaInt2in}{decode an integer number}
{
 \action{Convert free-format input into an integer.}
 \call{slaInt2in( string, \&nstrt, \&ireslt, \&jflag );}
}
\args{GIVEN}
{
 \spec{string}{char*}{string containing number to be decoded} \\
 \spec{nstrt}{int*}{index to where decoding is to commence} \\
 \spec{ireslt}{int*}{current value of result}
}
\args{RETURNED}
{
 \spec{nstrt}{int*}{advanced to next number} \\
 \spec{ireslt}{int*}{result} \\
 \spec{jflag}{int*}{status: $-$1 = $-$OK, 0~=~+OK,
                            1~=~null result, 2~=~error}
}
\notes
{
 \begin{enumerate}
 \item The reason {\tt slaInt2in} has separate ``OK'' status values
       for `+' and `$-$'
       is to enable minus zero to be detected.
       This is of crucial importance
       when decoding mixed-radix numbers.  For example, an angle
       expressed as degrees, arcminutes and arcseconds may have a
       leading minus sign but a zero degrees field.
 \item A TAB is interpreted as a space.
 \item The basic format is the sequence of fields $\pm n$,
       where $\pm$ is a sign
       character \fstring{+} or \fstring{-}, and $n$ means a string of decimal digits.
 \item Spaces:
       \begin{itemize}
       \item Leading spaces are ignored.
       \item Spaces between the sign and the number are allowed.
       \item Trailing spaces are ignored;  the first signifies
             end of decoding and subsequent ones are skipped.
       \end{itemize}
 \item Delimiters:
       \begin{itemize}
       \item Any character other than
             {\tt +},
             {\tt -},
             {\tt 0}-{\tt 9} or
             space may be
             used to signal the end of the number and terminate decoding.
       \item Comma is recognized by {\tt slaInt2in} as a special case; it
             is skipped, leaving the index on the next character.  See
             9, below.
       \item Decoding will in all cases terminate if end of string
             is reached.
       \end{itemize}
 \item The sign is optional.  The default is +.
 \item A {\it null result}\/ occurs when the string of characters
       being decoded does not begin with
       {\tt +},
       {\tt -} or
       {\tt 0}-{\tt 9}, or
       consists entirely of spaces.  When this condition is
       detected, {\tt jflag} is set to 1 and
       {\tt ireslt} is left untouched.
 \item {\tt nstrt}\,=\,1 for the first character in the string.
 \item On return from {\tt slaInt2in}, {\tt nstrt} is set ready for the next
       decode -- following trailing blanks and any comma.  If a
       delimiter other than comma is being used, {\tt nstrt} must be
       incremented before the next call to {\tt slaInt2in}, otherwise
       all subsequent calls will return a null result.
 \item Errors ({\tt jflag}\,=\,2) occur when:
       \begin{itemize}
       \item there is a \fstring{+} or \fstring{-} but no number; or
       \item the number is greater than $2^{31}-1$.
       \end{itemize}
 \item When an error has been detected, {\tt nstrt} is left
       pointing to the character following the last
       one used before the error came to light.
 \item See also {\tt slaIntin}, {\tt slaFlotin} and {\tt slaDfltin}.
 \end{enumerate}
}
%-----------------------------------------------------------------------
\routine{slaIntin}{decode a long integer number}
{
 \action{Convert free-format input into an integer.}
 \call{slaIntin( string, \&nstrt, \&lreslt, \&jflag );}
}
\args{GIVEN}
{
 \spec{string}{char*}{string containing number to be decoded} \\
 \spec{nstrt}{int*}{index to where decoding is to commence} \\
 \spec{lreslt}{long*}{current value of result}
}
\args{RETURNED}
{
 \spec{nstrt}{int*}{advanced to next number} \\
 \spec{lreslt}{long*}{result} \\
 \spec{jflag}{int*}{status: $-$1 = $-$OK, 0~=~+OK,
                            1~=~null result, 2~=~error}
}
\notes
{
 \begin{enumerate}
 \item The reason {\tt slaIntin} has separate ``OK'' status values
       for `+' and `$-$'
       is to enable minus zero to be detected.
       This is of crucial importance
       when decoding mixed-radix numbers.  For example, an angle
       expressed as degrees, arcminutes and arcseconds may have a
       leading minus sign but a zero degrees field.
 \item A TAB is interpreted as a space.
 \item The basic format is the sequence of fields $\pm n$,
       where $\pm$ is a sign
       character \fstring{+} or \fstring{-}, and $n$ means a string of decimal digits.
 \item Spaces:
       \begin{itemize}
       \item Leading spaces are ignored.
       \item Spaces between the sign and the number are allowed.
       \item Trailing spaces are ignored;  the first signifies
             end of decoding and subsequent ones are skipped.
       \end{itemize}
 \item Delimiters:
       \begin{itemize}
       \item Any character other than
             {\tt +},
             {\tt -},
             {\tt 0}-{\tt 9} or space may be
             used to signal the end of the number and terminate decoding.
       \item Comma is recognized by {\tt slaIntin} as a special case; it
             is skipped, leaving the index on the next character.  See
             9, below.
       \item Decoding will in all cases terminate if end of string
             is reached.
       \end{itemize}
 \item The sign is optional.  The default is \fstring{+}.
 \item A {\it null result}\/ occurs when the string of characters
       being decoded does not begin with
       {\tt +},
       {\tt -} or
       {\tt 0}-{\tt 9}, or
       consists entirely of spaces.  When this condition is
       detected, {\tt jflag} is set to 1 and {\tt lreslt} is left untouched.
 \item {\tt nstrt}\,=\,1 for the first character in the string.
 \item On return from {\tt slaIntin}, {\tt nstrt} is set ready for the next
       decode -- following trailing blanks and any comma.  If a
       delimiter other than comma is being used, {\tt nstrt} must be
       incremented before the next call to {\tt slaIntin}, otherwise
       all subsequent calls will return a null result.
 \item Errors ({\tt jflag}\,=\,2) occur when:
       \begin{itemize}
       \item there is a \fstring{+} or \fstring{-} but no number; or
       \item the number is greater than $2^{31}-1$.
       \end{itemize}
 \item When an error has been detected, {\tt nstrt} is left
       pointing to the character following the last
       one used before the error came to light.
 \item See also {\tt slaInt2in}, {\tt slaFlotin} and {\tt slaDfltin}.
 \end{enumerate}
}
%-----------------------------------------------------------------------
\routine{slaInvf}{invert linear model}
{
 \action{Invert a linear model of the type produced by the
         slaFitxy function.}
 \call{slaInvf( fwds, bkwds, \&j );}
}
\args{GIVEN}
{
 \spec{fwds}{double[6]}{model coefficients}
}
\args{RETURNED}
{
 \spec{bkwds}{double[6]}{inverse model} \\
 \spec{j}{int*}{status:  0 = OK, $-$1 = no inverse}
}
\notes
{
 \begin{enumerate}
  \item The models relate two sets of \xy\ coordinates as follows.
        Naming the six elements of {\tt fwds} $a,b,c,d,e$ \& $f$,
        where two sets of coordinates $[x_{1},y_{1}]$ and
        $[x_{2},y_{2}\,]$ are related thus:
        \begin{verse}
         $x_{2} = a + bx_{1} + cy_{1}$ \\
         $y_{2} = d + ex_{1} + fy_{1}$
        \end{verse}
        The present function generates a new set of coefficients
        $p,q,r,s,t$ \& $u$ (the array {\tt bkwds}) such that:
        \begin{verse}
         $x_{1} = p + qx_{2} + ry_{2}$ \\
         $y_{1} = s + tx_{2} + uy_{2}$
        \end{verse}
  \item Two successive calls to this function will deliver a set
        of coefficients equal to the starting values.
  \item {\tt fwds} and {\tt bkwds} can be the same array.
  \item See also {\tt slaFitxy}, {\tt slaPxy}, {\tt slaXy2xy}, {\tt slaDcmpf}.
 \end{enumerate}
}
%-----------------------------------------------------------------------
\routine{slaKbj}{select epoch prefix}
{
 \action{Select epoch prefix `B' or `J'.}
 \call{slaKbj( jb, e, \&k, \&j );}
}
\args{GIVEN}
{
 \spec{jb}{int}{{\tt slaDbjin} prefix status:
                 0\,=\,none, 1\,=\,\fstring{B}, 2\,=\,\fstring{J}} \\
 \spec{e}{double}{epoch -- Besselian or Julian}
}
\args{RETURNED}
{
 \spec{k}{char*}{\fstring{B} or \fstring{J}} \\
 \spec{j}{int*}{status:  0\,=\,OK}
}
\anote{The function is mainly intended for use in conjunction with the
       {\tt slaDbjin} function.  If the value of {\tt jb}
       indicates that an explicit
       \fstring{B} or \fstring{J}
       prefix was detected by {\tt slaDbjin}, a
       \fstring{B} or \fstring{J}
       is returned to match.  If {\tt jb}
       indicates that no explicit \fstring{B} or \fstring{J}
       was supplied, the choice is made on the basis of the epoch
       itself;  \fstring{B} is assumed for ${\tt e} < 1984$,
       otherwise \fstring{J}.}
%-----------------------------------------------------------------------
\routine{slaM2av}{rotation matrix to axial vector}
{
 \action{From a rotation matrix, determine the corresponding axial vector
         (single precision).}
 \call{slaM2av( rmat, axvec );}
}
\args{GIVEN}
{
 \spec{rmat}{float[3][3]}{rotation matrix}
}
\args{RETURNED}
{
 \spec{axvec}{float[3]}{axial vector (radians)}
}
\notes
{
 \begin{enumerate}
 \item A rotation matrix describes a rotation about some arbitrary axis,
       called the Euler axis.  The {\it axial vector} returned by
       this function has the same direction as the Euler axis, and its
       magnitude is the amount of rotation in radians.
 \item The magnitude and direction of the axial vector can be separated
       by means of the function {\tt slaVn}.
 \item The reference frame rotates clockwise as seen looking along
       the axial vector from the origin.
 \item If {\tt rmat} is null, so is the result.
 \end{enumerate}
}
%-----------------------------------------------------------------------
\routine{slaMap}{mean to apparent}
{
 \action{Transform star \radec\ from mean place to geocentric apparent.
         The reference systems and time-scales used are post IAU~1976.}
 \call{slaMap( rm, dm, pr, pd, px, rv, eq, date, \&ra, \&da );}
}
\args{GIVEN}
{
 \spec{rm,dm}{double}{mean \radec\ (radians)} \\
 \spec{pr,pd}{double}{proper motions:  \radec\ changes per Julian year} \\
 \spec{px}{double}{parallax (arcsec)} \\
 \spec{rv}{double}{radial velocity (km~s$^{-1}$, +ve if receding)} \\
 \spec{eq}{double}{epoch and equinox of star data (Julian)} \\
 \spec{date}{double}{TDB for apparent place (JD$-$2400000.5)}
}
\args{RETURNED}
{
 \spec{ra,da}{double*}{apparent \radec\ (radians)}
}
\notes
{
 \begin{enumerate}
  \item {\tt eq} is the Julian epoch specifying both the reference
        system and the epoch of the position -- usually 2000.
        For positions where the epoch and equinox are
        different, use the function {\tt slaPm} to apply proper
        motion corrections before using this function.
  \item The distinction between the required TDB and TT is
        always negligible.  Moreover, for all but the most
        critical applications UTC is adequate.
  \item The $\alpha$ proper motions are $\dot{\alpha}$ rather than
        $\dot{\alpha}\cos\delta$, and are per year rather than per century.
  \item This function may be wasteful for some applications
        because it recomputes the Earth position/velocity and
        the precession-nutation matrix each time, and because
        it allows for parallax and proper motion.  Where
        multiple transformations are to be carried out for one
        epoch, a faster method is to call the {\tt slaMappa} function
        once and then either the {\tt slaMapqk} function (which includes
        parallax and proper motion) or {\tt slaMapqkz} (which assumes
        zero parallax and FK5 proper motion).
  \item The accuracy, starting from ICRS star data,
        is limited to about 1~mas by the
        precession-nutation model used, SF2001
        (IAU~1976 precession, Shirai \& Fukushima 2001 forced nutation and
        precession corrections).  A different precession-nutation model
        can be introduced by using {\tt slaMappa} and {\tt slaMapqk} (see
        the previous note) and replacing the precession-nutation
        matrix into the parameter array directly.
  \item The accuracy is further limited by the function {\tt slaEvp}, called
        by {\tt slaMappa}, which computes the Earth position and
        velocity using the methods of Stumpff.  The maximum
        error is about 0.3~milliarcsecond.
 \end{enumerate}
}
\refs
{
 \begin{enumerate}
  \item 1984 {\it Astronomical Almanac}, pp B39-B41.
  \item Lederle \& Schwan 1984, {\it Astr.Astrophys.}\ {\bf 134}, 1-6.
  \item Shirai, T. \& Fukushima, T.\ 2001, {\it Astron.J.}, {\bf 121},
        3270-3283.
 \end{enumerate}
}
%-----------------------------------------------------------------------
\routine{slaMappa}{mean to apparent parameters}
{
 \action{Compute star-independent parameters in preparation for
         transformations between mean place and geocentric apparent place.
         The parameters produced by this function are required in the
         parallax, light deflection, aberration, and precession-nutation
         parts of the mean/apparent transformations.
         The reference systems and time-scales used are post IAU~1976.}
 \call{slaMappa( eq, date, amprms );}
}
\args{GIVEN}
{
 \spec{eq}{double}{epoch of mean equinox to be used (Julian)} \\
 \spec{date}{double}{TDB (JD$-$2400000.5)}
}
\args{RETURNED}
{
 \spec{amprms}{double[21]}{star-independent mean-to-apparent parameters:} \\
 \specel{0}     {time interval for proper motion (Julian years)} \\
 \specel{1-3}   {barycentric position of the Earth (AU)} \\
 \specel{4-6}   {heliocentric direction of the Earth (unit vector)} \\
 \specel{7}     {(gravitational radius of
                  Sun)$\times 2 / $(Sun-Earth distance)} \\
 \specel{8-10}  {{\bf v}: barycentric Earth velocity in units of c} \\
 \specel{11}    {$(1-\left|\mbox{\bf v}\right|^2)^{1/2}$} \\
 \specel{12-20} {precession-nutation $3\times3$ matrix}
}
\notes
{
 \begin{enumerate}
  \item For {\tt date}, the distinction between the required TDB and TT
        is always negligible.  Moreover, for all but the most
        critical applications UTC is adequate.
  \item The vectors {\tt amprms[1-3]} and
        {\tt amprms[4-6]} are (in essence) referred to
        the mean equinox and equator of epoch {\tt eq}.  For
        {\tt eq}\,=\,2000.0, they are referred to the ICRS.
  \item The parameters produced by this function are used by
        {\tt slaMapqk}, {\tt slaMapqkz} and {\tt slaAmpqk}.
  \item The accuracy, starting from ICRS star data,
        is limited to about 1~mas by the precession-nutation
        model used, SF2001
        (IAU~1976 precession, Shirai \& Fukushima 2001 forced nutation and
        precession corrections).  A different precession-nutation model
        can be introduced by first calling the present function
        and then replacing the precession-nutation
        matrix in {\tt amprms[12-20]} directly.
  \item A further limit to the accuracy of functions using the
        parameter array {\tt amprms} is
        imposed by the function {\tt slaEvp}, used here to compute the
        Earth position and velocity by the methods of Stumpff.
        The maximum error in the resulting aberration corrections is
        about 0.3 milliarcsecond.
 \end{enumerate}
}
\refs
{
 \begin{enumerate}
  \item 1984 {\it Astronomical Almanac}, pp B39-B41.
  \item Lederle \& Schwan 1984, {\it Astr.Astrophys.}\ {\bf 134}, 1-6.
  \item Shirai, T. \& Fukushima, T.\ 2001, {\it Astron.J.}, {\bf 121},
        3270-3283.
 \end{enumerate}
}
%-----------------------------------------------------------------------
\routine{slaMapqk}{quick mean to apparent}
{
 \action{Quick mean to apparent place:  transform a star \radec\ from
         mean place to geocentric apparent place, given the
         star-independent parameters.  The reference systems and
         time-scales used are post IAU~1976.}
 \call{slaMapqk( rm, dm, pr, pd, px, rv, amprms, \&ra, \&da );}
}
\args{GIVEN}
{
 \spec{rm,dm}{double}{mean \radec\ (radians)} \\
 \spec{pr,pd}{double}{proper motions:  \radec\ changes per Julian year} \\
 \spec{px}{double}{parallax (arcsec)} \\
 \spec{rv}{double}{radial velocity (km~s$^{-1}$, +ve if receding)} \\
 \spec{amprms}{double[21]}{star-independent mean-to-apparent parameters:} \\
 \specel{1}     {time interval for proper motion (Julian years)} \\
 \specel{1-3}   {barycentric position of the Earth (AU)} \\
 \specel{4-6}   {heliocentric direction of the Earth (unit vector)} \\
 \specel{7}     {(gravitational radius of
                  Sun)$\times 2 / $(Sun-Earth distance)} \\
 \specel{8-10}  {{\bf v}: barycentric Earth velocity in units of c} \\
 \specel{11}    {$(1-\left|\mbox{\bf v}\right|^2)^{1/2}$} \\
 \specel{12-20} {precession-nutation $3\times3$ matrix}
}
\args{RETURNED}
{
 \spec{ra,da}{double*}{apparent \radec\ (radians)}
}
\notes
{
 \begin{enumerate}
  \item Use of this function is appropriate when efficiency is important
        and where many star positions, all referred to the same equator
        and equinox, are to be transformed for one epoch.  The
        star-independent parameters can be obtained by calling the
        {\tt slaMappa} function.
  \item If the parallax and proper motions are zero the {\tt slaMapqkz}
        function can be used instead.
  \item The vectors {\tt amprms[1-3]} and
        {\tt amprms[4-6]} are (in essence) referred to
        the mean equinox and equator of epoch {\tt eq}.  For
        {\tt eq}\,=\,2000.0, they are referred to the ICRS.
  \item Strictly speaking, the function is not valid for solar-system
        sources, though the error will usually be extremely small.
        However, to prevent gross errors in the case where the
        position of the Sun is specified, the gravitational
        deflection term is restrained within about \arcseci{920} of the
        centre of the Sun's disc.  The term has a maximum value of
        about \arcsec{1}{85} at this radius, and decreases to zero as
        the centre of the disc is approached.
 \end{enumerate}
}
\refs
{
 \begin{enumerate}
  \item 1984 {\it Astronomical Almanac}, pp B39-B41.
  \item Lederle \& Schwan 1984, {\it Astr.Astrophys.}\ {\bf 134}, 1-6.
 \end{enumerate}
}
%-----------------------------------------------------------------------
\routine{slaMapqkz}{quick mean to apparent, no P.M.~{\it etc.}}
{
 \action{Quick mean to apparent place:  transform a star \radec\ from
         mean place to geocentric apparent place, given the
         star-independent parameters, and assuming zero parallax
         and FK5 proper motion.
         The reference systems and time-scales used are post IAU~1976.}
 \call{slaMapqkz( rm, dm, amprms, \&ra, \&da );}
}
\args{GIVEN}
{
 \spec{rm,dm}{double}{mean \radec\ (radians)} \\
 \spec{amprms}{double[21]}{star-independent mean-to-apparent parameters:} \\
 \specel{0}     {time interval for proper motion (Julian years)} \\
 \specel{1-3}   {barycentric position of the Earth (AU)} \\
 \specel{4-6}   {heliocentric direction of the Earth (unit vector)} \\
 \specel{7}     {(gravitational radius of
                  Sun)$\times 2 / $(Sun-Earth distance)} \\
 \specel{8-10}  {{\bf v}: barycentric Earth velocity in units of c} \\
 \specel{11}    {$(1-\left|\mbox{\bf v}\right|^2)^{1/2}$} \\
 \specel{12-20} {precession-nutation $3\times3$ matrix}
}
\args{RETURNED}
{
 \spec{ra,da}{double*}{apparent \radec\ (radians)}
}
\notes
{
 \begin{enumerate}
  \item Use of this function is appropriate when efficiency is important
        and where many star positions, all with parallax and proper
        motion either zero or already allowed for, and all referred to
        the same equator and equinox, are to be transformed for one
        epoch.  The star-independent parameters can be obtained by
        calling the {\tt slaMappa} function.
  \item The corresponding function for the case of non-zero parallax
        and ICRS proper motion is {\tt slaMapqk}.
  \item The vectors {\tt amprms[1-3]} and
        {\tt amprms[4-6]} are (in essence) referred to
        the mean equinox and equator of epoch {\tt eq}.  For
        {\tt eq}\,=\,2000.0, they are referred to the ICRS.
  \item Strictly speaking, the function is not valid for solar-system
        sources, though the error will usually be extremely small.
        However, to prevent gross errors in the case where the
        position of the Sun is specified, the gravitational
        deflection term is restrained within about \arcseci{920} of the
        centre of the Sun's disc.  The term has a maximum value of
        about \arcsec{1}{85} at this radius, and decreases to zero as
        the centre of the disc is approached.
 \end{enumerate}
}
\refs
{
 \begin{enumerate}
  \item 1984 {\it Astronomical Almanac}, pp B39-B41.
  \item Lederle \& Schwan 1984, {\it Astr.Astrophys.}\ {\bf 134}, 1-6.
 \end{enumerate}
}
%-----------------------------------------------------------------------
\routine{slaMoon}{approximate Moon position and velocity}
{
 \action{Approximate geocentric position and velocity of the Moon
         (single precision).}
 \call{slaMoon( iy, id, fd, pv );}
}
\args{GIVEN}
{
 \spec{iy}{int}{year} \\
 \spec{id}{int}{day in year (1 = Jan 1st)} \\
 \spec{fd}{R }{fraction of day}
}
\args{RETURNED}
{
 \spec{pv}{float[6]}{Moon \xyzxyzd, mean equator and equinox of
                 date (AU, AU~s$^{-1}$)}
}
\notes
{
 \begin{enumerate}
  \item The date and time is TDB (TT will do, but not UTC)
        in a Julian calendar
        which has been aligned to the ordinary Gregorian
        calendar for the interval 1900~March~1 to 2100~February~28.
        The year and day can be obtained by calling {\tt slaCalyd} or
        {\tt slaClyd}.
  \item The position is accurate to better than 0.5~arcminute
        in direction and 1000~km in distance.  The velocity
        is accurate to better than \arcsec{0}{5} per hour in direction
        and 4~metres per second in distance.  (RMS figures with respect
        to JPL DE200 for the interval 1960-2025 are \arcseci{14} and
        \arcsec{0}{2} per hour in longitude, \arcseci{9} and \arcsec{0}{2}
        per hour in latitude, 350~km and 2~metres per second in distance.)
        Note that the distance accuracy is comparatively poor because this
        function is principally intended for computing topocentric direction.
  \item This function is only a partial implementation of the original
        Meeus algorithm (reference below), which offers 4 times the
        accuracy in direction and 20 times the accuracy in distance
        when fully implemented (as it is in {\tt slaDmoon}).
 \end{enumerate}
}
\aref{Meeus, {\it l'Astronomie}, June 1984, p348.}
%-----------------------------------------------------------------------
\routine{slaMxm}{product of two $3\times3$ matrices}
{
 \action{Product of two $3\times3$ matrices (single precision).}
 \call{slaMxm( a, b, c );}
}
\args{GIVEN}
{
 \spec{a}{float[3][3]}{matrix {\bf A}} \\
 \spec{b}{float[3][3]}{matrix {\bf B}}
}
\args{RETURNED}
{
 \spec{c}{float[3][3]}{matrix result: {\bf A}$\times${\bf B}}
}
\anote{The arguments do not have to be different arrays.}
%-----------------------------------------------------------------------
\routine{slaMxv}{apply 3D rotation}
{
 \action{Multiply a 3-vector by a rotation matrix (single precision).}
 \call{slaMxv( rm, va, vb );}
}
\args{GIVEN}
{
 \spec{rm}{float[3][3]}{rotation matrix} \\
 \spec{va}{float[3]}{vector to be rotated}
}
\args{RETURNED}
{
 \spec{vb}{float[3]}{result vector}
}
\notes
{
 \begin{enumerate}
  \item This function performs the operation:
        \begin{verse}
         ${\bf b} = {\bf M}\times{\bf a}$
        \end{verse}
        where {\bf a} and {\bf b} are the 3-vectors
        {\tt va} and {\tt vb}
        respectively, and {\bf M} is the $3\times3$ matrix {\tt rm}.
  \item The main function of this function is apply a
        rotation;  under these circumstances, ${\bf M}$ is a
        {\it proper real orthogonal}\/ matrix.
  \item {\tt va} and {\tt vb} do not have to be
        different arrays.
 \end{enumerate}
}
%------------------------------------------------------------------------------
\routine{slaNut}{nutation matrix}
{
 \action{Form the matrix of nutation (SF2001 theory) for a given date.}
 \call{slaNut( date, rmatn );}
}
\args{GIVEN}
{
 \spec{date}{double}{TDB as Modified Julian Date
                                          (JD$-$2400000.5)}
}
\args{RETURNED}
{
 \spec{rmatn}{double[3][3]}{nutation matrix}
}
\notes{
 \begin{enumerate}
 \item The matrix is in the sense:
       \begin{verse}
        ${\bf v}_{true} = {\bf M}\times{\bf v}_{mean}$
       \end{verse}
       where ${\bf v}_{true}$ is the star vector relative to the
       true equator and equinox of date, {\bf M} is the
       $3\times3$ matrix {\tt rmatn} and
       ${\bf v}_{mean}$ is the star vector relative to the
       mean equator and equinox of date.
 \item The matrix represents forced nutation (but not free core nutation)
       plus corrections to the IAU~1976 precession model.
 \item Earth attitude predictions made by combining the present nutation
       matrix with IAU~1976 precession are accurate to 1~mas (with respect
       to the ICRS) for a few decades around 2000.
 \item The distinction between the required TDB and TT is
       always negligible.  Moreover, for all but the most
       critical applications UTC is adequate.
 \end{enumerate}
}
\refs
{
 \begin{enumerate}
  \item Kaplan, G.H., 1981, {\it USNO circular No.~163}, pA3-6.
  \item Shirai, T. \& Fukushima, T.\ 2001, {\it Astron.J.}, {\bf 121},
        3270-3283.
 \end{enumerate}
}
%-----------------------------------------------------------------------
\routine{slaNutc}{nutation components}
{
 \action{Nutation (SF2001 theory):  longitude \& obliquity
         components, and mean obliquity.}
 \call{slaNutc ( date, \&dpsi, \&deps, \&eps0 );}
}
\args{GIVEN}
{
 \spec{date}{double}{TDB as Modified Julian Date
                                           (JD$-$2400000.5)}
}
\args{RETURNED}
{
 \spec{dpsi,deps}{double*}{nutation in longitude and obliquity (radians)} \\
 \spec{eps0}{double*}{mean obliquity (radians)}
}
\notes
{
 \begin{enumerate}
  \item The function predicts forced nutation (but not free core nutation)
        plus corrections to the IAU~1976 precession model.
  \item Earth attitude predictions made by combining the present nutation
        model with IAU~1976 precession are accurate to 1~mas (with respect
        to the ICRS) for a few decades around 2000.
  \item The {\tt slaNutc80} function is the equivalent of the present function
        but using the IAU~1980 nutation theory.  The older theory is less
        accurate, leading to errors as large as 350\,mas over the interval
        1900-2100, mainly because of the error in the IAU~1976 precession.
  \item The distinction between the required TDB and TT is
        always negligible.  Moreover, for all but the most
        critical applications UTC is adequate.
 \end{enumerate}
}
\refs
{
 \begin{enumerate}
  \item Shirai, T. \& Fukushima, T.\ 2001, {\it Astron.J.}, {\bf 121},
        3270-3283.
  \item Fukushima, T.\ 1991, {\it Astron.Astrophys.}, {\bf 244}, L11.
  \item Simon, J. L., Bretagnon, P., Chapront, J., Chapront-Touze, M.,
        Francou, G. \& Laskar, J.\ 1994, {\it Astron.Astrophys.},
        {\bf 282}, 663.
 \end{enumerate}
}
%-----------------------------------------------------------------------
\routine{slaNutc80}{nutation components, IAU 1980}
{
 \action{Nutation (IAU 1980 theory):  longitude \& obliquity
         components, and mean obliquity.}
 \call{slaNutc80 ( date, \&dpsi, \&deps, \&eps0 );}
}
\args{GIVEN}
{
 \spec{date}{double}{TDB as Modified Julian Date
                                           (JD$-$2400000.5)}
}
\args{RETURNED}
{
 \spec{dpsi,deps}{double*}{nutation in longitude and obliquity (radians)} \\
 \spec{eps0}{double*}{mean obliquity (radians)}
}
\notes
{
 \begin{enumerate}
  \item The IAU 1980 theory used in the present function has
        errors as large as 350\,mas over the interval
        1900-2100, mainly because of the error in the IAU~1976
        precession.  For more accurate results, either the corrections
        published in IERS {\it Bulletin~B}\/
        must be applied, or the
        {\tt slaNutc} function can be used.  The latter is based upon the
        more recent SF2001 nutation theory and is of better
        than 1\,mas accuracy.
  \item The distinction between the required TDB and TT is
        always negligible.  Moreover, for all but the most
        critical applications UTC is adequate.
 \end{enumerate}
}
\refs
{
 \begin{enumerate}
  \item Final report of the IAU Working Group on Nutation,
        chairman P.K.Seidelmann, 1980.
  \item Kaplan, G.H.\ 1981, {\it USNO circular no.\ 163}, pA3-6.
 \end{enumerate}
}
%------------------------------------------------------------------------------
\routine{slaOap}{observed to apparent}
{
 \action{Observed to apparent place.}
 \call{slaOap( \vtop{
         \hbox{type, ob1, ob2, date, dut, elongm, phim,}
         \hbox{hm, xp, yp, tdk, pmb, rh, wl, tlr, \&rap, \&dap);}}}
}
\args{GIVEN}
{
 \spec{type}{char*}{type of coordinates --
                    \fstring{R},
                    \fstring{H} or
                    \fstring{A} (see below)} \\
 \spec{ob1}{double}{observed Az, HA or RA (radians;
                    Az is N\,=\,0, E\,=\,$90^{\circ}$)} \\
 \spec{ob2}{double}{observed zenith distance or $\delta$ (radians)} \\
 \spec{date}{D }{UTC date/time (Modified Julian Date, JD$-$2400000.5)} \\
 \spec{dut}{double}{$\Delta$UT:  UT1$-$UTC (UTC seconds)} \\
 \spec{elongm}{double}{observer's mean longitude (radians, east +ve)} \\
 \spec{phim}{double}{observer's mean geodetic latitude (radians)} \\
 \spec{hm}{double}{observer's height above sea level (metres)} \\
 \spec{xp,yp}{double}{polar motion \xy\ coordinates (radians)} \\
 \spec{tdk}{double}{local ambient temperature (K; std=273.15)} \\
 \spec{pmb}{double}{local atmospheric pressure (mb; std=1013.25)} \\
 \spec{rh}{double}{local relative humidity (in the range 0.0\,--\,1.0)} \\
 \spec{wl}{double}{effective wavelength ($\mu{\rm m}$, {\it e.g.}\ 0.55)} \\
 \spec{tlr}{double}{tropospheric lapse rate (K per metre,
                                                {\it e.g.}\ 0.0065)}
}
\args{RETURNED}
{
 \spec{rap,dap}{double*}{geocentric apparent \radec}
}
\notes
{
 \begin{enumerate}
  \item Only the first character of the
        {\tt type} argument is significant.
        \fstring{R} or \fstring{r} indicates that
        {\tt obs1} and {\tt obs2} are the observed right
        ascension and declination;
        \fstring{H} or \fstring{h} indicates that they are
        hour angle (west positive) and declination;
        anything else (\fstring{A} or \fstring{a}
        is recommended) indicates that
        {\tt obs1} and {\tt obs2} are azimuth
        (north zero, east $90^{\circ}$) and zenith distance.  (Zenith
        distance is used rather than elevation in order to reflect the
        fact that no allowance is made for depression of the horizon.)
  \item The accuracy of the result is limited by the corrections for
        refraction.  Providing the meteorological parameters are
        known accurately and there are no gross local effects, the
        predicted azimuth and elevation should be within about
        \arcsec{0}{1} for $\zeta<70^{\circ}$.  Even
        at a topocentric zenith distance of
        $90^{\circ}$, the accuracy in elevation should be better than
        1~arcminute;  useful results are available for a further
        $3^{\circ}$, beyond which the {\tt slaRefro} function returns a
        fixed value of the refraction.  The complementary
        functions {\tt slaAop} (or {\tt slaAopqk}) and {\tt slaOap} (or {\tt slaOapqk})
        are self-consistent to better than 1~microarcsecond all over
        the celestial sphere.
  \item It is advisable to take great care with units, as even
        unlikely values of the input parameters are accepted and
        processed in accordance with the models used.
  \item {\it Observed}\/ \azel\ means the position that would be seen by a
        perfect theodolite located at the observer.  This is
        related to the observed \hadec\ via the standard rotation, using
        the geodetic latitude (corrected for polar motion), while the
        observed HA and RA are related simply through the local
        apparent ST.  {\it Observed}\/ \radec\ or \hadec\ thus means the
        position that would be seen by a perfect equatorial located
        at the observer and with its polar axis aligned to the
        Earth's axis of rotation ({\it n.b.}\ not to the refracted pole).
        By removing from the observed place the effects of
        atmospheric refraction and diurnal aberration, the
        geocentric apparent \radec\ is obtained.
  \item Frequently, {\it mean}\/ rather than {\it apparent}\,
        \radec\ will be required,
        in which case further transformations will be necessary.  The
        {\tt slaAmp} {\it etc.}\ functions will convert
        the apparent \radec\ produced
        by the present function into an FK5 J2000 mean place, by
        allowing for the Sun's gravitational lens effect, annual
        aberration, nutation and precession.  Should FK4 B1950
        coordinates be required, the functions
        {\tt slaFk524} {\it etc.}\ will also
        have to be applied.
  \item To convert to apparent \radec\ the coordinates read from a
        real telescope, corrections would have to be applied for
        encoder zero points, gear and encoder errors, tube flexure,
        the position of the rotator axis and the pointing axis
        relative to it, non-perpendicularity between the mounting
        axes, and finally for the tilt of the azimuth or polar axis
        of the mounting (with appropriate corrections for mount
        flexures).  Some telescopes would, of course, exhibit other
        properties which would need to be accounted for at the
        appropriate point in the sequence.
  \item This function takes time to execute, due mainly to the
        rigorous integration used to evaluate the refraction.
        For processing multiple stars for one location and time,
        call {\tt slaAoppa} once followed by one call per star to {\tt slaOapqk}.
        Where a range of times within a limited period of a few hours
        is involved, and the highest precision is not required, call
        {\tt slaAoppa} once, followed by a call to {\tt slaAoppat} each time the
        time changes, followed by one call per star to {\tt slaOapqk}.
  \item The DATE argument is UTC expressed as an MJD.  This is,
        strictly speaking, wrong, because of leap seconds.  However,
        as long as the $\Delta$UT and the UTC are consistent there
        are no difficulties, except during a leap second.  In this
        case, the start of the 61st second of the final minute should
        begin a new MJD day and the old pre-leap $\Delta$UT should
        continue to be used.  As the 61st second completes, the MJD
        should revert to the start of the day as, simultaneously,
        the $\Delta$UT changes by one second to its post-leap new value.
  \item The $\Delta$UT (UT1$-$UTC) is tabulated in IERS circulars and
        elsewhere.  It increases by exactly one second at the end of
        each UTC leap second, introduced in order to keep $\Delta$UT
        within $\pm$\tsec{0}{9}.
  \item {\bf Important : take care with the longitude sign convention.}  The
        longitude required by the present function is {\bf east-positive},
        in accordance with geographical convention (and right-handed).
        In particular, note that the longitudes returned by the
        {\tt slaObs} function are west-positive (as in the {\it Astronomical
        Almanac}\/ before 1984) and must be reversed in sign before use
        in the present function.
  \item The polar coordinates {\tt xp},{\tt yp} can be obtained from IERS
        circulars and equivalent publications.  The
        maximum amplitude is about \arcsec{0}{3}.  If
        {\tt xp},{\tt yp} values
        are unavailable, use {\tt xp}\,=\,{\tt yp}\,=\,0.0.
        See page B60 of the 1988
        {\it Astronomical Almanac}\/ for a definition of the two angles.
  \item The height above sea level of the observing
        station, {\tt hm},
        can be obtained from the {\it Astronomical Almanac}\/ (Section J
        in the 1988 edition), or via the function {\tt slaObs}.
        If {\tt p}, the pressure in millibars (the same as hPa),
        is available, an adequate
        estimate of {\tt hm} can be obtained from the following expression:
        \begin{quote}
         {\tt hm = -29.3*tsl*log(p/1013.25)}
        \end{quote}
        where {\tt tsl} is the approximate sea-level air temperature in K
        (see {\it Astrophysical Quantities}, C.W.Allen, 3rd~edition,
        \S 52).  Similarly, if the pressure {\tt p} is not known,
        it can be estimated from the height of the observing
        station, {\tt hm} as follows:
        \begin{quote}
         {\tt p = 1013.25*exp(-hm/(29.3*tsl))}
        \end{quote}
        Note, however, that the refraction is nearly proportional to the
        pressure and that an accurate {\tt p} value is important for
        precise work.
  \item The azimuths {\it etc.}\ used by the present function are with
        respect to the celestial pole.
        The correction needed to refer instead to terrestrial north
        can be computed using {\tt slaPolmo}.
 \end{enumerate}
}
%-----------------------------------------------------------------------
\routine{slaOapqk}{quick observed to apparent}
{
 \action{Quick observed to apparent place.}
 \call{slaOapqk( type, ob1, ob2, aoprms, \&rap, \&dap );}
}
\args{GIVEN}
{
 \spec{type}{char*}{type of coordinates --
                    \fstring{R},
                    \fstring{H} or
                    \fstring{A} (see below)} \\
 \spec{ob1}{double}{observed Az, HA or RA (radians;
                    Az is N\,=\,0, E\,=\,$90^{\circ}$)} \\
 \spec{ob2}{double}{observed zenith distance or $\delta$ (radians)} \\
 \spec{aoprms}{double[14]}{star-independent apparent-to-observed parameters:} \\
 \specel{0}     {geodetic latitude (radians)} \\
 \specel{1,2}   {sine and cosine of geodetic latitude} \\
 \specel{3}     {magnitude of diurnal aberration vector} \\
 \specel{4}     {height above sea level (metres)} \\
 \specel{5}     {ambient temperature (K; std=273.15)} \\
 \specel{6}     {pressure (mb = hPa)} \\
 \specel{7}     {relative humidity (in the range 0.0\,--\,1.0)} \\
 \specel{8}     {wavelength ($\mu{\rm m}$)} \\
 \specel{9}     {tropospheric lapse rate (K per metre)} \\
 \specel{10,11} {refraction constants A and B (radians)} \\
 \specel{12}    {longitude + eqn of equinoxes +
                   ``sidereal $\Delta$UT'' (radians)} \\
 \specel{13}    {local apparent sidereal time (radians)}
}
\args{RETURNED}
{
 \spec{rap,dap}{double*}{geocentric apparent \radec}
}
\notes
{
 \begin{enumerate}
  \item Only the first character of the
        {\tt type} argument is significant.
        \fstring{R} or \fstring{r} indicates that
        {\tt obs1} and {\tt obs2} are the observed right
        ascension and declination;
        \fstring{H} or \fstring{h} indicates that they are
        hour angle (west positive) and declination;
        anything else (\fstring{A} or \fstring{a}
        is recommended) indicates that
        {\tt obs1} and {\tt obs2} are azimuth
        (north zero, east $90^{\circ}$) and zenith distance.  (Zenith
        distance is used rather than elevation in order to reflect the
        fact that no allowance is made for depression of the horizon.)
  \item The accuracy of the result is limited by the corrections for
        refraction.  Providing the meteorological parameters are
        known accurately and there are no gross local effects, the
        predicted azimuth and elevation should be within about
        \arcsec{0}{1} for $\zeta<70^{\circ}$.  Even
        at a topocentric zenith distance of
        $90^{\circ}$, the accuracy in elevation should be better than
        1~arcminute;  useful results are available for a further
        $3^{\circ}$, beyond which the {\tt slaRefro} function returns a
        fixed value of the refraction.  The complementary
        functions {\tt slaAop} (or {\tt slaAopqk}) and {\tt slaOap} (or {\tt slaOapqk})
        are self-consistent to better than 1~microarcsecond all over
        the celestial sphere.
  \item It is advisable to take great care with units, as even
        unlikely values of the input parameters are accepted and
        processed in accordance with the models used.
  \item {\it Observed}\/ \azel\ means the position that would be seen by a
        perfect theodolite located at the observer.  This is
        related to the observed \hadec\ via the standard rotation, using
        the geodetic latitude (corrected for polar motion), while the
        observed HA and RA are related simply through the local
        apparent ST.  {\it Observed}\/ \radec\ or \hadec\ thus means the
        position that would be seen by a perfect equatorial located
        at the observer and with its polar axis aligned to the
        Earth's axis of rotation ({\it n.b.}\ not to the refracted pole).
        By removing from the observed place the effects of
        atmospheric refraction and diurnal aberration, the
        geocentric apparent \radec\ is obtained.
  \item Frequently, {\it mean}\/ rather than {\it apparent}\,
        \radec\ will be required,
        in which case further transformations will be necessary.  The
        {\tt slaAmp} {\it etc.}\ functions will convert
        the apparent \radec\ produced
        by the present function into an FK5 J2000 mean place, by
        allowing for the Sun's gravitational lens effect, annual
        aberration, nutation and precession.  Should FK4 B1950
        coordinates be required, the functions
        {\tt slaFk524} {\it etc.}\ will also
        have to be applied.
  \item To convert to apparent \radec\ the coordinates read from a
        real telescope, corrections would have to be applied for
        encoder zero points, gear and encoder errors, tube flexure,
        the position of the rotator axis and the pointing axis
        relative to it, non-perpendicularity between the mounting
        axes, and finally for the tilt of the azimuth or polar axis
        of the mounting (with appropriate corrections for mount
        flexures).  Some telescopes would, of course, exhibit other
        properties which would need to be accounted for at the
        appropriate point in the sequence.
  \item The star-independent apparent-to-observed-place parameters
        in {\tt aoprms} may be computed by means of the {\tt slaAoppa} function.
        If nothing has changed significantly except the time, the
        {\tt slaAoppat} function may be used to perform the requisite
        partial recomputation of {\tt aoprms}.
  \item The azimuths {\it etc.}\ used by the present function are with
        respect to the celestial pole.
        The correction needed to refer instead to terrestrial north
        can be computed using {\tt slaPolmo}.
 \end{enumerate}
}
%-----------------------------------------------------------------------
\routine{slaObs}{observatory parameters}
{
 \action{Look up an entry in a standard list of
         groundbased observing stations parameters.}
 \call{slaObs( n, c, name, \&w, \&p, \&h );}
}
\args{GIVEN}
{
 \spec{n}{int}{number specifying observing station}
}
\args{GIVEN or RETURNED}
{
 \spec{c}{char*}{identifier specifying observing station}
}
\args{RETURNED}
{
 \spec{name}{char*}{name of specified observing station} \\
 \spec{w}{double*}{longitude (radians, west +ve)} \\
 \spec{p}{double*}{geodetic latitude (radians, north +ve)} \\
 \spec{h}{double*}{height above sea level (metres)}
}
\notes
{
 \begin{enumerate}
  \item Station identifiers {\tt c} may be up to 10 characters long,
        and station names {\tt name} may be up to 40 characters long.
  \item {\tt c} and {\tt n} are
        {\it alternative}\/ ways of specifying the observing
        station.  The {\tt c} option, which is the most generally useful,
        may be selected by specifying an {\tt n} value of zero or less.
        If {\tt n} is 1 or more,
        the parameters of the {\tt n}th station
        in the currently supported list are interrogated, and
        the station identifier {\tt c}
        is returned as well as {\tt name}, {\tt w},
        {\tt p} and {\tt h}.
  \item If the station parameters are not available, either because
        the station identifier {\tt c} is not recognized, or because an
        {\tt n} value greater than the number of stations supported is
        given, a name of \cstring{?} is returned
        and {\tt w}, {\tt p} and {\tt h} are left in
        their current states.
  \item Programs can obtain a list of all currently supported
        stations by calling the function repeatedly, with
        {\tt n}\,=\,1,2,3\ldots
        When {\tt name}\,=\,\cstring{?}
        is seen, the list of stations has been
        exhausted.  The stations at the time of writing are listed
        below.
  \item Station numbers, identifiers, names and other details are
        subject to change and should not be hardwired into
        application programs.
  \item All station identifiers {\tt c} are uppercase only;  lower case
        characters must be converted to uppercase by the calling
        program.  The station names returned may contain both upper-
        and lowercase.  All characters up to the first space are
        checked;  thus an abbreviated ID will return the parameters
        for the first station in the list which matches the
        abbreviation supplied, and no station in the list will ever
        contain embedded spaces.  {\tt c} must not have leading spaces.
  \item {\bf Important : take care with the longitude sign convention.}
        The longitude returned by {\tt slaObs} is
        {\bf west-positive}, following the pre-1984 {\it Astronomical
        Almanac}.  However, this sign convention is left-handed and is
        the opposite of the one now used; elsewhere in
        SLALIB the preferable east-positive convention is used.  In
        particular, note that for use in {\tt slaAop}, {\tt slaAoppa} and
        {\tt slaOap} the sign of the longitude must be reversed.
  \item Feedback is invited, to inform the author of
        possible improvements and extensions.  For example:
        \begin{itemize}
         \item typographical corrections
         \item more accurate parameters
         \item better station identifiers or names
         \item additional stations
        \end{itemize}
 \end{enumerate}
Stations supported by slaObs at the time of writing:
\begin{tabbing}
xxxxxxxxxxxxxxxxx \= \kill
{\it ID} \> {\it name} \\ \\
{\tt AAT} \> Anglo-Australian 3.9m Telescope \\
{\tt ANU2.3} \> Siding Spring 2.3m \\
{\tt APO3.5} \> Apache Point 3.5m \\
{\tt ARECIBO} \> Arecibo 1000 foot \\
{\tt ATCA} \> Australia Telescope Compact Array \\
{\tt BLOEMF} \> Bloemfontein 1.52m \\
{\tt BOSQALEGRE} \> Bosque Alegre 1.54m \\
{\tt CAMB1MILE} \> Cambridge 1 mile \\
{\tt CAMB5KM} \> Cambridge 5 km \\
{\tt CATALINA61} \> Catalina 61 inch \\
{\tt CFHT} \> Canada-France-Hawaii 3.6m Telescope \\
{\tt CSO} \> Caltech Sub-mm Observatory, Mauna Kea \\
{\tt DAO72} \> DAO Victoria BC 1.85m \\
{\tt DUNLAP74} \> David Dunlap 74 inch \\
{\tt DUPONT} \> Du Pont 2.5m Telescope, Las Campanas \\
{\tt EFFELSBERG} \> Effelsberg 100m \\
{\tt ESO3.6} \> ESO 3.6m \\
{\tt ESONTT} \> ESO 3.5m NTT \\
{\tt ESOSCHM} \> ESO 1m Schmidt, La Silla \\
{\tt FCRAO} \> Five College Radio Astronomy Obs \\
{\tt FLAGSTF61} \> USNO 61 inch astrograph, Flagstaff \\
{\tt GBVA140} \> Greenbank 140 foot \\
{\tt GBVA300} \> Greenbank 300 foot \\
{\tt GEMININ} \> Gemini North 8m \\
{\tt GEMINIS} \> Gemini South 8m \\
{\tt HARVARD} \> Harvard College Observatory 1.55m \\
{\tt HPROV1.52} \> Haute Provence 1.52m \\
{\tt HPROV1.93} \> Haute Provence 1.93m \\
{\tt IRTF} \> NASA IR Telescope Facility, Mauna Kea \\
{\tt JCMT} \> JCMT 15m \\
{\tt JODRELL1} \> Jodrell Bank 250 foot \\
{\tt KECK1} \> Keck 10m Telescope 1 \\
{\tt KECK2} \> Keck 10m Telescope 2 \\
{\tt KISO} \> Kiso 1.05m Schmidt, Japan \\
{\tt KOSMA3M} \> Cologne Submillimeter Observatory 3m \\
{\tt KOTTAMIA} \> Kottamia 74 inch \\
{\tt KPNO158} \> Kitt Peak 158 inch \\
{\tt KPNO36FT} \> Kitt Peak 36 foot \\
{\tt KPNO84} \> Kitt Peak 84 inch \\
{\tt KPNO90} \> Kitt Peak 90 inch \\
{\tt LICK120} \> Lick 120 inch \\
{\tt LOWELL72} \> Perkins 72 inch, Lowell \\
{\tt LPO1} \> Jacobus Kapteyn 1m Telescope \\
{\tt LPO2.5} \> Isaac Newton 2.5m Telescope \\
{\tt LPO4.2} \> William Herschel 4.2m Telescope \\
{\tt MAGELLAN1} \> Magellan 1, 6.5m, Las Campanas \\
{\tt MAGELLAN2} \> Magellan 2, 6.5m, Las Campanas \\
{\tt MAUNAK88} \> Mauna Kea 88 inch \\
{\tt MCDONLD2.1} \> McDonald 2.1m \\
{\tt MCDONLD2.7} \> McDonald 2.7m \\
{\tt MMT} \> MMT, Mt Hopkins \\
{\tt MOPRA} \> ATNF Mopra Observatory \\
{\tt MTEKAR} \> Mt Ekar 1.82m \\
{\tt MTHOP1.5} \> Mt Hopkins 1.5m \\
{\tt MTLEMMON60} \> Mt Lemmon 60 inch \\
{\tt NOBEYAMA} \> Nobeyama 45m \\
{\tt OKAYAMA} \> Okayama 1.88m \\
{\tt PALOMAR200} \> Palomar 200 inch \\
{\tt PALOMAR48} \> Palomar 48-inch Schmidt \\
{\tt PALOMAR60} \> Palomar 60 inch \\
{\tt PARKES} \> Parkes 64m \\
{\tt QUEBEC1.6} \> Quebec 1.6m \\
{\tt SAAO74} \> Sutherland 74 inch \\
{\tt SANPM83} \> San Pedro Martir 83 inch \\
{\tt ST.ANDREWS} \> St Andrews University Observatory \\
{\tt STEWARD90} \> Steward 90 inch \\
{\tt STROMLO74} \> Mount Stromlo 74 inch \\
{\tt SUBARU} \> Subaru 8m \\
{\tt SUGARGROVE} \> Sugar Grove 150 foot \\
{\tt TAUTNBG} \> Tautenburg 2m \\
{\tt TAUTSCHM} \> Tautenberg 1.34m Schmidt \\
{\tt TIDBINBLA} \> Tidbinbilla 64m \\
{\tt TOLOLO1.5M} \> Cerro Tololo 1.5m \\
{\tt TOLOLO4M} \> Cerro Tololo 4m \\
{\tt UKIRT} \> UK Infra Red Telescope \\
{\tt UKST} \> UK 1.2m Schmidt, Siding Spring \\
{\tt USSR6} \> USSR 6m \\
{\tt USSR600} \> USSR 600 foot \\
{\tt VLA} \> Very Large Array \\
{\tt VLT1} \> ESO VLT 8m, UT1 \\
{\tt VLT2} \> ESO VLT 8m, UT2 \\
{\tt VLT3} \> ESO VLT 8m, UT3 \\
{\tt VLT4} \> ESO VLT 8m, UT4
\end{tabbing}
}
%-----------------------------------------------------------------------
\routine{slaPa}{\hadec\ to parallactic angle}
{
 \action{Hour angle and declination to parallactic angle
         (double precision).}
 \call{d~=~slaPa( ha, dec, phi );}
}
\args{GIVEN}
{
 \spec{ha}{double}{hour angle in radians (geocentric apparent)} \\
 \spec{dec}{double}{declination in radians (geocentric apparent)} \\
 \spec{phi}{double}{latitude in radians (geodetic)}
}
\args{RETURNED}
{
 \spec{}{double}{parallactic angle (radians, in the range $\pm \pi$)}
}
\notes
{
 \begin{enumerate}
  \item The parallactic angle at a point in the sky is the position
        angle of the vertical, {\it i.e.}~the angle between the direction to
        the pole and to the zenith.  In precise applications care must
        be taken only to use geocentric apparent \hadec\ and to consider
        separately the effects of atmospheric refraction and telescope
        mount errors.
  \item At the pole a zero result is returned.
  \item There is some debate about whether ``parallactic angle'' is
        the correct term to use here.  What is certain, however, is that
        the meaning is conventionally as presented in Note~1.
 \end{enumerate}
}
%-----------------------------------------------------------------------
\routine{slaPav}{position-angle between two directions}
{
 \action{Returns the bearing (position angle) of one celestial
         direction with respect to another (single precision).}
 \call{r~=~slaPav( v1, v2 );}
}
\args{GIVEN}
{
 \spec{v1}{float[3]}{vector to one point} \\
 \spec{v2}{float[3]}{vector to the other point}
}
\args{RETURNED}
{
 \spec{}{float}{position-angle of second point with respect to first}
}
\notes
{
 \begin{enumerate}
 \item The coordinate frames correspond to \radec,
       \ecl\ {\it etc.}
 \item The result is the bearing (position angle), in radians,
        of point {\tt v2} as seen
       from point {\tt v1}.  It is in the range $\pm \pi$.  The sense
       is such that if {\tt v2}
       is a small distance due east of {\tt v1} the result
       is about $+\pi/2$. Zero is returned
       if the two points are coincident.
 \item There is no requirement for either vector to be of unit length.
 \item The function {\tt slaBear} performs an equivalent function except
       that the points are specified in the form of spherical coordinates.
 \end{enumerate}
}
%------------------------------------------------------------------------------
\routine{slaPcd}{apply radial distortion}
{
 \action{Apply pincushion/barrel distortion to a tangent-plane \xy.}
 \call{slaPcd( disco, \&x, \&y );}
}
\args{GIVEN}
{
 \spec{disco}{double}{pincushion/barrel distortion coefficient} \\
 \spec{x,y}{double*}{tangent-plane \xy}
}
\args{RETURNED}
{
 \spec{x,y}{double*}{distorted \xy}
}
\notes
{
 \begin{enumerate}
  \item The distortion is of the form
        $\rho = r (1 + c r^{2})$, where $r$ is
        the radial distance from the tangent point,
        $c$ is the {\tt disco}
        argument, and $\rho$ is the radial
        distance in the presence of
        the distortion.
  \item For {\it pincushion}\/ distortion, $c$ is positive;  for
        {\it barrel}\/ distortion, $c$ is negative.
  \item For {\tt x},{\tt y} in units of
        one projection radius (in the case of
        a photographic plate, the focal length), the following
        {\tt disco} values apply:
 
        \vspace{2ex}
 
        \hspace{5em}
        \begin{tabular}{|l|c|} \hline
         Geometry & {\tt disco} \\ \hline \hline
         astrograph & 0.0 \\ \hline
         Schmidt & $-$0.3333 \\ \hline
         AAT PF doublet & +147.069 \\ \hline
         AAT PF triplet & +178.585 \\ \hline
         AAT f/8 & +21.20 \\ \hline
         JKT f/8 & +14.6 \\ \hline
        \end{tabular}
 
        \vspace{2ex}
 
  \item There is a companion function, {\tt slaUnpcd}, which performs the
        inverse operation.
 \end{enumerate}
}
%-----------------------------------------------------------------------
\routine{slaPda2h}{hour angle for a given azimuth}
{
 \action{Hour angle corresponding to a given azimuth (double precision).}
 \call{slaPda2h( p, d, a, \&h1, \&j1, \&h2, \&j2 );}
}
\args{GIVEN}
{
 \spec{p}{double}{latitude} \\
 \spec{d}{double}{declination} \\
 \spec{a}{double}{azimuth}
}
\args{RETURNED}
{
 \spec{h1}{double*}{hour angle:  first solution if any} \\
 \spec{j1}{int*}{flag: 0 = solution 1 is valid} \\
 \spec{h2}{double*}{hour angle:  second solution if any} \\
 \spec{j2}{int*}{flag: 0 = solution 2 is valid}
}
%-----------------------------------------------------------------------
\routine{slaPdq2h}{hour angle for a given parallactic angle}
{
 \action{Hour angle corresponding to a given parallactic angle
         (double precision).}
 \call{slaPdq2h( p, d, q, \&h1, \&j1, \&h2, \&j2 );}
}
\args{GIVEN}
{
 \spec{p}{double}{latitude} \\
 \spec{d}{double}{declination} \\
 \spec{q}{double}{azimuth}
}
\args{RETURNED}
{
 \spec{h1}{double*}{hour angle:  first solution if any} \\
 \spec{j1}{int*}{flag: 0 = solution 1 is valid} \\
 \spec{h2}{double*}{hour angle:  second solution if any} \\
 \spec{j2}{int*}{flag: 0 = solution 2 is valid}
}
\anote{The {\it parallactic angle}\/ at a point in the sky
       (as conventionally defined --
       there is some dispute about whether the term is
       correctly used for this purpose)
       is the position angle of the vertical,
       {\it i.e.}~the angle between the directions to
       the pole and zenith.}
%-----------------------------------------------------------------------
\routine{slaPermut}{next permutation}
{
 \action{Generate the next permutation of a specified number of items.}
 \call{slaPermut( n, istate, iorder, \&j );}
}
\args{GIVEN}
{
 \spec{n}{int}{number of items:  there will be {\tt n}! permutations} \\
 \spec{istate}{int[n]}{state, ${\tt istate[0]}=-1$ to initialize}
}
\args{RETURNED}
{
 \spec{istate}{int[n]}{state, updated ready for next time} \\
 \spec{iorder}{int[n]}{next permutation of numbers 1,2,\ldots,{\tt n}} \\
 \spec{j}{int*}{status:} \\
 \spec{}{}{\hspace{1.5em} $-$1 = illegal {\tt n}
             (zero or less is illegal)} \\
 \spec{}{}{\hspace{2.3em}    0 = OK} \\
 \spec{}{}{\hspace{1.5em} $+$1 = no more permutations available}
}
\notes
{
 \begin{enumerate}
  \item This function returns, in the
        {\tt iorder} array, the integers 1 to {\tt n}
        inclusive, in an order that depends on the current contents of
        the {\tt istate} array.  Before calling the function for the first
        time, the caller must set the first element
        of the {\tt istate} array
        to $-1$ (any negative number will do)
        to cause the {\tt istate} array
        to be fully initialized.
  \item The first permutation to be generated is:
        \begin{verse}
           ${\tt iorder[0]}={\tt n}$,
           ${\tt iorder[1]}={\tt n}-1$,\ldots,
           ${\tt iorder[n-1]}=1$
        \end{verse}
        This is also the permutation returned
        for the ``finished'' (${\tt j}=1$) case.
        The final permutation to be generated is:
        \begin{verse}
           ${\tt iorder[0]}=1$,
           ${\tt iorder[1]}=2$,\ldots,
           ${\tt iorder[n-1]}={\tt n}$
        \end{verse}
  \item If the ``finished'' (${\tt j}=1$) status
        is ignored, the function continues
        to deliver permutations, the pattern repeating
        every~{\tt n}!\,~calls.
 \end{enumerate}
}
%------------------------------------------------------------------------------
\routine{slaPertel}{perturbed orbital elements}
{
 \action{Update the osculating elements of an asteroid or comet by
         applying planetary perturbations.}
 \call{slaPertel( \vtop{
         \hbox{jform, date0, date1,}
         \hbox{epoch0, orbi0, anode0, perih0, aorq0, e0, am0,}
         \hbox{\&epoch1, \&orbi1, \&anode1, \&perih1, \&aorq1, \&e1, \&am1,}
         \hbox{\&jstat);}}}
}
\args{GIVEN (format and dates)}
{
 \spec{jform}{int}{choice of element set (2 or 3; Note~1)} \\
 \spec{date0}{double}{date of osculation (TDB MJD) for the given} \\
 \spec{}{}{\hspace{1.5em} elements} \\
 \spec{date1}{double}{date of osculation (TDB MJD) for the updated} \\
 \spec{}{}{\hspace{1.5em} elements}
}
\args{GIVEN (the unperturbed elements)}
{
 \spec{epoch0}{double}{epoch of the given element set
                            ($t_0$ or $T$, TDB MJD;} \\
 \spec{}{}{\hspace{1.5em} Note~2)} \\
 \spec{orbi0}{double}{inclination ($i$, radians)} \\
 \spec{anode0}{double}{longitude of the ascending node ($\Omega$, radians)} \\
 \spec{perih0}{double}{argument of perihelion
                            ($\omega$, radians)} \\
 \spec{aorq0}{double}{mean distance or perihelion distance ($a$ or $q$, AU)} \\
 \spec{e0}{double}{eccentricity ($e$)} \\
 \spec{am0}{double}{mean anomaly ($M$, radians, ${\tt jform}=2$ only)}
}
\args{RETURNED (the updated elements)}
{
 \spec{epoch1}{double}{epoch of the updated element set
                            ($t_0$ or $T$,} \\
 \spec{}{}{\hspace{1.5em} TDB MJD; Note~2)} \\
 \spec{orbi1}{double}{inclination ($i$, radians)} \\
 \spec{anode1}{double}{longitude of the ascending node ($\Omega$, radians)} \\
 \spec{perih1}{double}{argument of perihelion
                            ($\omega$, radians)} \\
 \spec{aorq1}{double}{mean distance or perihelion distance ($a$ or $q$, AU)} \\
 \spec{e1}{double}{eccentricity ($e$)} \\
 \spec{am1}{double}{mean anomaly ($M$, radians, ${\tt jform}=2$ only)}
}
\args{RETURNED (status flag)}
{
 \spec{jstat}{int*}{status:} \\
 \spec{}{}{\hspace{0.5em}+102 = warning, distant epoch} \\
 \spec{}{}{\hspace{0.5em}+101 = warning, large timespan
                                            ($>100$ years)} \\
 \spec{}{}{\hspace{-1.8em}+1 to +10 = coincident with major planet
                                                  (Note~6)} \\
 \spec{}{}{\hspace{1.95em}       0 = OK} \\
 \spec{}{}{\hspace{1.2em}    $-$1 = illegal {\tt jform}} \\
 \spec{}{}{\hspace{1.2em}    $-$2 = illegal {\tt e0}} \\
 \spec{}{}{\hspace{1.2em}    $-$3 = illegal {\tt aorq0}} \\
 \spec{}{}{\hspace{1.2em}    $-$4 = internal error} \\
 \spec{}{}{\hspace{1.2em}    $-$5 = numerical error}
}
\notes
{
 \begin{enumerate}
  \item Two different element-format options are supported, as follows. \\
 
        ${\tt jform}=2$, suitable for minor planets:
 
        \begin{tabbing}
        xxx \= xxxxxxxx \= xx \= \kill
        \> {\tt epoch}  \> = \> epoch of elements $t_0$ (TDB MJD) \\
        \> {\tt orbinc} \> = \> inclination $i$ (radians) \\
        \> {\tt anode}  \> = \> longitude of the ascending node $\Omega$ (radians) \\
        \> {\tt perih}  \> = \> argument of perihelion $\omega$ (radians) \\
        \> {\tt aorq}   \> = \> mean distance $a$ (AU) \\
        \> {\tt e}      \> = \> eccentricity $e$ $( 0 \leq e < 1 )$ \\
        \> {\tt aorl}   \> = \> mean anomaly $M$ (radians)
        \end{tabbing}
 
        ${\tt jform}=3$, suitable for comets:
 
        \begin{tabbing}
        xxx \= xxxxxxxx \= xx \= \kill
        \> {\tt epoch}  \> = \> epoch of perihelion $T$ (TDB MJD) \\
        \> {\tt orbinc} \> = \> inclination $i$ (radians) \\
        \> {\tt anode}  \> = \> longitude of the ascending node
                                $\Omega$ (radians) \\
        \> {\tt perih}  \> = \> argument of perihelion $\omega$ (radians) \\
        \> {\tt aorq}   \> = \> perihelion distance $q$ (AU) \\
        \> {\tt e}      \> = \> eccentricity $e$ $( 0 \leq e \leq 10 )$
        \end{tabbing}
 \item {\tt date0}, {\tt date1}, {\tt epoch0} and
       {\tt epoch1} are all instants of time in
       the TDB time-scale (TT will do), expressed
       as Modified Julian Dates (JD$-$2400000.5).
       \begin{itemize}
       \item {\tt date0} is the instant at which the given
             ({\it i.e.}~unperturbed) osculating elements are correct.
       \item {\tt date1} is the specified instant at
             which the updated osculating
             elements are correct.
       \item {\tt epoch0} and {\tt epoch1} will
             be the same as {\tt date0} and {\tt date1}
             (respectively) for the ${\tt jform}=2$ case,
             normally used for minor
             planets.  For the ${\tt jform}=3$ case,
             the two epochs will refer to
             perihelion passage and so will not,
             in general, be the same as
             {\tt date0} and/or {\tt date1}
             though they may be similar to one another.
       \end{itemize}
 \item The elements are with respect to the
       J2000 ecliptic and mean equinox.
 \item Unused elements ({\tt am0} and {\tt am1}
       for ${\tt jform}=3$) are not accessed.
 \item See the {\tt slaPertue} function for details of the algorithm used.
 \item This function is not intended to be used for major planets, which
       is why ${\tt jform}=1$ is not available
       and why there is no opportunity
       to specify either the longitude of perihelion or the daily
       motion.  However, if ${\tt jform}=2$
       elements are somehow obtained for a
       major planet and supplied to the function, sensible results will,
       in fact, be produced.  This happens because the {\tt slaPertue} function
       that is called to perform the calculations checks the separation
       between the body and each of the planets and interprets a
       suspiciously small value (0.001~AU) as an attempt to apply it to
       the planet concerned.  If this condition is detected, the
       contribution from that planet is ignored, and the status is set to
       the planet number (1--10 = Mercury, Venus, EMB, Mars, Jupiter,
       Saturn, Uranus, Neptune, Earth, Moon) as a warning.
 \end{enumerate}
}
\aref{Sterne, Theodore E., {\it An Introduction to Celestial Mechanics,}\/
      Interscience Publishers, 1960.  Section 6.7, p199.}
%------------------------------------------------------------------------------
\routine{slaPertue}{perturbed universal elements}
{
 \action{Update the universal elements of an asteroid or comet by
         applying planetary perturbations.}
 \call{slaPertue( date, u, \&jstat );}
}
\args{GIVEN}
{
 \spec{date}{double}{final epoch (TDB MJD) for the updated elements}
}
\args{GIVEN and RETURNED}
{
 \spec{u}{double[13]}{universal elements (updated in place)} \\
 \specel{0}     {combined mass ($M+m$)} \\
 \specel{1}     {total energy of the orbit ($\alpha$)} \\
 \specel{2}     {reference (osculating) epoch ($t_0$)} \\
 \specel{3-5}   {position at reference epoch (${\rm \bf r}_0$)} \\
 \specel{6-8}   {velocity at reference epoch (${\rm \bf v}_0$)} \\
 \specel{9}     {heliocentric distance at reference epoch} \\
 \specel{10}    {${\rm \bf r}_0.{\rm \bf v_0}$} \\
 \specel{11}    {date ($t$)} \\
 \specel{12}    {universal eccentric anomaly ($\psi$) of date, approx}
}
\args{RETURNED}
{
 \spec{jstat}{int*}{status:} \\
 \spec{}{}{\hspace{0.5em}+102 = warning, distant epoch} \\
 \spec{}{}{\hspace{0.5em}+101 = warning, large timespan
                                            ($>100$ years)} \\
 \spec{}{}{\hspace{-1.8em}+1 to +10 = coincident with major planet
                                                  (Note~5)} \\
 \spec{}{}{\hspace{1.95em}       0 = OK} \\
 \spec{}{}{\hspace{1.2em}    $-$1 = numerical error}
}
\notes
{
 \begin{enumerate}
  \setlength{\parskip}{\medskipamount}
  \item The ``universal'' elements are those which define the orbit for the
        purposes of the method of universal variables (see reference 2).
        They consist of the combined mass of the two bodies, an epoch,
        and the position and velocity vectors (arbitrary reference frame)
        at that epoch.  The parameter set used here includes also various
        quantities that can, in fact, be derived from the other
        information.  This approach is taken to avoiding unnecessary
        computation and loss of accuracy.  The supplementary quantities
        are (i)~$\alpha$, which is proportional to the total energy of the
        orbit, (ii)~the heliocentric distance at epoch,
        (iii)~the outwards component of the velocity at the given epoch,
        (iv)~an estimate of $\psi$, the ``universal eccentric anomaly'' at a
        given date and (v)~that date.
 
  \item The universal elements are with respect to the J2000 equator and
        equinox.
 
  \item The epochs {\tt date}, {\tt u[2]} and {\tt u[11]}
        are all TDBs (TT will do) as Modified Julian Dates (JD$-$2400000.5).
 
  \item The algorithm is a simplified form of Encke's method.  It takes as
        a basis the unperturbed motion of the body, and numerically
        integrates the perturbing accelerations from the major planets.
        The expression used is essentially Sterne's 6.7-2 (reference 1).
        Everhart \& Pitkin (reference 2) suggest rectifying the orbit at
        each integration step by propagating the new perturbed position
        and velocity as the new universal variables.  In the present
        function the orbit is rectified less frequently than this, in order
        to gain a slight speed advantage.  However, the rectification is
        done directly in terms of position and velocity, as suggested by
        Everhart \& Pitkin, bypassing the use of conventional orbital
        elements.
 
        The $f(q)$ part of the full Encke method is not used.  The purpose
        of this part is to avoid subtracting two nearly equal quantities
        when calculating the ``indirect member'', which takes account of the
        small change in the Sun's attraction due to the slightly displaced
        position of the perturbed body.  A simpler, direct calculation in
        double precision proves to be faster and not significantly less
        accurate.
 
        Apart from employing a variable timestep, and occasionally
        ``rectifying the orbit'' to keep the indirect member small, the
        integration is done in a fairly straightforward way.  The
        acceleration estimated for the middle of the timestep is assumed
        to apply throughout that timestep;  it is also used in the
        extrapolation of the perturbations to the middle of the next
        timestep, to predict the new disturbed position.  There is no
        iteration within a timestep.
 
        Measures are taken to reach a compromise between execution time
        and accuracy.  The starting-point is the goal of achieving
        arcsecond accuracy for ordinary minor planets over a ten-year
        timespan.  This goal dictates how large the timesteps can be,
        which in turn dictates how frequently the unperturbed motion has
        to be recalculated from the osculating elements.
 
        Within predetermined limits, the timestep for the numerical
        integration is varied in length in inverse proportion to the
        magnitude of the net acceleration on the body from the major
        planets.
 
        The numerical integration requires estimates of the major-planet
        motions.  Approximate positions for the major planets (Pluto
        alone is omitted) are obtained from the function {\tt slaPlanet}.  Two
        levels of interpolation are used, to enhance speed without
        significantly degrading accuracy.  At a low frequency, the function
        {\tt slaPlanet} is called to generate updated position+velocity ``state
        vectors''.  The only task remaining to be carried out at the full
        frequency ({\it i.e.}~at each integration step) is to use the state
        vectors to extrapolate the planetary positions.  In place of a
        strictly linear extrapolation, some allowance is made for the
        curvature of the orbit by scaling back the radius vector as the
        linear extrapolation goes off at a tangent.
 
        Various other approximations are made.  For example, perturbations
        by Pluto and the minor planets are neglected and relativistic
        effects are not taken into account.
 
        In the interests of simplicity, the background calculations for
        the major planets are carried out {\it en masse.}
        The mean elements and
        state vectors for all the planets are refreshed at the same time,
        without regard for orbit curvature, mass or proximity.
 
        The Earth-Moon system is treated as a single body when the body is
        distant but as separate bodies when closer to the EMB than the
        (internal) parameter {\tt RNE}, which incurs a time penalty
        but improves accuracy
        for near-Earth objects.
 
  \item This function is not intended to be used for major planets.
        However, if major-planet elements are supplied, sensible results
        will, in fact, be produced.  This happens because the function
        checks the separation between the body and each of the planets and
        interprets a suspiciously small value (0.001~AU) as an attempt to
        apply the function to the planet concerned.  If this condition
        is detected, the
        contribution from that planet is ignored, and the status is set to
        the planet number (1--10~=~Mercury, Venus, EMB,
        Mars, Jupiter, Saturn, Uranus, Neptune, Earth, Moon) as a warning.
 \end{enumerate}
}
\refs{
   \begin{enumerate}
   \item Sterne, Theodore E.\ 1960,
         {\it An Introduction to Celestial Mechanics,}
         Interscience Publishers, Section 6.7, p199.
   \item Everhart, E. \& Pitkin, E.T.\ 1983, {\it Am.J.Phys.},
         {\bf 51}, 712.
   \end{enumerate}
}
%-----------------------------------------------------------------------
\routine{slaPlanel}{planet position from elements}
{
 \action{Heliocentric position and velocity of a planet,
         asteroid or comet, starting from orbital elements.}
 \call{slaPlanel( \vtop{
         \hbox{date, jform, epoch, orbinc, anode, perih,}
         \hbox{aorq, e, aorl, dm, pv, \&jstat);}}}
}
\args{GIVEN}
{
 \spec{date}{double}{TDB MJD of observation (JD$-$2400000.5,} \\
 \spec{}{}{\hspace{1.5em} Note~1)} \\
 \spec{jform}{int}{choice of element set (1-3, Note~3)} \\
 \spec{epoch}{double}{epoch of elements ($t_0$ or $T$, TDB MJD, Note~4)} \\
 \spec{orbinc}{double}{inclination ($i$, radians)} \\
 \spec{anode}{double}{longitude of the ascending node ($\Omega$, radians)} \\
 \spec{perih}{double}{longitude or argument of perihelion
                            ($\varpi$ or $\omega$,} \\
 \spec{}{}{\hspace{1.5em} radians)} \\
 \spec{aorq}{double}{mean distance or perihelion distance ($a$ or $q$, AU)} \\
 \spec{e}{double}{eccentricity ($e$)} \\
 \spec{aorl}{double}{mean anomaly or longitude
                               ($M$ or $L$, radians,} \\
 \spec{}{}{\hspace{1.5em} ${\tt jform}=1,2$ only)} \\
 \spec{dm}{double}{daily motion ($n$, radians, ${\tt jform}=1$ only)}
}
\args{RETURNED}
{
 \spec{pv}{double[6]}{heliocentric \xyzxyzd, equatorial, J2000} \\
 \spec{}{}{\hspace{1.5em} (AU, AU/s)} \\
 \spec{jstat}{int*}{status:} \\
 \spec{}{}{\hspace{2.3em}    0 = OK} \\
 \spec{}{}{\hspace{1.5em} $-$1 = illegal {\tt jform}} \\
 \spec{}{}{\hspace{1.5em} $-$2 = illegal {\tt e}} \\
 \spec{}{}{\hspace{1.5em} $-$3 = illegal {\tt aorq}} \\
 \spec{}{}{\hspace{1.5em} $-$4 = illegal {\tt dm}} \\
 \spec{}{}{\hspace{1.5em} $-$5 = numerical error}
}
\notes
{
 \begin{enumerate}
  \item DATE is the instant for which the prediction is required.  It is
        in the TDB time-scale (TT will do) and is a
        Modified Julian Date (JD$-$2400000.5).
  \item The elements are with respect to the J2000 ecliptic and equinox.
  \item A choice of three different element-format options is available, as
        follows. \\
 
        ${\tt jform}=1$, suitable for the major planets:
 
        \begin{tabbing}
        xxx \= xxxxxxx \= xx \= \kill
        \> {\tt epoch}  \> = \> epoch of elements $t_0$ (TDB MJD) \\
        \> {\tt orbinc} \> = \> inclination $i$ (radians) \\
        \> {\tt anode}  \> = \> longitude of the ascending node $\Omega$ (radians) \\
        \> {\tt perih}  \> = \> longitude of perihelion $\varpi$ (radians) \\
        \> {\tt aorq}   \> = \> mean distance $a$ (AU) \\
        \> {\tt e}      \> = \> eccentricity $e$ \\
        \> {\tt aorl}   \> = \> mean longitude $L$ (radians) \\
        \> {\tt dm}     \> = \> daily motion $n$ (radians)
        \end{tabbing}
 
        ${\tt jform}=2$, suitable for minor planets:
 
        \begin{tabbing}
        xxx \= xxxxxxx \= xx \= \kill
        \> {\tt epoch}  \> = \> epoch of elements $t_0$ (TDB MJD) \\
        \> {\tt orbinc} \> = \> inclination $i$ (radians) \\
        \> {\tt anode}  \> = \> longitude of the ascending node $\Omega$ (radians) \\
        \> {\tt perih}  \> = \> argument of perihelion $\omega$ (radians) \\
        \> {\tt aorq}   \> = \> mean distance $a$ (AU) \\
        \> {\tt e}      \> = \> eccentricity $e$ \\
        \> {\tt aorl}   \> = \> mean anomaly $M$ (radians)
        \end{tabbing}
 
        ${\tt jform}=3$, suitable for comets:
 
        \begin{tabbing}
        xxx \= xxxxxxx \= xx \= \kill
        \> {\tt epoch}  \> = \> epoch of perihelion $T$ (TDB MJD) \\
        \> {\tt orbinc} \> = \> inclination $i$ (radians) \\
        \> {\tt anode}  \> = \> longitude of the ascending node $\Omega$ (radians) \\
        \> {\tt perih}  \> = \> argument of perihelion $\omega$ (radians) \\
        \> {\tt aorq}   \> = \> perihelion distance $q$ (AU) \\
        \> {\tt e}      \> = \> eccentricity $e$
        \end{tabbing}
        Unused elements ({\tt dm} for ${\tt jform}=2$,
        {\tt aorl} and {\tt dm} for ${\tt jform}=3$) are
        not accessed.
 
  \item Each of the three element sets defines an unperturbed heliocentric
        orbit.  For a given epoch of observation, the position of the body
        in its orbit can be predicted from these elements, which are
        called {\it osculating elements,}\/
        using standard two-body analytical
        solutions.  However, due to planetary perturbations, a given set
        of osculating elements remains usable for only as long as the
        unperturbed orbit that it describes is an adequate approximation
        to reality.  Attached to such a set of elements is a date called
        the {\it osculating epoch,}\/
        at which the elements are, momentarily,
        a perfect representation of the instantaneous position and
        velocity of the body.
 
        \vspace{1ex}
 
        Therefore, for any given problem there are up to three different
        epochs in play, and it is vital to distinguish clearly between
        them:
        \begin{itemize}
        \item The epoch of observation:  the moment in time for which the
              position of the body is to be predicted.
        \item The epoch defining the position of the body:  the moment
              in time at which, in the absence of purturbations, the
              specified position---mean longitude, mean anomaly, or
              perihelion---is reached.
        \item The osculating epoch:  the moment in time at which the
              given elements are correct.
        \end{itemize}
        For the major-planet and minor-planet cases it is usual to make
        the epoch that defines the position of the body the same as the
        epoch of osculation.  Thus, only two different epochs are
        involved:  the epoch of the elements and the epoch of observation.
        For comets, the epoch of perihelion fixes the position in the
        orbit and in general a different epoch of osculation will be
        chosen.  Thus, all three types of epoch are involved.
 
        \vspace{1ex}
 
        \goodbreak
        For the present function:
        \begin{itemize}
        \item The epoch of observation is the argument {\tt date}.
        \item The epoch defining the position of the body is the argument
              {\tt epoch}.
        \item The osculating epoch is not used and is assumed to be
              close enough to the epoch of observation to deliver
              adequate accuracy. If not, a preliminary call to
              {\tt slaPertel} may be used to update the element-set (and
              its associated osculating epoch) by
              applying planetary perturbations.
        \end{itemize}
  \item The reference frame for the result is equatorial and is with
        respect to the mean equinox and ecliptic of epoch J2000.
  \item The algorithm was originally adapted from the {\tt EPHSLA}
        Fortran program of
        D.\,H.\,P.\,Jones (private communication, 1996).  The method
        is based on Stumpff's Universal Variables.
 \end{enumerate}
}
\aref{Everhart, E. \& Pitkin, E.T.\ 1983, {\it Am.J.Phys.}\
      {\bf 51}, 712.}
%------------------------------------------------------------------------------
\routine{slaPlanet}{planetary ephemerides, approximate}
{
 \action{Approximate heliocentric position and velocity of a planet.}
 \call{slaPlanet( date, np, pv, \&jstat );}
}
\args{GIVEN}
{
 \spec{date}{double}{Modified Julian Date (JD$-$2400000.5)} \\
 \spec{np}{int}{planet:} \\
 \spec{}{}{\hspace{1.5em} 1\,=\,Mercury} \\
 \spec{}{}{\hspace{1.5em} 2\,=\,Venus} \\
 \spec{}{}{\hspace{1.5em} 3\,=\,Earth-Moon Barycentre} \\
 \spec{}{}{\hspace{1.5em} 4\,=\,Mars} \\
 \spec{}{}{\hspace{1.5em} 5\,=\,Jupiter} \\
 \spec{}{}{\hspace{1.5em} 6\,=\,Saturn} \\
 \spec{}{}{\hspace{1.5em} 7\,=\,Uranus} \\
 \spec{}{}{\hspace{1.5em} 8\,=\,Neptune} \\
 \spec{}{}{\hspace{1.5em} 9\,=\,Pluto}
}
\args{RETURNED}
{
 \spec{pv}{double[6]}{heliocentric \xyzxyzd, equatorial, J2000} \\
 \spec{}{}{\hspace{1.5em} (AU, AU/s)} \\
 \spec{jstat}{int*}{status:} \\
 \spec{}{}{\hspace{1.5em} $+$1 = warning: date outside of range} \\
 \spec{}{}{\hspace{2.3em}    0 = OK} \\
 \spec{}{}{\hspace{1.5em} $-$1 = illegal {\tt np} (outside 1-9)} \\
 \spec{}{}{\hspace{1.5em} $-$2 = solution didn't converge}
}
\notes
{
 \begin{enumerate}
  \item The epoch, {\tt date},
        is in the TDB time scale and is in the form
        of a Modified Julian Date (JD$-$2400000.5).  TT will do,
        but not UTC.
  \item The reference frame is equatorial and is with respect to
        the mean equinox and ecliptic of epoch J2000.
  \item If a planet number, {\tt np},
        outside the range 1-9 is supplied, an error
        status is returned (${\tt jstat}=-1$)
        and the {\tt pv} vector is set to zeroes.
  \item The algorithm for obtaining the mean elements of the
        planets from Mercury to Neptune is due to
        J.\,L.\,Simon, P.\,Bretagnon, J.\,Chapront,
        M.\,Chapront-Touze, G.\,Francou and J.\,Laskar (Bureau des
        Longitudes, Paris, France).  The (completely different)
        algorithm for calculating the ecliptic coordinates of
        Pluto is by Meeus.
  \item Comparisons of the present function with the JPL DE200 ephemeris
        give the following RMS errors over the interval 1960-2025:
        \begin{tabbing}
         xxxxx \= xxxxxxxxxxxxxxxxx \= xxxxxxxxxxxxxx \= \kill
         \> \> {\it position (km)} \> {\it speed (metre/sec)} \\ \\
         \> Mercury \> \hspace{2em}334 \> \hspace{2.5em}0.437 \\
         \> Venus   \> \hspace{1.5em}1060 \> \hspace{2.5em}0.855 \\
         \> EMB     \> \hspace{1.5em}2010 \> \hspace{2.5em}0.815 \\
         \> Mars    \> \hspace{1.5em}7690 \> \hspace{2.5em}1.98 \\
         \> Jupiter \> \hspace{1em}71700 \> \hspace{2.5em}7.70 \\
         \> Saturn  \> \hspace{0.5em}199000 \> \hspace{2em}19.4 \\
         \> Uranus  \> \hspace{0.5em}564000 \> \hspace{2em}16.4 \\
         \> Neptune \> \hspace{0.5em}158000 \> \hspace{2em}14.4 \\
         \> Pluto \> \hspace{1em}36400 \> \hspace{2.5em}0.137
        \end{tabbing}
        From comparisons with DE102,
        Simon {\it et al.}\/ quote the following
        longitude accuracies over the interval 1800-2200:
        \begin{tabbing}
         xxxxx \= xxxxxxxxxxxxxxxxxxxx \= \kill
         \> Mercury \> \hspace{0.5em}\arcseci{4} \\
         \> Venus   \> \hspace{0.5em}\arcseci{5} \\
         \> EMB     \> \hspace{0.5em}\arcseci{6} \\
         \> Mars    \> \arcseci{17} \\
         \> Jupiter \> \arcseci{71} \\
         \> Saturn  \> \arcseci{81} \\
         \> Uranus  \> \arcseci{86} \\
         \> Neptune \> \arcseci{11}
        \end{tabbing}
        In the case of Pluto, Meeus quotes an accuracy of \arcsec{0}{6}
        in longitude and \arcsec{0}{2} in latitude for the period
        1885-2099.
 
        For all except Pluto, over the period 1000-3000,
        the accuracy is better than 1.5
        times that over 1800-2200.  Outside the interval 1000-3000 the
        accuracy declines.  For Pluto the accuracy declines rapidly
        outside the period 1885-2099.  Outside these ranges
        (1885-2099 for Pluto, 1000-3000 for the rest) a ``date out
        of range'' warning status (${\tt jstat}=+1$) is returned.
  \item The algorithms for (i)~Mercury through Neptune and
        (ii)~Pluto are completely independent.  In the Mercury
        through Neptune case, the present SLALIB
        implementation differs from the original
        Simon {\it et al.}\/ Fortran code in the following respects:
        \begin{itemize}
         \item The date is supplied as a Modified Julian Date rather
               a Julian Date (${\rm MJD} = {\rm JD} - 2400000.5$).
         \item The result is returned only in equatorial
               Cartesian form;  the ecliptic
               longitude, latitude and radius vector are not returned.
         \item The velocity is in AU per second, not AU per day.
         \item Different error/warning status values are used.
         \item Kepler's Equation is not solved inline.
         \item Polynomials in T are nested to minimize rounding errors.
         \item Explicit double-precision constants are used to avoid
               mixed-mode expressions.
         \item There are other, cosmetic, changes to comply with
               Starlink/SLALIB style guidelines.
        \end{itemize}
        None of the above changes affects the result significantly.
  \item For ${\tt np} = 3$ the result is
        for the Earth-Moon Barycentre.  To
        obtain the heliocentric position and velocity of the Earth,
        either use the SLALIB function {\tt slaEvp} (or {\tt slaEpv})
        or call {\tt slaDmoon} and
        subtract 0.012150581 times the geocentric Moon vector from
        the EMB vector produced by the present function.  (The Moon
        vector should be precessed to J2000 first, but this can
        be omitted for modern epochs without introducing significant
        inaccuracy.)
 \end{enumerate}
}
\refs
{
 \begin{enumerate}
  \item Simon {\it et al.} 1994, {\it Astron.Astrophys.}, {\bf 282}, 663.
  \item Meeus, J.\ 1991. {\it Astronomical Algorithms,} Willmann-Bell.
 \end{enumerate}
}
%------------------------------------------------------------------------------
\routine{slaPlante}{\radec\ of planet from elements}
{
 \action{Topocentric apparent \radec\ of a Solar-System object whose
         heliocentric orbital elements are known.}
 \call{slaPlante( \vtop{
         \hbox{date, elong, phi, jform, epoch, orbinc, anode, perih,}
         \hbox{aorq, e, aorl, dm, \&ra, \&dec, \&r, \&jstat);}}}
}
\args{GIVEN}
{
 \spec{date}{double}{TDB MJD of observation (JD$-$2400000.5,} \\
 \spec{}{}{\hspace{1.5em} Notes~1,5)} \\
 \spec{elong,phi}{double}{observer's longitude (east +ve) and latitude} \\
 \spec{}{}{\hspace{1.5em} (radians, Note~2)} \\
 \spec{jform}{int}{choice of element set (1-3, Notes~3-6)} \\
 \spec{epoch}{double}{epoch of elements ($t_0$ or $T$, TDB MJD, Note~5)} \\
 \spec{orbinc}{double}{inclination ($i$, radians)} \\
 \spec{anode}{double}{longitude of the ascending node ($\Omega$, radians)} \\
 \spec{perih}{double}{longitude or argument of perihelion
                            ($\varpi$ or $\omega$,} \\
 \spec{}{}{\hspace{1.5em} radians)} \\
 \spec{aorq}{double}{mean distance or perihelion distance ($a$ or $q$, AU)} \\
 \spec{e}{double}{eccentricity ($e$)} \\
 \spec{aorl}{double}{mean anomaly or longitude ($M$ or $L$, radians,} \\
 \spec{}{}{\hspace{1.5em} ${\tt jform}=1,2$ only)} \\
 \spec{dm}{double}{daily motion ($n$, radians, ${\tt jform}=1$ only)}
}
\args{RETURNED}
{
 \spec{ra,dec}{double*}{topocentric apparent \radec\ (radians)} \\
 \spec{r}{double*}{distance from observer (AU)} \\
 \spec{jstat}{int*}{status:} \\
 \spec{}{}{\hspace{2.3em}    0 = OK} \\
 \spec{}{}{\hspace{1.5em} $-$1 = illegal {\tt jform}} \\
 \spec{}{}{\hspace{1.5em} $-$2 = illegal {\tt e}} \\
 \spec{}{}{\hspace{1.5em} $-$3 = illegal {\tt aorq}} \\
 \spec{}{}{\hspace{1.5em} $-$4 = illegal {\tt dm}} \\
 \spec{}{}{\hspace{1.5em} $-$5 = numerical error}
}
\notes
{
 \begin{enumerate}
  \item {\tt date} is the instant for which the prediction is
        required.  It is in the TDB time-scale (TT will do) and is a
        Modified Julian Date (JD$-$2400000.5).
  \item The longitude and latitude allow correction for geocentric
        parallax.  This is usually a small effect, but can become
        important for near-Earth asteroids.  Geocentric positions
        can be generated by appropriate use of the functions
        {\tt slaEvp} (or {\tt slaEpv}) and {\tt slaPlanel}.
  \item The elements are with respect to the J2000 ecliptic and equinox.
  \item A choice of three different element-format options is available, as
        follows. \\
 
        ${\tt jform}=1$, suitable for the major planets:
 
        \begin{tabbing}
        xxx \= xxxxxxx \= xx \= \kill
        \> {\tt epoch}  \> = \> epoch of elements $t_0$ (TDB MJD) \\
        \> {\tt orbinc} \> = \> inclination $i$ (radians) \\
        \> {\tt anode}  \> = \> longitude of the ascending node $\Omega$ (radians) \\
        \> {\tt perih}  \> = \> longitude of perihelion $\varpi$ (radians) \\
        \> {\tt aorq}   \> = \> mean distance $a$ (AU) \\
        \> {\tt e}      \> = \> eccentricity $e$ \\
        \> {\tt aorl}   \> = \> mean longitude $L$ (radians) \\
        \> {\tt dm}     \> = \> daily motion $n$ (radians)
        \end{tabbing}
 
        ${\tt jform}=2$, suitable for minor planets:
 
        \begin{tabbing}
        xxx \= xxxxxxx \= xx \= \kill
        \> {\tt epoch}  \> = \> epoch of elements $t_0$ (TDB MJD) \\
        \> {\tt orbinc} \> = \> inclination $i$ (radians) \\
        \> {\tt anode}  \> = \> longitude of the ascending node $\Omega$ (radians) \\
        \> {\tt perih}  \> = \> argument of perihelion $\omega$ (radians) \\
        \> {\tt aorq}   \> = \> mean distance $a$ (AU) \\
        \> {\tt e}      \> = \> eccentricity $e$ \\
        \> {\tt aorl}   \> = \> mean anomaly $M$ (radians)
        \end{tabbing}
 
        ${\tt jform}=3$, suitable for comets:
 
        \begin{tabbing}
        xxx \= xxxxxxx \= xx \= \kill
        \> {\tt epoch}  \> = \> epoch of perihelion $T$ (TDB MJD) \\
        \> {\tt orbinc} \> = \> inclination $i$ (radians) \\
        \> {\tt anode}  \> = \> longitude of the ascending node $\Omega$ (radians) \\
        \> {\tt perih}  \> = \> argument of perihelion $\omega$ (radians) \\
        \> {\tt aorq}   \> = \> perihelion distance $q$ (AU) \\
        \> {\tt e}      \> = \> eccentricity $e$
        \end{tabbing}
        Unused elements ({\tt dm} for ${\tt jform}=2$,
        {\tt aorl} and {\tt dm} for ${\tt jform}=3$) are
        not accessed.
 
  \item Each of the three element sets defines an unperturbed heliocentric
        orbit.  For a given epoch of observation, the position of the body
        in its orbit can be predicted from these elements, which are
        called {\it osculating elements,}\/
        using standard two-body analytical
        solutions.  However, due to planetary perturbations, a given set
        of osculating elements remains usable for only as long as the
        unperturbed orbit that it describes is an adequate approximation
        to reality.  Attached to such a set of elements is a date called
        the {\it osculating epoch,}\/
        at which the elements are, momentarily,
        a perfect representation of the instantaneous position and
        velocity of the body.
 
        \vspace{1ex}
 
        Therefore, for any given problem there are up to three different
        epochs in play, and it is vital to distinguish clearly between
        them:
        \begin{itemize}
        \item The epoch of observation:  the moment in time for which the
              position of the body is to be predicted.
        \item The epoch defining the position of the body:  the moment
              in time at which, in the absence of purturbations, the
              specified position---mean longitude, mean anomaly, or
              perihelion---is reached.
        \item The osculating epoch:  the moment in time at which the
              given elements are correct.
        \end{itemize}
        For the major-planet and minor-planet cases it is usual to make
        the epoch that defines the position of the body the same as the
        epoch of osculation.  Thus, only two different epochs are
        involved:  the epoch of the elements and the epoch of observation.
        For comets, the epoch of perihelion fixes the position in the
        orbit and in general a different epoch of osculation will be
        chosen.  Thus, all three types of epoch are involved.
 
        \vspace{1ex}
 
        \goodbreak
        For the present function:
        \begin{itemize}
        \item The epoch of observation is the argument {\tt date}.
        \item The epoch defining the position of the body is
              the argument {\tt epoch}.
        \item The osculating epoch is not used and is assumed to be
              close enough to the epoch of observation to deliver
              adequate accuracy. If not, a preliminary call to
              {\tt slaPertel} may be used to update the element-set (and
              its associated osculating epoch) by
              applying planetary perturbations.
        \end{itemize}
  \item Two important sources for orbital elements are {\it Horizons,}\/
        operated by the Jet Propulsion Laboratory, Pasadena,
        and the {\it Minor Planet Center,}\/ operated by the Center for
        Astrophysics, Harvard.  For further details, see Section~\ref{ephem}.
 \end{enumerate}
}
%------------------------------------------------------------------------------
\routine{slaPlantu}{\radec\ from universal elements}
{
 \action{Topocentric apparent \radec\ of a Solar-System object whose
         heliocentric universal orbital elements are known.}
 \call{slaPlantu( date, elong, phi, u, \&ra, \&dec, \&r, \&jstat );}
}
\args{GIVEN}
{
 \spec{date}{double}{TDB MJD of observation (JD$-$2400000.5)} \\
 \spec{elong,phi}{double}{observer's longitude (east +ve) and latitude} \\
 \spec{}{}{\hspace{1.5em} radians)}
}
\args{GIVEN and RETURNED}
{
 \spec{u}{double[13]}{universal orbital elements} \\
 \specel{0}     {combined mass ($M+m$)} \\
 \specel{1}     {total energy of the orbit ($\alpha$)} \\
 \specel{2}     {reference (osculating) epoch ($t_0$)} \\
 \specel{3-5}   {position at reference epoch (${\rm \bf r}_0$)} \\
 \specel{6-8}   {velocity at reference epoch (${\rm \bf v}_0$)} \\
 \specel{9}     {heliocentric distance at reference epoch} \\
 \specel{10}    {${\rm \bf r}_0.{\rm \bf v_0}$} \\
 \specel{11}    {date ($t$)} \\
 \specel{12}    {universal eccentric anomaly ($\psi$) of date, approx}
}
\args{RETURNED}
{
 \spec{ra,dec}{double*}{topocentric apparent \radec\ (radians)} \\
 \spec{r}{double*}{distance from observer (AU)} \\
 \spec{jstat}{int*}{status:} \\
 \spec{}{}{\hspace{2.3em}    0 = OK} \\
 \spec{}{}{\hspace{1.5em} $-$1 = radius vector zero} \\
 \spec{}{}{\hspace{1.5em} $-$2 = failed to converge}
}
\notes
{
 \begin{enumerate}
  \item {\tt date} is the instant for which the prediction is
        required.  It is in the TDB time-scale (TT will do) and is a
        Modified Julian Date (JD$-$2400000.5).
  \item The longitude and latitude allow correction for geocentric
        parallax.  This is usually a small effect, but can become
        important for near-Earth asteroids.  Geocentric positions
        can be generated by appropriate use of the functions
        {\tt slaEvp} (or {\tt slaEpv}) and {\tt slaUe2pv}.
  \item The ``universal'' elements are those which define the orbit for the
        purposes of the method of universal variables (see reference 2).
        They consist of the combined mass of the two bodies, an epoch,
        and the position and velocity vectors (arbitrary reference frame)
        at that epoch.  The parameter set used here includes also various
        quantities that can, in fact, be derived from the other
        information.  This approach is taken to avoiding unnecessary
        computation and loss of accuracy.  The supplementary quantities
        are (i)~$\alpha$, which is proportional to the total energy of the
        orbit, (ii)~the heliocentric distance at epoch,
        (iii)~the outwards component of the velocity at the given epoch,
        (iv)~an estimate of $\psi$, the ``universal eccentric anomaly'' at a
        given date and (v)~that date.
  \item The universal elements are with respect to the J2000 ecliptic
        and equinox.
 \end{enumerate}
}
\refs{
   \begin{enumerate}
   \item Sterne, Theodore E.\ 1960,
         {\it An Introduction to Celestial Mechanics,}
         Interscience Publishers, Section 6.7, p199.
   \item Everhart, E.\ \& Pitkin, E.T.\ 1983, {\it Am.J.Phys.},
         {\bf 51}, 712.
   \end{enumerate}
}
%------------------------------------------------------------------------------
\routine{slaPm}{proper motion}
{
 \action{Apply corrections for proper motion to a star \radec.}
 \call{slaPm( r0, d0, pr, pd, px, rv, ep0, ep1, \&r1, \&d1 );}
}
\args{GIVEN}
{
 \spec{r0,d0}{double}{\radec\ at epoch {\tt ep0} (radians)} \\
 \spec{pr,pd}{double}{proper motions:  rate of change of
                 \radec\  (radians per year)} \\
 \spec{px}{double}{parallax (arcsec)} \\
 \spec{rv}{double}{radial velocity (km~s$^{-1}$, +ve if receding)} \\
 \spec{ep0}{double}{start epoch in years ({\it e.g.}\ Julian epoch)} \\
 \spec{ep1}{double}{end epoch in years (same system as {\tt ep0})}
}
\args{RETURNED}
{
 \spec{r1,d1}{double*}{\radec\ at epoch {\tt ep1} (radians)}
}
\notes
{
\begin{enumerate}
\item The $\alpha$ proper motions are $\dot{\alpha}$ rather than
      $\dot{\alpha}\cos\delta$, and are in the same coordinate
      system as {\tt r0},{\tt d0}.
\item If the available proper motions are pre-FK5 they will be per
      tropical year rather than per Julian year, and so the epochs
      must both be Besselian rather than Julian.  In such cases, a
      scaling factor of 365.2422/365.25 should be applied to the
      radial velocity before use also.
\end{enumerate}
}
\refs
{
 \begin{enumerate}
  \item 1984 {\it Astronomical Almanac}, pp B39-B41.
  \item Lederle \& Schwan 1984, {\it Astr. Astrophys.},
        {\bf 134}, 1-6.
 \end{enumerate}
}
%-----------------------------------------------------------------------
\routine{slaPolmo}{polar motion}
{
 \action{Polar motion:  correct site longitude and latitude for polar
         motion and calculate azimuth difference between celestial and
         terrestrial poles.}
 \call{slaPolmo( elongm, phim, xp, yp, \&elong, \&phi, \&daz );}
}
\args{GIVEN}
{
 \spec{elongm}{double}{mean longitude of the site (radians, east +ve)} \\
 \spec{phim}{double}{mean geodetic latitude of the site (radians)} \\
 \spec{xp}{double}{polar motion $x$-coordinate (radians)} \\
 \spec{yp}{double}{polar motion $y$-coordinate (radians)}
}
\args{RETURNED}
{
 \spec{elong}{double*}{true longitude of the site (radians, east +ve)} \\
 \spec{phi}{double*}{true geodetic latitude of the site (radians)} \\
 \spec{daz}{double*}{azimuth correction (terrestrial$-$celestial, radians)}
}
\notes
{
\begin{enumerate}
\item Polar motion can be regarded as those components of the
      changing orientation of the Earth's axis that are around 1~day
      in period, compared with the much slower precession-nutation
      effects.  The simplest way of dealing with polar motion is to apply
      a full 3D coordinate rotation, transforming
      between the International Celestial
      Reference System (ITRS) and the Terrestrial Intermediate Reference
      System.  However, it has long been a tradition instead to regard
      polar motion as something that leads to small variations
      in the latitude and
      longitude of the observing site, and the SLALIB routines
      {\tt slaAop} {\it etc.}\/ work in this way.  However, the traditional
      treatment omits one component of the transformation, namely the
      changing azimuth of the observing site reckoned with respect
      to the Celestial Intermediate Pole (CIP).  The present routine
      addresses that missing element.
\item ``Mean'' longitude and latitude are the (quasi-fixed) values for the
      site's location with respect to the ITRS;
      the latitude is geodetic.
      {\bf Important : take care with the longitude sign convention.}
      The longitudes used by the present function
      are east-positive, in accordance with geographical convention
      (and right-handed).  In particular, note that the longitudes
      returned by the {\tt slaObs} function are west-positive, following
      astronomical usage, and must be reversed in sign before use in
      the present function.
\item {\tt xp} and {\tt yp} are the (changing)
      coordinates of the CIP with respect to the ITRS pole.
      {\tt xp} is positive along the meridian at longitude $0^\circ$,
      and {\tt yp} is positive along the meridian at longitude
      $270^\circ$ ({\it i.e.}~$90^\circ$ west).  Values for
      {\tt xp},{\tt yp} can
      be obtained from IERS circulars and equivalent publications;
      the maximum amplitude observed so far is about \arcsec{0}{3}.
\item ``True'' longitude and latitude are the (changing) values for
      the site's location with respect to the CIP
      and the ITRS meridian which corresponds to the Greenwich
      apparent sidereal time (for right ascensions referred to the
      equinox; Earth Rotation Angle for right ascensions
      referred to the Celestial Intermediate Origin).
      The true longitude and latitude
      link the terrestrial coordinates with the standard celestial
      models (for precession, nutation, sidereal time {\it etc}).
\item The azimuths produced by {\tt slaAop} and {\tt slaAopqk} are with
      respect to due north as defined by the CIP,
      and can therefore be called ``celestial azimuths''.
      However, a telescope fixed to the Earth measures azimuth
      essentially with respect to due north as defined by the
      ITRS Pole, and can therefore be called ``terrestrial
      azimuth''.  Uncorrected, this would manifest itself as a
      changing ``azimuth zero-point error''.  The value {\tt daz} is the
      correction to be added to a celestial azimuth to produce
      a terrestrial azimuth.
\item The present function is rigorous.  For most practical
      purposes, the following simplified formulae provide an
      adequate approximation: \\[2ex]
      \hspace*{1em}\begin{tabular}{lll}
        {\tt elong} & {\tt =} &
             {\tt elongm+xp*cos(elongm)-yp*sin(elongm);} \\
        {\tt phi  } & {\tt =} &
             {\tt phim+(xp*sin(elongm)+yp*cos(elongm))*tan(phim);} \\
        {\tt daz  } & {\tt =} &
             {\tt -SQRT(xp*xp+yp*yp)*cos(elongm-atan2(xp,yp))/cos(phim);} \\
      \end{tabular} \\[2ex]
      An alternative formulation for daz is:\\[2ex]
      \hspace*{1em}\begin{tabular}{lll}
        {\tt x  } & {\tt =} & {\tt cos(elongm)*cos(phim);} \\
        {\tt y  } & {\tt =} & {\tt sin(elongm)*cos(phim);} \\
        {\tt daz} & {\tt =} & {\tt atan2(-x*yp-y*xp,x*x+y*y);} \\
      \end{tabular}
\end{enumerate}
}
\aref{Seidelmann, P.K.\ (ed), 1992.  {\it Explanatory
      Supplement to the Astronomical Almanac,}\/ ISBN~0-935702-68-7,
      sections 3.27, 4.25, 4.52.}
%-----------------------------------------------------------------------
\routine{slaPrebn}{precession matrix (FK4)}
{
 \action{Generate the matrix of precession between two epochs,
         using the old, pre IAU~1976, Bessel-Newcomb model, in
         Andoyer's formulation.}
 \call{slaPrebn( bep0, bep1, rmatp );}
}
\args{GIVEN}
{
 \spec{bep0}{double}{beginning Besselian epoch} \\
 \spec{bep1}{double}{ending Besselian epoch}
}
\args{RETURNED}
{
 \spec{rmatp}{double[3][3]}{precession matrix}
}
\anote{The matrix is in the sense:
       \begin{verse}
        ${\bf v}_{1} =  {\bf M}\times{\bf v}_{0}$
       \end{verse}
       where ${\bf v}_{1}$ is the star vector relative to the
       mean equator and equinox of epoch {\tt bep1}, {\bf M} is the
       $3\times3$ matrix {\tt rmatp} and
       ${\bf v}_{0}$ is the star vector relative to the
       mean equator and equinox of epoch {\tt bep0}.}
\aref{Smith {\it et al.}\ 1989, {\it Astr.J.}\ {\bf 97}, 269.}
%-----------------------------------------------------------------------
\routine{slaPrec}{precession matrix (IAU 1976)}
{
 \action{Form the matrix of precession between two epochs (IAU 1976, FK5).}
 \call{slaPrec( ep0, ep1, rmatp );}
}
\args{GIVEN}
{
 \spec{ep0}{double}{beginning epoch} \\
 \spec{ep1}{double}{ending epoch}
}
\args{RETURNED}
{
 \spec{rmatp}{double[3][3]}{precession matrix}
}
\notes
{
 \begin{enumerate}
  \item The epochs are TDB Julian epochs.  TT (or even UTC) will do.
  \item The matrix is in the sense:
        \begin{verse}
         ${\bf v}_{1} = {\bf M}\times{\bf v}_{0}$
        \end{verse}
        where {\bf v}$_{1}$ is the star vector relative to the
        mean equator and equinox of epoch {\tt ep1}, {\bf M} is the
        $3\times3$ matrix {\tt rmatp} and
        {\bf v}$_{0}$ is the star vector relative to the
        mean equator and equinox of epoch {\tt ep0}.
  \item Though the matrix method itself is rigorous, the precession
        angles are expressed through canonical polynomials which are
        valid only for a limited time span.  There are also
        errors in the IAU 1976 precession rate at the level of
        \arcsec{0}{3} per century.
        The absolute accuracy
        of the present formulation is better
        than \arcsec{0}{1} from
        1960\,AD to 2040\,AD, better than
        \arcseci{1} from 1640\,AD to 2360\,AD,
        and remains below \arcseci{3} for the whole of the period
        500\,BC to 3000\,AD.  The errors exceed \arcseci{10} outside the
        range 1200\,BC to 3900\,AD, exceed \arcseci{100} outside 4200\,BC to
        5600\,AD and exceed \arcseci{1000} outside 6800\,BC to 8200\,AD.
        The SLALIB function {\tt slaPrecl} implements a more elaborate
        model which is suitable for problems spanning several
        thousand years.
 \end{enumerate}
}
\refs
{
 \begin{enumerate}
  \item Lieske, J.H.\ 1979, {\it Astr.Astrophys.}, {\bf 73}, 282;
        equations 6 \& 7, p283.
  \item Kaplan, G.H.\ 1981, {\it USNO circular no.\ 163}, pA2.
 \end{enumerate}
}
%-----------------------------------------------------------------------
\routine{slaPreces}{precession}
{
 \action{Precession -- either the old ``FK4'' (Bessel-Newcomb, pre~IAU~1976)
         or newer ``FK5'' (Fricke, post~IAU~1976) as required.}
 \call{slaPreces( system, ep0, ep1, \&ra, \&dc );}
}
\args{GIVEN}
{
 \spec{system}{C}{precession to be applied:
                  \cstring{FK4} or \cstring{FK5}} \\
 \spec{ep0,ep1}{double}{starting and ending epoch} \\
 \spec{ra,dc}{double}{\radec, mean equator \& equinox of epoch {tt ep0}}
}
\args{RETURNED}
{
 \spec{ra,dc}{double*}{\radec, mean equator \& equinox of epoch {tt ep1}}
}
\notes
{
 \begin{enumerate}
  \item Lowercase characters in {\tt system} are acceptable.
  \item The epochs are Besselian if
        {\tt system}\,=\,\cstring{FK4} and Julian if \cstring{FK5}.
        For example, to precess coordinates in the old system from
        equinox 1900.0 to 1950.0 the call would be:
        \begin{quote}
         {\tt slaPreces( "FK4", 1900.0, 1950.0, \&ra, \&dc )}
        \end{quote}
  \item This function will {\bf NOT} correctly convert between the old and
        the new systems -- for example transformation from B1950 to J2000.
        For these purposes see {\tt slaFk425}, {\tt slaFk524}, {\tt slaFk45z} and
        {\tt slaFk54z}.
  \item If an invalid {\tt system}
        is supplied, values of $-$99.0,$-$99.0 are
        returned for both {\tt ra} and {\tt dc}.
 \end{enumerate}
}
%-----------------------------------------------------------------------
\routine{slaPrecl}{precession matrix (more accurate)}
{
 \action{Form the matrix of precession between two epochs, using the
         model of Simon {\it et al}.\ (1994), which is suitable for long
         periods of time.}
 \call{slaPrecl( ep0, ep1, rmatp );}
}
\args{GIVEN}
{
 \spec{ep0}{double}{beginning epoch} \\
 \spec{ep1}{double}{ending epoch}
}
\args{RETURNED}
{
 \spec{rmatp}{double[3][3]}{precession matrix}
}
\notes
{
 \begin{enumerate}
  \item The epochs are TDB Julian epochs.  TT (or even UTC) will do.
  \item The matrix is in the sense:
        \begin{verse}
         ${\bf v}_{1} = {\bf M}\times{\bf v}_{0}$
        \end{verse}
        where {\bf v}$_{1}$ is the star vector relative to the
        mean equator and equinox of epoch {\tt ep1}, {\bf M} is the
        $3\times3$ matrix {\tt rmatp} and
        {\bf v}$_{0}$ is the star vector relative to the
        mean equator and equinox of epoch {\tt ep0}.
  \item The absolute accuracy of the model is limited by the
        uncertainty in the general precession, about \arcsec{0}{3} per
        1000~years.  The remainder of the formulation provides a
        precision of 1~milliarcsecond over the interval from 1000\,AD
        to 3000\,AD, \arcsec{0}{1} from 1000\,BC to 5000\,AD and
        \arcseci{1} from 4000\,BC to 8000\,AD.
 \end{enumerate}
}
\aref{Simon, J.L.\ {\it et al}.\ 1994, {\it Astr.Astrophys.}\ {\bf 282},
      663.}
%-----------------------------------------------------------------------
\routine{slaPrenut}{precession-nutation matrix}
{
 \action{Form the matrix of precession and nutation (SF2001).}
 \call{slaPrenut( epoch, date, rmatpn );}
}
\args{GIVEN}
{
 \spec{epoch}{double}{Julian Epoch for mean coordinates} \\
 \spec{date}{double}{Modified Julian Date (JD$-$2400000.5)
                       for true coordinates}
}
\args{RETURNED}
{
 \spec{rmatpn}{double[3][3]}{combined precession-nutation matrix}
}
\notes
{
 \begin{enumerate}
  \item The epoch and date are TDB.  TT (or even UTC) will do.
  \item The matrix is in the sense:
        \begin{verse}
         ${\bf v}_{true} =  {\bf M}\times{\bf v}_{mean}$
        \end{verse}
        where {\bf v}$_{true}$ is the star vector relative to the
        true equator and equinox of {\tt date}, {\bf M} is the
        $3\times3$ matrix {\tt rmatpn} and
        {\bf v}$_{mean}$ is the star vector relative to the
        mean equator and equinox of {\tt epoch}.
 \end{enumerate}
}
%-----------------------------------------------------------------------
\routine{slaPv2el}{orbital elements from position \& velocity}
{
 \action{Heliocentric osculating elements obtained from instantaneous
         position and velocity.}
 \call{slaPv2el( \vtop{
         \hbox{pv, date, pmass, jformr, \&jform, \&epoch, \&orbinc,}
         \hbox{\&anode, \&perih, \&aorq, \&e, \&aorl, \&dm, \&jstat);}}}
}
\args{GIVEN}
{
 \spec{pv}{double[6]}{heliocentric \xyzxyzd, equatorial, J2000} \\
 \spec{}{}{\hspace{1.5em} (AU, AU/s; Note~1)} \\
 \spec{date}{double}{date (TT Modified Julian Date = JD$-$2400000.5)} \\
 \spec{pmass}{double}{mass of the planet (Sun = 1; Note~2)} \\
 \spec{jformr}{int}{requested element set (1-3; Note~3)}
}
\args{RETURNED}
{
 \spec{jform}{int*}{element set actually returned (1-3; Note~4)} \\
 \spec{epoch}{double*}{epoch of elements ($t_0$ or $T$, TDB MJD)} \\
 \spec{orbinc}{double*}{inclination ($i$, radians)} \\
 \spec{anode}{double*}{longitude of the ascending node ($\Omega$, radians)} \\
 \spec{perih}{double*}{longitude or argument of perihelion
                            ($\varpi$ or $\omega$,} \\
 \spec{}{}{\hspace{1.5em} radians)} \\
 \spec{aorq}{double}{mean distance or perihelion distance ($a$ or $q$, AU)} \\
 \spec{e}{double}{eccentricity ($e$)} \\
 \spec{aorl}{double}{mean anomaly or longitude
                               ($M$ or $L$, radians,} \\
 \spec{}{}{\hspace{1.5em} ${\tt jform}=1,2$ only)} \\
 \spec{dm}{double}{daily motion ($n$, radians, ${\tt jform}=1$ only)} \\
 \spec{jstat}{int*}{status:} \\
 \spec{}{}{\hspace{2.3em}    0 = OK} \\
 \spec{}{}{\hspace{1.5em} $-$1 = illegal {\tt pmass}} \\
 \spec{}{}{\hspace{1.5em} $-$2 = illegal {\tt jformr}} \\
 \spec{}{}{\hspace{1.5em} $-$3 = position/velocity out of allowed range}
}
\notes
{
 \begin{enumerate}
  \item The {\tt pv} 6-vector is with respect
        to the mean equator and equinox of
        epoch J2000.  The orbital elements produced are with respect to
        the J2000 ecliptic and mean equinox.
  \item The mass, {\tt pmass}, is important
        only for the larger planets.  For
        most purposes ({\it e.g.}~asteroids) use 0.0.
        Values less than zero are illegal.
  \item Three different element-format options are supported, as
        follows. \\
 
        ${\tt jform}=1$, suitable for the major planets:
 
        \begin{tabbing}
        xxx \= xxxxxxxx \= xx \= \kill
        \> {\tt epoch}  \> = \> epoch of elements $t_0$ (TDB MJD) \\
        \> {\tt orbinc} \> = \> inclination $i$ (radians) \\
        \> {\tt anode}  \> = \> longitude of the
                                ascending node $\Omega$ (radians) \\
        \> {\tt perih}  \> = \> longitude of perihelion
                                $\varpi$ (radians) \\
        \> {\tt aorq}   \> = \> mean distance $a$ (AU) \\
        \> {\tt e}      \> = \> eccentricity $e$ $( 0 \leq e < 1 )$ \\
        \> {\tt aorl}   \> = \> mean longitude $L$ (radians) \\
        \> {\tt dm}     \> = \> daily motion $n$ (radians)
        \end{tabbing}
 
        ${\tt jform}=2$, suitable for minor planets:
 
        \begin{tabbing}
        xxx \= xxxxxxxx \= xx \= \kill
        \> {\tt epoch}  \> = \> epoch of elements $t_0$ (TDB MJD) \\
        \> {\tt orbinc} \> = \> inclination $i$ (radians) \\
        \> {\tt anode}  \> = \> longitude of the ascending node
                                $\Omega$ (radians) \\
        \> {\tt perih}  \> = \> argument of perihelion
                                $\omega$ (radians) \\
        \> {\tt aorq}   \> = \> mean distance $a$ (AU) \\
        \> {\tt e}      \> = \> eccentricity $e$ $( 0 \leq e < 1 )$ \\
        \> {\tt aorl}   \> = \> mean anomaly $M$ (radians)
        \end{tabbing}
 
        ${\tt jform}=3$, suitable for comets:
 
        \begin{tabbing}
        xxx \= xxxxxxxx \= xx \= \kill
        \> {\tt epoch}  \> = \> epoch of perihelion $T$ (TDB MJD) \\
        \> {\tt orbinc} \> = \> inclination $i$ (radians) \\
        \> {\tt anode}  \> = \> longitude of the ascending node $\Omega$ (radians) \\
        \> {\tt perih}  \> = \> argument of perihelion $\omega$ (radians) \\
        \> {\tt aorq}   \> = \> perihelion distance $q$ (AU) \\
        \> {\tt e}   \> = \> eccentricity $e$ $( 0 \leq e \leq 10 )$
        \end{tabbing}
  \item It may not be possible to generate elements in the form
        requested through {\tt jformr}.  The caller is notified of the form
        of elements actually returned by means of the {\tt jform} argument:
 
        \begin{tabbing}
        xx \= xxxxxxxxxx \= xxxxxxxxxxx \= \kill
        \> {\tt jformr} \> {\tt jform} \> meaning \\ \\
        \> ~~~~~1   \> ~~~~1  \> OK: elements are in the requested format \\
        \> ~~~~~1   \> ~~~~2  \> never happens \\
        \> ~~~~~1   \> ~~~~3  \> orbit not elliptical \\
        \> ~~~~~2   \> ~~~~1  \> never happens \\
        \> ~~~~~2   \> ~~~~2  \> OK: elements are in the requested format \\
        \> ~~~~~2   \> ~~~~3  \> orbit not elliptical \\
        \> ~~~~~3   \> ~~~~1  \> never happens \\
        \> ~~~~~3   \> ~~~~2  \> never happens \\
        \> ~~~~~3   \> ~~~~3  \> OK: elements are in the requested format
        \end{tabbing}
  \item The arguments returned for each value of {\tt jform}
        ({\it cf.}\/ Note~5:
        {\tt jform} might not be the same as {\tt jformr}) are as follows:
 
        \begin{tabbing}
        xxx \= xxxxxxxxxxxx \= xxxxxx \= xxxxxx \= \kill
        \> {\tt jform}  \> 1        \> 2        \> 3 \\ \\
        \> {\tt epoch}  \> $t_0$    \> $t_0$    \> $T$ \\
        \> {\tt orbinc} \> $i$      \> $i$      \> $i$ \\
        \> {\tt anode}  \> $\Omega$ \> $\Omega$ \> $\Omega$ \\
        \> {\tt perih}  \> $\varpi$ \> $\omega$ \> $\omega$ \\
        \> {\tt aorq}   \> $a$      \> $a$      \> $q$ \\
        \> {\tt e}      \> $e$      \> $e$      \> $e$ \\
        \> {\tt aorl}   \> $L$      \> $M$      \> - \\
        \> {\tt dm}     \> $n$      \> -        \> -
        \end{tabbing}
 
        where:
        \begin{tabbing}
        xxx \= xxxxxxxx \= xxx \= \kill
        \> $t_0$    \> is the epoch of the elements (MJD, TT) \\
        \> $T$      \> is the epoch of perihelion (MJD, TT) \\
        \> $i$      \> is the inclination (radians) \\
        \> $\Omega$ \> is the longitude of the ascending node (radians) \\
        \> $\varpi$ \> is the longitude of perihelion (radians) \\
        \> $\omega$ \> is the argument of perihelion (radians) \\
        \> $a$      \> is the mean distance (AU) \\
        \> $q$      \> is the perihelion distance (AU) \\
        \> $e$      \> is the eccentricity \\
        \> $L$      \> is the longitude (radians, $0-2\pi$) \\
        \> $M$      \> is the mean anomaly (radians, $0-2\pi$) \\
        \> $n$      \> is the daily motion (radians) \\
        \> - \> means no value is set
        \end{tabbing}
  \item At very small inclinations, the longitude of the ascending node
        {\tt anode} becomes indeterminate
        and under some circumstances may be
        set arbitrarily to zero.  Similarly, if the orbit is close to
        circular, the true anomaly becomes indeterminate and under some
        circumstances may be set arbitrarily to zero.  In such cases,
        the other elements are automatically adjusted to compensate,
        and so the elements remain a valid description of the orbit.
  \item The osculating epoch for the returned elements is the argument
        {\tt date}.
 \end{enumerate}
}
\aref{Sterne, Theodore E.\ 1960,
      {\it An Introduction to Celestial Mechanics,}\/
      Interscience Publishers.}
%-----------------------------------------------------------------------
\routine{slaPv2ue}{position and velocity to universal elements}
{
 \action{Construct a universal element set based on an instantaneous
         position and velocity.}
 \call{slaPv2ue( pv, date, pmass, u, \&jstat );}
}
\args{GIVEN}
{
 \spec{pv}{double[6]}{heliocentric \xyzxyzd, equatorial, J2000} \\
 \spec{}{}{\hspace{1.5em} (AU, AU/s; Note~1)} \\
 \spec{date}{double}{date (TT Modified Julian Date = JD$-$2400000.5)} \\
 \spec{pmass}{double}{mass of the planet (Sun = 1; Note~2)}
}
\args{RETURNED}
{
 \spec{u}{double[13]}{universal orbital elements (Note~3)} \\
 \specel{0}     {combined mass ($M+m$)} \\
 \specel{1}     {total energy of the orbit ($\alpha$)} \\
 \specel{2}     {reference (osculating) epoch ($t_0$)} \\
 \specel{3-5}   {position at reference epoch (${\rm \bf r}_0$)} \\
 \specel{6-8}   {velocity at reference epoch (${\rm \bf v}_0$)} \\
 \specel{9}     {heliocentric distance at reference epoch} \\
 \specel{10}    {${\rm \bf r}_0.{\rm \bf v}_0$} \\
 \specel{11}    {date ($t$)} \\
 \specel{12}    {universal eccentric anomaly ($\psi$) of date, approx} \\
 \spec{jstat}{int*}{status:} \\
 \spec{}{}{\hspace{1.95em}      0 = OK} \\
 \spec{}{}{\hspace{1.2em}    $-$1 = illegal {\tt pmass}} \\
 \spec{}{}{\hspace{1.2em}    $-$2 = too close to Sun} \\
 \spec{}{}{\hspace{1.2em}    $-$3 = too slow}
}
\notes
{
 \begin{enumerate}
  \item The {\tt pv} 6-vector can be with
        respect to any chosen inertial frame,
        and the resulting universal-element set will be with respect to
        the same frame.  A common choice will be mean equator and ecliptic
        of epoch J2000.
  \item The mass, {\tt pmass}, is important only for the larger planets.  For
        most purposes ({\it e.g.}~asteroids) use 0.0.  Values less than zero
        are illegal.
  \item The ``universal'' elements are those which define the orbit for the
        purposes of the method of universal variables (see reference).
        They consist of the combined mass of the two bodies, an epoch,
        and the position and velocity vectors (arbitrary reference frame)
        at that epoch.  The parameter set used here includes also various
        quantities that can, in fact, be derived from the other
        information.  This approach is taken to avoiding unnecessary
        computation and loss of accuracy.  The supplementary quantities
        are (i)~$\alpha$, which is proportional to the total energy of the
        orbit, (ii)~the heliocentric distance at epoch,
        (iii)~the outwards component of the velocity at the given epoch,
        (iv)~an estimate of $\psi$, the ``universal eccentric anomaly'' at a
        given date and (v)~that date.
 \end{enumerate}
}
\aref{Everhart, E. \& Pitkin, E.T.\ 1983, {\it Am.J.Phys.}\
      {\bf 51}, 712.}
%-----------------------------------------------------------------------
\routine{slaPvobs}{observatory position \& velocity}
{
 \action{Position and velocity of an observing station.}
 \call{slaPvobs( p, h, stl, pv );}
}
\args{GIVEN}
{
 \spec{p}{double}{latitude (geodetic, radians)} \\
 \spec{h}{double}{height above reference spheroid (geodetic, metres)} \\
 \spec{stl}{double}{local apparent sidereal time (radians)}
}
\args{RETURNED}
{
 \spec{pv}{double[6]}{\xyzxyzd\ (AU, AU~s$^{-1}$, true equator and equinox
                                                            of date)}
}
\anote{IAU 1976 constants are used.}
%-----------------------------------------------------------------------
\routine{slaPxy}{apply linear model}
{
 \action{Given arrays of {\it expected}\/ and {\it measured}\,
         \xy\ coordinates, and a
         linear model relating them (as produced by slaFitxy), compute
         the array of {\it predicted}\/ coordinates and the RMS residuals.}
 \call{slaPxy( np, xye, xym, coeffs, xyp, \&xrms, \&yrms, \&rrms );}
}
\args{GIVEN}
{
 \spec{np}{int}{number of samples} \\
 \spec{xye}{double[2][np]}{expected \xy\ for each sample} \\
 \spec{xym}{double[2][np]}{measured \xy\ for each sample} \\
 \spec{coeffs}{double[6]}{coefficients of model (see below)}
}
\args{RETURNED}
{
 \spec{xyp}{double[2][np]}{predicted \xy\ for each sample} \\
 \spec{xrms}{double*}{RMS in $x$} \\
 \spec{yrms}{double*}{RMS in $y$} \\
 \spec{rrms}{double*}{total RMS (RSS of {\tt xrms} and {\tt yrms})}
}
\notes
{
 \begin{enumerate}
  \item The model is supplied in the array {\tt coeffs}.  Naming the
        six elements of {\tt coeffs} $a,b,c,d,e$ \& $f$,
        the model transforms {\it measured}\/ coordinates
        $[x_{m},y_{m}\,]$ into {\it predicted}\/ coordinates
        $[x_{p},y_{p}\,]$ as follows:
        \begin{verse}
         $x_{p} = a + bx_{m} + cy_{m}$ \\
         $y_{p} = d + ex_{m} + fy_{m}$
        \end{verse}
  \item The residuals are $(x_{p}-x_{e})$ and $(y_{p}-y_{e})$.
  \item If {\tt np} is less than or equal to zero, no coordinates are
        transformed, and the RMS residuals are all zero.
  \item See also {\tt slaFitxy}, {\tt slaInvf}, {\tt slaXy2xy}, {\tt slaDcmpf}
 \end{enumerate}
}
%-----------------------------------------------------------------------
\routine{slaRange}{put angle into range $\pm\pi$}
{
 \action{Normalize an angle into the range $\pm\pi$ (single precision).}
 \call{r~=~slaRange( angle );}
}
\args{GIVEN}
{
 \spec{angle}{float}{angle in radians}
}
\args{RETURNED}
{
 \spec{}{float}{{\tt angle} expressed in the range $\pm\pi$.}
}
%-----------------------------------------------------------------------
\routine{slaRanorm}{put angle into range $0\!-\!2\pi$}
{
 \action{Normalize an angle into the range $0\!-\!2\pi$ (single precision).}
 \call{r~=~slaRanorm( angle );}
}
\args{GIVEN}
{
 \spec{angle}{float}{angle in radians}
}
\args{RETURNED}
{
 \spec{}{float}{{\tt angle} expressed in the range $0\!-\!2\pi$}
}
%-----------------------------------------------------------------------
\routine{slaRcc}{barycentric coordinate time}
{
 \call{d~=~slaRcc( tdb, ut1, wl, u, v );}
 \action{The relativistic clock correction:
         the difference between {\it proper time}\/ at
         a point on the Earth and
         {\it coordinate time}\/ in the solar
         system barycentric space-time frame of reference.
         The proper time is Terrestrial Time, TT;
         the coordinate time is an implementation of Barycentric
         Dynamical Time, TDB.}
}
\args{GIVEN}
{
 \spec{tdb}{double}{TDB (MJD: JD$-$2400000.5)} \\
 \spec{ut1}{double}{universal time (fraction of one day)} \\
 \spec{wl}{double}{clock longitude (radians west)} \\
 \spec{u}{double}{clock distance from Earth spin axis (km)} \\
 \spec{v}{double}{clock distance north of Earth equatorial plane (km)}
}
\args{RETURNED}
{
 \spec{}{double}{TDB$-$TT (s)}
}
\notes
{
 \begin{enumerate}
  \item TDB is coordinate time in the solar system barycentre frame
        of reference, in units chosen to eliminate the scale difference
        with respect to terrestrial time.  TT is the proper
        time for clocks at mean sea level on the Earth.
  \item The number returned by {\tt slaRcc} is dominated by an
        annual sinusoidal term of amplitude
        approximately 1.66\,ms.  Its second harmonic is at
        14\,$\mu$s amplitude, slightly below the largest
        planetary term which is of 21\,$\mu$s amplitude.
        Diurnal terms start at 2\,$\mu$s.  The
        variation arises from the transverse Doppler effect and the
        gravitational red-shift as the observer varies in speed and
        moves through different gravitational potentials.
  \item The argument {\tt tdb}
        is, strictly, the barycentric coordinate time;
        however, the terrestrial time (TT) can in practice be used without
        significant loss of accuracy.
  \item The geocentric model is that of Fairhead \& Bretagnon (1990), in
        its full form.  It was supplied by Fairhead (private communication)
        as a Fortran subroutine.  A number of coding changes were made to
        this subroutine in order
        match the calling sequence of previous versions of the present
        function, to comply with Starlink programming standards and to
        avoid compilation problems on certain machines.  The
        numerical results are essentially unaffected by the
        changes.
  \item The topocentric model is from Moyer (1981) and Murray (1983).
        It is an approximation to the expression
        \[\frac{{\bf v}_e \cdot ( {\bf x} - {\bf x}_e )}{c^2}\]
        where ${\bf v}_e$ is the barycentric velocity of
        the Earth, ${\bf x}$ and ${\bf x}_e$ are the barycentric positions
        of the observer and the Earth respectively, and
        c is the speed of light.
        It can be disabled, if necessary, by setting the arguments
        {\tt u} and {\tt v} to zero.
  \item During the interval 1950-2050, the absolute accuracy
        is better than $\pm3$~nanoseconds
        relative to direct numerical integrations using the JPL DE200/LE200
        solar system ephemeris.
  \item The IAU 1976 definition of TDB
        was that it must differ from TT only by
        periodic terms.  Though practical, this is an imprecise definition
        that ignores the existence of very long-period and secular effects
        in the dynamics of the solar system.  As a consequence, different
        implementations of TDB will, in general, differ in zero-point and
        will drift linearly relative to one other.  In 1991
        the IAU introduced
        new time-scales designed to overcome these objections:
        geocentric coordinate
        time, TCG, and barycentric coordinate time, TCB.
        In principle, therefore,
        TDB is obsolete.  However, {\tt slaRcc}
        can be used to implement the periodic part of TCB$-$TCG.
  \item TDB is for all practical purposes the same as the ${\rm T}_{eph}$
        time-scale used with JPL solar-system ephemerides.
 \end{enumerate}
}
\refs
{
 \begin{enumerate}
  \item Fairhead,\,L., \& Bretagnon,\,P.\ 1990, {\it Astron.Astrophys.},
        {\bf 229}, 240-247.
  \item Moyer,\,T.D.\ 1981, {\it Cel.Mech.}, {\bf 23}, 33.
  \item Murray,\,C.A.\ 1983, {\it Vectorial Astrometry,} Adam Hilger.
  \item Seidelmann,\,P.K.\ {\it et al.}\ 1992,
        {\it Explanatory Supplement to the
        Astronomical Almanac}, Chapter 2, University Science Books.
  \item Simon,\,J.L., Bretagnon,\,P., Chapront,\,J., Chapront-Touze,\,M.,
        Francou,\,G.\ \& Laskar,\,J.\ 1994, {\it Astron.Astrophys.},
        {\bf 282}, 663-683.
 \end{enumerate}
}
%------------------------------------------------------------------------------
\routine{slaRdplan}{apparent \radec\ of planet}
{
 \action{Approximate topocentric apparent \radec\ and angular
         size of a planet.}
 \call{slaRdplan( date, np, elong, phi, \&ra, \&dec, \&diam );}
}
\args{GIVEN}
{
 \spec{date}{double}{MJD of observation (JD$-$2400000.5)} \\
 \spec{np}{int}{planet:} \\
 \spec{}{}{\hspace{1.5em} 1\,=\,Mercury} \\
 \spec{}{}{\hspace{1.5em} 2\,=\,Venus} \\
 \spec{}{}{\hspace{1.5em} 3\,=\,Moon} \\
 \spec{}{}{\hspace{1.5em} 4\,=\,Mars} \\
 \spec{}{}{\hspace{1.5em} 5\,=\,Jupiter} \\
 \spec{}{}{\hspace{1.5em} 6\,=\,Saturn} \\
 \spec{}{}{\hspace{1.5em} 7\,=\,Uranus} \\
 \spec{}{}{\hspace{1.5em} 8\,=\,Neptune} \\
 \spec{}{}{\hspace{1.5em} 9\,=\,Pluto} \\
 \spec{}{}{\hspace{0.44em} else\,=\,Sun} \\
 \spec{elong,phi}{double}{observer's longitude (east +ve) and latitude
                     (radians)}
}
\args{RETURNED}
{
 \spec{ra,dec}{double*}{topocentric apparent \radec\ (radians)} \\
 \spec{diam}{double*}{angular diameter (equatorial, radians)}
}
\notes
{
 \begin{enumerate}
  \item The date is in a dynamical time-scale (TDB)
        and is in the form of a Modified
        Julian Date (JD$-$2400000.5).  For all practical purposes, TT can
        be used instead of TDB, and for many applications UTC will do
        (except for the Moon).
  \item The longitude and latitude allow correction for geocentric
        parallax.  This is a major effect for the Moon, but in the
        context of the limited accuracy of the present function its
        effect on planetary positions is small (negligible for the
        outer planets).  Geocentric positions can be generated by
        appropriate use of the functions {\tt slaDmoon} and {\tt slaPlanet}.
  \item The direction accuracy (arcsec, 1000-3000\,AD) is of order:
        \begin{tabbing}
         xxxxxxx \= xxxxxxxxxxxxxxxxxx \= \kill
         \> Sun     \>  \hspace{0.5em}5 \\
         \> Mercury \>  \hspace{0.5em}2 \\
         \> Venus   \> 10 \\
         \> Moon    \> 30 \\
         \> Mars    \> 50 \\
         \> Jupiter \> 90 \\
         \> Saturn  \> 90 \\
         \> Uranus  \> 90 \\
         \> Neptune \> 10 \\
         \> Pluto \> \hspace{0.5em}1~~~(1885-2099\,AD only)
        \end{tabbing}
        The angular diameter accuracy is about 0.4\% for the Moon,
        and 0.01\% or better for the Sun and planets.
        For more information on accuracy,
        refer to the functions {\tt slaPlanet} and {\tt slaDmoon},
        which the present function uses.
 \end{enumerate}
}
%-----------------------------------------------------------------------
\routine{slaRefco}{refraction constants}
{
 \action{Determine the constants $A$ and $B$ in the
         atmospheric refraction model
         $\Delta \zeta = A \tan \zeta + B \tan^{3} \zeta$,
         where $\zeta$ is the {\it observed}\/ zenith distance
         ({\it i.e.}~affected by refraction) and $\Delta \zeta$ is
         what to add to $\zeta$ to give the {\it topocentric}\/
         ({\it i.e.\ in vacuo}) zenith distance.}
 \call{slaRefco( hm, tdk, pmb, rh, wl, phi, tlr, eps, \&refa, \&refb );}
}
\args{GIVEN}
{
 \spec{hm}{double}{height of the observer above sea level (metre)} \\
 \spec{tdk}{double}{ambient temperature at the observer K)} \\
 \spec{pmb}{double}{pressure at the observer (mb)} \\
 \spec{rh}{double}{relative humidity at the observer (range 0\,--\,1)} \\
 \spec{wl}{double}{effective wavelength of the source ($\mu{\rm m}$)} \\
 \spec{phi}{double}{latitude of the observer (radian, astronomical)} \\
 \spec{tlr}{double}{temperature lapse rate in the troposphere
                                     (K per metre)} \\
 \spec{eps}{double}{precision required to terminate iteration (radian)}
}
\args{RETURNED}
{
 \spec{refa}{double*}{$A$: the $\tan \zeta$ coefficient (radians)} \\
 \spec{refb}{double*}{$B$: the $\tan^{3} \zeta$ coefficient (radians)}
}
\notes
{
 \begin{enumerate}
  \item Suggested values for the {\tt tlr} and
        {\tt eps} arguments are 0.0065 and $10^{-8}$
        respectively.  The signs of both are immaterial.
  \item The radio refraction is chosen by specifying
        ${\tt wl}>100$~$\mu{\rm m}$.
  \item The function is a slower but more accurate alternative to the
        {\tt slaRefcoq} function.  The constants it produces give perfect
        agreement with {\tt slaRefro} at zenith distances
        $\tan^{-1} 1$ ($45^\circ$) and $\tan^{-1} 4$ ($\sim 76^\circ$).
        At other zenith distances, the model achieves:
        \arcsec{0}{5} accuracy for $\zeta<80^{\circ}$,
        \arcsec{0}{01} accuracy for $\zeta<60^{\circ}$, and
        \arcsec{0}{001} accuracy for $\zeta<45^{\circ}$.
  \item The sign convention for $B$ is that it is a negative
        number.
 \end{enumerate}
}
%-----------------------------------------------------------------------
\routine{slaRefcoq}{refraction constants (fast)}
{
 \action{Determine the constants $A$ and $B$ in the
         atmospheric refraction model
         $\Delta \zeta = A \tan \zeta + B \tan^{3} \zeta$,
         where $\zeta$ is the {\it observed}\/ zenith distance
         ({\it i.e.}~affected by refraction) and $\Delta \zeta$ is
         what to add to $\zeta$ to give the {\it topocentric}\/
         ({\it i.e.\ in vacuo}) zenith distance. (This is a fast
         alternative to the slaRefco function -- see notes.)}
 \call{slaRefcoq( tdk, pmb, rh, wl, \&refa, \&refb );}
}
\args{GIVEN}
{
 \spec{tdk}{double}{ambient temperature at the observer (K)} \\
 \spec{pmb}{double}{pressure at the observer (mb)} \\
 \spec{rh}{double}{relative humidity at the observer (range 0\,--\,1)} \\
 \spec{wl}{double}{effective wavelength of the source ($\mu{\rm m}$)}
}
\args{RETURNED}
{
 \spec{refa}{double*}{$A$: the $\tan \zeta$ coefficient (radians)} \\
 \spec{refb}{double*}{$B$: the $\tan^{3} \zeta$ coefficient (radians)}
}
\notes
{
 \begin{enumerate}
  \item The radio refraction is chosen by specifying
        ${\tt wl}>100$~$\mu{\rm m}$.
  \item The model is an approximation, for moderate zenith distances,
        to the predictions of the {\tt slaRefro} function.  The approximation
        is maintained across a range of conditions, and applies to
        both optical/IR and radio.
  \item The algorithm is a fast alternative to the {\tt slaRefco} function.
        The latter calls the {\tt slaRefro} function itself:  this involves
        integrations through a model atmosphere, and is costly in
        processor time.  However, the model which is produced is precisely
        correct for two zenith distances ($45^\circ$ and $\sim\!76^\circ$)
        and at other zenith distances is limited in accuracy only by the
        $\Delta \zeta = A \tan \zeta + B \tan^{3} \zeta$ formulation
        itself.  The present function is not as accurate, though it
        satisfies most practical requirements.
  \item The model omits the effects of (i)~height above sea level (apart
        from the reduced pressure itself), (ii)~latitude ({\it i.e.}~the
        flattening of the Earth) and (iii)~variations in tropospheric
        lapse rate.
  \item The model has been tested using the following range of conditions:
        \begin{itemize}
        \item [$\cdot$] lapse rates 0.0055, 0.0065, 0.0075~K per metre
        \item [$\cdot$] latitudes $0^\circ$,
                        $25^\circ$, $50^\circ$, $75^\circ$
        \item [$\cdot$] heights 0, 2500, 5000 metres above sea level
        \item [$\cdot$] pressures, mean for height $-10$\% to $+5$\%
                        in steps of $5$\%
        \item [$\cdot$] temperatures $-10^\circ$ to $+20^\circ$
                        with respect to $280$K at sea level
        \item [$\cdot$] relative humidity 0, 0.5, 1
        \item [$\cdot$] wavelength 0.4, 0.6, \ldots\ $2\mu{\rm m}$, + radio
        \item [$\cdot$] zenith distances $15^\circ$, $45^\circ$, $75^\circ$
        \end{itemize}
        For the above conditions, the comparison with {\tt slaRefro}
        was as follows:
 
        \vspace{2ex}
 
        ~~~~~~~~~~
        \begin{tabular}{|r|r|r|} \hline
              & {\it worst} & {\it RMS} \\ \hline
              optical/IR & 62 & 8 \\
              radio & 319 & 49 \\ \hline
              & mas & mas \\ \hline
        \end{tabular}
 
        \vspace{3ex}
 
        For this particular set of conditions:
        \begin{itemize}
        \item [$\cdot$] lapse rate 6.5 K km$^{-1}$
        \item [$\cdot$] latitude $50^\circ$
        \item [$\cdot$] sea level
        \item [$\cdot$] pressure 1005\,mb
        \item [$\cdot$] temperature $7^\circ$C
        \item [$\cdot$] humidity 80\%
        \item [$\cdot$] wavelength 5740\,\.{A}
        \end{itemize}
        the results were as follows:
 
        \vspace{2ex}
 
        ~~~~~~~~~~
        \begin{tabular}{|r|r|r|r|} \hline
        \multicolumn{1}{|c}{$\zeta$} &
        \multicolumn{1}{|c}{{\tt slaRefro}} &
        \multicolumn{1}{|c}{{\tt slaRefcoq}} &
        \multicolumn{1}{|c|}{Saastamoinen} \\ \hline
        10 &  10.27 &  10.27 &  10.27 \\
        20 &  21.19 &  21.20 &  21.19 \\
        30 &  33.61 &  33.61 &  33.60 \\
        40 &  48.82 &  48.83 &  48.81 \\
        45 &  58.16 &  58.18 &  58.16 \\
        50 &  69.28 &  69.30 &  69.27 \\
        55 &  82.97 &  82.99 &  82.95 \\
        60 & 100.51 & 100.54 & 100.50 \\
        65 & 124.23 & 124.26 & 124.20 \\
        70 & 158.63 & 158.68 & 158.61 \\
        72 & 177.32 & 177.37 & 177.31 \\
        74 & 200.35 & 200.38 & 200.32 \\
        76 & 229.45 & 229.43 & 229.42 \\
        78 & 267.44 & 267.29 & 267.41 \\
        80 & 319.13 & 318.55 & 319.10 \\ \hline
        deg & arcsec & arcsec & arcsec \\ \hline
        \end{tabular}
 
        \vspace{3ex}
 
        The values for Saastamoinen's formula (which includes terms
        up to $\tan^5$) are taken from Hohenkerk \& Sinclair (1985).
 
        The results from the much slower but more accurate {\tt slaRefco}
        function have not been included in the tabulation as they are
        identical to those in the {\tt slaRefro} column to the \arcsec{0}{01}
        resolution used.
  \item Outlandish input parameters are silently limited
        to mathematically safe values.  Zero pressure is permissible,
        and causes zeroes to be returned.
  \item The algorithm draws on several sources, as follows:
        \begin{itemize}
        \item The formula for the saturation vapour pressure of water as
              a function of temperature and temperature is taken from
              expressions A4.5-A4.7 of Gill (1982).
        \item The formula for the water vapour pressure, given the
              saturation pressure and the relative humidity is from
              Crane (1976), expression 2.5.5.
        \item The refractivity of air is a function of temperature,
              total pressure, water-vapour pressure and, in the case
              of optical/IR but not radio, wavelength.  The formulae
              for the two cases are developed from Hohenkerk \& Sinclair
              (1985) and Rueger (2002).
        \item The formula for $\beta~(=H_0/r_0)$ is
              an adaption of expression 9 from Stone (1996).  The
              adaptations, arrived at empirically, consist of (i)~a
              small adjustment to the coefficient and (ii)~a humidity
              term for the radio case only.
        \item The formulae for the refraction constants as a function of
              $n-1$ and $\beta$ are from Green (1987), expression 4.31.
        \end{itemize}
        The first three items are as used in the {\tt slaRefro} function.
  \item The sign convention for $B$ is that it is a negative number.
 \end{enumerate}
}
\refs
{
 \begin{enumerate}
  \item Crane, R.K.\ 1976,
        ``Refraction Effects in the Neutral Atmosphere'',
        {\it Methods of Experimental Physics: Astrophysics 12B},
        Meeks, M.L.\ (ed),
        Academic Press.
  \item Gill, Adrian E.\ 1982, {\it Atmosphere-Ocean Dynamics},
        Academic Press.
  \item Green, R.M.\ 1987, {\it Spherical Astronomy}, Cambridge
        University Press.
  \item Hohenkerk, C.Y., \& Sinclair, A.T.\ 1985,
        {\it NAO Technical Note}\/
        No.~63, Royal Greenwich Observatory.
  \item Rueger, J.M.\ 2002, {\it Refractive Index Formulae for
        Electronic Distance Measurement with Radio and Millimetre
        Waves}, in Unisurv Report S-68, School of Surveying
        and Spatial Information Systems, University of New South
        Wales, Sydney, Australia.
  \item Stone, Ronald C., P.A.S.P.~{\bf 108} 1051-1058, 1996.
 \end{enumerate}
}
%-----------------------------------------------------------------------
\routine{slaRefro}{refraction}
{
 \action{Atmospheric refraction, for radio or optical/IR wavelengths.}
 \call{slaRefro( zobs, hm, tdk, pmb, rh, wl, phi, tlr, eps, \&ref );}
}
\args{GIVEN}
{
 \spec{zobs}{double}{observed zenith distance of the source (radians)} \\
 \spec{hm}{double}{height of the observer above sea level (metre)} \\
 \spec{tdk}{double}{ambient temperature at the observer (K)} \\
 \spec{pmb}{double}{pressure at the observer (mb)} \\
 \spec{rh}{double}{relative humidity at the observer (range 0\,--\,1)} \\
    \spec{wl}{double}{effective wavelength of the source ($\mu{\rm m}$)} \\
 \spec{phi}{double}{latitude of the observer (radian, astronomical)} \\
 \spec{tlr}{double}{temperature lapse rate in the troposphere
                                     (K per metre)} \\
 \spec{eps}{double}{precision required to terminate iteration (radian)}
}
\args{RETURNED}
{
 \spec{ref}{double*}{refraction: {\it in vacuo}\/ ZD
                     minus observed ZD (radians)}
}
\notes
{
 \begin{enumerate}
 \item A suggested value for the {\tt tlr} argument
       is 0.0065 (sign immaterial).
       The refraction is significantly affected by {\tt tlr},
       and if studies of the local atmosphere have been carried
       out a better value may be available.
 \item A suggested value for the {\tt eps}
       argument is $10^{-8}$.  The result is
       usually at least two orders of magnitude more computationally
       precise than the supplied {\tt eps} value.
 \item The function computes the refraction for zenith distances up
       to and a little beyond $90^\circ$ using the method of Hohenkerk
       \& Sinclair (1985), subsequently adopted in the {\it Explanatory
       Supplement to the Astronomical Almanac}\/
       (Seidelmann 1992 -- see section 3.281).
 \item The code is based on the {\tt AREF}
       optical/IR refraction subroutine
       (HMNAO, September 1984, RGO: Hohenkerk 1985),
       with extensions to
       support the radio case.  The modifications to the original HMNAO
       optical/IR refraction code which affect the results are:
       \begin{itemize}
        \item The angle arguments have been changed to radians,
              any value of {\tt zobs} is allowed (see Note~6, below) and
              other argument values have been limited to safe values.
        \item Revised values for the gas constants are used, from
              Murray (1983).
        \item A better model for $P_s(T)$ has been adopted,
              from Gill (1982).
        \item More accurate expressions for $Pw_o$ have been adopted
              (again from Gill 1982).
        \item The formula for the water vapour pressure, given the
              saturation pressure and the relative humidity, is from
              Crane (1976), expression 2.5.5.
        \item Provision for radio wavelengths has been added using
              expressions devised by A.\,T.\,Sinclair, RGO
              (Sinclair 1989).  The refractivity model
              is from Rueger (2002).
        \item The optical refractivity for dry air is from IAG (1999).
       \end{itemize}
 \item The radio refraction is chosen by specifying
       ${\tt wl}>100\mu{\rm m}$.
       Because the algorithm takes no account of the ionosphere, the
       accuracy deteriorates at low frequencies, below about 30\,MHz.
 \item Before use, the value of
       {\tt zobs} is expressed in the range $\pm\pi$.
       If this ranged {\tt zobs} is negative, the result
       {\tt ref} is computed from its
       absolute value before being made negative to match.  In addition, if
       it has an absolute value greater than $93^\circ$,
       a fixed {\tt ref} value
       equal to the result for ${\tt zobs}=93^\circ$ is returned,
       appropriately signed.
 \item As in the original Hohenkerk \& Sinclair algorithm, fixed values
       of the water vapour polytrope exponent, the height of the
       tropopause, and the height at which refraction is negligible are
       used.
 \item The radio refraction has been tested against work done by
       Iain~Coulson, JACH, (1995) for the
       James Clerk Maxwell Telescope, Mauna Kea.  For typical conditions,
       agreement at the \arcsec{0}{1} level is achieved for moderate
       zenith distances, worsening to perhaps
       \arcsec{0}{5}\,--\,\arcsec{1}{0} at ZD $80^\circ$.
       At hot and humid sea-level sites the accuracy will not be as good.
 \item It should be noted that the relative humidity
       {\tt rh} is formally
       defined in terms of ``mixing ratio'' rather than pressures or
       densities as is often stated.  It is the mass of water per unit
       mass of dry air divided by that for saturated air at the same
       temperature and pressure (see Gill 1982).  The familiar
       $\nu=p_w/p_s$ or $\nu=\rho_w/\rho_s$ expressions can differ from
       the formal definition by several percent, significant in the
       radio case.
 \item The algorithm is designed for observers in the troposphere.  The
       supplied temperature, pressure and lapse rate are assumed to be
       for a point in the troposphere and are used to define a model
       atmosphere with the tropopause at 11\,km altitude and a constant
       temperature above that.  However, in practice, the refraction
       values returned for stratospheric observers, at altitudes up to
       25km, are quite usable.
  \end{enumerate}
}
\refs
{
 \begin{enumerate}
  \item Coulsen, I.\ 1995, private communication.
  \item Crane, R.K.\ 1976,
        ``Refraction Effects in the Neutral Atmosphere'',
        {\it Methods of Experimental Physics: Astrophysics 12B},
        Meeks, M.L.\ (ed),
        Academic Press.
  \item Gill, Adrian E.\ 1982, {\it Atmosphere-Ocean Dynamics},
        Academic Press.
  \item Hohenkerk, C.Y.\ 1985, private communication.
  \item Hohenkerk, C.Y., \& Sinclair, A.T.\ 1985,
        {\it NAO Technical Note}\/
        No.~63, Royal Greenwich Observatory.
  \item International Association of Geodesy,
        XXIIth General Assembly, Birmingham, UK, 1999,
        Resolution 3.
  \item Murray, C.A.\ 1983, {\it Vectorial Astrometry,}
        Adam Hilger, Bristol.
  \item Seidelmann,\,P.K.\ {\it et al.}\ 1992,
        {\it Explanatory Supplement to the
        Astronomical Almanac}, Chapter 3, University Science Books.
  \item Rueger, J.M.\ 2002, {\it Refractive Index Formulae for
        Electronic Distance Measurement with Radio and Millimetre
        Waves}, in Unisurv Report S-68, School of Surveying
        and Spatial Information Systems, University of New South
        Wales, Sydney, Australia.
  \item Sinclair, A.T.\ 1989, private communication.
 \end{enumerate}
}
%-----------------------------------------------------------------------
\routine{slaRefv}{apply refraction to vector}
{
 \action{Adjust an unrefracted Cartesian vector to include the effect of
         atmospheric refraction, using the simple
         $\Delta \zeta = A \tan \zeta + B \tan^{3} \zeta$ model.}
 \call{slaRefv( vu, refa, refb, vr );}
}
\args{GIVEN}
{
 \spec{vu}{double[3]}{unrefracted position of the source (\azel\ 3-vector)} \\
 \spec{refa}{double}{$\tan \zeta$ coefficient (radians)} \\
 \spec{refb}{double}{$\tan^{3} \zeta$ coefficient (radians)}
}
\args{RETURNED}
{
 \spec{vr}{double[3]}{refracted position of the source (\azel\ 3-vector)}
}
\notes
{
 \begin{enumerate}
  \item This function applies the adjustment for refraction in the
        opposite sense to the usual one -- it takes an unrefracted
        ({\it in vacuo}\/) position and produces an observed (refracted)
        position, whereas the
        $\Delta \zeta = A \tan \zeta + B \tan^{3} \zeta$
        model strictly
        applies to the case where an observed position is to have the
        refraction removed.  The unrefracted to refracted case is
        harder, and requires an inverted form of the text-book
        refraction models;  the algorithm used here is equivalent to
        one iteration of the Newton-Raphson method applied to the
        above formula.
  \item Though optimized for speed rather than precision, the present
        function achieves consistency with the refracted-to-unrefracted
        $\Delta \zeta = A \tan \zeta + B \tan^{3} \zeta$
        model at better than 1~microarcsecond within
        $30^\circ$ of the zenith and remains within 1~milliarcsecond to
        $\zeta=70^\circ$.  The inherent accuracy of the model is, of
        course, far worse than this -- see the documentation for {\tt slaRefco}
        for more information.
  \item At low elevations (below about $3^\circ$) the refraction
        correction is held back to prevent arithmetic problems and
        wildly wrong results.  For optical/IR wavelengths, over a wide
        range of observer heights and corresponding temperatures and
        pressures, the following levels of accuracy (worst case)
        are achieved, relative to numerical integration through a model
        atmosphere:
        \begin{center}
        \begin{tabular}{ccl}
              $\zeta_{obs}$ & {\it error} \\ \\
              $80^\circ$ & \arcsec{0}{7}  \\
              $81^\circ$ & \arcsec{1}{3}  \\
              $82^\circ$ & \arcsec{2}{5}  \\
              $83^\circ$ & \arcseci{5}    \\
              $84^\circ$ & \arcseci{10}    \\
              $85^\circ$ & \arcseci{20}   \\
              $86^\circ$ & \arcseci{55}   \\
              $87^\circ$ & \arcseci{160}  \\
              $88^\circ$ & \arcseci{360}  \\
              $89^\circ$ & \arcseci{640}  \\
              $90^\circ$ & \arcseci{1100} \\
              $91^\circ$ & \arcseci{1700} & $<$ high-altitude \\
              $92^\circ$ & \arcseci{2600} & $<$ sites only \\
        \end{tabular}
        \end{center}
        The results for radio are slightly worse over most of the range,
        becoming significantly worse below $\zeta = 88^\circ$
        and unusable beyond $\zeta = 90^\circ$.
  \item See also the function {\tt slaRefz}, which performs the adjustment to
        the zenith distance rather than in \xyz.
        The present function is faster than {\tt slaRefz} and,
        except very low down,
        is equally accurate for all practical purposes.  However, beyond
        about $\zeta=84^\circ$ {\tt slaRefz} should be used, and for the utmost
        accuracy iterative use of {\tt slaRefro} should be considered.
 \end{enumerate}
}
%-----------------------------------------------------------------------
\routine{slaRefz}{apply refraction to ZD}
{
 \action{Adjust an unrefracted zenith distance to include the effect of
         atmospheric refraction, using the simple
         $\Delta \zeta = A \tan \zeta + B \tan^{3} \zeta$ model.}
 \call{slaRefz( zu, refa, refb, \&zr );}
}
\args{GIVEN}
{
 \spec{zu}{double}{unrefracted zenith distance of the source (radians)} \\
 \spec{refa}{double}{$\tan \zeta$ coefficient (radians)} \\
 \spec{refb}{double}{$\tan^{3} \zeta$ coefficient (radians)}
}
\args{RETURNED}
{
 \spec{zr}{double*}{refracted zenith distance (radians)}
}
\notes
{
 \begin{enumerate}
  \item This function applies the adjustment for refraction in the
        opposite sense to the usual one -- it takes an unrefracted
        ({\it in vacuo}\/) position and produces an observed (refracted)
        position, whereas the
        $\Delta \zeta = A \tan \zeta + B \tan^{3} \zeta$
        model strictly
        applies to the case where an observed position is to have the
        refraction removed.  The unrefracted to refracted case is
        harder, and requires an inverted form of the text-book
        refraction models;  the formula used here is based on the
        Newton-Raphson method.  For the utmost numerical consistency
        with the refracted to unrefracted model, two iterations are
        carried out, achieving agreement at the $10^{-11}$~arcsecond level
        for $\zeta=80^\circ$.  The inherent accuracy of the model
        is, of course, far worse than this -- see the documentation for
        {\tt slaRefco} for more information.
  \item At $\zeta=83^\circ$, the rapidly-worsening
        $\Delta \zeta = A \tan \zeta + B \tan^{3} \zeta$
        model is abandoned and an empirical formula takes over:
 
          \[\Delta \zeta = F \left(
  \frac{0^\circ\hspace{-0.37em}.\hspace{0.02em}55445
                - 0^\circ\hspace{-0.37em}.\hspace{0.02em}01133 E
                          + 0^\circ\hspace{-0.37em}.\hspace{0.02em}00202 E^2}
             {1 + 0.28385 E +0.02390 E^2} \right) \]
        where $E=90^\circ-\zeta_{true}$
        and $F$ is a factor chosen to meet the
        $\Delta \zeta = A \tan \zeta + B \tan^{3} \zeta$
        formula at $\zeta=83^\circ$.
 
        For optical/IR wavelengths, over a wide range of observer heights
        and corresponding temperatures and pressures, the following levels
        of accuracy (worst case) are achieved,
        relative to numerical integration through a model atmosphere:
 
        \begin{center}
        \begin{tabular}{ccl}
              $\zeta_{obs}$ & {\it error} \\ \\
              $80^\circ$ & \arcsec{0}{7}  \\
              $81^\circ$ & \arcsec{1}{3}  \\
              $82^\circ$ & \arcsec{2}{4}  \\
              $83^\circ$ & \arcsec{4}{7}  \\
              $84^\circ$ & \arcsec{6}{2}  \\
              $85^\circ$ & \arcsec{6}{4}  \\
              $86^\circ$ & \arcseci{8}    \\
              $87^\circ$ & \arcseci{10}   \\
              $88^\circ$ & \arcseci{15}   \\
              $89^\circ$ & \arcseci{30}   \\
              $90^\circ$ & \arcseci{60}   \\
              $91^\circ$ & \arcseci{150} & $<$ high-altitude \\
              $92^\circ$ & \arcseci{400} & $<$ sites only \\
        \end{tabular}
        \end{center}
        For radio wavelengths the errors are typically 50\% larger than
        the optical figures and by $\zeta = 85^\circ$ are twice as bad,
        worsening rapidly below that.  To maintain \arcseci{1} accuracy
        down to $\zeta = 85^\circ$ at the Green Bank site, Condon (2004)
        has suggested amplifying the amount of refraction predicted by
        {\tt slaRefz} below \degree{10}{8} elevation by the factor
        $(1+0.00195*(10.8-E_{topo}))$, where $E_{topo}$ is the
        unrefracted elevation in degrees.
 
        The high-ZD model is scaled to match the normal model at the
        transition point;  there is no glitch.
  \item See also the function {\tt slaRefv}, which performs the adjustment in
        \xyz, and with the emphasis on speed rather than numerical accuracy.
 \end{enumerate}
}
\aref{Condon,\,J.J.\ 2004, {\it Refraction Corrections for the GBT,}
      PTCS/PN/35.2, NRAO Green Bank.}
%-----------------------------------------------------------------------
\routine{slaRverot}{radial velocity correction to Earth centre}
{
 \action{Velocity component in a given direction due to Earth rotation.}
 \call{r~=~slaRverot( phi, ra, da, st );}
}
\args{GIVEN}
{
 \spec{phi}{float}{geodetic latitude of observing station (radians)} \\
 \spec{ra,da}{float}{apparent \radec\ (radians)} \\
 \spec{st}{float}{local apparent sidereal time (radians)}
}
\args{RETURNED}
{
 \spec{}{float}{Component of Earth rotation in
                       direction [{\tt ra},{\tt da}]~(km~s$^{-1}$)}
}
\notes
{
 \begin{enumerate}
  \item Sign convention: the result is positive when the observatory
        is receding from the given point on the sky.
  \item Accuracy: the simple algorithm used assumes a spherical Earth and
        an observing station at sea level;  for actual observing
        sites, the error is unlikely to be greater than 0.0005~km~s$^{-1}$.
        For applications requiring greater accuracy, use the function
        {\tt slaPvobs}.
 \end{enumerate}
}
%-----------------------------------------------------------------------
\routine{slaRvgalc}{radial velocity correction to galactic centre}
{
 \action{Velocity component in a given direction due to the rotation
         of the Galaxy.}
 \call{r~=~slaRvgalc( r2000, d2000 );}
}
\args{GIVEN}
{
 \spec{r2000,d2000}{float}{J2000.0 mean \radec\ (radians)}
}
\args{RETURNED}
{
 \spec{}{float}{Component of dynamical LSR motion in direction
                       [{\tt r2000},{\tt d2000}]~(km~s$^{-1}$)}
}
\notes
{
 \begin{enumerate}
  \item Sign convention: the result is positive when the LSR
        is receding from the given point on the sky.
  \item The Local Standard of Rest used here is a point in the
        vicinity of the Sun which is in a circular orbit around
        the galactic centre.  Sometimes called the {\it dynamical}\/ LSR,
        it is not to be confused with a {\it kinematical}\/ LSR, which
        is the mean standard of rest of star catalogues or stellar
        populations.
  \item The dynamical LSR velocity due to galactic rotation is assumed to
        be 220~km~s$^{-1}$ towards $l^{I\!I}\,=\,90^{\circ}$,
                                   $b^{I\!I}\,=\,0$.
 \end{enumerate}
}
\aref{Kerr F.J., \& Lynden-Bell, D.\ 1986, {\it Mon.Not.R.astr.Soc.,}
      {\bf 221}, 1023.}
%-----------------------------------------------------------------------
\routine{slaRvlg}{radial velocity correction to local group}
{
 \action{Velocity component in a given direction due to the combination
         of the rotation of the Galaxy and the motion of the Galaxy
         relative to the mean motion of the local group.}
 \call{r~=~slaRvlg( r2000, d2000 );}
}
\args{GIVEN}
{
 \spec{r2000,d2000}{float}{J2000.0 mean \radec\ (radians)}
}
\args{RETURNED}
{
 \spec{}{float}{Component of {\bf solar} ({\it n.b.})
                     motion in direction
                     [{\tt r2000},{\tt d2000}]~(km~s$^{-1}$)}
}
\anote{Sign convention: the result is positive when
       the Sun is receding from the given point on the sky.}
\aref{{\it IAU Trans.}\ 1976, {\bf 16B}, p201.}
%-----------------------------------------------------------------------
\routine{slaRvlsrd}{radial velocity correction to dynamical LSR}
{
 \action{Velocity component in a given direction due to the Sun's
         motion with respect to the ``dynamical'' Local Standard of Rest.}
 \call{r~=~slaRvlsrd( r2000, d2000 );}
}
\args{GIVEN}
{
 \spec{r2000,d2000}{float}{J2000.0 mean \radec\ (radians)}
}
\args{RETURNED}
{
 \spec{}{float}{Component of {\it peculiar}\/ solar motion
                in direction [{\tt r2000},{\tt d2000}]~(km~s$^{-1}$)}
}
\notes
{
 \begin{enumerate}
  \item Sign convention: the result is positive when
        the Sun is receding from the given point on the sky.
  \item The Local Standard of Rest used here is the {\it dynamical}\/ LSR,
        a point in the vicinity of the Sun which is in a circular
        orbit around the galactic centre.  The Sun's motion with
        respect to the dynamical LSR is called the {\it peculiar}\/ solar
        motion.
  \item There is another type of LSR, called a {\it kinematical}\/ LSR.  A
        kinematical LSR is the mean standard of rest of specified star
        catalogues or stellar populations, and several slightly
        different kinematical LSRs are in use.  The Sun's motion with
        respect to an agreed kinematical LSR is known as the
        {\it standard}\/ solar motion.
        The dynamical LSR is seldom used by observational astronomers,
        who conventionally use a kinematical LSR such as the one implemented
        in the function {\tt slaRvlsrk}.
  \item The peculiar solar motion is from Delhaye (1965):
        in galactic Cartesian
        coordinates $(+9,+12,+7)\,{\rm km}\,s^{-1}$.
        This corresponds to about $16.6\,{\rm km}\,s^{-1}$
        towards galactic coordinates
        $l^{I\!I}=53^{\circ},~b^{I\!I}=+25^{\circ}$.
 \end{enumerate}
}
\aref{Delhaye, J., 1965, in {\it Stars and Stellar Systems}, vol~5,
      {\it Galactic Structure,} ed.~Blaauw \& Schmidt, Univ.\ of Chicago Press,
      pp73-4.}
%-----------------------------------------------------------------------
\routine{slaRvlsrk}{radial velocity correction to kinematical LSR}
{
 \action{Velocity component in a given direction due to the Sun's
         motion with respect to a kinematical Local Standard of Rest.}
 \call{r~=~slaRvlsrk( r2000, d2000 );}
}
\args{GIVEN}
{
 \spec{r2000,d2000}{float}{J2000.0 mean \radec\ (radians)}
}
\args{RETURNED}
{
 \spec{}{float}{Component of {\it standard}\/ solar motion
                in direction [{\tt r2000},{\tt d2000}]~(km~s$^{-1}$)}
}
\notes
{
 \begin{enumerate}
  \item Sign convention: the result is positive when
        the Sun is receding from the given point on the sky.
  \item The Local Standard of Rest used here is one of several
        {\it kinematical}\/ LSRs in common use.  A kinematical LSR is the
        mean standard of rest of specified star catalogues or stellar
        populations.  The Sun's motion with respect to a kinematical
        LSR is known as the {\it standard}\/ solar motion.
  \item There is another sort of LSR, seldom used by observational
        astronomers, called the {\it dynamical}\/ LSR.  This is a
        point in the vicinity of the Sun which is in a circular orbit
        around the galactic centre.  The Sun's motion with respect to
        the dynamical LSR is called the {\it peculiar}\/ solar motion.  To
        obtain a radial velocity correction with respect to the
        dynamical LSR use the function {\tt slaRvlsrd}.
  \item The adopted standard solar motion is 20~km~s$^{-1}$
        towards $\alpha=18^{\rm h},\,\delta=+30^{\circ}$ (1900).
 \end{enumerate}
}
\refs
{
 \begin{enumerate}
  \item Delhaye, J., 1965, in {\it Stars and Stellar Systems}, vol~5,
        {\it Galactic Structure,} ed.~Blaauw \& Schmidt,
        Univ.\ of Chicago Press, pp73-4.
  \item {\it Methods of Experimental Physics}\/ (ed Meeks), vol~12,
        part~C, sec~6.1.5.2, p281.
 \end{enumerate}
}
%-----------------------------------------------------------------------
\routine{slaS2tp}{spherical to tangent plane}
{
 \action{Projection of spherical coordinates onto the tangent plane
         (single precision).}
 \call{slaS2tp( ra, dec, raz, decz, \&xi, \&eta, \&j );}
}
\args{GIVEN}
{
 \spec{ra,dec}{float}{spherical coordinates of star (radians)} \\
 \spec{raz,decz}{float}{spherical coordinates of tangent point (radians)}
}
\args{RETURNED}
{
 \spec{xi,eta}{float*}{tangent plane coordinates (radians)} \\
 \spec{j}{int*}{status:} \\
 \spec{}{}{\hspace{1.5em} 0 = OK, star on tangent plane} \\
 \spec{}{}{\hspace{1.5em} 1 = error, star too far from axis} \\
 \spec{}{}{\hspace{1.5em} 2 = error, antistar on tangent plane} \\
 \spec{}{}{\hspace{1.5em} 3 = error, antistar too far from axis}
}
\notes
{
 \begin{enumerate}
  \item The projection is called the {\it gnomonic}\/ projection;  the
        Cartesian coordinates \xieta\ are called
        {\it standard coordinates.}\/  The latter
        are in units of the distance from the tangent plane to the projection
        point, {\it i.e.}~radians near the origin.
  \item When working in \xyz\ rather than spherical coordinates, the
        equivalent Cartesian function {\tt slaV2tp} is available.
 \end{enumerate}
}
%-----------------------------------------------------------------------
\routine{slaSep}{angle between two points on sphere}
{
 \action{Angle between two points on a sphere (single precision).}
 \call{r~=~slaSep( a1, b1, a2, b2 );}
}
\args{GIVEN}
{
 \spec{a1,b1}{float}{spherical coordinates of one point (radians)} \\
 \spec{a2,b2}{float}{spherical coordinates of the other point (radians)}
}
\args{RETURNED}
{
 \spec{}{float}{angle between [{\tt a1},{\tt b1}]
                and [{\tt a2},{\tt b2}] in radians}
}
\notes
{
 \begin{enumerate}
  \item The spherical coordinates are right ascension and declination,
  longitude and latitude, {\it etc.}\ in radians.
  \item The result is always positive.
 \end{enumerate}
}
%-----------------------------------------------------------------------
\routine{slaSepv}{angle between two vectors}
{
 \action{Angle between two vectors (single precision).}
 \call{r~=~slaSepv( v1, v2 );}
}
\args{GIVEN}
{
 \spec{v1}{float[3]}{first vector} \\
 \spec{v2}{float[3]}{second vector}
}
\args{RETURNED}
{
 \spec{}{float}{angle between {\tt v1} and {\tt v2} in radians}
}
\notes
{
 \begin{enumerate}
  \item There is no requirement for either vector to be of unit length.
  \item If either vector is null, zero is returned.
  \item The result is always positive.
 \end{enumerate}
}
%-----------------------------------------------------------------------
\routine{slaSmat}{solve simultaneous equations}
{
 \action{Matrix inversion and solution of simultaneous equations
         (single precision).}
 \call{slaSmat( n, a, y, \&d, \&jf, iw );}
}
\args{GIVEN}
{
 \spec{n}{int}{number of unknowns} \\
 \spec{a}{R(N,N)}{matrix} \\
 \spec{y}{R(N)}{vector}
}
\args{RETURNED}
{
 \spec{a}{float[n][n]}{matrix inverse} \\
 \spec{y}{float[n]}{solution} \\
 \spec{d}{float*}{determinant} \\
 \spec{jf}{int*}{singularity flag: 0\,=\,OK} \\
 \spec{iw}{int[n]}{workspace}
}
\notes
{
 \begin{enumerate}
  \item For the set of $n$ simultaneous linear equations in $n$ unknowns:
        \begin{verse}
         ${\bf A}\times{\bf y} = {\bf x}$
        \end{verse}
        where:
        \begin{itemize}
         \item {\bf A} is a non-singular $n \times n$ matrix,
         \item {\bf y} is the vector of $n$ unknowns, and
         \item {\bf x} is the known vector,
        \end{itemize}
        {\tt slaSmat} computes:
        \begin{itemize}
         \item the inverse of matrix {\bf A},
         \item the determinant of matrix {\bf A}, and
         \item the vector of $n$ unknowns {\bf y}.
        \end{itemize}
        Argument {\tt n} is the order $n$,
        {\tt a} (given) is the matrix {\bf A},
        {\tt y} (given) is the vector {\bf x} and {\tt y} (returned)
        is the vector {\bf y}.
        The argument {\tt a} (returned) is the
        inverse matrix ${\bf A}^{-1}$,
        and {\tt d} is $\det{\bf A}$.
  \item {\tt jf} is the singularity flag.  If the matrix is non-singular,
        ${\tt jf}=0$ is returned.
        If the matrix is singular, ${\tt jf}=-1$
        and ${\tt d}=0.0$ are returned.
        In the latter case, the contents
        of array {\tt a} on return are undefined.
  \item The algorithm is Gaussian elimination with partial pivoting.
        This method is very fast;  some much slower algorithms can give
        better accuracy, but only by a small factor.
 \end{enumerate}
}
%-----------------------------------------------------------------------
\routine{slaSubet}{remove E-terms}
{
 \action{Remove the E-terms (elliptic component of annual aberration)
         from a pre IAU~1976 catalogue \radec\ to give a mean place.}
 \call{slaSubet( rc, dc, eq, \&rm, \&dm );}
}
\args{GIVEN}
{
 \spec{rc,dc}{double}{\radec\ with E-terms included (radians)} \\
 \spec{eq}{double}{Besselian epoch of mean equator and equinox}
}
\args{RETURNED}
{
 \spec{rm,dm}{double*}{\radec\ without E-terms (radians)}
}
\anote{Most star positions from pre-1984 optical catalogues (or
       obtained by astrometry with respect to such stars) have the
       E-terms built-in.  This function converts such a position to a
       formal mean place (allowing, for example, comparison with a
       pulsar timing position).}
\aref{Seidelmann,\,P.K.\ {\it et al.}\ 1992,
      {\it Explanatory Supplement to the
      Astronomical Almanac}, University Science Books,
      section 2D, p48.}
%-----------------------------------------------------------------------
\routine{slaSupgal}{supergalactic to galactic}
{
 \action{Transformation from de Vaucouleurs supergalactic coordinates
         to IAU 1958 galactic coordinates.}
 \call{slaSupgal( dsl, dsb, \&dl, \&db );}
}
\args{GIVEN}
{
 \spec{dsl,dsb}{double}{supergalactic longitude and latitude (radians)}
}
\args{RETURNED}
{
 \spec{dl,db}{double*}{galactic longitude and latitude \gal\ (radians)}
}
\refs
{
 \begin{enumerate}
  \item de Vaucouleurs, de Vaucouleurs, \& Corwin, 1976,
        {\it Second Reference Catalogue of Bright Galaxies},
        U.Texas, p8.
  \item Systems \& Applied Sciences Corp., documentation for the
        machine-readable version of the above catalogue,
        Contract NAS 5-26490.
 \end{enumerate}
 (These two references give different values for the galactic
 longitude of the supergalactic origin.  Both are wrong;  the
 correct value is $l^{I\!I}=137.37^{\circ}$.)
}
%------------------------------------------------------------------------------
\routine{slaSvd}{singular value decomposition}
{
 \action{Singular value decomposition.
         This function expresses a given matrix {\bf A} as the product of
         three matrices {\bf U}, {\bf W}, {\bf V}$^{T}$:
         \begin{tabbing}
         XXXXXX \= \kill
         \> {\bf A} = {\bf U} $\cdot$ {\bf W} $\cdot$ {\bf V}$^{T}$
         \end{tabbing}
         where:
         \begin{tabbing}
         XXXXXX \= XXXX \= \kill
         \> {\bf A} \> is any $m$ (rows) $\times n$ (columns) matrix,
                       where $m \geq n$ \\
         \> {\bf U} \> is an $m \times n$ column-orthogonal matrix \\
         \> {\bf W} \> is an $n \times n$ diagonal matrix with
                       $w_{ii} \geq 0$ \\
         \> {\bf V}$^{T}$ \> is the transpose of an $n \times n$
                             orthogonal matrix
\end{tabbing}
}
 \call{slaSvd( m, n, mp, np, a, w, v, work, \&jstat );}
}
\args{GIVEN}
{
 \spec{m,n}{int}{$m$, $n$, the numbers of
                 rows and columns in matrix {\bf A}} \\
 \spec{mp,np}{int}{physical dimensions of array
                   containing matrix {\bf A}} \\
 \spec{a}{double[mp*np]}{array containing $m \times n$ matrix {\bf A}}
}
\args{RETURNED}
{
 \spec{a}{double[mp*np]}{array containing $m \times n$ column-orthogonal
                         matrix {\bf U}} \\
 \spec{w}{double[n]}{$n \times n$ diagonal matrix {\bf W}
                     (diagonal elements only)} \\
 \spec{v}{double[np*np]}{array containing $n \times n$ orthogonal
                         matrix {\bf V} ({\it n.b.}\ not
                         {\bf V}$^{T}$)} \\
 \spec{work}{double[n]}{workspace} \\
 \spec{jstat}{int*}{0~=~OK, $-$1~=~array {\tt a} wrong shape,
                    $>$0~=~index of {\tt w}
                    for which convergence failed (see note~3, below)}
}
\notes
{
 \begin{enumerate}
  \item {\tt m} and {\tt n}
        are the {\it logical}\/ dimensions of the
        matrices and vectors concerned, which can be located in
        arrays of larger {\it physical}\/ dimensions,
        given by {\tt mp} and {\tt np}.
  \item {\tt v} contains matrix {\bf V},
        not the transpose of matrix {\bf V}.
  \item If the status {\tt jstat} is greater than zero, this need not
        necessarily be treated as a failure.  It means that, due to
        chance properties of the matrix {\bf A}, the QR transformation
        phase of the function did not fully converge in a predefined
        number of iterations, something that very seldom occurs.
        When this condition does arise, it is possible that the
        elements of the diagonal matrix
        {\bf W} have not been correctly
        found.  However, in practice the results are likely to
        be trustworthy.  Applications should report the condition
        as a warning, but then proceed normally.
  \item The algorithm is an adaptation of the
        function SVD in the {\it EISPACK}\,
        library (Garbow~{\it et~al.}\ 1977),
        which is a FORTRAN~66 implementation of the Algol
        function SVD of Wilkinson \& Reinsch (1971).
        These
        references give full details of the algorithm used here.
        A good account of the use of SVD
        in least squares problems is given
        in Press~{\it et~al.}\ (1987), which includes another
        variant of the EISPACK code.
  \item The lack of an ``adjustable dimension'' feature in C means that
        appropriate casts are needed in the application code.  Here is
        an example:
 \end{enumerate}
}
       \vspace{-5ex}
       \begin{verbatim}
          :
         double a[mp][np], w[np], v[np][np], work[np];
         int m, n, j;
          :
         slaSvd ( m, n, mp, np, (double*) a, w, (double*) v, work, &j );
          :
       \end{verbatim}
       \vspace{-6ex}
\refs{
 \begin{enumerate}
  \item Garbow~{\it et~al.}\ 1977, {\it EISPACK Guide Extension},
        Springer Verlag.
  \item Wilkinson \& Reinsch 1971, {\it Handbook for Automatic
        Computation}, vol~2, ed,\ Bauer~{\it et~al.},
        Springer Verlag).
  \item Press~{\it et~al.}\ 1987, {\it Numerical Recipes},
        Cambridge University Press.
 \end{enumerate}
}
%-----------------------------------------------------------------------
\routine{slaSvdcov}{covariance matrix from SVD}
{
 \action{From the {\bf W} and {\bf V} matrices from the SVD
         factorization of a matrix
         (as obtained from the {\tt slaSvd} function), obtain
         the covariance matrix.}
 \call{slaSvdcov( n, np, nc, w, v, work, cvm );}
}
\args{GIVEN}
{
 \spec{n}{int}{$n$, the number of rows and columns in
             matrices {\bf W} and {\bf V}} \\
 \spec{np}{int}{first dimension of array containing $n \times n$
              matrix {\bf V}} \\
 \spec{nc}{int}{first dimension of covariance matrix} \\
 \spec{w}{double[n]}{$n \times n$ diagonal matrix {\bf W}
                (diagonal elements only)} \\
 \spec{v}{double[np*np]}{array containing
          $n \times n$ orthogonal matrix {\bf V}}
}
\args{RETURNED}
{
 \spec{work}{double[n]}{workspace} \\
 \spec{cvm}{double[nc*nc]}{array to receive covariance matrix}
}
\anote{The lack of an ``adjustable dimension'' feature in C means that
       appropriate casts are needed in the application code.  Here is
       an example:}
       \vspace{-5ex}
       \begin{verbatim}
          :
         double w[np], v[np][np], work[np], c[nc][nc];
         int n;
          :
         slaSvdcov ( n, np, nc, w, (double *) v, work, (double *) c );
          :
       \end{verbatim}
       \vspace{-6ex}
\aref{Press~{\it et~al.}\ 1987, {\it Numerical Recipes},
      Cambridge University Press, \S 14.3.}
%-----------------------------------------------------------------------
\routine{slaSvdsol}{solution vector from SVD}
{
 \action{From a given vector and the SVD of a matrix (as obtained from
         the slaSvd function), obtain the solution vector.
         This function solves the equation:
         \begin{tabbing}
         XXXXXX \= \kill
         \> {\bf A} $\cdot$ {\bf x} = {\bf b}
         \end{tabbing}
         where:
         \begin{tabbing}
         XXXXXX \= XXXX \= \kill
         \> {\bf A} \> is a given $m$ (rows) $\times n$ (columns)
                       matrix, where $m \geq n$ \\
         \> {\bf x} \> is the $n$-vector we wish to find, and \\
         \> {\bf b} \> is a given $m$-vector
         \end{tabbing}
         by means of the {\it Singular Value Decomposition}\/ method (SVD).}
 \call{slaSvdsol( m, n, mp, np, b, u, w, v, work, x );}
}
\args{GIVEN}
{
 \spec{m,n}{int}{$m$, $n$, the numbers of rows and columns in matrix {\bf A}} \\
 \spec{mp,np}{int}{physical dimensions of array containing matrix {\bf A}} \\
 \spec{b}{D(M)}{known vector {\bf b}} \\
 \spec{u}{D(MP,NP)}{array containing $m \times n$ matrix {\bf U}} \\
 \spec{w}{D(N)}{$n \times n$ diagonal matrix {\bf W}
                (diagonal elements only)} \\
 \spec{v}{D(NP,NP)}{array containing $n \times n$ orthogonal matrix {\bf V}}
}
\args{RETURNED}
{
 \spec{work}{double[n]}{workspace} \\
 \spec{x}{double[n]}{unknown vector {\bf x}}
}
\notes
{
 \begin{enumerate}
  \item In the Singular Value Decomposition method (SVD),
        the matrix {\bf A} is first factorized (for example by
        the function {\tt slaSvd}) into the following components:
        \begin{tabbing}
        XXXXXX \= \kill
        \> {\bf A} = {\bf U} $\cdot$ {\bf W} $\cdot$ {\bf V}$^{T}$
        \end{tabbing}
        where:
        \begin{tabbing}
        XXXXXX \= XXXX \= \kill
        \> {\bf A} \> is any $m$ (rows) $\times n$ (columns) matrix,
                      where $m > n$ \\
        \> {\bf U} \> is an $m \times n$ column-orthogonal matrix \\
        \> {\bf W} \> is an $n \times n$ diagonal matrix with
                      $w_{ii} \geq 0$ \\
        \> {\bf V}$^{T}$ \> is the transpose of an $n \times n$
                            orthogonal matrix
        \end{tabbing}
        Note that $m$ and $n$ are the {\it logical}\/ dimensions of the
        matrices and vectors concerned, which can be located in
        arrays of larger {\it physical}\/ dimensions MP and NP.
        The solution is then found from the expression:
        \begin{tabbing}
        XXXXXX \= \kill
        \> {\bf x} = {\bf V} $\cdot~[diag(1/${\bf W}$_{j})]
           \cdot (${\bf U}$^{T} \cdot${\bf b})
        \end{tabbing}
  \item If matrix {\bf A} is square, and if the diagonal matrix {\bf W} is not
        altered, the method is equivalent to conventional solution
        of simultaneous equations.
  \item If $m > n$, the result is a least-squares fit.
  \item If the solution is poorly determined, this shows up in the
        SVD factorization as very small or zero {\bf W}$_{j}$ values.  Where
        a {\bf W}$_{j}$ value is small but non-zero it can be set to zero to
        avoid ill effects.  The present function detects such zero
        {\bf W}$_{j}$ values and produces a sensible solution, with highly
        correlated terms kept under control rather than being allowed
        to elope to infinity, and with meaningful values for the
        other terms.
  \item The lack of an ``adjustable dimension'' feature in C means that
        appropriate casts are needed in the application code.  Here is
        an example:
 \end{enumerate}
}
       \vspace{-5ex}
       \begin{verbatim}
         double a[mp][np], w[np], v[np][np], work[np], b[mp], x[np];
         int m, n;
          :
         slaSvdsol ( m, n, mp, np, b, (double *) a, w, (double *) v, work, x );
          :
       \end{verbatim}
       \vspace{-6ex}
\aref{Press~{\it et~al.}\ 1987, {\it Numerical Recipes},
      Cambridge University Press, \S 2.9.}
%-----------------------------------------------------------------------
\routine{slaTp2s}{tangent plane to spherical}
{
 \action{Transform tangent plane coordinates into spherical
         coordinates (single precision)}
 \call{slaTp2s( xi, eta, raz, decz, \&ra, \&dec );}
}
\args{GIVEN}
{
 \spec{xi,eta}{float}{tangent plane rectangular coordinates (radians)} \\
 \spec{raz,decz}{float}{spherical coordinates of tangent point (radians)}
}
\args{RETURNED}
{
 \spec{ra,dec}{float*}{spherical coordinates (radians)}
}
\notes
{
 \begin{enumerate}
  \item The projection is called the {\it gnomonic}\/ projection;  the
        Cartesian coordinates \xieta\ are called
        {\it standard coordinates.}\/  The latter
        are in units of the distance from the tangent plane to the projection
        point, {\it i.e.}~radians near the origin.
  \item When working in \xyz\ rather than spherical coordinates, the
        equivalent Cartesian function {\tt slaTp2v} is available.
 \end{enumerate}
}
%-----------------------------------------------------------------------
\routine{slaTp2v}{tangent plane to \xyz}
{
 \action{Given the tangent-plane coordinates of a star and the direction
         cosines of the tangent point, determine the direction cosines
         of the star
         (single precision).}
 \call{slaTp2v( xi, eta, v0, v );}
}
\args{GIVEN}
{
 \spec{xi,eta}{float}{tangent plane coordinates of star (radians)} \\
 \spec{v0}{float[3]}{direction cosines of tangent point}
}
\args{RETURNED}
{
 \spec{v}{float[3]}{direction cosines of star}
}
\notes
{
 \begin{enumerate}
  \item If vector {\tt v0} is not of unit length,
        the returned vector {\tt v} will
        be wrong.
  \item If vector {\tt v0} points at a pole,
        the returned vector {\tt v} will be
        based on the arbitrary assumption that $\alpha=0$ at
        the tangent point.
  \item The projection is called the {\it gnomonic}\/ projection;  the
        Cartesian coordinates \xieta\ are called
        {\it standard coordinates.}\/  The latter
        are in units of the distance from the tangent plane to the projection
        point, {\it i.e.}~radians near the origin.
  \item This function is the Cartesian equivalent of the function {\tt slaTp2s}.
 \end{enumerate}
}
%-----------------------------------------------------------------------
\routine{slaTps2c}{plate centre from \xieta\ and \radec}
{
 \action{From the tangent plane coordinates of a star of known \radec,
        determine the \radec\ of the tangent point (single precision)}
 \call{slaTps2c( xi, eta, ra, dec, \&raz1, \&decz1, \&raz2, \&decz2, \&n );}
}
\args{GIVEN}
{
 \spec{xi,eta}{float}{tangent plane rectangular coordinates (radians)} \\
 \spec{ra,dec}{float}{spherical coordinates (radians)}
}
\args{RETURNED}
{
 \spec{raz1,decz1}{float*}{spherical coordinates of tangent point,
                      solution 1} \\
 \spec{raz2,decz2}{float*}{spherical coordinates of tangent point,
                      solution 2} \\
 \spec{n}{int*}{number of solutions:} \\
 \spec{}{}{\hspace{1em} 0 = no solutions returned  (Note~2)} \\
 \spec{}{}{\hspace{1em} 1 = only the first solution is useful (Note~3)} \\
 \spec{}{}{\hspace{1em} 2 = there are two useful solutions (Note~3)}
}
\notes
{
 \begin{enumerate}
  \item The {\tt raz1} and {\tt raz2} values
        returned are in the range $0\!-\!2\pi$.
  \item Cases where there is no solution can only arise near the poles.
        For example, it is clearly impossible for a star at the pole
        itself to have a non-zero $\xi$ value, and hence it is
        meaningless to ask where the tangent point would have to be
        to bring about this combination of $\xi$ and $\delta$.
  \item Also near the poles, cases can arise where there are two useful
        solutions.  The argument {\tt n}
        indicates whether the second of the
        two solutions returned is useful.  ${\tt n}=1$
        indicates only one useful solution, the usual case;  under
        these circumstances, the second solution corresponds to the
        ``beyond the pole'' case, and this is reflected in the values
        of {\tt raz2} and {\tt decz2} which are returned.
  \item The {\tt decz1} and {\tt decz2} values
        returned are in the range $\pm\pi$,
        but in the ordinary, non-pole-crossing, case, the range is
        $\pm\pi/2$.
  \item {\tt ra}, {\tt dec},
        {\tt raz1}, {\tt decz1},
        {\tt raz2}, {\tt decz2} are all in radians.
  \item The projection is called the {\it gnomonic}\/ projection;  the
        Cartesian coordinates \xieta\ are called
        {\it standard coordinates.}\/  The latter
        are in units of the distance from
        the tangent plane to the projection
        point, {\it i.e.}~radians near the origin.
  \item When working in \xyz\ rather than spherical coordinates, the
        equivalent Cartesian function {\tt slaTpv2c} is available.
 \end{enumerate}
}
%-----------------------------------------------------------------------
\routine{slaTpv2c}{plate centre from \xieta\ and \xyz}
{
 \action{From the tangent plane coordinates of a star of known
         direction cosines, determine the direction cosines
         of the tangent point (single precision)}
 \call{slaTpv2c( xi, eta, v, v01, v02, \&n );}
}
\args{GIVEN}
{
 \spec{xi,eta}{float}{tangent plane coordinates of star (radians)} \\
 \spec{v}{float[3]}{direction cosines of star}
}
\args{RETURNED}
{
 \spec{v01}{float[3]}{direction cosines of tangent point, solution 1} \\
 \spec{v01}{float[3]}{direction cosines of tangent point, solution 2} \\
 \spec{n}{int*}{number of solutions:} \\
 \spec{}{}{\hspace{1em} 0 = no solutions returned  (Note~2)} \\
 \spec{}{}{\hspace{1em} 1 = only the first solution is useful (Note~3)} \\
 \spec{}{}{\hspace{1em} 2 = there are two useful solutions (Note~3)}
}
\notes
{
 \begin{enumerate}
  \item The vector {\tt v}
        must be of unit length or the result will be wrong.
  \item Cases where there is no solution can only arise near the poles.
        For example, it is clearly impossible for a star at the pole
        itself to have a non-zero $\xi$ value.
  \item Also near the poles, cases can arise where there are two useful
        solutions.  The argument {\tt n}
        indicates whether the second of the
        two solutions returned is useful.
        ${\tt n}=1$
        indicates only one useful solution, the usual case;  under these
        circumstances, the second solution can be regarded as valid if
        the vector {\tt v02} is interpreted as
        the ``beyond the pole'' case.
  \item The projection is called the {\it gnomonic}\/ projection;  the
        Cartesian coordinates \xieta\ are called
        {\it standard coordinates.}\/  The latter
        are in units of the distance from the
        tangent plane to the projection
        point, {\it i.e.}~radians near the origin.
  \item This function is the Cartesian equivalent
        of the function {\tt slaTps2c}.
 \end{enumerate}
}
%-----------------------------------------------------------------------
\routine{slaUe2el}{universal to conventional elements}
{
 \action{Transform universal elements into conventional heliocentric
         osculating elements.}
 \call{slaUe2el( \vtop{
         \hbox{u, jformr,}
         \hbox{\&jform, \&epoch, \&orbinc, \&anode, \&perih,}
         \hbox{\&aorq, \&e, \&aorl, \&dm, \&jstat);}}}
}
\args{GIVEN}
{
 \spec{u}{double[13]}{universal orbital elements (updated; Note~1)} \\
 \specel{0}     {combined mass ($M+m$)} \\
 \specel{1}     {total energy of the orbit ($\alpha$)} \\
 \specel{2}     {reference (osculating) epoch ($t_0$)} \\
 \specel{3-5}   {position at reference epoch (${\rm \bf r}_0$)} \\
 \specel{6-8}   {velocity at reference epoch (${\rm \bf v}_0$)} \\
 \specel{9}     {heliocentric distance at reference epoch} \\
 \specel{10}    {${\rm \bf r}_0.{\rm \bf v}_0$} \\
 \specel{11}    {date ($t$)} \\
 \specel{12}    {universal eccentric anomaly ($\psi$) of date, approx} \\ \\
 \spec{jformr}{int}{requested element set (1-3; Note~3)}
}
\args{RETURNED}
{
 \spec{jform}{int*}{element set actually returned (1-3; Note~4)} \\
 \spec{epoch}{double*}{epoch of elements ($t_0$ or $T$, TDB MJD)} \\
 \spec{orbinc}{double*}{inclination ($i$, radians)} \\
 \spec{anode}{double*}{longitude of the ascending node ($\Omega$, radians)} \\
 \spec{perih}{double*}{longitude or argument of perihelion
                            ($\varpi$ or $\omega$,} \\
 \spec{}{}{\hspace{1.5em} radians)} \\
 \spec{aorq}{double*}{mean distance or perihelion distance ($a$ or $q$, AU)} \\
 \spec{e}{double*}{eccentricity ($e$)} \\
 \spec{aorl}{double*}{mean anomaly or longitude
                               ($M$ or $L$, radians,} \\
 \spec{}{}{\hspace{1.5em} ${\tt jform}=1,2$ only)} \\
 \spec{dm}{double*}{daily motion ($n$, radians, ${\tt jform}=1$ only)} \\
 \spec{jstat}{int*}{status:} \\
 \spec{}{}{\hspace{2.3em}    0 = OK} \\
 \spec{}{}{\hspace{1.5em} $-$1 = illegal {\tt pmass}} \\
 \spec{}{}{\hspace{1.5em} $-$2 = illegal {\tt jformr}} \\
 \spec{}{}{\hspace{1.5em} $-$3 = position/velocity out of allowed range}
}
\notes
{
 \begin{enumerate}
  \setlength{\parskip}{\medskipamount}
  \item The ``universal'' elements are those which define the orbit for the
        purposes of the method of universal variables (see reference 2).
        They consist of the combined mass of the two bodies, an epoch,
        and the position and velocity vectors (arbitrary reference frame)
        at that epoch.  The parameter set used here includes also various
        quantities that can, in fact, be derived from the other
        information.  This approach is taken to avoiding unnecessary
        computation and loss of accuracy.  The supplementary quantities
        are (i)~$\alpha$, which is proportional to the total energy of the
        orbit, (ii)~the heliocentric distance at epoch,
        (iii)~the outwards component of the velocity at the given epoch,
        (iv)~an estimate of $\psi$, the ``universal eccentric anomaly'' at a
        given date and (v)~that date.
  \item The universal elements are with respect to the mean equator and
        equinox of epoch J2000.  The orbital elements produced are with
        respect to the J2000 ecliptic and mean equinox.
  \item Three different element-format options are supported, as
        follows. \\
 
        ${\tt jform}=1$, suitable for the major planets:
 
        \begin{tabbing}
        xxx \= xxxxxxxx \= xx \= \kill
        \> {\tt epoch}  \> = \> epoch of elements $t_0$ (TDB MJD) \\
        \> {\tt orbinc} \> = \> inclination $i$ (radians) \\
        \> {\tt anode}  \> = \> longitude of the
                                ascending node $\Omega$ (radians) \\
        \> {\tt perih}  \> = \> longitude of perihelion
                                $\varpi$ (radians) \\
        \> {\tt aorq}   \> = \> mean distance $a$ (AU) \\
        \> {\tt e}      \> = \> eccentricity $e$ $( 0 \leq e < 1 )$ \\
        \> {\tt aorl}   \> = \> mean longitude $L$ (radians) \\
        \> {\tt dm}     \> = \> daily motion $n$ (radians)
        \end{tabbing}
 
        ${\tt jform}=2$, suitable for minor planets:
 
        \begin{tabbing}
        xxx \= xxxxxxxx \= xx \= \kill
        \> {\tt epoch}  \> = \> epoch of elements $t_0$ (TDB MJD) \\
        \> {\tt orbinc} \> = \> inclination $i$ (radians) \\
        \> {\tt anode}  \> = \> longitude of the ascending node
                                $\Omega$ (radians) \\
        \> {\tt perih}  \> = \> argument of perihelion
                                $\omega$ (radians) \\
        \> {\tt aorq}   \> = \> mean distance $a$ (AU) \\
        \> {\tt e}      \> = \> eccentricity $e$ $( 0 \leq e < 1 )$ \\
        \> {\tt aorl}   \> = \> mean anomaly $M$ (radians)
        \end{tabbing}
 
        ${\tt jform}=3$, suitable for comets:
 
        \begin{tabbing}
        xxx \= xxxxxxxx \= xx \= \kill
        \> {\tt epoch}  \> = \> epoch of perihelion $T$ (TDB MJD) \\
        \> {\tt orbinc} \> = \> inclination $i$ (radians) \\
        \> {\tt anode}  \> = \> longitude of the ascending node
                                $\Omega$ (radians) \\
        \> {\tt perih}  \> = \> argument of perihelion $\omega$ (radians) \\
        \> {\tt aorq}   \> = \> perihelion distance $q$ (AU) \\
        \> {\tt e}   \> = \> eccentricity $e$ $( 0 \leq e \leq 10 )$
        \end{tabbing}
  \item It may not be possible to generate elements in the form
        requested through {\tt jformr}.  The caller is notified of the form
        of elements actually returned by means of the {\tt jform} argument:
 
        \begin{tabbing}
        xx \= xxxxxxxxxx \= xxxxxxxxxxx \= \kill
        \> {\tt jformr} \> {\tt jform} \> meaning \\ \\
        \> ~~~~~1   \> ~~~~1  \> OK: elements are in the requested format \\
        \> ~~~~~1   \> ~~~~2  \> never happens \\
        \> ~~~~~1   \> ~~~~3  \> orbit not elliptical \\
        \> ~~~~~2   \> ~~~~1  \> never happens \\
        \> ~~~~~2   \> ~~~~2  \> OK: elements are in the requested format \\
        \> ~~~~~2   \> ~~~~3  \> orbit not elliptical \\
        \> ~~~~~3   \> ~~~~1  \> never happens \\
        \> ~~~~~3   \> ~~~~2  \> never happens \\
        \> ~~~~~3   \> ~~~~3  \> OK: elements are in the requested format
        \end{tabbing}
  \item The arguments returned for each value of
        {\tt jform} ({\it cf.}\/ Note~5:
        {\tt jform} might not be the same as {\tt jformr}) are as follows:
 
        \begin{tabbing}
        xxx \= xxxxxxxxxxxx \= xxxxxx \= xxxxxx \= \kill
        \> {\tt jform}  \> 1        \> 2        \> 3 \\ \\
        \> {\tt epoch}  \> $t_0$    \> $t_0$    \> $T$ \\
        \> {\tt orbinc} \> $i$      \> $i$      \> $i$ \\
        \> {\tt anode}  \> $\Omega$ \> $\Omega$ \> $\Omega$ \\
        \> {\tt perih}  \> $\varpi$ \> $\omega$ \> $\omega$ \\
        \> {\tt aorq}   \> $a$      \> $a$      \> $q$ \\
        \> {\tt e}      \> $e$      \> $e$      \> $e$ \\
        \> {\tt aorl}   \> $L$      \> $M$      \> - \\
        \> {\tt dm}     \> $n$      \> -        \> -
        \end{tabbing}
 
        where:
        \begin{tabbing}
        xxx \= xxxxxxxx \= xxx \= \kill
        \> $t_0$    \> is the epoch of the elements (MJD, TT) \\
        \> $T$      \> is the epoch of perihelion (MJD, TT) \\
        \> $i$      \> is the inclination (radians) \\
        \> $\Omega$ \> is the longitude of the ascending node (radians) \\
        \> $\varpi$ \> is the longitude of perihelion (radians) \\
        \> $\omega$ \> is the argument of perihelion (radians) \\
        \> $a$      \> is the mean distance (AU) \\
        \> $q$      \> is the perihelion distance (AU) \\
        \> $e$      \> is the eccentricity \\
        \> $L$      \> is the longitude (radians, $0-2\pi$) \\
        \> $M$      \> is the mean anomaly (radians, $0-2\pi$) \\
        \> $n$      \> is the daily motion (radians) \\
        \> - \> means no value is set
        \end{tabbing}
  \item At very small inclinations, the longitude of the ascending node
        {\tt anode} becomes indeterminate
        and under some circumstances may be
        set arbitrarily to zero.  Similarly, if the orbit is close to
        circular, the true anomaly becomes indeterminate and under some
        circumstances may be set arbitrarily to zero.  In such cases,
        the other elements are automatically adjusted to compensate,
        and so the elements remain a valid description of the orbit.
  \item The various epochs are in the TDB time-scale (TT will do) and are
        Modified Julian Dates (JD$-$2400000.5).
 \end{enumerate}
}
\refs{
   \begin{enumerate}
   \item Sterne, Theodore E.\ 1960
         {\it An Introduction to Celestial Mechanics},
         Interscience Publishers, Section 6.7, p199.
   \item Everhart, E. \& Pitkin, E.T.\ 1983, {\it Am.J.Phys.},
         {\bf 51}, 712.
   \end{enumerate}
}
%-----------------------------------------------------------------------
\routine{slaUe2pv}{position \& velocity from universal elements}
{
 \action{Heliocentric position and velocity of a planet, asteroid or comet,
         starting from orbital elements in the ``universal variables'' form.}
 \call{slaUe2pv( date, u, pv, \&jstat );}
}
\args{GIVEN}
{
 \spec{date}{double}{date (TT Modified Julian Date = JD$-$2400000.5)}
}
\args{GIVEN and RETURNED}
{
 \spec{u}{double[13]}{universal orbital elements (updated; Note~1)} \\
 \specel{0}     {combined mass ($M+m$)} \\
 \specel{1}     {total energy of the orbit ($\alpha$)} \\
 \specel{2}     {reference (osculating) epoch ($t_0$)} \\
 \specel{3-5}   {position at reference epoch (${\rm \bf r}_0$)} \\
 \specel{6-8}   {velocity at reference epoch (${\rm \bf v}_0$)} \\
 \specel{9}     {heliocentric distance at reference epoch} \\
 \specel{10}    {${\rm \bf r}_0.{\rm \bf v}_0$} \\
 \specel{11}    {date ($t$)} \\
 \specel{12}    {universal eccentric anomaly ($\psi$) of date, approx}
}
\args{RETURNED}
{
 \spec{pv}{double[6]}{heliocentric \xyzxyzd, equatorial, J2000} \\
 \spec{}{}{\hspace{1.5em} (AU, AU/s; Note~1)} \\
 \spec{jstat}{int*}{status:} \\
 \spec{}{}{\hspace{1.95em}      0 = OK} \\
 \spec{}{}{\hspace{1.2em}    $-$1 = radius vector zero} \\
 \spec{}{}{\hspace{1.2em}    $-2$ = failed to converge}
}
\notes
{
 \begin{enumerate}
  \setlength{\parskip}{\medskipamount}
  \item The ``universal'' elements are those which define the orbit for the
        purposes of the method of universal variables (see reference).
        They consist of the combined mass of the two bodies, an epoch,
        and the position and velocity vectors (arbitrary reference frame)
        at that epoch.  The parameter set used here includes also various
        quantities that can, in fact, be derived from the other
        information.  This approach is taken to avoiding unnecessary
        computation and loss of accuracy.  The supplementary quantities
        are (i)~$\alpha$, which is proportional to the total energy of the
        orbit, (ii)~the heliocentric distance at epoch,
        (iii)~the outwards component of the velocity at the given epoch,
        (iv)~an estimate of $\psi$, the ``universal eccentric anomaly'' at a
        given date and (v)~that date.
  \item The companion function is {\tt slaEl2ue}.  This takes the conventional
        orbital elements and transforms them into the set of numbers
        needed by the present function.  A single prediction requires one
        one call to {\tt slaEl2ue} followed by one call to the present function;
        for convenience, the two calls are packaged as the function
        {\tt slaPlanel}.  Multiple predictions may be made by again
        calling {\tt slaEl2ue} once, but then calling the present function
        multiple times, which is faster than multiple calls to {\tt slaPlanel}.
 
        It is not obligatory to use {\tt slaEl2ue} to obtain the parameters.
        However, it should be noted that because {\tt slaEl2ue} performs its
        own validation, no checks on the contents of the array U are made
        by the present function.
  \item {\tt date} is the instant for which the
        prediction is required.  It is
        in the TDB time-scale (TT will do) and is a
        Modified Julian Date (JD$-$2400000.5).
  \item The universal elements supplied in the array
        {\tt u} are in canonical
        units (solar masses, AU and canonical days).  The position and
        velocity are not sensitive to the choice of reference frame.  The
        {\tt slaEl2ue} function in fact produces coordinates with respect to the
        J2000 equator and equinox.
  \item The algorithm was originally adapted from the
        {\tt EPHSLA} Fortran program of
        D.\,H.\,P.\,Jones (private communication, 1996).  The method
        is based on Stumpff's Universal Variables.
 \end{enumerate}
}
\aref{Everhart, E. \& Pitkin, E.T.\ 1983, {\it Am.J.Phys.}\ {\bf 51}, 712.}
%-----------------------------------------------------------------------
\routine{slaUnpcd}{remove radial distortion}
{
 \action{Remove pincushion/barrel distortion from a distorted
         \xy\  to give tangent-plane \xy.}
 \call{slaUnpcd( disco, \&x, \&y );}
}
\args{GIVEN}
{
 \spec{disco}{double}{pincushion/barrel distortion coefficient} \\
 \spec{x,y}{double*}{distorted \xy}
}
\args{RETURNED}
{
 \spec{x,y}{double*}{tangent-plane \xy}
}
\notes
{
 \begin{enumerate}
  \item The distortion is of the form $\rho = r (1 + c r^{2})$, where $r$ is
        the radial distance from the tangent point, $c$ is the {\tt disco}
        argument, and $\rho$ is the radial distance in the presence of
        the distortion.
  \item For {\it pincushion}\/ distortion,
        $c$ is positive;  for
        {\it barrel}\/ distortion, $c$ is negative.
  \item For {\tt x},{\tt y} in units
        of one projection radius (in the case of
        a photographic plate, the focal length), the following
        {\tt disco} values apply:
 
        \vspace{2ex}
 
        \hspace{5em}
        \begin{tabular}{|l|c|} \hline
         {\it geometry} & {\tt disco} \\ \hline \hline
         astrograph & 0.0 \\ \hline
         Schmidt & $-$0.3333 \\ \hline
         AAT PF doublet & +147.069 \\ \hline
         AAT PF triplet & +178.585 \\ \hline
         AAT f/8 & +21.20 \\ \hline
         JKT f/8 & +14.6 \\ \hline
        \end{tabular}
 
        \vspace{2ex}
 
  \item The present function is a rigorous inverse of the companion
        function {\tt slaPcd}.  The expression for $\rho$ in Note~1
        is rewritten in the form $x^3 + ax + b = 0$ and solved by
        standard techniques.
 
  \item Cases where the cubic has multiple real roots can sometimes
        occur, corresponding to extreme instances of barrel distortion
        where up to three different undistorted \xy s all produce the
        same distorted \xy.  However, only one solution is returned,
        the one that produces the smallest change in \xy.
 \end{enumerate}
}
%-----------------------------------------------------------------------
\routine{slaV2tp}{\xyz\ to tangent plane}
{
 \action{Given the direction cosines of a star and of the tangent point,
         determine the star's tangent-plane coordinates
         (single precision).}
 \call{slaV2tp( v, v0, \&xi, \&eta, \&j );}
}
\args{GIVEN}
{
 \spec{v}{float[3]}{direction cosines of star} \\
 \spec{v0}{float[3]}{direction cosines of tangent point}
}
\args{RETURNED}
{
 \spec{xi,eta}{float*}{tangent plane coordinates (radians)} \\
 \spec{j}{int*}{status:} \\
 \spec{}{}{\hspace{1.5em} 0 = OK, star on tangent plane} \\
 \spec{}{}{\hspace{1.5em} 1 = error, star too far from axis} \\
 \spec{}{}{\hspace{1.5em} 2 = error, antistar on tangent plane} \\
 \spec{}{}{\hspace{1.5em} 3 = error, antistar too far from axis}
}
\notes
{
 \begin{enumerate}
  \item If vector {\tt v0} is not of unit length,
        or if vector {\tt v} is of zero
        length, the results will be wrong.
  \item If {\tt v0} points at a pole, the returned $\xi,\eta$
        will be based on the
        arbitrary assumption that $\alpha=0$ at the tangent point.
  \item The projection is called the {\it gnomonic}\/ projection;  the
        Cartesian coordinates \xieta\ are called
        {\it standard coordinates.}\/  The latter
        are in units of the distance from the tangent plane to the projection
        point, {\it i.e.}~radians near the origin.
  \item This function is the Cartesian equivalent of the function {\tt slaS2tp}.
 \end{enumerate}
}
%-----------------------------------------------------------------------
\routine{slaVdv}{scalar product}
{
 \action{Scalar product of two 3-vectors (single precision).}
 \call{r~=~slaVdv( va, vb );}
}
\args{GIVEN}
{
 \spec{va}{float[3]}{first vector} \\
 \spec{vb}{float[3]}{second vector}
}
\args{RETURNED}
{
 \spec{}{float*}{scalar product {\tt va.vb}}
}
%-----------------------------------------------------------------------
\routine{slaVn}{normalize vector}
{
 \action{Normalize a 3-vector, also giving the modulus (single precision).}
 \call{slaVn( v, uv, \&vm );}
}
\args{GIVEN}
{
 \spec{v}{float[3]}{vector}
}
\args{RETURNED}
{
 \spec{uv}{float[3]}{unit vector $\hat{\tt v}$} \\
 \spec{vm}{float*}{modulus $\mid\!{\tt v}\!\mid$}
}
\anote{If the modulus of {\tt v} is zero,
       {\tt uv} is set to the null vector.}
%-----------------------------------------------------------------------
\routine{slaVxv}{vector product}
{
 \action{Vector product of two 3-vectors (single precision).}
 \call{slaVxv( va, vb, vc );}
}
\args{GIVEN}
{
 \spec{va}{float[3]}{first vector} \\
 \spec{vb}{float[3]}{second vector}
}
\args{RETURNED}
{
 \spec{vc}{float[3]}{vector product {\tt va}$\times${\tt vb}}
}
%-----------------------------------------------------------------------
\routine{slaXy2xy}{apply linear model to an \xy}
{
 \action{Transform one \xy\ into another using a linear model of the type
         produced by the slaFitxy function.}
 \call{slaXy2xy( x1, y1, coeffs, \&x2, \&y2 );}
}
\args{GIVEN}
{
 \spec{x1,y1}{double}{\xy\ before transformation} \\
 \spec{coeffs}{double[6]}{transformation coefficients (see note~1)}
}
\args{RETURNED}
{
 \spec{x2,y2}{double*}{\xy\ after transformation}
}
\notes
{
 \begin{enumerate}
  \item The model relates two sets of \xy\ coordinates as follows.
        Naming the six elements of {\tt coeffs} $a,b,c,d,e$ \& $f$,
        the present function performs the transformation:
        \begin{verse}
          $x_{2} = a + bx_{1} + cy_{1}$ \\
          $y_{2} = d + ex_{1} + fy_{1}$
        \end{verse}
  \item See also {\tt slaFitxy}, {\tt slaPxy}, {\tt slaInvf}, {\tt slaDcmpf}.
 \end{enumerate}
}
%-----------------------------------------------------------------------
\routine{slaZd}{\hadec\ to zenith distance}
{
 \action{Hour angle and declination to zenith distance.}
 \call{d~=~slaZd( ha, dec, phi );}
}
\args{GIVEN}
{
 \spec{ha}{double}{hour angle in radians} \\
 \spec{dec}{double}{declination in radians} \\
 \spec{phi}{double}{latitude in radians}
}
\args{RETURNED}
{
 \spec{}{double}{zenith distance (radians, $0\!-\!\pi$)}
}
\notes
{
 \begin{enumerate}
  \item The latitude must be geodetic.  In critical applications,
        corrections for polar motion should be applied (see {\tt slaPolmo}).
  \item In some applications it will be important to specify the
        correct type of hour angle and declination in order to
        produce the required type
        of zenith distance.  In particular, it may be
        important to distinguish between the zenith distance
        as affected by refraction, which would require the
        {\it observed}\/ \hadec, and the zenith distance {\it in vacuo},
        which would require the {\it topocentric}\/ \hadec.  If
        the effects of diurnal aberration can be neglected, the
        {\it apparent}\/ \hadec\ may be used instead of the
        {\it topocentric}\/ \hadec.
  \item No range checking of arguments is done.
  \item In applications which involve many zenith distance calculations,
        rather than calling the present function it will be more
        efficient to use inline code, having previously computed fixed
        terms such as sine and cosine of latitude, and perhaps sine and
        cosine of declination.
 \end{enumerate}
}
\fi
\vfill
\pagebreak

\section{SUMMARY OF CALLS}
The basic trigonometrical and numerical facilities are supplied in both
single precision ({\tt float}) and double precision ({\tt double}) versions.
Most of the more esoteric position and time functions use double precision
arguments only, even in cases where single precision would normally
be adequate in practice.
Certain functions with modest accuracy objectives are supplied in
single precision versions only.
In the calling sequences which follow, no attempt has been made
to distinguish between single and double precision argument names,
and frequently the same name is used on different occasions to
mean different things.
However, none of the functions uses a mixture of single and
double precision arguments;  each function is either wholly
single precision or wholly double precision.
 
In the classified list, below, functions with calls beginning \fstring{r=}
return a {\tt float} whereas those beginning \fstring{d=}
return a {\tt double}; the rest are {\tt void}.
 
The list is, of course, merely for quick reference;  inexperienced
users {\bf must} refer to the detailed specifications given later.
In particular, {\bf don't guess} whether arguments are single or
double precision; the result could be a program that happens to
works on one platform but not on another.
 
\callhead{String Decoding}
\begin{callset}
\subq{slaInt2in( string, \&nstrt, \&ireslt, \&j );}
     {slaIntin( string, \&nstrt, \&lreslt, \&j );}
   Convert free-format string into integer
\subq{slaFlotin( string, \&nstrt, \&reslt, \&j );}
     {slaDfltin( string, \&nstrt, \&dreslt, \&j );}
   Convert free-format string into floating-point number
\subq{slaAfin( string, \&nstrt, \&reslt, \&j );}
     {slaDafin( string, \&nstrt, \&dreslt, \&j );}
   Convert free-format string from deg,arcmin,arcsec to radians
\end{callset}
 
\callhead{Sexagesimal Conversions}
\begin{callset}
\subq{slaCtf2d( ihour, imin, sec, \&days, \&j );}
     {slaDtf2d( ihour, imin, sec, \&days, \&j );}
   Hours, minutes, seconds to days
\subq{slaCd2tf( ndp, days, \&sign, ihmsf );}
     {slaDd2tf( ndp, days, \&sign, ihmsf );}
   Days to hours, minutes, seconds
\subq{slaCtf2r( ihour, imin, sec, \&rad, \&j );}
     {slaDtf2r( ihour, imin, sec, \&rad, \&j );}
   Hours, minutes, seconds to radians
\subq{slaCr2tf( ndp, angle, \&sign, ihmsf );}
     {slaDr2tf( ndp, angle, \&sign, ihmsf );}
   Radians to hours, minutes, seconds
\subq{slaCaf2r( ideg, iamin, asec, \&rad, \&j );}
     {slaDaf2r( ideg, iamin, asec, \&rad, \&j );}
   Degrees, arcminutes, arcseconds to radians
\subq{slaCr2af( ndp, angle, \&sign, idmsf );}
     {slaDr2af( ndp, angle, \&sign, idmsf );}
   Radians to degrees, arcminutes, arcseconds
\end{callset}
 
\callhead{Angles, Vectors and Rotation Matrices}
\begin{callset}
\subq{r~=~slaRange( angle );}
     {d~=~slaDrange( angle );}
   Normalize angle into range $\pm\pi$
\subq{r~=~slaRanorm( angle );}
     {d~=~slaDranrm( angle );}
   Normalize angle into range $0\!-\!2\pi$
\subq{slaCs2c( a, b, v );}
     {slaDcs2c( a, b, v );}
   Spherical coordinates to \xyz
\subq{slaCc2s( v, \&a, \&b );}
     {slaDcc2s( v, \&a, \&b );}
   \xyz\ to spherical coordinates
\subq{r~=~slaVdv( va, vb );}
     {d~=~slaDvdv( va, vb );}
   Scalar product of two 3-vectors
\subq{slaVxv( va, vb, vc );}
     {slaDvxv( va, vb, vc );}
   Vector product of two 3-vectors
\subq{slaVn( v, uv, \&vm );}
     {slaDvn( v, uv, \&vm );}
   Normalize a 3-vector also giving the modulus
\subq{r~=~slaSep( a1, b1, a2, b2 );}
     {d~=~slaDsep( a1, b1, a2, b2 );}
   Angle between two points on a sphere
\subq{r~=~slaSepv( v1, v2 );}
     {d~=~slaDsepv( v1, v2 );}
   Angle between two \xyz\ vectors
\subq{r~=~slaBear( a1, b1, a2, b2 );}
     {d~=~slaDbear( a1, b1, a2, b2 );}
   Direction of one point on a sphere seen from another
\subq{r~=~slaPav( v1, v2 );}
     {d~=~slaDpav( v1, v2 );}
   Position-angle of one \xyz\ with respect to another
\subq{slaEuler( order, phi, theta, psi, rmat );}
     {slaDeuler( order, phi, theta, psi, rmat );}
   Form rotation matrix from three Euler angles
\subq{slaAv2m( axvec, rmat );}
     {slaDav2m( axvec, rmat );}
   Form rotation matrix from axial vector
\subq{slaM2av( rmat, axvec );}
     {slaDm2av( rmat, axvec );}
   Determine axial vector from rotation matrix
\subq{slaMxv( rm, va, vb );}
     {slaDmxv( dm, va, vb );}
   Rotate vector forwards
\subq{slaImxv( rm, va, vb );}
     {slaDimxv( dm, va, vb );}
   Rotate vector backwards
\subq{slaMxm( a, b, c );}
     {slaDmxm( a, b, c );}
   Product of two $3\times3$ matrices
\subq{slaCs2c6( a, b, r, ad, bd, rd, v );}
     {slaDs2c6( a, b, r, ad, bd, rd, v );}
   Transformation of position and velocity in spherical
     coordinates to Cartesian coordinates
\subq{slaCc62s( v, \&a, \&b, \&r, \&ad, \&bd, \&rd );}
     {slaDc62s( v, \&a, \&b, \&r, \&ad, \&bd, \&rd );}
   Transformation of position and velocity in Cartesian
     coordinates to spherical coordinates
\end{callset}
 
\callhead{Calendars}
\begin{callset}
\subp{slaCldj( iy, im, id, \&djm, \&j );}
   Gregorian Calendar to Modified Julian Date
\subp{slaCaldj( iy, im, id, \&djm, \&j );}
   Gregorian Calendar to Modified Julian Date,
     permitting century default
\subp{slaDjcal( ndp, djm, iymdf, \&j );}
   Modified Julian Date to Gregorian Calendar,
     in a form convenient for formatted output
\subp{slaDjcl( djm, \&iy, \&im, \&id, \&fd, \&j );}
   Modified Julian Date to Gregorian Year, Month, Day, Fraction
\subp{slaCalyd( iy, im, id, \&ny, \&nd, \&j );}
   Calendar to year and day in year, permitting century default
\subp{slaClyd( iy, im, id, \&ny, \&nd, \&j );}
   Calendar to year and day in year
\subp{d~=~slaEpb( date );}
   Modified Julian Date to Besselian Epoch
\subp{d~=~slaEpb2d( epb );}
   Besselian Epoch to Modified Julian Date
\subp{d~=~slaEpj( date );}
   Modified Julian Date to Julian Epoch
\subp{d~=~slaEpj2d( epj );}
   Julian Epoch to Modified Julian Date
\end{callset}
 
\callhead{Time-scales}
\begin{callset}
\subp{d~=~slaDat( utc );}
   Offset of Atomic Time from Coordinated Universal Time: TAI$-$UTC
\subp{d~=~slaDt( epoch );}
   Approximate offset between dynamical time and universal time
\subp{d~=~slaDtt( utc );}
   Offset of Terrestrial Time from Coordinated Universal Time: TT$-$UTC
\subp{d~=~slaRcc( tdb, ut1, wl, u, v );}
   Relativistic clock correction: TDB$-$TT
\end{callset}
 
\callhead{Sidereal Time}
\begin{callset}
\subp{d~=~slaGmst( ut1 );}
   Transformation from Universal Time to sidereal time
\subp{d~=~slaGmsta( date, ut1 );}
   Transformation from Universal Time to sidereal time,
   rounding errors minimized
\subp{d~=~slaEqeqx( date );}
   Equation of the equinoxes
\end{callset}
 
\callhead{Precession and Nutation}
\begin{callset}
\subp{slaNut( date, rmatn );}
   Nutation matrix
\subp{slaNutc( date, \&dpsi, \&deps, \&eps0 );}
   Longitude and obliquity components of nutation, and
     mean obliquity
\subp{slaNutc80( date, \&dpsi, \&deps, \&eps0 );}
   Longitude and obliquity components of nutation, and
     mean obliquity, IAU 1980
\subp{slaPrec( ep0, ep1, rmatp );}
   Precession matrix ( IAU )
\subp{slaPrecl( ep0, ep1, rmatp );}
   Precession matrix ( suitable for long periods )
\subp{slaPrenut( epoch, date, rmatpn );}
   Combined precession-nutation matrix
\subp{slaPrebn( bep0, bep1, rmatp );}
   Precession matrix, old system
\subp{slaPreces( system, ep0, ep1, \&ra, \&dc );}
   Precession, in either the old or the new system
\end{callset}
 
\callhead{Proper Motion}
\begin{callset}
\subp{slaPm( r0, d0, pr, pd, px, rv, ep0, ep1, \&r1, \&d1 );}
   Adjust for proper motion
\end{callset}
 
\callhead{FK4/FK5/Hipparcos Transformations}
\begin{callset}
\subp{slaFk425( \vtop
  {\hbox{r1950, d1950, dr1950, dd1950, p1950, v1950,}
   \hbox{\&r2000, \&d2000, \&dr2000, \&dd2000, \&p2000, \&v2000 );}}} \\[-4pt]
   Convert B1950.0 FK4 star data to J2000.0 FK5
\subp{slaFk45z( r1950, d1950, epoch, \&r2000, \&d2000 );}
   Convert B1950.0 FK4 position to J2000.0 FK5 assuming zero
   FK5 proper motion and no parallax
\subp{slaFk524( \vtop
  {\hbox{r2000, d2000, dr2000, dd2000, p2000, v2000,}
   \hbox{\&r1950, \&d1950, \&dr1950, \&dd1950, \&p1950, \&v1950 );}}} \\[-4pt]
   Convert J2000.0 FK5 star data to B1950.0 FK4
\subp{slaFk54z( r2000, d2000, bepoch,
               \&r1950, \&d1950, \&dr1950, \&dd1950 );}
   Convert J2000.0 FK5 position to B1950.0 FK4 assuming zero
   FK5 proper motion and no parallax
\subp{slaFk52h( r5, d5, dr5, dd5, \&rh, \&dh, \&drh, \&ddh );}
   Convert J2000.0 FK5 star data to Hipparcos
\subp{slaFk5hz( r5, d5, epoch, \&rh, \&dh  );}
   Convert J2000.0 FK5 position to Hipparcos assuming zero Hipparcos
   proper motion
\subp{slaH2fk5( rh, dh, drh, ddh, \&r5, \&d5, \&dr5, \&dd5 );}
   Convert Hipparcos star data to J2000.0 FK5
\subp{slaHfk5z( rh, dh, epoch, \&r5, \&d5, \&dr5, \&dd5 );}
   Convert Hipparcos position to J2000.0 FK5 assuming zero Hipparcos
   proper motion
\subp{slaDbjin( string, \&nstrt, \&dreslt, \&j1, \&j2 );}
   Like slaDfltin but with extensions to accept leading `B' and `J'
\subp{slaKbj( jb, e, \&k, \&j );}
   Select epoch prefix `B' or `J'
\subp{d~=~slaEpco ( k0, k, e );}
   Convert an epoch into the appropriate form -- `B' or `J'
\end{callset}
 
\callhead{Elliptic Aberration}
\begin{callset}
\subp{slaEtrms( ep, ev );}
   E-terms
\subp{slaSubet( rc, dc, eq, \&rm, \&dm );}
   Remove the E-terms
\subp{slaAddet( rm, dm, eq, \&rc, \&dc );}
   Add the E-terms
\end{callset}
 
\callhead{Geographical and Geocentric Coordinates}
\begin{callset}
\subp{slaObs( number, id, name, \&wlong, \&phi, \&height );}
   Interrogate list of observatory parameters
\subp{slaGeoc( p, h, \&r, \&z );}
   Convert geodetic position to geocentric
\subp{slaPolmo( elongm, phim, xp, yp, \&elong, \&phi, \&daz );}
   Polar motion
\subp{slaPvobs( p, h, stl, pv );}
   Position and velocity of observatory
\end{callset}
 
\callhead{Apparent and Observed Place}
\begin{callset}
\subp{slaMap( rm, dm, pr, pd, px, rv, eq, date, \&ra, \&da );}
   Mean place to geocentric apparent place
\subp{slaMappa( eq, date, amprms );}
   Precompute mean to apparent parameters
\subp{slaMapqk( rm, dm, pr, pd, px, rv, amprms, \&ra, \&da );}
   Mean to apparent using precomputed parameters
\subp{slaMapqkz( rm, dm, amprms, \&ra, \&da );}
   Mean to apparent using precomputed parameters, for zero proper
     motion, parallax and radial velocity
\subp{slaAmp( ra, da, date, eq, \&rm, \&dm );}
   Geocentric apparent place to mean place
\subp{slaAmpqk( ra, da, amprms, \&rm, \&dm );}
   Apparent to mean using precomputed parameters
\subp{slaAop( \vtop
               {\hbox{rap, dap, utc, dut, elongm, phim, hm, xp, yp,}
                \hbox{tdk, pmb, rh, wl, tlr,
                      \&aob, \&zob, \&hob, \&dob, \&rob );}}} \\[-4pt]
   Apparent place to observed place
\subp{slaAoppa( \vtop
                {\hbox{utc, dut, elongm, phim, hm, xp, yp,}
                 \hbox{tdk, pmb, rh, wl, tlr, aoprms );}}} \\[-4pt]
   Precompute apparent to observed parameters
\subp{slaAoppat( utc, aoprms );}
   Update sidereal time in apparent to observed parameters
\subp{slaAopqk( rap, dap, aoprms, \&aob, \&zob, \&hob, \&dob, \&rob );}
   Apparent to observed using precomputed parameters
\subp{slaOap( \vtop
              {\hbox{type, ob1, ob2, utc, dut, elongm, phim, hm, xp, yp,}
               \hbox{tdk, pmb, rh, wl, tlr, \&rap, \&dap );}}} \\[-4pt]
   Observed to apparent
\subp{slaOapqk( type, ob1, ob2, aoprms, \&ra, \&da );}
   Observed to apparent using precomputed parameters
\end{callset}
 
\callhead{Azimuth and Elevation}
\begin{callset}
\subp{slaAltaz( \vtop
               {\hbox{ha, dec, phi,}
                \hbox{\&az, \&azd, \&azdd,
                      \&el, \&eld, \&eldd,
                      \&pa, \&pad, \&padd );}}} \\[-4pt]
   Positions, velocities {\it etc.}\ for an altazimuth mount
\subq{slaE2h( ha, dec, phi, \&az, \&el );}
     {slaDe2h( ha, dec, phi, \&az, \&el );}
   \hadec\ to \azel
\subq{slaH2e( az, el, phi, \&ha, \&dec );}
     {slaDh2e( az, el, phi, \&ha, \&dec );}
   \azel\ to \hadec
\subp{slaPda2h( p, d, a, \&h1, \&j1, \&h2, \&j2 );}
   Hour angle corresponding to a given azimuth
\subp{slaPdq2h( p, d, q, \&h1, \&j1, \&h2, \&j2 );}
   Hour angle corresponding to a given parallactic angle
\subp{d~=~slaPa( ha, dec, phi );}
   \hadec\ to parallactic angle
\subp{d~=~slaZd( ha, dec, phi );}
   \hadec\ to zenith distance
\end{callset}
 
\callhead{Refraction and Air Mass}
\begin{callset}
\subp{slaRefro( zobs, hm, tdk, pmb, rh, wl, phi, tlr, eps, \&ref );}
   Change in zenith distance due to refraction
\subp{slaRefco( hm, tdk, pmb, rh, wl, phi, tlr, eps, \&refa, \&refb );}
   Constants for simple refraction model ( accurate )
\subp{slaRefcoq( tdk, pmb, rh, wl, \&refa, \&refb );}
   Constants for simple refraction model ( fast )
\subp{slaAtmdsp(  tdk, pmb, rh, wl1, refa1, refb1, wl2, \&refa2, \&refb2  );}
   Adjust refraction constants for colour
\subp{slaRefz( zu, refa, refb, \&zr );}
   Unrefracted to refracted ZD, simple model
\subp{slaRefv( vu, refa, refb, vr );}
   Unrefracted to refracted \azel\ vector, simple model
\subp{d~=~slaAirmas( zd );}
   Air mass
\end{callset}
 
\callhead{Ecliptic Coordinates}
\begin{callset}
\subp{slaEcmat( date, rmat );}
   Equatorial to ecliptic rotation matrix
\subp{slaEqecl( dr, dd, date, \&dl, \&db );}
   J2000.0 `FK5' to ecliptic coordinates
\subp{slaEcleq( dl, db, date, \&dr, \&dd );}
   Ecliptic coordinates to J2000.0 `FK5'
\end{callset}
 
\callhead{Galactic Coordinates}
\begin{callset}
\subp{slaEg50( dr, dd, \&dl, \&db );}
   B1950.0 `FK4' to galactic
\subp{slaGe50( dl, db, \&dr, \&dd );}
   Galactic to B1950.0 `FK4'
\subp{slaEqgal( dr, dd, \&dl, \&db );}
   J2000.0 `FK5' to galactic
\subp{slaGaleq( dl, db, \&dr, \&dd );}
   Galactic to J2000.0 `FK5'
\end{callset}
 
\callhead{Supergalactic Coordinates}
\begin{callset}
\subp{slaGalsup( dl, db, \&dsl, \&dsb );}
   Galactic to supergalactic
\subp{slaSupgal( dsl, dsb, \&dl, \&db );}
   Supergalactic to galactic
\end{callset}
 
\callhead{Ephemerides}
\begin{callset}
\subp{slaDmoon( date, pv );}
   Approximate geocentric position and velocity of the Moon
\subp{slaEarth( iy, id, fd, pv );}
   Approximate heliocentric position and velocity of the Earth
\subp{slaEpv( date, dph, dvh, dpb, dvb  );}
   Heliocentric and barycentric position and velocity of the Earth
   (high-accuracy)
\subp{slaEvp( date, deqx, dvb, dpb, dvh, dph );}
   Barycentric and heliocentric velocity and position of the Earth
\subp{slaMoon( iy, id, fd, pv );}
   Approximate geocentric position and velocity of the Moon
\subp{slaPlanet( date, np, pv, \&jstat );}
   Approximate heliocentric position and velocity of a planet
\subp{slaRdplan( date, np, elong, phi, \&ra, \&dec, \&diam );}
   Approximate topocentric apparent place of a planet
\subp{slaPlanel( \vtop
   {\hbox{date, jform, epoch, orbinc, anode, perih,}
    \hbox{aorq, e, aorl, dm, pv, \&jstat );}}} \\[-4pt]
   Heliocentric position and velocity of a planet, asteroid or
   comet, starting from orbital elements
\subp{slaPlante( \vtop
   {\hbox{date, elong, phi, jform, epoch, orbinc, anode,}
    \hbox{perih, aorq, e, aorl, dm, \&ra, \&dec, \&r, \&jstat );}}} \\[-4pt]
   Topocentric apparent place of a Solar-System object whose
   heliocentric orbital elements are known
\subp{slaPlantu( date, elong, phi, u, \&ra, \&dec, \&r, \&jstat );}
   Topocentric apparent place of a Solar-System object whose
   heliocentric universal orbital elements are known
\subp{slaPv2el( \vtop
   {\hbox{pv, date, pmass, jformr, \&jform, \&epoch, \&orbinc,}
    \hbox{\&anode, \&perih, \&aorq, \&e, \&aorl, \&dm, \&jstat );}}} \\[-4pt]
   Orbital elements of a planet from instantaneous position and velocity
\subp{slaPertel( \vtop
   {\hbox{jform, date0, date1,}
    \hbox{epoch0, orbi0, anode0, perih0, aorq0, e0, am0,}
    \hbox{\&epoch1, \&orbi1, \&anode1, \&perih1, \&aorq1, \&e1, \&am1,}
    \hbox{\&jstat );}}} \\[-4pt]
   Update elements by applying perturbations
\subp{slaEl2ue( \vtop
   {\hbox{date, jform, epoch, orbinc, anode,}
    \hbox{perih, aorq, e, aorl, dm,}
    \hbox{u, \&jstat );}}} \\[-4pt]
   Transform conventional elements to universal elements
\subp{slaUe2el( \vtop
    {\hbox{u, jformr,}
     \hbox{\&jform, \&epoch, \&orbinc, \&anode, \&perih,}
     \hbox{\&aorq, \&e, \&aorl, \&dm, \&jstat );}}} \\[-4pt]
   Transform universal elements to conventional elements
\subp{slaPv2ue( pv, date, pmass, u, \&jstat );}
   Package a position and velocity for use as universal elements
\subp{slaUe2pv( date, u, pv, \&jstat );}
   Extract the position and velocity from universal elements
\subp{slaPertue( date, u, \&jstat );}
   Update universal elements by applying perturbations
\subp{r~=~slaRverot( phi, ra, da, st );}
   Velocity component due to rotation of the Earth
\subp{slaEcor( rm, dm, iy, id, fd, \&rv, \&tl );}
   Components of velocity and light time due to Earth orbital motion
\subp{r~=~slaRvlsrd( r2000, d2000 );}
   Velocity component due to solar motion wrt dynamical LSR
\subp{r~=~slaRvlsrk( r2000, d2000 );}
   Velocity component due to solar motion wrt kinematical LSR
\subp{r~=~slaRvgalc( r2000, d2000 );}
   Velocity component due to rotation of the Galaxy
\subp{r~=~slaRvlg( r2000, d2000 );}
   Velocity component due to rotation and translation of the
   Galaxy, relative to the mean motion of the local group
\end{callset}
 
\callhead{Astrometry}
\begin{callset}
\subq{slaS2tp( ra, dec, raz, decz, \&xi, \&eta, \&j );}
     {slaDs2tp( ra, dec, raz, decz, \&xi, \&eta, \&j );}
   Transform spherical coordinates into tangent plane
\subq{slaV2tp( v, v0, \&xi, \&eta, \&j );}
     {slaDv2tp( v, v0, \&xi, \&eta, \&j );}
   Transform \xyz\ into tangent plane coordinates
\subq{slaDtp2s( xi, eta, raz, decz, \&ra, \&dec );}
     {slaTp2s( xi, eta, raz, decz, \&ra, \&dec );}
   Transform tangent plane coordinates into spherical coordinates
\subq{slaDtp2v( xi, eta, v0, v );}
     {slaTp2v( xi, eta, v0, v );}
   Transform tangent plane coordinates into \xyz
\subq{slaDtps2c( xi, eta, ra, dec, \&raz1, \&decz1, \&raz2, \&decz2, \&n );}
     {slaTps2c( xi, eta, ra, dec, \&raz1, \&decz1, \&raz2, \&decz2, \&n );}
   Get plate centre from star \radec\ and tangent plane coordinates
\subq{slaDtpv2c( xi, eta, v, v01, v02, \&n );}
     {slaTpv2c( xi, eta, v, v01, v02, \&n );}
   Get plate centre from star \xyz\ and tangent plane coordinates
\subp{slaPcd( disco, \&x, \&y );}
   Apply pincushion/barrel distortion
\subp{slaUnpcd( disco, \&x, \&y );}
   Remove pincushion/barrel distortion
\subp{slaFitxy( itype, np, xye, xym, coeffs, \&j );}
   Fit a linear model to relate two sets of \xy\ coordinates
\subp{slaPxy( np, xye, xym, coeffs, xyp, \&xrms, \&yrms, \&rrms );}
   Compute predicted coordinates and residuals
\subp{slaInvf( fwds, bkwds, \&j );}
   Invert a linear model
\subp{slaXy2xy( x1, y1, coeffs, \&x2, \&y2 );}
   Transform one \xy
\subp{slaDcmpf( coeffs, \&xz, \&yz, \&xs, \&ys, \&perp, \&orient );}
   Decompose a linear fit into scales {\it etc.}
\end{callset}
 
\callhead{Numerical Methods}
\begin{callset}
\subp{slaCombn( nsel, ncand, list, \&j );}
   Next combination ( subset from a specified number of items )
\subp{slaPermut( n, istate, iorder, \&j );}
   Next permutation of a specified number of items
\subq{slaSmat( n, a, y, \&d, \&jf, iw );}
     {slaDmat( n, a, y, \&d, \&jf, iw );}
   Matrix inversion and solution of simultaneous equations
\subp{slaSvd( m, n, mp, np, a, w, v, work, \&jstat );}
   Singular value decomposition of a matrix
\subp{slaSvdsol( m, n, mp, np, b, u, w, v, work, x );}
   Solution from given vector plus SVD
\subp{slaSvdcov( n, np, nc, w, v, work, cvm );}
   Covariance matrix from SVD
\end{callset}
 
\end{document}
