\documentclass[12pt]{article}

%\usepackage{times} %for Times New Roman, if required
\usepackage[top=1in, bottom=1in, left=1in, right=1in]{geometry} %adjust margins. TODO: hack to fix large top margin
\usepackage{setspace} %allows doublespacing, onehalfspacing, singlespacing
\usepackage{enumitem} %for continuing lists
\usepackage{titling} %for moving the title
\usepackage[normalem]{ulem} %for underlining
\usepackage{graphics}

%redefine section numbering to match Mayron's style
\renewcommand{\thesection}{\arabic{section}} %make sections use numbers
\renewcommand{\thesubsection}{\alph{subsection}} %make subsections use letters
\renewcommand{\thesubsubsection}{\roman{subsubsection}} %subsubsections use romans


\begin{document}

\begin{spacing}{.4}
\setlength{\droptitle}{-7em}
\title{CSE 485 Semester Report \\  Team 1, Friday 10:30am}
\author{Connor Alfheim \and Ryan Dougherty \and David Ganey \and Dylan Lusi \and Joseph North \and Ben Roos}
\maketitle
\end{spacing}

\begin{spacing}{1.5}
\noindent
Project sponsors: Dr. Judd Bowman and Dr. Danny Jacobs \\
Project description: A virtual observatory for the Murchison Widefield Array radio telescope.
\newpage

\tableofcontents

\section{Executive Summary}

\section{Introduction}
\subsection{Project description}
This semester, we produced a Web site that functions as a data browser for the MWA telescope's observations. It allows a user to log in with their account and select a date range using selectors at the top of the page and see a list of telescope observations that happened in that range (or scheduled observations that will happen, if the range extends into the future). Specifically, the user can see the observation ID (the unique identifier for the observation in the system), the observation name, and who scheduled the observation. Additionally, the user can see a graph that shows the status of the telescope's data pipeline over that date range. This graph shows the total observation hours scheduled for the telescope versus how many hours have actually been observed, and how many hours' worth of observation data have been transferred from the telescope to databases at MIT.
\subsection{Purpose of project}

\section{Scope}
\subsection{Original Definition}
\subsection{Change of scope and reason for change}

\section{User Overview}
\subsection{Use case diagram}
\subsection{Description of actors}
\subsection{Use cases/user stories}

\section{Project Plan}
\subsection{First semester}
\subsection{Second Semester}

\section{Development Approach}

\section{Design Overview and Decisions}

\section{Technology and Tools}
fdsfsdfsd
\section{Preliminary results}

\section{Problems and risks}

\section{Summary of Tasks}
\subsection{Connor Alfheim}
\subsubsection{Team Presentation}
\subsubsection{Report}
\subsubsection{Product}
\subsubsection{Initialization Document}
\subsubsection{Team Management}
\subsection{Ryan Dougherty}
\subsubsection{Team Presentation}
\subsubsection{Report}
fdsfsdsdf
\subsubsection{Product}
\subsubsection{Initialization Document}
\subsubsection{Team Management}
\subsection{David Ganey}
\subsubsection{Team Presentation}
\subsubsection{Report}
\subsubsection{Product}
\subsubsection{Initialization Document}
\subsubsection{Team Management}
\subsection{Dylan Lusi}
\subsubsection{Team Presentation}
\subsubsection{Report}
\subsubsection{Product}
\subsubsection{Initialization Document}
\subsubsection{Team Management}
\subsection{Joseph North}
\subsubsection{Team Presentation}
\subsubsection{Report}
\subsubsection{Product}
\subsubsection{Initialization Document}
\subsubsection{Team Management}
\subsection{Ben Roos}
\subsubsection{Team Presentation}
\subsubsection{Report}
fdsfdsa
\subsubsection{Product}
\subsubsection{Initialization Document}
\subsubsection{Team Management}



\section{Conclusions}
\subsection{Success So Far}
\subsection{Lessons Learned}
\subsection{Future Work}


\end{spacing}
\end{document}

